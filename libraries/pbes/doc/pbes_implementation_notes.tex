%--- From tcilatex.tex ---%
%\def\dsum{\mathop{\displaystyle \sum }}%
%\def\dprod{\mathop{\displaystyle \prod }}%
%\def\dbigcap{\mathop{\displaystyle \bigcap }}%
%\def\dbigwedge{\mathop{\displaystyle \bigwedge }}%
%\def\dbigoplus{\mathop{\displaystyle \bigoplus }}%
%\def\dbigodot{\mathop{\displaystyle \bigodot }}%
%\def\dbigsqcup{\mathop{\displaystyle \bigsqcup }}%
%\def\dcoprod{\mathop{\displaystyle \coprod }}%
%\def\dbigcup{\mathop{\displaystyle \bigcup }}%
%\def\dbigvee{\mathop{\displaystyle \bigvee }}%
%\def\dbigotimes{\mathop{\displaystyle \bigotimes }}%
%\def\dbiguplus{\mathop{\displaystyle \biguplus }}%
%\def\QQfnmark#1{\footnotemark}
%\def\QQfntext#1#2{\addtocounter{footnote}{#1}\footnotetext{#2}}
%--------------------------%


\documentclass{article}
%%%%%%%%%%%%%%%%%%%%%%%%%%%%%%%%%%%%%%%%%%%%%%%%%%%%%%%%%%%%%%%%%%%%%%%%%%%%%%%%%%%%%%%%%%%%%%%%%%%%%%%%%%%%%%%%%%%%%%%%%%%%%%%%%%%%%%%%%%%%%%%%%%%%%%%%%%%%%%%%%%%%%%%%%%%%%%%%%%%%%%%%%%%%%%%%%%%%%%%%%%%%%%%%%%%%%%%%%%%%%%%%%%%%%%%%%%%%%%%%%%%%%%%%%%%%
\usepackage{fullpage}
\usepackage{amsmath}
\usepackage{amsfonts}
\usepackage{amssymb}

\setcounter{MaxMatrixCols}{10}
%TCIDATA{OutputFilter=LATEX.DLL}
%TCIDATA{Version=5.50.0.2890}
%TCIDATA{<META NAME="SaveForMode" CONTENT="1">}
%TCIDATA{BibliographyScheme=Manual}
%TCIDATA{Created=Tuesday, January 15, 2008 14:02:49}
%TCIDATA{LastRevised=Monday, December 22, 2008 16:03:40}
%TCIDATA{<META NAME="GraphicsSave" CONTENT="32">}
%TCIDATA{<META NAME="DocumentShell" CONTENT="Standard LaTeX\Blank - Standard LaTeX Article">}
%TCIDATA{CSTFile=40 LaTeX article.cst}

\font \aap cmmi10
\newcommand{\at}[1]{\mbox{\aap ,} #1}
\newcommand{\ap}{{:}}
\newcommand{\tuple}[1]{\ensuremath{\langle {#1} \rangle}}
\newtheorem{theorem}{Theorem}
\newtheorem{acknowledgement}[theorem]{Acknowledgement}
\newtheorem{algorithm}[theorem]{Algorithm}
\newtheorem{axiom}[theorem]{Axiom}
\newtheorem{case}[theorem]{Case}
\newtheorem{claim}[theorem]{Claim}
\newtheorem{conclusion}[theorem]{Conclusion}
\newtheorem{condition}[theorem]{Condition}
\newtheorem{conjecture}[theorem]{Conjecture}
\newtheorem{corollary}[theorem]{Corollary}
\newtheorem{criterion}[theorem]{Criterion}
\newtheorem{definition}[theorem]{Definition}
\newtheorem{example}[theorem]{Example}
\newtheorem{exercise}[theorem]{Exercise}
\newtheorem{lemma}[theorem]{Lemma}
\newtheorem{notation}[theorem]{Notation}
\newtheorem{problem}[theorem]{Problem}
\newtheorem{proposition}[theorem]{Proposition}
\newtheorem{remark}[theorem]{Remark}
\newtheorem{solution}[theorem]{Solution}
\newtheorem{summary}[theorem]{Summary}
\newenvironment{proof}[1][Proof]{\noindent\textbf{#1.} }{\ \rule{0.5em}{0.5em}}
% Macros for Scientific Word 4.0 documents saved with the LaTeX filter.
% Copyright (C) 2002 Mackichan Software, Inc.

\typeout{TCILATEX Macros for Scientific Word 5.0 <13 Feb 2003>.}
\typeout{NOTICE:  This macro file is NOT proprietary and may be 
freely copied and distributed.}
%
\makeatletter

%%%%%%%%%%%%%%%%%%%%%
% pdfTeX related.
\ifx\pdfoutput\relax\let\pdfoutput=\undefined\fi
\newcount\msipdfoutput
\ifx\pdfoutput\undefined
\else
 \ifcase\pdfoutput
 \else 
    \msipdfoutput=1
    \ifx\paperwidth\undefined
    \else
      \ifdim\paperheight=0pt\relax
      \else
        \pdfpageheight\paperheight
      \fi
      \ifdim\paperwidth=0pt\relax
      \else
        \pdfpagewidth\paperwidth
      \fi
    \fi
  \fi  
\fi

%%%%%%%%%%%%%%%%%%%%%
% FMTeXButton
% This is used for putting TeXButtons in the 
% frontmatter of a document. Add a line like
% \QTagDef{FMTeXButton}{101}{} to the filter 
% section of the cst being used. Also add a
% new section containing:
%     [f_101]
%     ALIAS=FMTexButton
%     TAG_TYPE=FIELD
%     TAG_LEADIN=TeX Button:
%
% It also works to put \defs in the preamble after 
% the \input tcilatex
\def\FMTeXButton#1{#1}
%
%%%%%%%%%%%%%%%%%%%%%%
% macros for time
\newcount\@hour\newcount\@minute\chardef\@x10\chardef\@xv60
\def\tcitime{
\def\@time{%
  \@minute\time\@hour\@minute\divide\@hour\@xv
  \ifnum\@hour<\@x 0\fi\the\@hour:%
  \multiply\@hour\@xv\advance\@minute-\@hour
  \ifnum\@minute<\@x 0\fi\the\@minute
  }}%

%%%%%%%%%%%%%%%%%%%%%%
% macro for hyperref and msihyperref
%\@ifundefined{hyperref}{\def\hyperref#1#2#3#4{#2\ref{#4}#3}}{}

\def\x@hyperref#1#2#3{%
   % Turn off various catcodes before reading parameter 4
   \catcode`\~ = 12
   \catcode`\$ = 12
   \catcode`\_ = 12
   \catcode`\# = 12
   \catcode`\& = 12
   \y@hyperref{#1}{#2}{#3}%
}

\def\y@hyperref#1#2#3#4{%
   #2\ref{#4}#3
   \catcode`\~ = 13
   \catcode`\$ = 3
   \catcode`\_ = 8
   \catcode`\# = 6
   \catcode`\& = 4
}

\@ifundefined{hyperref}{\let\hyperref\x@hyperref}{}
\@ifundefined{msihyperref}{\let\msihyperref\x@hyperref}{}




% macro for external program call
\@ifundefined{qExtProgCall}{\def\qExtProgCall#1#2#3#4#5#6{\relax}}{}
%%%%%%%%%%%%%%%%%%%%%%
%
% macros for graphics
%
\def\FILENAME#1{#1}%
%
\def\QCTOpt[#1]#2{%
  \def\QCTOptB{#1}
  \def\QCTOptA{#2}
}
\def\QCTNOpt#1{%
  \def\QCTOptA{#1}
  \let\QCTOptB\empty
}
\def\Qct{%
  \@ifnextchar[{%
    \QCTOpt}{\QCTNOpt}
}
\def\QCBOpt[#1]#2{%
  \def\QCBOptB{#1}%
  \def\QCBOptA{#2}%
}
\def\QCBNOpt#1{%
  \def\QCBOptA{#1}%
  \let\QCBOptB\empty
}
\def\Qcb{%
  \@ifnextchar[{%
    \QCBOpt}{\QCBNOpt}%
}
\def\PrepCapArgs{%
  \ifx\QCBOptA\empty
    \ifx\QCTOptA\empty
      {}%
    \else
      \ifx\QCTOptB\empty
        {\QCTOptA}%
      \else
        [\QCTOptB]{\QCTOptA}%
      \fi
    \fi
  \else
    \ifx\QCBOptA\empty
      {}%
    \else
      \ifx\QCBOptB\empty
        {\QCBOptA}%
      \else
        [\QCBOptB]{\QCBOptA}%
      \fi
    \fi
  \fi
}
\newcount\GRAPHICSTYPE
%\GRAPHICSTYPE 0 is for TurboTeX
%\GRAPHICSTYPE 1 is for DVIWindo (PostScript)
%%%(removed)%\GRAPHICSTYPE 2 is for psfig (PostScript)
\GRAPHICSTYPE=\z@
\def\GRAPHICSPS#1{%
 \ifcase\GRAPHICSTYPE%\GRAPHICSTYPE=0
   \special{ps: #1}%
 \or%\GRAPHICSTYPE=1
   \special{language "PS", include "#1"}%
%%%\or%\GRAPHICSTYPE=2
%%%  #1%
 \fi
}%
%
\def\GRAPHICSHP#1{\special{include #1}}%
%
% \graffile{ body }                                  %#1
%          { contentswidth (scalar)  }               %#2
%          { contentsheight (scalar) }               %#3
%          { vertical shift when in-line (scalar) }  %#4

\def\graffile#1#2#3#4{%
%%% \ifnum\GRAPHICSTYPE=\tw@
%%%  %Following if using psfig
%%%  \@ifundefined{psfig}{\input psfig.tex}{}%
%%%  \psfig{file=#1, height=#3, width=#2}%
%%% \else
  %Following for all others
  % JCS - added BOXTHEFRAME, see below
    \bgroup
	   \@inlabelfalse
       \leavevmode
       \@ifundefined{bbl@deactivate}{\def~{\string~}}{\activesoff}%
        \raise -#4 \BOXTHEFRAME{%
           \hbox to #2{\raise #3\hbox to #2{\null #1\hfil}}}%
    \egroup
}%
%
% A box for drafts
\def\draftbox#1#2#3#4{%
 \leavevmode\raise -#4 \hbox{%
  \frame{\rlap{\protect\tiny #1}\hbox to #2%
   {\vrule height#3 width\z@ depth\z@\hfil}%
  }%
 }%
}%
%
\newcount\@msidraft
\@msidraft=\z@
\let\nographics=\@msidraft
\newif\ifwasdraft
\wasdraftfalse

%  \GRAPHIC{ body }                                  %#1
%          { draft name }                            %#2
%          { contentswidth (scalar)  }               %#3
%          { contentsheight (scalar) }               %#4
%          { vertical shift when in-line (scalar) }  %#5
\def\GRAPHIC#1#2#3#4#5{%
   \ifnum\@msidraft=\@ne\draftbox{#2}{#3}{#4}{#5}%
   \else\graffile{#1}{#3}{#4}{#5}%
   \fi
}
%
\def\addtoLaTeXparams#1{%
    \edef\LaTeXparams{\LaTeXparams #1}}%
%
% JCS -  added a switch BoxFrame that can 
% be set by including X in the frame params.
% If set a box is drawn around the frame.

\newif\ifBoxFrame \BoxFramefalse
\newif\ifOverFrame \OverFramefalse
\newif\ifUnderFrame \UnderFramefalse

\def\BOXTHEFRAME#1{%
   \hbox{%
      \ifBoxFrame
         \frame{#1}%
      \else
         {#1}%
      \fi
   }%
}


\def\doFRAMEparams#1{\BoxFramefalse\OverFramefalse\UnderFramefalse\readFRAMEparams#1\end}%
\def\readFRAMEparams#1{%
 \ifx#1\end%
  \let\next=\relax
  \else
  \ifx#1i\dispkind=\z@\fi
  \ifx#1d\dispkind=\@ne\fi
  \ifx#1f\dispkind=\tw@\fi
  \ifx#1t\addtoLaTeXparams{t}\fi
  \ifx#1b\addtoLaTeXparams{b}\fi
  \ifx#1p\addtoLaTeXparams{p}\fi
  \ifx#1h\addtoLaTeXparams{h}\fi
  \ifx#1X\BoxFrametrue\fi
  \ifx#1O\OverFrametrue\fi
  \ifx#1U\UnderFrametrue\fi
  \ifx#1w
    \ifnum\@msidraft=1\wasdrafttrue\else\wasdraftfalse\fi
    \@msidraft=\@ne
  \fi
  \let\next=\readFRAMEparams
  \fi
 \next
 }%
%
%Macro for In-line graphics object
%   \IFRAME{ contentswidth (scalar)  }               %#1
%          { contentsheight (scalar) }               %#2
%          { vertical shift when in-line (scalar) }  %#3
%          { draft name }                            %#4
%          { body }                                  %#5
%          { caption}                                %#6


\def\IFRAME#1#2#3#4#5#6{%
      \bgroup
      \let\QCTOptA\empty
      \let\QCTOptB\empty
      \let\QCBOptA\empty
      \let\QCBOptB\empty
      #6%
      \parindent=0pt
      \leftskip=0pt
      \rightskip=0pt
      \setbox0=\hbox{\QCBOptA}%
      \@tempdima=#1\relax
      \ifOverFrame
          % Do this later
          \typeout{This is not implemented yet}%
          \show\HELP
      \else
         \ifdim\wd0>\@tempdima
            \advance\@tempdima by \@tempdima
            \ifdim\wd0 >\@tempdima
               \setbox1 =\vbox{%
                  \unskip\hbox to \@tempdima{\hfill\GRAPHIC{#5}{#4}{#1}{#2}{#3}\hfill}%
                  \unskip\hbox to \@tempdima{\parbox[b]{\@tempdima}{\QCBOptA}}%
               }%
               \wd1=\@tempdima
            \else
               \textwidth=\wd0
               \setbox1 =\vbox{%
                 \noindent\hbox to \wd0{\hfill\GRAPHIC{#5}{#4}{#1}{#2}{#3}\hfill}\\%
                 \noindent\hbox{\QCBOptA}%
               }%
               \wd1=\wd0
            \fi
         \else
            \ifdim\wd0>0pt
              \hsize=\@tempdima
              \setbox1=\vbox{%
                \unskip\GRAPHIC{#5}{#4}{#1}{#2}{0pt}%
                \break
                \unskip\hbox to \@tempdima{\hfill \QCBOptA\hfill}%
              }%
              \wd1=\@tempdima
           \else
              \hsize=\@tempdima
              \setbox1=\vbox{%
                \unskip\GRAPHIC{#5}{#4}{#1}{#2}{0pt}%
              }%
              \wd1=\@tempdima
           \fi
         \fi
         \@tempdimb=\ht1
         %\advance\@tempdimb by \dp1
         \advance\@tempdimb by -#2
         \advance\@tempdimb by #3
         \leavevmode
         \raise -\@tempdimb \hbox{\box1}%
      \fi
      \egroup%
}%
%
%Macro for Display graphics object
%   \DFRAME{ contentswidth (scalar)  }               %#1
%          { contentsheight (scalar) }               %#2
%          { draft label }                           %#3
%          { name }                                  %#4
%          { caption}                                %#5
\def\DFRAME#1#2#3#4#5{%
  \vspace\topsep
  \hfil\break
  \bgroup
     \leftskip\@flushglue
	 \rightskip\@flushglue
	 \parindent\z@
	 \parfillskip\z@skip
     \let\QCTOptA\empty
     \let\QCTOptB\empty
     \let\QCBOptA\empty
     \let\QCBOptB\empty
	 \vbox\bgroup
        \ifOverFrame 
           #5\QCTOptA\par
        \fi
        \GRAPHIC{#4}{#3}{#1}{#2}{\z@}%
        \ifUnderFrame 
           \break#5\QCBOptA
        \fi
	 \egroup
  \egroup
  \vspace\topsep
  \break
}%
%
%Macro for Floating graphic object
%   \FFRAME{ framedata f|i tbph x F|T }              %#1
%          { contentswidth (scalar)  }               %#2
%          { contentsheight (scalar) }               %#3
%          { caption }                               %#4
%          { label }                                 %#5
%          { draft name }                            %#6
%          { body }                                  %#7
\def\FFRAME#1#2#3#4#5#6#7{%
 %If float.sty loaded and float option is 'h', change to 'H'  (gp) 1998/09/05
  \@ifundefined{floatstyle}
    {%floatstyle undefined (and float.sty not present), no change
     \begin{figure}[#1]%
    }
    {%floatstyle DEFINED
	 \ifx#1h%Only the h parameter, change to H
      \begin{figure}[H]%
	 \else
      \begin{figure}[#1]%
	 \fi
	}
  \let\QCTOptA\empty
  \let\QCTOptB\empty
  \let\QCBOptA\empty
  \let\QCBOptB\empty
  \ifOverFrame
    #4
    \ifx\QCTOptA\empty
    \else
      \ifx\QCTOptB\empty
        \caption{\QCTOptA}%
      \else
        \caption[\QCTOptB]{\QCTOptA}%
      \fi
    \fi
    \ifUnderFrame\else
      \label{#5}%
    \fi
  \else
    \UnderFrametrue%
  \fi
  \begin{center}\GRAPHIC{#7}{#6}{#2}{#3}{\z@}\end{center}%
  \ifUnderFrame
    #4
    \ifx\QCBOptA\empty
      \caption{}%
    \else
      \ifx\QCBOptB\empty
        \caption{\QCBOptA}%
      \else
        \caption[\QCBOptB]{\QCBOptA}%
      \fi
    \fi
    \label{#5}%
  \fi
  \end{figure}%
 }%
%
%
%    \FRAME{ framedata f|i tbph x F|T }              %#1
%          { contentswidth (scalar)  }               %#2
%          { contentsheight (scalar) }               %#3
%          { vertical shift when in-line (scalar) }  %#4
%          { caption }                               %#5
%          { label }                                 %#6
%          { name }                                  %#7
%          { body }                                  %#8
%
%    framedata is a string which can contain the following
%    characters: idftbphxFT
%    Their meaning is as follows:
%             i, d or f : in-line, display, or floating
%             t,b,p,h   : LaTeX floating placement options
%             x         : fit contents box to contents
%             F or T    : Figure or Table. 
%                         Later this can expand
%                         to a more general float class.
%
%
\newcount\dispkind%

\def\makeactives{
  \catcode`\"=\active
  \catcode`\;=\active
  \catcode`\:=\active
  \catcode`\'=\active
  \catcode`\~=\active
}
\bgroup
   \makeactives
   \gdef\activesoff{%
      \def"{\string"}%
      \def;{\string;}%
      \def:{\string:}%
      \def'{\string'}%
      \def~{\string~}%
      %\bbl@deactivate{"}%
      %\bbl@deactivate{;}%
      %\bbl@deactivate{:}%
      %\bbl@deactivate{'}%
    }
\egroup

\def\FRAME#1#2#3#4#5#6#7#8{%
 \bgroup
 \ifnum\@msidraft=\@ne
   \wasdrafttrue
 \else
   \wasdraftfalse%
 \fi
 \def\LaTeXparams{}%
 \dispkind=\z@
 \def\LaTeXparams{}%
 \doFRAMEparams{#1}%
 \ifnum\dispkind=\z@\IFRAME{#2}{#3}{#4}{#7}{#8}{#5}\else
  \ifnum\dispkind=\@ne\DFRAME{#2}{#3}{#7}{#8}{#5}\else
   \ifnum\dispkind=\tw@
    \edef\@tempa{\noexpand\FFRAME{\LaTeXparams}}%
    \@tempa{#2}{#3}{#5}{#6}{#7}{#8}%
    \fi
   \fi
  \fi
  \ifwasdraft\@msidraft=1\else\@msidraft=0\fi{}%
  \egroup
 }%
%
% This macro added to let SW gobble a parameter that
% should not be passed on and expanded. 

\def\TEXUX#1{"texux"}

%
% Macros for text attributes:
%
\def\BF#1{{\bf {#1}}}%
\def\NEG#1{\leavevmode\hbox{\rlap{\thinspace/}{$#1$}}}%
%
%%%%%%%%%%%%%%%%%%%%%%%%%%%%%%%%%%%%%%%%%%%%%%%%%%%%%%%%%%%%%%%%%%%%%%%%
%
%
% macros for user - defined functions
\def\limfunc#1{\mathop{\rm #1}}%
\def\func#1{\mathop{\rm #1}\nolimits}%
% macro for unit names
\def\unit#1{\mathord{\thinspace\rm #1}}%

%
% miscellaneous 
\long\def\QQQ#1#2{%
     \long\expandafter\def\csname#1\endcsname{#2}}%
\@ifundefined{QTP}{\def\QTP#1{}}{}
\@ifundefined{QEXCLUDE}{\def\QEXCLUDE#1{}}{}
\@ifundefined{Qlb}{\def\Qlb#1{#1}}{}
\@ifundefined{Qlt}{\def\Qlt#1{#1}}{}
\def\QWE{}%
\long\def\QQA#1#2{}%
\def\QTR#1#2{{\csname#1\endcsname {#2}}}%
\long\def\TeXButton#1#2{#2}%
\long\def\QSubDoc#1#2{#2}%
\def\EXPAND#1[#2]#3{}%
\def\NOEXPAND#1[#2]#3{}%
\def\PROTECTED{}%
\def\LaTeXparent#1{}%
\def\ChildStyles#1{}%
\def\ChildDefaults#1{}%
\def\QTagDef#1#2#3{}%

% Constructs added with Scientific Notebook
\@ifundefined{correctchoice}{\def\correctchoice{\relax}}{}
\@ifundefined{HTML}{\def\HTML#1{\relax}}{}
\@ifundefined{TCIIcon}{\def\TCIIcon#1#2#3#4{\relax}}{}
\if@compatibility
  \typeout{Not defining UNICODE  U or CustomNote commands for LaTeX 2.09.}
\else
  \providecommand{\UNICODE}[2][]{\protect\rule{.1in}{.1in}}
  \providecommand{\U}[1]{\protect\rule{.1in}{.1in}}
  \providecommand{\CustomNote}[3][]{\marginpar{#3}}
\fi

\@ifundefined{lambdabar}{
      \def\lambdabar{\errmessage{You have used the lambdabar symbol. 
                      This is available for typesetting only in RevTeX styles.}}
   }{}

%
% Macros for style editor docs
\@ifundefined{StyleEditBeginDoc}{\def\StyleEditBeginDoc{\relax}}{}
%
% Macros for footnotes
\def\QQfnmark#1{\footnotemark}
\def\QQfntext#1#2{\addtocounter{footnote}{#1}\footnotetext{#2}}
%
% Macros for indexing.
%
\@ifundefined{TCIMAKEINDEX}{}{\makeindex}%
%
% Attempts to avoid problems with other styles
\@ifundefined{abstract}{%
 \def\abstract{%
  \if@twocolumn
   \section*{Abstract (Not appropriate in this style!)}%
   \else \small 
   \begin{center}{\bf Abstract\vspace{-.5em}\vspace{\z@}}\end{center}%
   \quotation 
   \fi
  }%
 }{%
 }%
\@ifundefined{endabstract}{\def\endabstract
  {\if@twocolumn\else\endquotation\fi}}{}%
\@ifundefined{maketitle}{\def\maketitle#1{}}{}%
\@ifundefined{affiliation}{\def\affiliation#1{}}{}%
\@ifundefined{proof}{\def\proof{\noindent{\bfseries Proof. }}}{}%
\@ifundefined{endproof}{\def\endproof{\mbox{\ \rule{.1in}{.1in}}}}{}%
\@ifundefined{newfield}{\def\newfield#1#2{}}{}%
\@ifundefined{chapter}{\def\chapter#1{\par(Chapter head:)#1\par }%
 \newcount\c@chapter}{}%
\@ifundefined{part}{\def\part#1{\par(Part head:)#1\par }}{}%
\@ifundefined{section}{\def\section#1{\par(Section head:)#1\par }}{}%
\@ifundefined{subsection}{\def\subsection#1%
 {\par(Subsection head:)#1\par }}{}%
\@ifundefined{subsubsection}{\def\subsubsection#1%
 {\par(Subsubsection head:)#1\par }}{}%
\@ifundefined{paragraph}{\def\paragraph#1%
 {\par(Subsubsubsection head:)#1\par }}{}%
\@ifundefined{subparagraph}{\def\subparagraph#1%
 {\par(Subsubsubsubsection head:)#1\par }}{}%
%%%%%%%%%%%%%%%%%%%%%%%%%%%%%%%%%%%%%%%%%%%%%%%%%%%%%%%%%%%%%%%%%%%%%%%%
% These symbols are not recognized by LaTeX
\@ifundefined{therefore}{\def\therefore{}}{}%
\@ifundefined{backepsilon}{\def\backepsilon{}}{}%
\@ifundefined{yen}{\def\yen{\hbox{\rm\rlap=Y}}}{}%
\@ifundefined{registered}{%
   \def\registered{\relax\ifmmode{}\r@gistered
                    \else$\m@th\r@gistered$\fi}%
 \def\r@gistered{^{\ooalign
  {\hfil\raise.07ex\hbox{$\scriptstyle\rm\text{R}$}\hfil\crcr
  \mathhexbox20D}}}}{}%
\@ifundefined{Eth}{\def\Eth{}}{}%
\@ifundefined{eth}{\def\eth{}}{}%
\@ifundefined{Thorn}{\def\Thorn{}}{}%
\@ifundefined{thorn}{\def\thorn{}}{}%
% A macro to allow any symbol that requires math to appear in text
\def\TEXTsymbol#1{\mbox{$#1$}}%
\@ifundefined{degree}{\def\degree{{}^{\circ}}}{}%
%
% macros for T3TeX files
\newdimen\theight
\@ifundefined{Column}{\def\Column{%
 \vadjust{\setbox\z@=\hbox{\scriptsize\quad\quad tcol}%
  \theight=\ht\z@\advance\theight by \dp\z@\advance\theight by \lineskip
  \kern -\theight \vbox to \theight{%
   \rightline{\rlap{\box\z@}}%
   \vss
   }%
  }%
 }}{}%
%
\@ifundefined{qed}{\def\qed{%
 \ifhmode\unskip\nobreak\fi\ifmmode\ifinner\else\hskip5\p@\fi\fi
 \hbox{\hskip5\p@\vrule width4\p@ height6\p@ depth1.5\p@\hskip\p@}%
 }}{}%
%
\@ifundefined{cents}{\def\cents{\hbox{\rm\rlap c/}}}{}%
\@ifundefined{tciLaplace}{\def\tciLaplace{\ensuremath{\mathcal{L}}}}{}%
\@ifundefined{tciFourier}{\def\tciFourier{\ensuremath{\mathcal{F}}}}{}%
\@ifundefined{textcurrency}{\def\textcurrency{\hbox{\rm\rlap xo}}}{}%
\@ifundefined{texteuro}{\def\texteuro{\hbox{\rm\rlap C=}}}{}%
\@ifundefined{euro}{\def\euro{\hbox{\rm\rlap C=}}}{}%
\@ifundefined{textfranc}{\def\textfranc{\hbox{\rm\rlap-F}}}{}%
\@ifundefined{textlira}{\def\textlira{\hbox{\rm\rlap L=}}}{}%
\@ifundefined{textpeseta}{\def\textpeseta{\hbox{\rm P\negthinspace s}}}{}%
%
\@ifundefined{miss}{\def\miss{\hbox{\vrule height2\p@ width 2\p@ depth\z@}}}{}%
%
\@ifundefined{vvert}{\def\vvert{\Vert}}{}%  %always translated to \left| or \right|
%
\@ifundefined{tcol}{\def\tcol#1{{\baselineskip=6\p@ \vcenter{#1}} \Column}}{}%
%
\@ifundefined{dB}{\def\dB{\hbox{{}}}}{}%        %dummy entry in column 
\@ifundefined{mB}{\def\mB#1{\hbox{$#1$}}}{}%   %column entry
\@ifundefined{nB}{\def\nB#1{\hbox{#1}}}{}%     %column entry (not math)
%
\@ifundefined{note}{\def\note{$^{\dag}}}{}%
%
\def\newfmtname{LaTeX2e}
% No longer load latexsym.  This is now handled by SWP, which uses amsfonts if necessary
%
\ifx\fmtname\newfmtname
  \DeclareOldFontCommand{\rm}{\normalfont\rmfamily}{\mathrm}
  \DeclareOldFontCommand{\sf}{\normalfont\sffamily}{\mathsf}
  \DeclareOldFontCommand{\tt}{\normalfont\ttfamily}{\mathtt}
  \DeclareOldFontCommand{\bf}{\normalfont\bfseries}{\mathbf}
  \DeclareOldFontCommand{\it}{\normalfont\itshape}{\mathit}
  \DeclareOldFontCommand{\sl}{\normalfont\slshape}{\@nomath\sl}
  \DeclareOldFontCommand{\sc}{\normalfont\scshape}{\@nomath\sc}
\fi

%
% Greek bold macros
% Redefine all of the math symbols 
% which might be bolded	 - there are 
% probably others to add to this list

\def\alpha{{\Greekmath 010B}}%
\def\beta{{\Greekmath 010C}}%
\def\gamma{{\Greekmath 010D}}%
\def\delta{{\Greekmath 010E}}%
\def\epsilon{{\Greekmath 010F}}%
\def\zeta{{\Greekmath 0110}}%
\def\eta{{\Greekmath 0111}}%
\def\theta{{\Greekmath 0112}}%
\def\iota{{\Greekmath 0113}}%
\def\kappa{{\Greekmath 0114}}%
\def\lambda{{\Greekmath 0115}}%
\def\mu{{\Greekmath 0116}}%
\def\nu{{\Greekmath 0117}}%
\def\xi{{\Greekmath 0118}}%
\def\pi{{\Greekmath 0119}}%
\def\rho{{\Greekmath 011A}}%
\def\sigma{{\Greekmath 011B}}%
\def\tau{{\Greekmath 011C}}%
\def\upsilon{{\Greekmath 011D}}%
\def\phi{{\Greekmath 011E}}%
\def\chi{{\Greekmath 011F}}%
\def\psi{{\Greekmath 0120}}%
\def\omega{{\Greekmath 0121}}%
\def\varepsilon{{\Greekmath 0122}}%
\def\vartheta{{\Greekmath 0123}}%
\def\varpi{{\Greekmath 0124}}%
\def\varrho{{\Greekmath 0125}}%
\def\varsigma{{\Greekmath 0126}}%
\def\varphi{{\Greekmath 0127}}%

\def\nabla{{\Greekmath 0272}}
\def\FindBoldGroup{%
   {\setbox0=\hbox{$\mathbf{x\global\edef\theboldgroup{\the\mathgroup}}$}}%
}

\def\Greekmath#1#2#3#4{%
    \if@compatibility
        \ifnum\mathgroup=\symbold
           \mathchoice{\mbox{\boldmath$\displaystyle\mathchar"#1#2#3#4$}}%
                      {\mbox{\boldmath$\textstyle\mathchar"#1#2#3#4$}}%
                      {\mbox{\boldmath$\scriptstyle\mathchar"#1#2#3#4$}}%
                      {\mbox{\boldmath$\scriptscriptstyle\mathchar"#1#2#3#4$}}%
        \else
           \mathchar"#1#2#3#4% 
        \fi 
    \else 
        \FindBoldGroup
        \ifnum\mathgroup=\theboldgroup % For 2e
           \mathchoice{\mbox{\boldmath$\displaystyle\mathchar"#1#2#3#4$}}%
                      {\mbox{\boldmath$\textstyle\mathchar"#1#2#3#4$}}%
                      {\mbox{\boldmath$\scriptstyle\mathchar"#1#2#3#4$}}%
                      {\mbox{\boldmath$\scriptscriptstyle\mathchar"#1#2#3#4$}}%
        \else
           \mathchar"#1#2#3#4% 
        \fi     	    
	  \fi}

\newif\ifGreekBold  \GreekBoldfalse
\let\SAVEPBF=\pbf
\def\pbf{\GreekBoldtrue\SAVEPBF}%
%

\@ifundefined{theorem}{\newtheorem{theorem}{Theorem}}{}
\@ifundefined{lemma}{\newtheorem{lemma}[theorem]{Lemma}}{}
\@ifundefined{corollary}{\newtheorem{corollary}[theorem]{Corollary}}{}
\@ifundefined{conjecture}{\newtheorem{conjecture}[theorem]{Conjecture}}{}
\@ifundefined{proposition}{\newtheorem{proposition}[theorem]{Proposition}}{}
\@ifundefined{axiom}{\newtheorem{axiom}{Axiom}}{}
\@ifundefined{remark}{\newtheorem{remark}{Remark}}{}
\@ifundefined{example}{\newtheorem{example}{Example}}{}
\@ifundefined{exercise}{\newtheorem{exercise}{Exercise}}{}
\@ifundefined{definition}{\newtheorem{definition}{Definition}}{}


\@ifundefined{mathletters}{%
  %\def\theequation{\arabic{equation}}
  \newcounter{equationnumber}  
  \def\mathletters{%
     \addtocounter{equation}{1}
     \edef\@currentlabel{\theequation}%
     \setcounter{equationnumber}{\c@equation}
     \setcounter{equation}{0}%
     \edef\theequation{\@currentlabel\noexpand\alph{equation}}%
  }
  \def\endmathletters{%
     \setcounter{equation}{\value{equationnumber}}%
  }
}{}

%Logos
\@ifundefined{BibTeX}{%
    \def\BibTeX{{\rm B\kern-.05em{\sc i\kern-.025em b}\kern-.08em
                 T\kern-.1667em\lower.7ex\hbox{E}\kern-.125emX}}}{}%
\@ifundefined{AmS}%
    {\def\AmS{{\protect\usefont{OMS}{cmsy}{m}{n}%
                A\kern-.1667em\lower.5ex\hbox{M}\kern-.125emS}}}{}%
\@ifundefined{AmSTeX}{\def\AmSTeX{\protect\AmS-\protect\TeX\@}}{}%
%

% This macro is a fix to eqnarray
\def\@@eqncr{\let\@tempa\relax
    \ifcase\@eqcnt \def\@tempa{& & &}\or \def\@tempa{& &}%
      \else \def\@tempa{&}\fi
     \@tempa
     \if@eqnsw
        \iftag@
           \@taggnum
        \else
           \@eqnnum\stepcounter{equation}%
        \fi
     \fi
     \global\tag@false
     \global\@eqnswtrue
     \global\@eqcnt\z@\cr}


\def\TCItag{\@ifnextchar*{\@TCItagstar}{\@TCItag}}
\def\@TCItag#1{%
    \global\tag@true
    \global\def\@taggnum{(#1)}}
\def\@TCItagstar*#1{%
    \global\tag@true
    \global\def\@taggnum{#1}}
%
%%%%%%%%%%%%%%%%%%%%%%%%%%%%%%%%%%%%%%%%%%%%%%%%%%%%%%%%%%%%%%%%%%%%%
%
\def\QATOP#1#2{{#1 \atop #2}}%
\def\QTATOP#1#2{{\textstyle {#1 \atop #2}}}%
\def\QDATOP#1#2{{\displaystyle {#1 \atop #2}}}%
\def\QABOVE#1#2#3{{#2 \above#1 #3}}%
\def\QTABOVE#1#2#3{{\textstyle {#2 \above#1 #3}}}%
\def\QDABOVE#1#2#3{{\displaystyle {#2 \above#1 #3}}}%
\def\QOVERD#1#2#3#4{{#3 \overwithdelims#1#2 #4}}%
\def\QTOVERD#1#2#3#4{{\textstyle {#3 \overwithdelims#1#2 #4}}}%
\def\QDOVERD#1#2#3#4{{\displaystyle {#3 \overwithdelims#1#2 #4}}}%
\def\QATOPD#1#2#3#4{{#3 \atopwithdelims#1#2 #4}}%
\def\QTATOPD#1#2#3#4{{\textstyle {#3 \atopwithdelims#1#2 #4}}}%
\def\QDATOPD#1#2#3#4{{\displaystyle {#3 \atopwithdelims#1#2 #4}}}%
\def\QABOVED#1#2#3#4#5{{#4 \abovewithdelims#1#2#3 #5}}%
\def\QTABOVED#1#2#3#4#5{{\textstyle 
   {#4 \abovewithdelims#1#2#3 #5}}}%
\def\QDABOVED#1#2#3#4#5{{\displaystyle 
   {#4 \abovewithdelims#1#2#3 #5}}}%
%
% Macros for text size operators:
%
\def\tint{\mathop{\textstyle \int}}%
\def\tiint{\mathop{\textstyle \iint }}%
\def\tiiint{\mathop{\textstyle \iiint }}%
\def\tiiiint{\mathop{\textstyle \iiiint }}%
\def\tidotsint{\mathop{\textstyle \idotsint }}%
\def\toint{\mathop{\textstyle \oint}}%
\def\tsum{\mathop{\textstyle \sum }}%
\def\tprod{\mathop{\textstyle \prod }}%
\def\tbigcap{\mathop{\textstyle \bigcap }}%
\def\tbigwedge{\mathop{\textstyle \bigwedge }}%
\def\tbigoplus{\mathop{\textstyle \bigoplus }}%
\def\tbigodot{\mathop{\textstyle \bigodot }}%
\def\tbigsqcup{\mathop{\textstyle \bigsqcup }}%
\def\tcoprod{\mathop{\textstyle \coprod }}%
\def\tbigcup{\mathop{\textstyle \bigcup }}%
\def\tbigvee{\mathop{\textstyle \bigvee }}%
\def\tbigotimes{\mathop{\textstyle \bigotimes }}%
\def\tbiguplus{\mathop{\textstyle \biguplus }}%
%
%
%Macros for display size operators:
%
\def\dint{\mathop{\displaystyle \int}}%
\def\diint{\mathop{\displaystyle \iint}}%
\def\diiint{\mathop{\displaystyle \iiint}}%
\def\diiiint{\mathop{\displaystyle \iiiint }}%
\def\didotsint{\mathop{\displaystyle \idotsint }}%
\def\doint{\mathop{\displaystyle \oint}}%
\def\dsum{\mathop{\displaystyle \sum }}%
\def\dprod{\mathop{\displaystyle \prod }}%
\def\dbigcap{\mathop{\displaystyle \bigcap }}%
\def\dbigwedge{\mathop{\displaystyle \bigwedge }}%
\def\dbigoplus{\mathop{\displaystyle \bigoplus }}%
\def\dbigodot{\mathop{\displaystyle \bigodot }}%
\def\dbigsqcup{\mathop{\displaystyle \bigsqcup }}%
\def\dcoprod{\mathop{\displaystyle \coprod }}%
\def\dbigcup{\mathop{\displaystyle \bigcup }}%
\def\dbigvee{\mathop{\displaystyle \bigvee }}%
\def\dbigotimes{\mathop{\displaystyle \bigotimes }}%
\def\dbiguplus{\mathop{\displaystyle \biguplus }}%

\if@compatibility\else
  % Always load amsmath in LaTeX2e mode
  \RequirePackage{amsmath}
\fi

\def\ExitTCILatex{\makeatother\endinput}

\bgroup
\ifx\ds@amstex\relax
   \message{amstex already loaded}\aftergroup\ExitTCILatex
\else
   \@ifpackageloaded{amsmath}%
      {\if@compatibility\message{amsmath already loaded}\fi\aftergroup\ExitTCILatex}
      {}
   \@ifpackageloaded{amstex}%
      {\if@compatibility\message{amstex already loaded}\fi\aftergroup\ExitTCILatex}
      {}
   \@ifpackageloaded{amsgen}%
      {\if@compatibility\message{amsgen already loaded}\fi\aftergroup\ExitTCILatex}
      {}
\fi
\egroup

%Exit if any of the AMS macros are already loaded.
%This is always the case for LaTeX2e mode.


%%%%%%%%%%%%%%%%%%%%%%%%%%%%%%%%%%%%%%%%%%%%%%%%%%%%%%%%%%%%%%%%%%%%%%%%%%
% NOTE: The rest of this file is read only if in LaTeX 2.09 compatibility
% mode. This section is used to define AMS-like constructs in the
% event they have not been defined.
%%%%%%%%%%%%%%%%%%%%%%%%%%%%%%%%%%%%%%%%%%%%%%%%%%%%%%%%%%%%%%%%%%%%%%%%%%
\typeout{TCILATEX defining AMS-like constructs in LaTeX 2.09 COMPATIBILITY MODE}
%%%%%%%%%%%%%%%%%%%%%%%%%%%%%%%%%%%%%%%%%%%%%%%%%%%%%%%%%%%%%%%%%%%%%%%%
%  Macros to define some AMS LaTeX constructs when 
%  AMS LaTeX has not been loaded
% 
% These macros are copied from the AMS-TeX package for doing
% multiple integrals.
%
\let\DOTSI\relax
\def\RIfM@{\relax\ifmmode}%
\def\FN@{\futurelet\next}%
\newcount\intno@
\def\iint{\DOTSI\intno@\tw@\FN@\ints@}%
\def\iiint{\DOTSI\intno@\thr@@\FN@\ints@}%
\def\iiiint{\DOTSI\intno@4 \FN@\ints@}%
\def\idotsint{\DOTSI\intno@\z@\FN@\ints@}%
\def\ints@{\findlimits@\ints@@}%
\newif\iflimtoken@
\newif\iflimits@
\def\findlimits@{\limtoken@true\ifx\next\limits\limits@true
 \else\ifx\next\nolimits\limits@false\else
 \limtoken@false\ifx\ilimits@\nolimits\limits@false\else
 \ifinner\limits@false\else\limits@true\fi\fi\fi\fi}%
\def\multint@{\int\ifnum\intno@=\z@\intdots@                          %1
 \else\intkern@\fi                                                    %2
 \ifnum\intno@>\tw@\int\intkern@\fi                                   %3
 \ifnum\intno@>\thr@@\int\intkern@\fi                                 %4
 \int}%                                                               %5
\def\multintlimits@{\intop\ifnum\intno@=\z@\intdots@\else\intkern@\fi
 \ifnum\intno@>\tw@\intop\intkern@\fi
 \ifnum\intno@>\thr@@\intop\intkern@\fi\intop}%
\def\intic@{%
    \mathchoice{\hskip.5em}{\hskip.4em}{\hskip.4em}{\hskip.4em}}%
\def\negintic@{\mathchoice
 {\hskip-.5em}{\hskip-.4em}{\hskip-.4em}{\hskip-.4em}}%
\def\ints@@{\iflimtoken@                                              %1
 \def\ints@@@{\iflimits@\negintic@
   \mathop{\intic@\multintlimits@}\limits                             %2
  \else\multint@\nolimits\fi                                          %3
  \eat@}%                                                             %4
 \else                                                                %5
 \def\ints@@@{\iflimits@\negintic@
  \mathop{\intic@\multintlimits@}\limits\else
  \multint@\nolimits\fi}\fi\ints@@@}%
\def\intkern@{\mathchoice{\!\!\!}{\!\!}{\!\!}{\!\!}}%
\def\plaincdots@{\mathinner{\cdotp\cdotp\cdotp}}%
\def\intdots@{\mathchoice{\plaincdots@}%
 {{\cdotp}\mkern1.5mu{\cdotp}\mkern1.5mu{\cdotp}}%
 {{\cdotp}\mkern1mu{\cdotp}\mkern1mu{\cdotp}}%
 {{\cdotp}\mkern1mu{\cdotp}\mkern1mu{\cdotp}}}%
%
%
%  These macros are for doing the AMS \text{} construct
%
\def\RIfM@{\relax\protect\ifmmode}
\def\text{\RIfM@\expandafter\text@\else\expandafter\mbox\fi}
\let\nfss@text\text
\def\text@#1{\mathchoice
   {\textdef@\displaystyle\f@size{#1}}%
   {\textdef@\textstyle\tf@size{\firstchoice@false #1}}%
   {\textdef@\textstyle\sf@size{\firstchoice@false #1}}%
   {\textdef@\textstyle \ssf@size{\firstchoice@false #1}}%
   \glb@settings}

\def\textdef@#1#2#3{\hbox{{%
                    \everymath{#1}%
                    \let\f@size#2\selectfont
                    #3}}}
\newif\iffirstchoice@
\firstchoice@true
%
%These are the AMS constructs for multiline limits.
%
\def\Let@{\relax\iffalse{\fi\let\\=\cr\iffalse}\fi}%
\def\vspace@{\def\vspace##1{\crcr\noalign{\vskip##1\relax}}}%
\def\multilimits@{\bgroup\vspace@\Let@
 \baselineskip\fontdimen10 \scriptfont\tw@
 \advance\baselineskip\fontdimen12 \scriptfont\tw@
 \lineskip\thr@@\fontdimen8 \scriptfont\thr@@
 \lineskiplimit\lineskip
 \vbox\bgroup\ialign\bgroup\hfil$\m@th\scriptstyle{##}$\hfil\crcr}%
\def\Sb{_\multilimits@}%
\def\endSb{\crcr\egroup\egroup\egroup}%
\def\Sp{^\multilimits@}%
\let\endSp\endSb
%
%
%These are AMS constructs for horizontal arrows
%
\newdimen\ex@
\ex@.2326ex
\def\rightarrowfill@#1{$#1\m@th\mathord-\mkern-6mu\cleaders
 \hbox{$#1\mkern-2mu\mathord-\mkern-2mu$}\hfill
 \mkern-6mu\mathord\rightarrow$}%
\def\leftarrowfill@#1{$#1\m@th\mathord\leftarrow\mkern-6mu\cleaders
 \hbox{$#1\mkern-2mu\mathord-\mkern-2mu$}\hfill\mkern-6mu\mathord-$}%
\def\leftrightarrowfill@#1{$#1\m@th\mathord\leftarrow
\mkern-6mu\cleaders
 \hbox{$#1\mkern-2mu\mathord-\mkern-2mu$}\hfill
 \mkern-6mu\mathord\rightarrow$}%
\def\overrightarrow{\mathpalette\overrightarrow@}%
\def\overrightarrow@#1#2{\vbox{\ialign{##\crcr\rightarrowfill@#1\crcr
 \noalign{\kern-\ex@\nointerlineskip}$\m@th\hfil#1#2\hfil$\crcr}}}%
\let\overarrow\overrightarrow
\def\overleftarrow{\mathpalette\overleftarrow@}%
\def\overleftarrow@#1#2{\vbox{\ialign{##\crcr\leftarrowfill@#1\crcr
 \noalign{\kern-\ex@\nointerlineskip}$\m@th\hfil#1#2\hfil$\crcr}}}%
\def\overleftrightarrow{\mathpalette\overleftrightarrow@}%
\def\overleftrightarrow@#1#2{\vbox{\ialign{##\crcr
   \leftrightarrowfill@#1\crcr
 \noalign{\kern-\ex@\nointerlineskip}$\m@th\hfil#1#2\hfil$\crcr}}}%
\def\underrightarrow{\mathpalette\underrightarrow@}%
\def\underrightarrow@#1#2{\vtop{\ialign{##\crcr$\m@th\hfil#1#2\hfil
  $\crcr\noalign{\nointerlineskip}\rightarrowfill@#1\crcr}}}%
\let\underarrow\underrightarrow
\def\underleftarrow{\mathpalette\underleftarrow@}%
\def\underleftarrow@#1#2{\vtop{\ialign{##\crcr$\m@th\hfil#1#2\hfil
  $\crcr\noalign{\nointerlineskip}\leftarrowfill@#1\crcr}}}%
\def\underleftrightarrow{\mathpalette\underleftrightarrow@}%
\def\underleftrightarrow@#1#2{\vtop{\ialign{##\crcr$\m@th
  \hfil#1#2\hfil$\crcr
 \noalign{\nointerlineskip}\leftrightarrowfill@#1\crcr}}}%
%%%%%%%%%%%%%%%%%%%%%

\def\qopnamewl@#1{\mathop{\operator@font#1}\nlimits@}
\let\nlimits@\displaylimits
\def\setboxz@h{\setbox\z@\hbox}


\def\varlim@#1#2{\mathop{\vtop{\ialign{##\crcr
 \hfil$#1\m@th\operator@font lim$\hfil\crcr
 \noalign{\nointerlineskip}#2#1\crcr
 \noalign{\nointerlineskip\kern-\ex@}\crcr}}}}

 \def\rightarrowfill@#1{\m@th\setboxz@h{$#1-$}\ht\z@\z@
  $#1\copy\z@\mkern-6mu\cleaders
  \hbox{$#1\mkern-2mu\box\z@\mkern-2mu$}\hfill
  \mkern-6mu\mathord\rightarrow$}
\def\leftarrowfill@#1{\m@th\setboxz@h{$#1-$}\ht\z@\z@
  $#1\mathord\leftarrow\mkern-6mu\cleaders
  \hbox{$#1\mkern-2mu\copy\z@\mkern-2mu$}\hfill
  \mkern-6mu\box\z@$}


\def\projlim{\qopnamewl@{proj\,lim}}
\def\injlim{\qopnamewl@{inj\,lim}}
\def\varinjlim{\mathpalette\varlim@\rightarrowfill@}
\def\varprojlim{\mathpalette\varlim@\leftarrowfill@}
\def\varliminf{\mathpalette\varliminf@{}}
\def\varliminf@#1{\mathop{\underline{\vrule\@depth.2\ex@\@width\z@
   \hbox{$#1\m@th\operator@font lim$}}}}
\def\varlimsup{\mathpalette\varlimsup@{}}
\def\varlimsup@#1{\mathop{\overline
  {\hbox{$#1\m@th\operator@font lim$}}}}

%
%Companion to stackrel
\def\stackunder#1#2{\mathrel{\mathop{#2}\limits_{#1}}}%
%
%
% These are AMS environments that will be defined to
% be verbatims if amstex has not actually been 
% loaded
%
%
\begingroup \catcode `|=0 \catcode `[= 1
\catcode`]=2 \catcode `\{=12 \catcode `\}=12
\catcode`\\=12 
|gdef|@alignverbatim#1\end{align}[#1|end[align]]
|gdef|@salignverbatim#1\end{align*}[#1|end[align*]]

|gdef|@alignatverbatim#1\end{alignat}[#1|end[alignat]]
|gdef|@salignatverbatim#1\end{alignat*}[#1|end[alignat*]]

|gdef|@xalignatverbatim#1\end{xalignat}[#1|end[xalignat]]
|gdef|@sxalignatverbatim#1\end{xalignat*}[#1|end[xalignat*]]

|gdef|@gatherverbatim#1\end{gather}[#1|end[gather]]
|gdef|@sgatherverbatim#1\end{gather*}[#1|end[gather*]]

|gdef|@gatherverbatim#1\end{gather}[#1|end[gather]]
|gdef|@sgatherverbatim#1\end{gather*}[#1|end[gather*]]


|gdef|@multilineverbatim#1\end{multiline}[#1|end[multiline]]
|gdef|@smultilineverbatim#1\end{multiline*}[#1|end[multiline*]]

|gdef|@arraxverbatim#1\end{arrax}[#1|end[arrax]]
|gdef|@sarraxverbatim#1\end{arrax*}[#1|end[arrax*]]

|gdef|@tabulaxverbatim#1\end{tabulax}[#1|end[tabulax]]
|gdef|@stabulaxverbatim#1\end{tabulax*}[#1|end[tabulax*]]


|endgroup
  

  
\def\align{\@verbatim \frenchspacing\@vobeyspaces \@alignverbatim
You are using the "align" environment in a style in which it is not defined.}
\let\endalign=\endtrivlist
 
\@namedef{align*}{\@verbatim\@salignverbatim
You are using the "align*" environment in a style in which it is not defined.}
\expandafter\let\csname endalign*\endcsname =\endtrivlist




\def\alignat{\@verbatim \frenchspacing\@vobeyspaces \@alignatverbatim
You are using the "alignat" environment in a style in which it is not defined.}
\let\endalignat=\endtrivlist
 
\@namedef{alignat*}{\@verbatim\@salignatverbatim
You are using the "alignat*" environment in a style in which it is not defined.}
\expandafter\let\csname endalignat*\endcsname =\endtrivlist




\def\xalignat{\@verbatim \frenchspacing\@vobeyspaces \@xalignatverbatim
You are using the "xalignat" environment in a style in which it is not defined.}
\let\endxalignat=\endtrivlist
 
\@namedef{xalignat*}{\@verbatim\@sxalignatverbatim
You are using the "xalignat*" environment in a style in which it is not defined.}
\expandafter\let\csname endxalignat*\endcsname =\endtrivlist




\def\gather{\@verbatim \frenchspacing\@vobeyspaces \@gatherverbatim
You are using the "gather" environment in a style in which it is not defined.}
\let\endgather=\endtrivlist
 
\@namedef{gather*}{\@verbatim\@sgatherverbatim
You are using the "gather*" environment in a style in which it is not defined.}
\expandafter\let\csname endgather*\endcsname =\endtrivlist


\def\multiline{\@verbatim \frenchspacing\@vobeyspaces \@multilineverbatim
You are using the "multiline" environment in a style in which it is not defined.}
\let\endmultiline=\endtrivlist
 
\@namedef{multiline*}{\@verbatim\@smultilineverbatim
You are using the "multiline*" environment in a style in which it is not defined.}
\expandafter\let\csname endmultiline*\endcsname =\endtrivlist


\def\arrax{\@verbatim \frenchspacing\@vobeyspaces \@arraxverbatim
You are using a type of "array" construct that is only allowed in AmS-LaTeX.}
\let\endarrax=\endtrivlist

\def\tabulax{\@verbatim \frenchspacing\@vobeyspaces \@tabulaxverbatim
You are using a type of "tabular" construct that is only allowed in AmS-LaTeX.}
\let\endtabulax=\endtrivlist

 
\@namedef{arrax*}{\@verbatim\@sarraxverbatim
You are using a type of "array*" construct that is only allowed in AmS-LaTeX.}
\expandafter\let\csname endarrax*\endcsname =\endtrivlist

\@namedef{tabulax*}{\@verbatim\@stabulaxverbatim
You are using a type of "tabular*" construct that is only allowed in AmS-LaTeX.}
\expandafter\let\csname endtabulax*\endcsname =\endtrivlist

% macro to simulate ams tag construct


% This macro is a fix to the equation environment
 \def\endequation{%
     \ifmmode\ifinner % FLEQN hack
      \iftag@
        \addtocounter{equation}{-1} % undo the increment made in the begin part
        $\hfil
           \displaywidth\linewidth\@taggnum\egroup \endtrivlist
        \global\tag@false
        \global\@ignoretrue   
      \else
        $\hfil
           \displaywidth\linewidth\@eqnnum\egroup \endtrivlist
        \global\tag@false
        \global\@ignoretrue 
      \fi
     \else   
      \iftag@
        \addtocounter{equation}{-1} % undo the increment made in the begin part
        \eqno \hbox{\@taggnum}
        \global\tag@false%
        $$\global\@ignoretrue
      \else
        \eqno \hbox{\@eqnnum}% $$ BRACE MATCHING HACK
        $$\global\@ignoretrue
      \fi
     \fi\fi
 } 

 \newif\iftag@ \tag@false
 
 \def\TCItag{\@ifnextchar*{\@TCItagstar}{\@TCItag}}
 \def\@TCItag#1{%
     \global\tag@true
     \global\def\@taggnum{(#1)}}
 \def\@TCItagstar*#1{%
     \global\tag@true
     \global\def\@taggnum{#1}}

  \@ifundefined{tag}{
     \def\tag{\@ifnextchar*{\@tagstar}{\@tag}}
     \def\@tag#1{%
         \global\tag@true
         \global\def\@taggnum{(#1)}}
     \def\@tagstar*#1{%
         \global\tag@true
         \global\def\@taggnum{#1}}
  }{}

\def\tfrac#1#2{{\textstyle {#1 \over #2}}}%
\def\dfrac#1#2{{\displaystyle {#1 \over #2}}}%
\def\binom#1#2{{#1 \choose #2}}%
\def\tbinom#1#2{{\textstyle {#1 \choose #2}}}%
\def\dbinom#1#2{{\displaystyle {#1 \choose #2}}}%

% Do not add anything to the end of this file.  
% The last section of the file is loaded only if 
% amstex has not been.
\makeatother
\endinput

\begin{document}

\title{PBES Implementation Notes}
\author{Wieger Wesselink}
\maketitle

This document contains details about data structures and algorithms of the
PBES Library of the mCRL2 toolset.

\section{Definitions}

Parameterised Boolean Equation Systems (PBESs) are empty (denoted $\epsilon $%
) or finite sequences of fixed point equations, where each equation is of
the form $(\mu X(d{:}D)=\phi $ or $(\nu X(d{:}D)=\phi $. The left-hand side
of each equation consists of a \emph{fixed point symbol}, where $\mu $
indicates a least and $\nu $ a greatest fixed point, and a sorted predicate
variable $X$ of sort $D\rightarrow B$, taken from some countable domain of
sorted predicate variables $\mathcal{X}$. The right-hand side of each
equation is a predicate formula as defined below.

\begin{definition}
\emph{Predicate formulae} $\phi $ are defined by the following grammar: 
\begin{equation*}
\phi ::=b~|~X(e)~|~\lnot \phi ~|~\phi \oplus \phi ~|~\mathsf{Q}d:D.~\phi
\end{equation*}%
where $\oplus \in \{\wedge ,\vee ,\Rightarrow \}$, $\mathsf{Q}\in \{\forall
,\exists \}$, $b$ is a data term of sort $\mathsf{B}$, $X$ is a predicate
variable, $d$ is a data variable of sort $D$ and $e$ is a vector of data
terms.
\end{definition}

The set of predicate variables that occur in a predicate formula $\phi $,
denoted by $\mathsf{occ}$, is defined recursively as follows, for any
formulae $\phi _{1},\phi _{2}$: 
\begin{equation*}
\begin{array}{llll}
\mathsf{occ(}{b)} & =_{def}\emptyset & \mathsf{occ(}{X(e))} & =_{def}\{X\}
\\ 
\mathsf{occ(}{\phi _{1}\oplus \phi _{2})} & =_{def}\mathsf{occ(}{\phi _{1})}%
\cup \mathsf{occ(}{\phi _{2})}\qquad & \mathsf{occ}(\mathsf{Q}d:D{.~\phi
_{1})} & =_{def}\mathsf{occ(}{\phi _{1})}.%
\end{array}%
\end{equation*}%
Extended to equation systems, $\mathsf{occ}{(}\mathcal{E}{)}$ is the union
of all variables occurring at the right-hand side of equations in $\mathcal{E%
}$. Likewise, the set of predicate variable instantiations that occur in a
predicate formula $\phi $ is denoted by $\mathsf{iocc}$, and is defined
recursively as follows%
\begin{equation*}
\begin{array}{llll}
\mathsf{iocc(}{b)} & =_{def}\emptyset & \mathsf{iocc(}{X(e))} & 
=_{def}\{X(e)\} \\ 
\mathsf{iocc(}{\phi _{1}\oplus \phi _{2})} & =_{def}\mathsf{iocc(}{\phi _{1})%
}\cup \mathsf{iocc(}{\phi _{2})}\qquad & \mathsf{iocc}(\mathsf{Q}d:D{.~\phi
_{1})} & =_{def}\mathsf{iocc(}{\phi _{1})}.%
\end{array}%
\end{equation*}

For any equation system $\mathcal{E}$, the set of \emph{binding predicate
variables}, $\mathsf{bnd(}\mathcal{E})$, is the set of variables occurring
at the left-hand side of some equation in $\mathcal{E}$. Formally, we
define: 
\begin{equation*}
\begin{array}{llll}
\mathsf{bnd(}{\epsilon )} & =_{def}\emptyset \qquad & \mathsf{bnd(}{(\sigma
X(d{:}D)=\phi )~\mathcal{E)}} & =_{def}\mathsf{bnd(}{\mathcal{E)}}\cup \{X\}
\\ 
\mathsf{occ(}{\epsilon )} & =_{def}\emptyset \qquad & \mathsf{occ(}{(\sigma
X(d{:}D)=\phi )~\mathcal{E)}} & =_{def}\mathsf{occ(}{\mathcal{E)}}\cup 
\mathsf{occ(}{\phi )}.%
\end{array}%
\end{equation*}%
Let $\mathsf{dvar}(d)$ be the set of \emph{free data variables} occurring in
a data term $d$. The function $\mathsf{dvar}$ is extended to predicate
formulae using%
\begin{equation*}
\begin{array}{llll}
\mathsf{dvar(}{X(e))} & =_{def}\mathsf{dvar}(e) & \mathsf{dvar}(\mathsf{Q}d:D%
{.~\phi _{1})} & =_{def}\mathsf{dvar(}{\phi _{1})}\setminus \mathsf{dvar(}d{)%
}. \\ 
\mathsf{dvar(}{\phi _{1}\oplus \phi _{2})} & =_{def}\mathsf{dvar(}{\phi _{1})%
}\cup \mathsf{iocc(}{\phi _{2}).}\qquad &  & 
\end{array}%
\end{equation*}

The set of freely occurring predicate variables in $\mathcal{E}$, denoted $%
\mathsf{pvar}(\mathcal{E})$ is defined as $\mathsf{occ(}{\mathcal{E)}}%
\setminus \mathsf{bnd(}{\mathcal{E)}}$. An equation system $\mathcal{E}$ is
said to be \emph{well-formed} iff every binding predicate variable occurs at
the left-hand side of precisely one equation of $\mathcal{E}$. We only
consider well-formed equation systems in this paper.

An equation system $\mathcal{E}$ is called \emph{closed} if $\mathsf{pvar}(%
\mathcal{E})=\emptyset $ and \emph{open} otherwise. An equation $(\sigma
X(d:D)=\phi )$, where $\sigma $ denotes either the fixed point sign $\mu $
or $\nu $, is called \emph{data-closed} if the set of data variables that
occur freely in $\phi $ is contained in the set of variables induced by the
vector of variables $d$. An equation system is called \emph{data-closed} iff
each of its equations is data-closed.\newline

\begin{definition}
\emph{Action formulae} $\alpha $ are defined by the following grammar:%
\begin{equation*}
\alpha ::=b~|~\lnot \alpha ~|~\alpha \oplus \alpha ~|~\mathsf{Q}d:D.\alpha
~|~a(d)~|~\alpha \mbox{\aap ,}t
\end{equation*}%
where $\oplus \in \{\wedge ,\vee ,\Rightarrow \}$, $\mathsf{Q}\in \{\forall
,\exists \}$, $b$ is a data term of sort $\mathsf{B}$, $X$ is a predicate
variable, $d$ is a data variable of sort $D$ and $a$ is an action label.
\end{definition}

\begin{definition}
\emph{State formulae} $\phi $ are defined by the following grammar:%
\begin{equation*}
\phi ::=b~|~X(e)~|~\lnot \phi ~|~\phi \oplus \phi ~|~\mathsf{Q}d:D.~\phi
~|~\langle \alpha \rangle \phi ~|~[\alpha ]\phi ~|~\Delta ~|~\Delta
(t)~|~\nabla ~|~\nabla (t)~|~\sigma X(d{:}D:=e)
\end{equation*}%
where $\oplus \in \{\wedge ,\vee ,\Rightarrow \}$, $\mathsf{Q}\in \{\forall
,\exists \}$, $\sigma \in \{\mu ,\nu \}$, $b$ is a data term of sort $%
\mathsf{B}$, $X$ is a predicate variable, $d$ is a data variable of sort $D$
and $e$ is a vector of data terms and $\alpha $ is an action formula.
\end{definition}

\newpage

\subsection{Monotonicity}

\begin{definition}
A state formula is called \emph{monotonous} if it can be rewritten such that
propositional variables are not inside the scope of a negation or an
implication. More formally, a state formula $\varphi $ is monotonous if $%
m(\varphi )=true$, where $m$ is defined as follows. This definition applies
to predicate formulae as well.
\end{definition}

\begin{equation*}
\begin{array}{lll}
m(\lnot b) & =_{def} & \mathsf{true} \\ 
m(\lnot \lnot \varphi ) & =_{def} & m(\varphi ) \\ 
m(\lnot (\varphi \wedge \psi )) & =_{def} & m(\lnot \varphi )\wedge m(\lnot
\psi ) \\ 
m(\lnot (\varphi \vee \psi )) & =_{def} & m(\lnot \varphi )\wedge m(\lnot
\psi ) \\ 
m(\lnot (\varphi \Rightarrow \psi )) & =_{def} & m(\varphi )\wedge m(\lnot
\psi ) \\ 
m(\lnot \forall d{{:}D}.\varphi ) & =_{def} & m(\lnot \varphi ) \\ 
m(\lnot \exists d{{:}D}.\varphi ) & =_{def} & m(\lnot \varphi ) \\ 
m(\lnot \lbrack \alpha ]\varphi ) & =_{def} & m(\lnot \varphi ) \\ 
m(\lnot \langle \alpha \rangle \varphi ) & =_{def} & m(\lnot \varphi ) \\ 
m(\lnot \nabla ) & =_{def} & \mathsf{true} \\ 
m(\lnot \nabla (t)) & =_{def} & \mathsf{true} \\ 
m(\lnot \Delta ) & =_{def} & \mathsf{true} \\ 
m(\lnot \Delta (t)) & =_{def} & \mathsf{true} \\ 
m(\lnot X(e)) & =_{def} & \mathsf{false} \\ 
m(\lnot \mu X(d{:}D:=e).~\varphi )~~~ & =_{def}~~~ & m(\lnot \varphi \lbrack
X:=\lnot X]) \\ 
m(\lnot \nu X(d{:}D:=e).~\varphi )~~~ & =_{def}~~~ & m(\lnot \varphi \lbrack
X:=\lnot X]) \\ 
m(b) & =_{def} & \mathsf{true} \\ 
m(\varphi \wedge \psi ) & =_{def} & m(\varphi )\wedge m(\psi ) \\ 
m(\varphi \vee \psi ) & =_{def} & m(\varphi )\wedge m(\psi ) \\ 
m(\varphi \Rightarrow \psi ) & =_{def} & m(\lnot \varphi )\wedge m(\psi ) \\ 
m(\forall d{{:}D}.\varphi ) & =_{def} & m(\varphi ) \\ 
m(\exists d{{:}D}.\varphi ) & =_{def} & m(\varphi ) \\ 
m([\alpha ]\varphi ) & =_{def} & m(\varphi ) \\ 
m(\langle \alpha \rangle \varphi ) & =_{def} & m(\varphi ) \\ 
m(\nabla ) & =_{def} & \mathsf{true} \\ 
m(\nabla (t)) & =_{def} & \mathsf{true} \\ 
m(\Delta ) & =_{def} & \mathsf{true} \\ 
m(\Delta (t)) & =_{def} & \mathsf{true} \\ 
m(X(e)) & =_{def} & \mathsf{true} \\ 
m(\mu X(d{:}D:=e).~\varphi )~~~ & =_{def}~~~ & m(\varphi ) \\ 
m(\nu X(d{:}D:=e).~\varphi )~~~ & =_{def}~~~ & m(\varphi )%
\end{array}%
\end{equation*}

\newpage

\subsection{Normalization}

The normalization function $h$ is a function that eliminates implications
from a state formula $\varphi $, and that 'pushes' negations inwards to the
level of data expressions. A precondition of $h$ is that $\varphi $ is
monotonous. If this is not the case, during the computation a term $\lnot
X(e)$ will be encountered.%
\begin{equation*}
\begin{array}{lll}
h(\lnot b) & =_{def} & \lnot b \\ 
h(\lnot \lnot \varphi ) & =_{def} & h(\varphi ) \\ 
h(\lnot (\varphi \wedge \psi )) & =_{def} & h(\lnot \varphi )\vee h(\lnot
\psi ) \\ 
h(\lnot (\varphi \vee \psi )) & =_{def} & h(\lnot \varphi )\wedge h(\lnot
\psi ) \\ 
h(\lnot (\varphi \Rightarrow \psi )) & =_{def} & h(\varphi )\wedge h(\lnot
\psi ) \\ 
h(\lnot \forall d{{:}D}.\varphi ) & =_{def} & \exists d{{:}D}.h(\lnot
\varphi ) \\ 
h(\lnot \exists d{{:}D}.\varphi ) & =_{def} & \forall d{{:}D}.h(\lnot
\varphi ) \\ 
h(\lnot \lbrack \alpha ]\varphi ) & =_{def} & [\alpha ]h(\lnot \varphi ) \\ 
h(\lnot \langle \alpha \rangle \varphi ) & =_{def} & \langle \alpha \rangle
h(\lnot \varphi ) \\ 
h(\lnot \nabla ) & =_{def} & \Delta \\ 
h(\lnot \nabla (t)) & =_{def} & \Delta (t) \\ 
h(\lnot \Delta ) & =_{def} & \nabla \\ 
h(\lnot \Delta (t)) & =_{def} & \nabla (t) \\ 
h(\lnot X(e)) & =_{def} & undefined \\ 
h(\lnot \mu X(d{:}D:=e).~\varphi )~~~ & =_{def}~~~ & \nu X(d{:}%
D:=e).~h(\lnot \varphi \lbrack X:=\lnot X]) \\ 
h(\lnot \nu X(d{:}D:=e).~\varphi )~~~ & =_{def}~~~ & \mu X(d{:}%
D:=e).~h(\lnot \varphi \lbrack X:=\lnot X]) \\ 
h(b) & =_{def} & b \\ 
h(\varphi \wedge \psi ) & =_{def} & h(\varphi )\wedge h(\psi ) \\ 
h(\varphi \vee \psi ) & =_{def} & h(\varphi )\vee h(\psi ) \\ 
h(\varphi \Rightarrow \psi ) & =_{def} & h(\lnot \varphi )\vee h(\psi ) \\ 
h(\mathsf{Q}d{{:}D}.\varphi ) & =_{def} & \mathsf{\forall }d{{:}D}.h(\varphi
) \\ 
h([\alpha ]\varphi ) & =_{def} & [\alpha ]h(\varphi ) \\ 
h(\langle \alpha \rangle \varphi ) & =_{def} & \langle \alpha \rangle
h(\varphi ) \\ 
h(\nabla ) & =_{def} & \nabla \\ 
h(\nabla (t)) & =_{def} & \nabla (t) \\ 
h(\Delta ) & =_{def} & \Delta \\ 
h(\Delta (t)) & =_{def} & \Delta (t) \\ 
h(X(d)) & =_{def} & X(d) \\ 
h(\sigma X(d{:}D:=e).~\varphi )~~~ & =_{def}~~~ & \sigma X(d{:}%
D:=e).~h(\varphi )%
\end{array}%
\end{equation*}%
\pagebreak

\subsection{The predicate formula normal form (PFNF)}

\begin{definition}
A predicate formula is said to be in \emph{Predicate Formula Normal Form}
(PFNF) if it has the following form: 
\begin{equation*}
\mathsf{Q}_{1}v_{1}{:}V_{1}.\cdots \mathsf{Q}_{n}v_{n}{:}V_{n}.~h\wedge
\bigwedge\limits_{i\in I}\left( g_{i}\implies \bigvee\limits_{j\in
J_{i}}~X^{j}(e^{j})\right)
\end{equation*}%
where $X^{j}\in \chi $ ($\chi $ is a countable of sorted predicate
variables), $\mathsf{Q}_{i}\in \{\forall ,\exists \}$, $I$ is a (possibly
empty) finite index set, each $J_{i}$ is a non-empty finite index set, and $%
h $ and every $g_{i}$ are simple formulae.
\end{definition}

Note that here $J_{i}$ is used to index a set of occurrences of not
necessarily different variables. For instance, $(n>0\implies X(3)\vee
X(5)\vee Y(6))$ is a formula complying to the definition of PFNF. So long as
it does not lead to confusion, we stick to the convention to drop the typing
of the quantified variables $v_{i}$. An algorithm to compute a PFNF is:

\begin{equation*}
\begin{array}{lll}
p(c) & =_{def} & c \\ 
p(X(d)) & =_{def} & X(d) \\ 
p(\forall {x{:}D}.\varphi ) & =_{def} & \forall {x{:}D}.p(\varphi ) \\ 
p(\exists {x{:}D}.\varphi ) & =_{def} & \exists {x{:}D}.p(\varphi ) \\ 
&  &  \\ 
p(\varphi \wedge \psi ) & =_{def} & 
\begin{array}{l}
\mathsf{Q}_{1}^{\varphi }\cdots \mathsf{Q}_{n^{\varphi }}^{\varphi }\mathsf{Q%
}_{1}^{\psi }\cdots \mathsf{Q}_{n^{\psi }}^{\psi }.~~\left( h^{\varphi
}\wedge h^{\psi }\right) \\ 
\wedge \bigwedge\limits_{i\in I^{\varphi }\cup I^{\psi }}\left(
g_{i}\implies \bigvee\limits_{j\in J_{i}}~X^{j}(e^{j})\right)%
\end{array}
\\ 
&  &  \\ 
p(\varphi \vee \psi ) & =_{def} & 
\begin{array}{l}
\mathsf{Q}_{1}^{\varphi }\cdots \mathsf{Q}_{n^{\varphi }}^{\varphi }\mathsf{Q%
}_{1}^{\psi }\cdots \mathsf{Q}_{n^{\psi }}^{\psi }.\left( h^{\varphi }\vee
h^{\psi }\right) \\ 
\wedge \bigwedge\limits_{i\in I^{\varphi }}\left( \left( \lnot h^{\psi
}\wedge g_{i}\right) \implies \bigvee\limits_{j\in J_{i}}~X^{j}(e^{j})\right)
\\ 
\wedge \bigwedge\limits_{i\in I^{\psi }}\left( \left( \lnot h^{\varphi
}\wedge g_{i}\right) \implies \bigvee\limits_{j\in J_{i}}~X^{j}(e^{j})\right)
\\ 
\wedge \bigwedge\limits_{i\in I^{\varphi },k\in I^{\psi }}\left( \left(
g_{i}\wedge g_{k}\right) \implies \bigvee\limits_{j\in J_{i},m\in
J_{k}}~X^{j}(e^{j})\vee X^{m}(e^{m})\right)%
\end{array}%
\end{array}%
\end{equation*}%
where

\begin{equation*}
\begin{array}{lll}
p(\varphi ) & = & \mathsf{Q}_{1}^{\varphi }\cdots \mathsf{Q}_{n^{\varphi
}}^{\varphi }.~h^{\varphi }\wedge \bigwedge\limits_{i\in I^{\varphi }}\left(
g_{i}\implies \bigvee\limits_{j\in J_{i}}~X^{j}(e^{j})\right) \\ 
p(\psi ) & = & \mathsf{Q}_{1}^{\psi }\cdots \mathsf{Q}_{n^{\psi }}^{\psi
}.~h^{\psi }\wedge \bigwedge\limits_{i\in I^{\psi }}\left( g_{i}\implies
\bigvee\limits_{j\in J_{i}}~X^{j}(e^{j})\right) ,%
\end{array}%
\end{equation*}%
under the assumption that $I^{\varphi }$ and $I^{\psi }$ are disjoint, and $%
v_{i}^{\varphi }\neq v_{j}^{\psi }$ for all $i,j$.\pagebreak

\section{Transforming a state formula to a PBES}

In this section we define the algorithm pbes\_translate that generates a
PBES from a state formula and an LPD. Let $\langle {D_p, d_0, P} \rangle $
be the LPD given by

\begin{equation*}
\begin{array}{lrl}
\mathbf{proc}~P(x{:}D_{p}) & = & \sum_{i\in
I}\sum_{y:E_{i}}c_{i}(x,y)\rightarrow a_{i}(f_{i}(x,y))\mbox{\aap ,}%
t_{i}(x,y)\cdot P(g_{i}(x,y)) \\ 
& + & \sum_{j\in J}\sum_{y:E_{j}}c_{j}(x,y)\rightarrow \delta \mbox{\aap ,}%
t_{j}(x,y);%
\end{array}%
\end{equation*}%
where $a_{i}(f_{i}(x,y))$ is a multiset of actions. Then we define 
\begin{equation*}
\mathbf{pbes\_translate}(\sigma X(x_{f}:D_{f}:=d).~\varphi ,\langle {%
D_{p},d_{0},P}\rangle )=\mathbf{E}(\varphi ),
\end{equation*}%
where the function $\mathbf{E}$ is inductively defined using the tables
below. The function $\varphi $ has to be in positive normal form, i.e. it
may not contain any $\lnot $ or $\Rightarrow $ symbols. This is done using
the function $h$, as given below. There is also an untimed variant of the
algorithm, which can be obtained by removing all time references. A formula $%
\varphi $ not of the form $\sigma X(x_{f}:D_{f}:=d).~\varphi $ is first
translated into $\nu X().~\varphi $. We assume that $T:\mathbb{R}$ is a
unique fresh time variable that is generated by the algorithm.

\vspace{1cm}

%-------------------------------------%
%           function Sat
%-------------------------------------%
Let $a=\{a_{1},\ldots ,a_{n}\}$ and $b=\{b_{1},\ldots ,b_{n}\}$ be two multi
actions. Let $A$ be the set of all permutations $[i_{1},\ldots ,i_{n}]$ of $%
[1,\ldots n]$ such that $name(a_{k})=name(b_{i_{k}})$ for $k=1\ldots n$.
Then we define the function $\mathbf{Sat}$ as follows:

\begin{equation*}
\begin{array}{lll}
\mathbf{Sat}(a \mbox{\aap ,} t, b) & =_{def} & \left\{ 
\begin{array}{cc}
\dbigvee\limits_{\lbrack i_{1},\ldots ,i_{n}]\in
A}\dbigwedge\limits_{k=1\ldots n}(a_{k}=b_{i_{k}}) & \text{if }A\neq
\emptyset \\ 
false & \text{otherwise}%
\end{array}%
\right. \\ 
\mathbf{Sat}(a \mbox{\aap ,} t, c) & =_{def} & c \\ 
\mathbf{Sat}(a \mbox{\aap ,} t, \alpha \mbox{\aap ,} u) & =_{def} & \mathbf{%
Sat}(a \mbox{\aap ,} t, \alpha) \wedge t \approx u \\ 
\mathbf{Sat}(a \mbox{\aap ,} t, \neg \alpha) & =_{def} & \neg \mathbf{Sat}(a %
\mbox{\aap ,} t, \alpha) \\ 
\mathbf{Sat}(a \mbox{\aap ,} t, \alpha \wedge \beta) & =_{def} & \mathbf{Sat}%
(a \mbox{\aap ,} t, \alpha) \wedge \mathbf{Sat}(a \mbox{\aap ,} t, \beta) \\ 
\mathbf{Sat}(a \mbox{\aap ,} t, \alpha \vee \beta) & =_{def} & \mathbf{Sat}%
(a \mbox{\aap ,} t, \alpha) \vee \mathbf{Sat}(a \mbox{\aap ,} t, \beta) \\ 
\mathbf{Sat}(a \mbox{\aap ,} t, \alpha \Rightarrow \beta) & =_{def} & 
\mathbf{Sat}(a \mbox{\aap ,} t, \alpha) \Rightarrow \mathbf{Sat}(a %
\mbox{\aap ,} t, \beta) \\ 
\mathbf{Sat}(a \mbox{\aap ,} t, \forall{x {:} D}. \alpha) & =_{def} & \forall%
{y {:} D}. (\mathbf{Sat}(a \mbox{\aap ,} t, \alpha[x := y]) \\ 
\mathbf{Sat}(a \mbox{\aap ,} t, \exists{x {:} D}. \alpha) & =_{def} & \exists%
{y {:} D}. (\mathbf{Sat}(a \mbox{\aap ,} t, \alpha[x := y]) \\ 
&  & 
\end{array}%
\end{equation*}

%-------------------------------------%
%           function Par
%-------------------------------------%

\begin{equation*}
\begin{array}{lll}
\mathbf{Par}_{X,l}(c) & =_{def} & [] \\ 
\mathbf{Par}_{X,l}(\lnot \varphi ) & =_{def} & \mathbf{Par}_{X,l}(\varphi )
\\ 
\mathbf{Par}_{X,l}(\varphi \wedge \psi ) & =_{def} & \mathbf{Par}%
_{X,l}(\varphi )++\mathbf{Par}_{X,l}(\psi ) \\ 
\mathbf{Par}_{X,l}(\varphi \vee \psi ) & =_{def} & \mathbf{Par}%
_{X,l}(\varphi )++\mathbf{Par}_{X,l}(\psi ) \\ 
\mathbf{Par}_{X,l}(\varphi \Rightarrow \psi ) & =_{def} & \mathbf{Par}%
_{X,l}(\varphi )++\mathbf{Par}_{X,l}(\psi ) \\ 
\mathbf{Par}_{X,l}([\alpha ]\varphi ) & =_{def} & \mathbf{Par}_{X,l}(\varphi
) \\ 
\mathbf{Par}_{X,l}(\langle \alpha \rangle \varphi ) & =_{def} & \mathbf{Par}%
_{X,l}(\varphi ) \\ 
\mathbf{Par}_{X,l}(\forall {x{:}D}.\varphi ) & =_{def} & \mathbf{Par}%
_{X,l++[x{:}D]}(\varphi ) \\ 
\mathbf{Par}_{X,l}(\exists {x{:}D}.\varphi ) & =_{def} & \mathbf{Par}%
_{X,l++[x{:}D]}(\varphi ) \\ 
\mathbf{Par}_{X,l}(Y(d_{f})) & =_{def} & [] \\ 
\mathbf{Par}_{X,l}(\sigma Y(x_{f}{:}D_{f}:=d).\varphi ) & =_{def} & \left\{ 
\begin{array}{cc}
l & \mathsf{if\ }Y=X \\ 
\mathbf{Par}_{X,l++[x_{f}{:}D_{f}]}(\varphi ) & \mathsf{if\ }Y\neq X%
\end{array}%
\right. \\ 
\mathbf{Par}_{X,l}(\nabla (t)) & =_{def} & [] \\ 
\mathbf{Par}_{X,l}(\Delta (t)) & =_{def} & [] \\ 
&  & 
\end{array}%
\end{equation*}

\pagebreak

%-------------------------------------%
%           function RHS
%-------------------------------------%
\begin{equation*}
\begin{array}{lll}
{\mathbf{RHS}}(c) & =_{def} & c \\ 
{\mathbf{RHS}}(\varphi \wedge \psi ) & =_{def} & {\mathbf{RHS}}(\varphi
)\wedge {\mathbf{RHS}}(\psi ) \\ 
{\mathbf{RHS}}(\varphi \vee \psi ) & =_{def} & {\mathbf{RHS}}(\varphi )\vee {%
\mathbf{RHS}}(\psi ) \\ 
{\mathbf{RHS}}(\varphi \Rightarrow \psi ) & =_{def} & {\mathbf{RHS}}(\lnot
\varphi )\vee {\mathbf{RHS}}(\psi ) \\ 
{\mathbf{RHS}}(\forall {x{:}D}.\varphi ) & =_{def} & \forall {x{:}D}.{%
\mathbf{RHS}}(\varphi ) \\ 
{\mathbf{RHS}}(\exists {x{:}D}.\varphi ) & =_{def} & \exists {x{:}D}.{%
\mathbf{RHS}}(\varphi ) \\ 
{\mathbf{RHS}}([\alpha ]\varphi ) & =_{def} & \bigwedge_{i{\in }I}\forall _{y%
{:}E_{i}}((\mathbf{Sat}_{true }(a_{i}(f_{i}(x_{p},y))\mbox{\aap ,}%
t_{i}(x_{p},y),\alpha )\ \wedge \\ 
&  & \phantom{\bigwedge_{i {\in} I} \forall_{y {:} E_i}(}c_{i}(x_{p},y)%
\wedge t_{i}(x_{p},y)>T\ )\Rightarrow \\ 
&  & \phantom{\bigwedge_{i {\in} I} \forall_{y {:} E_i}(} {\mathbf{RHS}}%
(\varphi )[T:=t_{i}(x_{p},y)][x_{p}:=g_{i}(x_{p},y)]) \\ 
{\mathbf{RHS}}(\langle \alpha \rangle \varphi ) & =_{def} & \bigvee_{i{\in }%
I}\exists _{y{:}E_{i}}(\mathbf{Sat}_{true }(a_{i}(f_{i}(x_{p},y))%
\mbox{\aap
,}t_{i}(x_{p},y),\alpha )\ \wedge \\ 
&  & \phantom{\bigvee_{i {\in} I} \exists_{y {:} E_i}(}c_{i}(x_{p},y)\wedge
t_{i}(x_{p},y)>T\ \wedge \\ 
&  & \phantom{\bigvee_{i {\in} I} \exists_{y {:} E_i}(} {\mathbf{RHS}}%
(\varphi )[T:=t_{i}(x_{p},y)][x_{p}:=g_{i}(x_{p},y)]) \\ 
{\mathbf{RHS}}(X(d)) & =_{def} & \tilde{X}(T,d,x_{p},\mathbf{Par}%
_{X,[]}(\varphi _{0})) \\ 
{\mathbf{RHS}}(\sigma X(x_{f}{:}D_{f}:=d).~\varphi ) & =_{def}~~~ & \tilde{X}%
(T,d,x_{p},\mathbf{Par}_{X,[]}(\varphi _{0})) \\ 
{\mathbf{RHS}}(\nabla (t)) & =_{def} & \left( \bigwedge_{i{\in }I\cup
J}\forall _{y{:}E_{i}}((\lnot c_{i}(x_{p},y)\vee t>t_{k}(x_{p},y))\right)
\wedge t>T \\ 
{\mathbf{RHS}}(\Delta (t)) & =_{def} & \left( \bigvee_{i{\in }I\cup
J}\exists _{y{:}E_{i}}((c_{i}(x_{p},y)\wedge t\leq t_{k}(x_{p},y))\right)
\vee t\leq T \\ 
{\mathbf{RHS}}(\lnot c) & =_{def} & \lnot {\mathbf{RHS}}(\lnot c)=\lnot c \\ 
{\mathbf{RHS}}(\lnot \lnot \varphi ) & =_{def} & \mathbf{RHS}(\varphi ) \\ 
{\mathbf{RHS}}(\lnot (\varphi \wedge \psi )) & =_{def} & {\mathbf{RHS}}%
(\lnot \varphi )\vee {\mathbf{RHS}}(\lnot \psi ) \\ 
{\mathbf{RHS}}(\lnot (\varphi \vee \psi )) & =_{def} & {\mathbf{RHS}}(\lnot
\varphi )\wedge {\mathbf{RHS}}(\lnot \psi ) \\ 
{\mathbf{RHS}}(\lnot (\varphi \Rightarrow \psi )) & =_{def} & {\mathbf{RHS}}%
(\varphi )\wedge {\mathbf{RHS}}(\lnot \psi ) \\ 
{\mathbf{RHS}}(\lnot (\forall {x{:}D}.\varphi )) & =_{def} & \exists {x{:}D}.%
{\mathbf{RHS}}(\lnot \varphi ) \\ 
{\mathbf{RHS}}(\lnot (\exists {x{:}D}.\varphi )) & =_{def} & \forall {x{:}D}.%
{\mathbf{RHS}}(\lnot \varphi ) \\ 
{\mathbf{RHS}}(\lnot ([\alpha ]\varphi )) & =_{def} & {\mathbf{RHS}}(\langle
\alpha \rangle (\lnot \varphi )) \\ 
{\mathbf{RHS}}(\lnot (\langle \alpha \rangle \varphi )) & =_{def} & {\mathbf{%
RHS}}([\alpha ](\lnot \varphi )) \\ 
{\mathbf{RHS}}(\lnot X(d)) & =_{def} & {\mathbf{RHS}}(X(d)) \\ 
{\mathbf{RHS}}(\lnot (\sigma X(x_{f}{:}D_{f}:=d).~\varphi )) & =_{def}~~~ & {%
\mathbf{RHS}}(\widetilde{\sigma }X(x_{f}{:}D_{f}:=d).~(\lnot \varphi \lbrack
X:=\lnot X]))={\mathbf{RHS}}((\sigma X(x_{f}{:}D_{f}:=d).~\lnot \varphi ))
\\ 
{\mathbf{RHS}}(\lnot \nabla (t)) & =_{def} & {\mathbf{RHS}}(\Delta (t)) \\ 
{\mathbf{RHS}}(\lnot \Delta (t)) & =_{def} & {\mathbf{RHS}}(\nabla (t))%
\end{array}%
\end{equation*}

\pagebreak

%-------------------------------------%
%           function E
%-------------------------------------%
\begin{equation*}
\begin{array}{lll}
{\mathbf{E}}(c) & =_{def} & \epsilon \\ 
{\mathbf{E}}(\varphi \wedge \psi ) & =_{def} & {\mathbf{E}}(\varphi ){%
\mathbf{E}}(\psi ) \\ 
{\mathbf{E}}(\varphi \vee \psi ) & =_{def} & {\mathbf{E}}(\varphi ){\mathbf{E%
}}(\psi ) \\ 
{\mathbf{E}}(\varphi \Rightarrow \psi ) & =_{def} & {\mathbf{E}}(\lnot
\varphi ){\mathbf{E}}(\psi ) \\ 
{\mathbf{E}}(\forall {x{:}D}.\varphi ) & =_{def} & {\mathbf{E}}(\varphi ) \\ 
{\mathbf{E}}(\exists {x{:}D}.\varphi ) & =_{def} & {\mathbf{E}}(\varphi ) \\ 
{\mathbf{E}}([\alpha ]\varphi ) & =_{def} & {\mathbf{E}}(\varphi ) \\ 
{\mathbf{E}}(\langle \alpha \rangle \varphi ) & =_{def} & {\mathbf{E}}%
(\varphi ) \\ 
{\mathbf{E}}(\nabla ) & =_{def} & \epsilon \\ 
{\mathbf{E}}(\nabla (t)) & =_{def} & \epsilon \\ 
{\mathbf{E}}(\Delta ) & =_{def} & \epsilon \\ 
{\mathbf{E}}(\Delta (t)) & =_{def} & \epsilon \\ 
{\mathbf{E}}(X(d)) & =_{def} & \epsilon \\ 
{\mathbf{E}}(\sigma X(x_{f}{:}D_{f}:=d).~\varphi )~~~ & =_{def}~~~ & (\sigma 
\tilde{X}(T:\mathbb{R},x_{f}{:}D_{f},x_{p}{:}D_{p},\mathbf{Par}%
_{X,[]}(\varphi _{0}))={\mathbf{RHS}}(\varphi )~)~{\mathbf{E}}(\varphi ) \\ 
{\mathbf{E}}(\lnot c) & =_{def} & \epsilon \\ 
{\mathbf{E}}(\lnot \lnot \varphi ) & =_{def} & {\mathbf{E}}(\varphi ) \\ 
{\mathbf{E}}(\lnot (\varphi \wedge \psi )) & =_{def} & {\mathbf{E}}(\lnot
\varphi ){\mathbf{E}}(\lnot \psi ) \\ 
{\mathbf{E}}(\lnot (\varphi \vee \psi )) & =_{def} & {\mathbf{E}}(\lnot
\varphi ){\mathbf{E}}(\lnot \psi ) \\ 
{\mathbf{E}}(\lnot (\varphi \Rightarrow \psi )) & =_{def} & {\mathbf{E}}%
(\varphi ){\mathbf{E}}(\lnot \psi ) \\ 
{\mathbf{E}}(\lnot (\forall {x{:}D}.\varphi )) & =_{def} & {\mathbf{E}}%
(\lnot \varphi ) \\ 
{\mathbf{E}}(\lnot (\exists {x{:}D}.\varphi )) & =_{def} & {\mathbf{E}}%
(\lnot \varphi ) \\ 
{\mathbf{E}}(\lnot ([\alpha ]\varphi )) & =_{def} & {\mathbf{E}}(\lnot
\varphi ) \\ 
{\mathbf{E}}(\lnot (\langle \alpha \rangle \varphi )) & =_{def} & {\mathbf{E}%
}(\lnot \varphi ) \\ 
{\mathbf{E}}(\lnot \nabla ) & =_{def} & \epsilon \\ 
{\mathbf{E}}(\lnot \nabla (t)) & =_{def} & \epsilon \\ 
{\mathbf{E}}(\lnot \Delta ) & =_{def} & \epsilon \\ 
{\mathbf{E}}(\lnot \Delta (t)) & =_{def} & \epsilon \\ 
{\mathbf{E}}(\lnot X(d)) & =_{def} & \epsilon \\ 
{\mathbf{E}}(\lnot \sigma X(x_{f}{:}D_{f}:=d).~\varphi )~~~ & =_{def}~~~ & (%
\widetilde{\sigma }\tilde{X}(T:\mathbb{R},x_{f}{:}D_{f},x_{p}{:}D_{p},%
\mathbf{Par}_{X,[]}(\varphi _{0}))={\mathbf{RHS}}(\lnot \varphi )~[X:=\lnot
X])~{\mathbf{E}}(\lnot \varphi ),%
\end{array}%
\end{equation*}%
\pagebreak where $\widetilde{\sigma }=\mu $ if $\sigma =\nu $ and $%
\widetilde{\sigma }=\nu $ if $\sigma =\mu $ and $\tilde{X}$ is a fresh
predicate variable.

\newpage

\section{Bisimulation algorithms}

Let%
\begin{eqnarray*}
M(d) &=&\sum\limits_{i\in I_{M}}\sum_{e:E_{i}}c_{i}(d,e)\rightarrow
a_{i}(d,e)\cdot M(g_{i}(d,e)) \\
S(d) &=&\sum\limits_{i\in I_{S}}\sum_{e:E_{i}}c_{i}(d,e)\rightarrow
a_{i}(d,e)\cdot M(g_{i}(d,e))
\end{eqnarray*}%
be two linear processes, such that $I_{M}\cap I_{S}=\emptyset $. $M$ is
called the model and $S$ the specification. The expression $a_{i}(d,e)$ can
be a multi-action, or have the special value $\tau $. We assume that there
are no $\delta $ summands. We define four pbes equation systems that express
some kind of bisimulation equivalence between $M$ and $S$.

\paragraph{Branching Bisimulation}

\emph{brbsim}$(M,S)=\nu E_{2}\mu E_{1}$, where%
\begin{equation*}
\begin{array}{ccl}
E_{2} & := & \{X^{M,S}(d,d^{\prime })=match^{M,S}(d,d^{\prime })\wedge
match^{S,M}(d^{\prime },d), \\ 
&  & X^{S,M}(d^{\prime },d)=X^{M,S}(d,d^{\prime })\} \\ 
E_{1} & := & \{Y_{i}^{M,S}(d,d^{\prime },e)=close_{i}^{M,S}(d,d^{\prime
},e)|i\in I_{M}, \\ 
&  & Y_{i}^{S,M}(d^{\prime },d,e)=close_{i}^{S,M}(d^{\prime },d,e)|i\in
I_{S}\}%
\end{array}%
\end{equation*}%
with for all $i\in I_{p}$ and $(p,q)\in \{(M,S),(S,M)\}$:%
\begin{eqnarray*}
match^{p,q}(d,d^{\prime }) &=&\bigwedge\limits_{i\in I_{p}}\forall
_{e:E_{i}}.(c_{i}(d,e)\Rightarrow Y_{i}^{p,q}(d,d^{\prime },e)) \\
close_{i}^{p,q}(d,d^{\prime },e) &=&\bigvee\limits_{\{j\in I_{q}|a_{j}=\tau
\}}\exists _{e^{\prime }:E_{j}}.(c_{j}(d^{\prime },e^{\prime })\wedge
Y_{i}^{p,q}(d,g_{j}(d^{\prime },e^{\prime }),e)) \\
&&\vee (X^{p,q}(d,d^{\prime })\wedge step_{i}^{p,q}(d,d^{\prime },e)) \\
step_{i}^{p,q}(d,d^{\prime },e) &=&\left\{ 
\begin{array}{cl}
a_{i}=\tau : & X^{p,q}(g_{i}(d,e),d^{\prime })\vee \bigvee\limits_{\{j\in
I_{q}|a_{j}=\tau \}}\exists _{e^{\prime }:E_{j}}.(c_{j}(d^{\prime
},e^{\prime })\wedge X^{p,q}(g_{i}(d,e),g_{j}(d^{\prime },e^{\prime })) \\ 
a_{i}\neq \tau : & \bigvee\limits_{\{j\in I_{q}|a_{j}=a_{i}\}}\exists
_{e^{\prime }:E_{j}}.(c_{j}(d^{\prime },e^{\prime })\wedge
(a_{i}(d,e)=a_{j}(d^{\prime },e^{\prime }))\wedge
X^{p,q}(g_{i}(d,e),g_{j}(d^{\prime },e^{\prime }))%
\end{array}%
\right.
\end{eqnarray*}

\paragraph{Strong Bisimulation}

\emph{sbisim}$(M,S)=\nu E$, where%
\begin{equation*}
\begin{array}{ccl}
E & := & \{X^{M,S}(d,d^{\prime })=match^{M,S}(d,d^{\prime })\wedge
match^{S,M}(d^{\prime },d), \\ 
&  & X^{S,M}(d^{\prime },d)=X^{M,S}(d,d^{\prime })\}%
\end{array}%
\end{equation*}%
with for all $i\in I_{p}$ and $(p,q)\in \{(M,S),(S,M)\}$:%
\begin{eqnarray*}
match^{p,q}(d,d^{\prime }) &=&\bigwedge\limits_{i\in I_{p}}\forall
_{e:E_{i}}.(c_{i}(d,e)\Rightarrow step_{i}^{p,q}(d,d^{\prime },e)) \\
step_{i}^{p,q}(d,d^{\prime },e) &=&\bigvee\limits_{j\in I_{q}}\exists
_{e^{\prime }:E_{j}}.(c_{j}(d^{\prime },e^{\prime })\wedge
(a_{i}(d,e)=a_{j}(d^{\prime },e^{\prime }))\wedge
X^{p,q}(g_{i}(d,e),g_{j}(d^{\prime },e^{\prime }))
\end{eqnarray*}

\paragraph{Weak Bisimulation}

\emph{wbisim}$(M,S)=\nu E_{2}\mu E_{1}$, where%
\begin{equation*}
\begin{array}{ccl}
E_{3} & := & \{X^{M,S}(d,d^{\prime })=match^{M,S}(d,d^{\prime })\wedge
match^{S,M}(d^{\prime },d), \\ 
&  & X^{S,M}(d^{\prime },d)=X^{M,S}(d,d^{\prime })\} \\ 
E_{2} & := & \{Y_{1,i}^{M,S}(d,d^{\prime },e)=close_{1,i}^{M,S}(d,d^{\prime
},e)|i\in I_{M}, \\ 
&  & Y_{2,i}^{M,S}(d,d^{\prime })=close_{2,i}^{M,S}(d,d^{\prime })|i\in
I_{M}, \\ 
&  & Y_{1,i}^{S,M}(d^{\prime },d,e)=close_{1,i}^{S,M}(d^{\prime },d,e)|i\in
I_{S}, \\ 
&  & Y_{2,i}^{S,M}(d^{\prime },d)=close_{2,i}^{S,M}(d^{\prime },d)|i\in
I_{S}\}%
\end{array}%
\end{equation*}%
with for all $i\in I_{p}$ and $(p,q)\in \{(M,S),(S,M)\}$:%
\begin{eqnarray*}
match^{p,q}(d,d^{\prime }) &=&\bigwedge\limits_{i\in I_{p}}\forall
_{e:E_{i}}.(c_{i}(d,e)\Rightarrow Y_{1,i}^{p,q}(d,d^{\prime },e)) \\
close_{1,i}^{p,q}(d,d^{\prime },e) &=&\left( \bigvee\limits_{\{j\in
I_{q}|a_{j}=\tau \}}\exists _{e^{\prime }:E_{j}}.(c_{j}(d^{\prime
},e^{\prime })\wedge Y_{1,i}^{p,q}(d,g_{j}(d^{\prime },e^{\prime
}),e))\right) \vee step_{i}^{p,q}(d,d^{\prime },e) \\
step_{i}^{p,q}(d,d^{\prime },e) &=&\left\{ 
\begin{array}{cl}
a_{i}=\tau : & close_{2,i}^{p,q}(g_{i}(d,e),d^{\prime }) \\ 
a_{i}\neq \tau : & \bigvee\limits_{j\in I_{q}}\exists _{e^{\prime
}:E_{j}}.\left( c_{j}(d^{\prime },e^{\prime })\wedge
a_{i}(d,e)=a_{j}(d^{\prime },e^{\prime })\wedge
close_{2,i}^{p,q}(g_{i}(d,e),g_{j}(d^{\prime },e^{\prime }))\right)%
\end{array}%
\right. \\
close_{2,i}^{p,q}(d,d^{\prime }) &=&X^{p,q}(d,d^{\prime })\vee
\bigvee\limits_{\{j\in I_{q}|a_{j}=\tau \}}\left( \exists _{e^{\prime
}:E_{j}}c_{j}(d^{\prime },e^{\prime })\wedge Y_{2,i}^{p,q}(d,g_{j}(d^{\prime
},e^{\prime }))\right)
\end{eqnarray*}

\paragraph{Branching Simulation Equivalence}

\emph{brbsim}$(M,S)=\nu E_{2}\mu E_{1}$, where%
\begin{equation*}
\begin{array}{ccl}
E_{2} & := & \{X^{M,S}(d,d^{\prime })=match^{M,S}(d,d^{\prime })\wedge
match^{S,M}(d^{\prime },d), \\ 
&  & X^{M,S}(d,d^{\prime })=X^{S,M}(d^{\prime },d), \\ 
&  & X^{S,M}(d^{\prime },d)=X^{M,S}(d,d^{\prime })\} \\ 
E_{1} & := & \{Y_{i}^{M,S}(d,d^{\prime },e)=close_{i}^{M,S}(d,d^{\prime
},e)|i\in I_{M}, \\ 
&  & Y_{i}^{S,M}(d^{\prime },d,e)=close_{i}^{S,M}(d^{\prime },d,e)|i\in
I_{S}\}%
\end{array}%
\end{equation*}%
with $match$, $close$, and $step$ defined exactly the same as in branching
bisimulation.\newpage

\section{PBES rewriters}

In this section we describe two PBES rewriters. We assume that a data
rewriter $\mathsf{datar}$ is given that rewrites data terms.

\subsection{Simplifying rewriter}

We define a simplifying PBES rewriter $\mathsf{pbesr}$ recursively as follows%
\begin{equation*}
\begin{array}{ccl}
\mathsf{pbesr}(b) & = & \mathsf{datar}(b) \\ 
\mathsf{pbesr}(\lnot \varphi ) & = & \lnot \mathsf{pbesr}(\varphi ) \\ 
&  &  \\ 
\mathsf{pbesr}(\varphi \wedge \psi ) & = & \left\{ 
\begin{array}{lcl}
false &  & \text{if }\mathsf{pbesr}(\varphi )=false\vee \mathsf{pbesr}(\psi
)=false \\ 
\mathsf{pbesr}(\psi ) &  & \text{if }\mathsf{pbesr}(\varphi )=true \\ 
\mathsf{pbesr}(\varphi ) &  & \text{if }\mathsf{pbesr}(\psi )=true \\ 
\mathsf{pbesr}(\varphi ) &  & \text{if }\mathsf{pbesr}(\varphi )=\mathsf{%
pbesr}(\psi ) \\ 
\mathsf{pbesr}(\varphi )\wedge \mathsf{pbesr}(\psi ) &  & \text{otherwise}%
\end{array}%
\right. \\ 
&  &  \\ 
\mathsf{pbesr}(\varphi \vee \psi ) & = & \left\{ 
\begin{array}{lcl}
true &  & \text{if }\mathsf{pbesr}(\varphi )=true\vee \mathsf{pbesr}(\psi
)=true \\ 
\mathsf{pbesr}(\psi ) &  & \text{if }\mathsf{pbesr}(\varphi )=false \\ 
\mathsf{pbesr}(\varphi ) &  & \text{if }\mathsf{pbesr}(\psi )=false \\ 
\mathsf{pbesr}(\varphi ) &  & \text{if }\mathsf{pbesr}(\varphi )=\mathsf{%
pbesr}(\psi ) \\ 
\mathsf{pbesr}(\varphi )\vee \mathsf{pbesr}(\psi ) &  & \text{otherwise}%
\end{array}%
\right. \\ 
&  &  \\ 
\mathsf{pbesr}(\varphi \rightarrow \psi ) & = & \left\{ 
\begin{array}{lcl}
\mathsf{pbesr}(\psi ) &  & \text{if }\mathsf{pbesr}(\varphi )=true \\ 
true &  & \text{if }\mathsf{pbesr}(\varphi )=false \\ 
true &  & \text{if }\mathsf{pbesr}(\psi )=true \\ 
\mathsf{pbesr}(\lnot \varphi ) &  & \text{if }\mathsf{pbesr}(\psi )=false \\ 
true &  & \text{if }\mathsf{pbesr}(\varphi )=\mathsf{pbesr}(\psi ) \\ 
\mathsf{pbesr}(\varphi )\rightarrow \mathsf{pbesr}(\psi ) &  & \text{%
otherwise}%
\end{array}%
\right. \\ 
&  &  \\ 
\mathsf{pbesr}(\forall _{d:D}.\varphi ) & = & \left\{ 
\begin{array}{lcl}
true &  & \text{if }\mathsf{pbesr}(\varphi )=true \\ 
false &  & \text{if }\mathsf{pbesr}(\varphi )=false\text{ and }D\text{ is
non-empty} \\ 
\mathsf{pbesr}(\varphi ) &  & \text{if }d\text{ does not occur in }\varphi
\\ 
\forall _{d:D}.\mathsf{pbesr}(\varphi ) &  & \text{otherwise}%
\end{array}%
\right. \\ 
&  &  \\ 
\mathsf{pbesr}(\exists _{d:D}.\varphi ) & = & \left\{ 
\begin{array}{lcl}
true &  & \text{if }\mathsf{pbesr}(\varphi )=true\text{ and }D\text{ is
non-empty} \\ 
false &  & \text{if }\mathsf{pbesr}(\varphi )=false \\ 
\mathsf{pbesr}(\varphi ) &  & \text{if }d\text{ does not occur in }\varphi
\\ 
\exists _{d:D}.\mathsf{pbesr}(\varphi ) &  & \text{otherwise}%
\end{array}%
\right. \\ 
&  &  \\ 
\mathsf{pbesr}(X(e)) & = & X(\mathsf{datar}(e))%
\end{array}%
\end{equation*}%
where $b$ is a data term of data sort $\mathbb{B}$, $true$ and $false$ are
elements of data sort $\mathbb{B}$, $X$ is a predicate variable, $e$consists
of zero or more data sorts and $d,d_{1},d_{2}$ are data variables of sort $D$%
.

\paragraph{\newpage Simplify}

The pbes expression rewrite system \textsc{Simplify} [Luc Engelen, 2007]
consists of the following rules\footnote{%
Todo: reformulate this rewrite system.}:%
\begin{eqnarray*}
false\wedge x &\rightarrow &false \\
x\wedge false &\rightarrow &false \\
true\wedge x &\rightarrow &x \\
x\wedge true &\rightarrow &x \\
\lnot true &\rightarrow &false \\
\lnot false &\rightarrow &true \\
ITE(true,x,y) &\rightarrow &x \\
ITE(false,x,y) &\rightarrow &y \\
x &=&x\rightarrow true \\
y &=&x\rightarrow x=y,\text{ provided }y\succ x
\end{eqnarray*}

\subsection{Quantifier Elimination Rewriter}

This section describes a rewriter on predicate formulae that eliminates
quantifiers. It is based on the following property%
\begin{equation*}
\begin{array}{cc}
\forall _{x:X}.\varphi \equiv \dbigwedge\limits_{y:X}\varphi \lbrack
x:=y]\quad & \exists _{x:X}.\varphi \equiv \dbigvee\limits_{y:X}\varphi
\lbrack x:=y],%
\end{array}%
\end{equation*}

where the conjunction and disjunction on the right hand sides may be
infinite. Because of this, the rewriter we describe here is not guaranteed
to terminate. However, in many practical cases the rewriter can compute a
finite expression even if the quantifier variables are of infinite sort. An
example of this is the formula $\forall _{n:\mathbb{N}}.(n>2)\vee X(n)$ that
can be rewritten into $X(0)\wedge X(1)\wedge X(2)$.

We assume that the sorts of quantifier variables can be enumerated. By this
we mean the existence of a function $enum$ that maps an arbitrary term $d:D$
to a finite set of terms $\{d_{1},\cdots ,d_{k}\}$, such that $%
range(d)=\dbigcup\limits_{i=1\cdots k}range(d_{i})$, where $range(d)$ is the
set of closed terms obtained from $d$ by substituting values for the free
variables of $d$. For example, if natural numbers are represented by $%
S^{n}(0)$, with $S$ a function that expresses the successor of a number,
then possible enumerations of the term $n$ are $\{0,S(n^{\prime })\}$ and $%
\{0,S(0),S(S(n^{\prime \prime }))\}$. Let $id$ be the identity function and
let $\sigma \lbrack d_{1}:=e_{1},\cdots ,d_{n}:=e_{n}]$ be the function $%
\sigma ^{\prime }$ with $\sigma ^{\prime }(x)=e_{i}$ if $x=d_{i}$ and $%
\sigma ^{\prime }(x)=\sigma (x)$ otherwise.

A parameter of the \textsc{EliminateQuantifiers }algorithm is a rewriter $R$
on quantifier free predicate formulae, that is expected to have the
following properties:%
\begin{eqnarray*}
R(\bot ) &=&\bot \\
(R(t) &=&R(t^{\prime }))\Rightarrow t\simeq t^{\prime }\text{,}
\end{eqnarray*}%
where $t\simeq t^{\prime }$ indicates that $t$ and $t^{\prime }$ are
equivalent. 
\begin{equation*}
\begin{tabular}{l}
\textsc{EliminateQuantifiers(}$Q_{d_{1}:D_{1},\ldots ,d_{n}:D_{n}}.\varphi
,R $\textsc{)} \\ 
$\text{\textbf{if }}freevars(R(\varphi ))\cap \{d_{1},\cdots
,d_{n}\}=\emptyset $ $\text{\textbf{then return }}R(\varphi )$ \\ 
$V:=\emptyset $ \\ 
$\text{\textbf{for }}i\in \{1,\ldots ,n\}\text{ \textbf{do }}%
E_{i}:=\{d_{i}\} $ \\ 
$\text{\textbf{do}}$ \\ 
$\qquad \text{\textbf{choose }}e_{k}\in E_{k}$, such that $\mathsf{dvar}%
(e_{k})\neq \emptyset $ \\ 
$\qquad E_{k}:=E_{k}\backslash \{e_{k}\}$ \\ 
$\qquad \text{\textbf{for }}e\in enum(e_{k}):$ \\ 
$\qquad \qquad W:=\emptyset $ \\ 
$\qquad \qquad \text{\textbf{for }}\sigma \in \{id[d_{1}:=f_{1},\cdots
,d_{k-1}:=f_{k-1},d_{k}:=e,d_{k+1}:=f_{k+1},\cdots ,d_{n}:=f_{n}]$ \\ 
$\qquad \qquad \qquad \wedge f_{i}\in E_{i}\quad (i=1,\cdots ,n)\}:$ \\ 
$\qquad \qquad \qquad W:=W\cup \{R(\sigma (\varphi )\}$ \\ 
$\qquad \qquad \text{\textbf{if }}stop_{Q}\in W\text{ \textbf{then return }}%
stop_{Q}$ \\ 
$\qquad \qquad V:=V\cup \{w\in W\ |\ \mathsf{dvar}(w)\subset \mathsf{dvar}%
(\varphi )\}$ \\ 
$\qquad \qquad \text{\textbf{if }}\{w\in W\ |\ \mathsf{dvar}(w)\varsubsetneq 
\mathsf{dvar}(\varphi )\}\neq \emptyset $ $\text{\textbf{then }}%
E_{k}:=E_{k}\cup \{e\}$ \\ 
$\qquad $\textbf{rof} \\ 
$\text{\textbf{while }}\forall _{i\in \{1,\ldots ,n\}}.E_{i}\neq \emptyset $
\\ 
$\text{\textbf{return} }\dbigoplus\limits_{v\in V}v,$%
\end{tabular}%
\end{equation*}%
where $stop_{Q}=\bot $ and $\dbigoplus =\dbigwedge $ in case $Q=\forall $,
and where $stop_{Q}=\top $ and $\dbigoplus =\dbigvee $ in case $Q=\exists $%
.\newpage

\section{PBES instantiation}

In this section we describe an implementation of the instantiation algorithm 
$Inst$. Let $\mathcal{E=(\sigma }_{1}X_{1}(d_{1}:D_{1})=\varphi _{1})\cdots 
\mathcal{(\sigma }_{n}X_{n}(d_{n}:D_{n})=\varphi _{n})$ be a PBES, and $%
X_{init}(e_{init})$ an initial state. The algorithm \textsc{Pbes2bes} uses
instantiation to compute a BES. It takes two extra parameters, an injective
function $\rho $ that renames proposition variables to predicate variables,
and a rewriter $R$ that eliminates quantifiers from predicate formulae. This
rewriter is described in the next section.%
\begin{equation*}
\begin{array}{l}
\text{\textsc{Pbes2bes(}}\mathcal{E}\text{, }X_{init}(e_{init})\text{, }R%
\text{, }\rho \text{\textsc{)}} \\ 
\text{\textbf{for }}i:=1\cdots n\text{ \textbf{do }}\mathcal{E}_{i}:=\epsilon
\\ 
todo:=\{R(X_{init}(e_{init}))\} \\ 
done:=\emptyset \\ 
\text{\textbf{while }}todo\neq \emptyset \text{ \textbf{do}} \\ 
\qquad \text{\textbf{choose }}X_{k}(e)\in todo \\ 
\qquad todo:=todo\ \backslash \ \{X_{k}(e)\} \\ 
\qquad done:=done\cup \{X_{k}(e)\} \\ 
\qquad X^{e}:=\rho (X_{k}(e)) \\ 
\qquad \psi ^{e}:=R(\varphi _{k}[d_{k}:=e]) \\ 
\qquad \mathcal{E}_{k}:=\mathcal{E}_{k}(\mathcal{\sigma }_{k}X^{e}=\rho
(\psi ^{e})) \\ 
\qquad todo:=todo\cup \{Y(f)\in \mathsf{occ}(\psi ^{e})\ |\ Y(f)\notin done\}
\\ 
\text{\textbf{return }}\mathcal{E}_{1}\cdots \mathcal{E}_{n},%
\end{array}%
\end{equation*}%
where $\rho $ is extended from predicate variables to quantifier free
predicate formulae using%
\begin{equation*}
\begin{array}{cc}
\rho (b)=_{def}b & \quad \rho (\varphi \oplus \psi )=_{def}\rho (\varphi
)\oplus \rho (\psi )%
\end{array}%
\end{equation*}

\newpage

\section{Constant Parameter Detection and Elimination}

Let $\mathcal{E=(\sigma }_{1}X_{1}(d_{X_{1}}:D_{X_{1}})=\varphi
_{X_{1}})\cdots \mathcal{(\sigma }_{n}X_{n}(d_{X_{n}}:D_{X_{n}})=\varphi
_{X_{n}})$ be a PBES, and $\kappa $ an initial state and let $\mathsf{eval}$
be an evaluator function on data expressions. We denote the $i$-th element
of a vector $x$ as $x[i]$. Then we define the algorithm \textsc{PbesConstelm}
as follows:$\qquad \qquad $%
\begin{equation*}
\begin{array}{l}
\text{\textsc{PbesConstelm(}}\mathcal{E}\text{, }\kappa \text{, }\mathsf{eval%
}\text{\textsc{)}} \\ 
\text{\textbf{for }}X\in \mathsf{occ}(\mathcal{E)}\text{ \textbf{do }}%
c_{X}:=d_{X} \\ 
\text{\textbf{for }}X(e)\in \mathsf{iocc}(\kappa \mathcal{)}\text{ \textbf{%
do }}c_{X}:=update_{X}(c_{X},\mathsf{eval}(e[d_{X}:=c_{X}])) \\ 
todo:=\mathsf{occ}(\kappa \mathcal{)} \\ 
\text{\textbf{while }}todo\neq \emptyset \text{ \textbf{do}} \\ 
\qquad \text{\textbf{choose }}X\in todo \\ 
\qquad todo:=todo\ \backslash \ \{X\} \\ 
\qquad \text{\textbf{for }}Y(e)\in \mathsf{iocc}(\varphi _{X}\mathcal{)}%
\text{ \textbf{do}} \\ 
\qquad \qquad \text{\textbf{if }}\mathsf{eval}(do%
\_update(Y(e)[d_{X}:=c_{X}]))\neq false\text{ \textbf{then}} \\ 
\qquad \qquad \qquad c_{Y}^{\prime }:=update_{X}(c_{Y},\mathsf{eval}%
(e[d_{X}:=c_{X}])) \\ 
\qquad \qquad \qquad \text{\textbf{if }}c_{Y}^{\prime }\neq c_{Y}\text{ 
\textbf{then}} \\ 
\qquad \qquad \qquad \qquad c_{Y}:=c_{Y}^{\prime } \\ 
\qquad \qquad \qquad \qquad todo:=todo\cup \{Y\} \\ 
redundant\_variables=\{X\in \mathsf{occ}(\mathcal{E)}\ |\ c_{X}=d_{X}\} \\ 
redundant\_parameters=\{(X,i)\ |\ c_{X}[i]\in D_{X}[i]\} \\ 
\text{\textbf{return }}\{redundant\_variables\text{, }redundant\_parameters\}%
\end{array}%
\end{equation*}

where $update$ is defined as follows:%
\begin{equation*}
update_{X}(c,c^{\prime })=_{def}c^{\prime \prime },\text{ with }c^{\prime
\prime }[i]=\left\{ 
\begin{array}{ll}
c[i] & \text{if }c[i]=c^{\prime }[i] \\ 
c[i] & \text{if }c[i]\notin D_{X}\cup \{d_{X}[i]\} \\ 
c^{\prime }[i] & \text{if }c[i]=d_{X}[i]\wedge c^{\prime }[i]\notin D_{X} \\ 
fresh\_var(D_{X}[i]) & \text{otherwise}%
\end{array}%
\right.
\end{equation*}

and where $do\_update$ is a boolean function that determines whether an
update should be performed. A safe choice for this function is the constant
function $true$, but a more sophisticated choice can be made (see [Simon
Janssen, 2008]).\pagebreak

\section{Gau\ss\ Elimination}

A predicate formula $\varphi $ is defined by the following grammar:%
\begin{equation*}
\varphi ::=b|X(e)|\lnot \varphi |\varphi \wedge \varphi |\varphi \vee
\varphi |\varphi \rightarrow \varphi |\forall d:D.\varphi |\exists
d:D.\varphi |true |false
\end{equation*}%
where $b$ is a data term of sort $\mathbb{B}$, $X$ is a predicate variable, $%
d$ is a data variable of sort $D$, $e$ is a data term, $true $ represents $%
true$, and $false $ represents $false$.

\begin{definition}
(Predicate Variable Substitution) Let $\varphi ,\psi $ be predicate formulae
and $X$ a predicate variable. Then we define $\psi \lbrack \varphi /X]$ as
the result of applying the substitution $X:=\varphi $ to the formula $\psi $%
. To make this more precise: suppose $X$ is declared as $X(d:D)$, then any
occurrence $X(\overline{d})$ in $\psi $ is replaced by $\varphi \lbrack d:=%
\overline{d}]$.
\end{definition}

\begin{lemma}
(Substitution) Let $\mathcal{E}$ be an equation system for which $X,Y\notin
bnd(\mathcal{E})$, then:%
\begin{equation*}
(\sigma X(d:D)=\varphi )\mathcal{E}(\sigma ^{\prime }Y(e:E)=\psi )\equiv
(\sigma X(d:D)=\varphi )[\psi /Y]\mathcal{E}(\sigma ^{\prime }Y(e:E)=\psi )
\end{equation*}
\end{lemma}

\begin{definition}
(Approximation) Let $\varphi ,\psi $ be predicate formulae and $X$ a
predicate variable. We inductively define $\psi \lbrack \varphi /X]^{k}$ as
follows:%
\begin{eqnarray*}
&&\psi \lbrack \varphi /X]^{0}\overset{def}{=}\varphi \\
&&\psi \lbrack \varphi /X]^{k+1}\overset{def}{=}\psi \lbrack \varphi /X]^{k}
\end{eqnarray*}
\end{definition}

Thus, $\psi \lbrack \varphi /X]^{k}$ represents the result of recursively
substituting $\varphi $ for $X$ in $\psi $.

\begin{lemma}
(Approximants as Solutions) Let $\varphi $ be a predicate formula and $k\in 
\mathbb{N}$ be a natural number. Then%
\begin{eqnarray*}
(\mu X(d &:&D)=\varphi \lbrack false /X]^{k})\Rrightarrow (\mu
X(d:D)=\varphi ) \\
(\nu X(d &:&D)=\varphi )\Rrightarrow (\nu X(d:D)=\varphi \lbrack true
/X]^{k})
\end{eqnarray*}
\end{lemma}

\begin{lemma}
(Stable Approximants as Solutions) Let $\varphi $ be a predicate formula and 
$k\in \mathbb{N}$ be a natural number. Then%
\begin{eqnarray*}
\text{if }\varphi \lbrack false/X]^{k} &\longleftrightarrow &\varphi \lbrack
false/X]^{k+1}\text{ then }(\mu X(d:D)=\varphi \lbrack false/X]^{k})\equiv
(\mu X(d:D)=\varphi ) \\
\text{if }\varphi \lbrack true/X]^{k} &\longleftrightarrow &\varphi \lbrack
true/X]^{k+1}\text{ then }(\nu X(d:D)=\varphi \lbrack true/X]^{k})\equiv
(\nu X(d:D)=\varphi )
\end{eqnarray*}
\end{lemma}

\subsection{Gau\ss\ Elimination Algorithm}

Let $\mathcal{E}$ be an equation system of the form%
\begin{equation*}
\mathcal{E=(}\sigma _{1}X_{1}(d_{1}:D_{1})=\varphi _{1})\cdots (\sigma
_{n}X_{n}(d_{n}:D_{n})=\varphi _{n}),
\end{equation*}%
and let $r$ be a rewrite function that maps a pbes expression $\varphi $ to
an equivalent expression $\varphi ^{\prime }$.

Then we define%
\begin{equation*}
\begin{array}{l}
\text{\textsc{Gau\textsc{\ss\ }Elimination(}}\mathcal{E},r\text{\textsc{)}}
\\ 
\mathcal{E}^{\prime }:=\varepsilon \\ 
i:=n \\ 
\text{\textbf{while} \textbf{not} }i=0 \\ 
\text{\textbf{do}} \\ 
\qquad (\sigma _{i}X_{i}=\psi _{i}):=\text{\textsc{SolveEquation(}}\sigma
_{i}X_{i}=\varphi _{i}\text{\textsc{)}} \\ 
\qquad \varphi _{i}:=\psi _{i} \\ 
\qquad \mathcal{E}^{\prime }:=\mathcal{E}^{\prime }(\sigma _{i}X_{i}=\varphi
_{i}) \\ 
\qquad \text{\textbf{for }}k=1\text{ \textbf{to }}i-1\text{ \textbf{do }}%
\varphi _{k}:=r(\varphi _{k}[\varphi _{i}/X_{i}])\text{ \textbf{od}} \\ 
\qquad i:=i-1 \\ 
\text{\textbf{od}} \\ 
\text{\textbf{return }}\mathcal{E}^{^{\prime }}%
\end{array}%
\end{equation*}%
Here \textsc{SolveEquation} is an algorithm that solves a pbes equation,
such that the resulting equation has no reference to the predicate variable
in its right hand side. An example of such a solve equation algorithm is 
\textsc{Approximate}.%
\begin{equation*}
\begin{array}{l}
\text{\textsc{Approximate(}}\sigma X=\varphi \text{\textsc{)}} \\ 
j:=0 \\ 
\text{\textbf{if }}\sigma =\nu \text{ \textbf{then} }\psi _{0}:=true \text{ 
\textbf{else} }\psi _{0}:=false \\ 
\text{\textbf{repeat}} \\ 
\qquad \psi _{j+1}:=\varphi \lbrack \psi _{j}/X] \\ 
\qquad j:=j+1 \\ 
\text{\textbf{until }}(\psi _{j}=\psi _{j+1}) \\ 
\text{\textbf{return }}\sigma X=\psi _{j}%
\end{array}%
\end{equation*}

Also pattern matching algorithms exist for this. The \textsc{Gau\textsc{\ss\ 
}Elimination} algorithm solves the equation system $\mathcal{E}$ for the
predicate variable $X_{1}$. To solve the system $\mathcal{E}$ for all
variables, the algorithm has to be applied repeatedly.

\subsection{Solving a BES}

If the equation system $\mathcal{E}$ is a BES (i.e. the predicate variables
have no parameters), then the following simple approximate function can be
used to solve it:%
\begin{equation*}
\begin{array}{l}
\text{\textsc{Approximate-BES(}}\sigma X=\varphi \text{\textsc{)}} \\ 
\text{\textbf{if }}\sigma =\nu \text{ \textbf{then} }\psi _{0}:=true \text{ 
\textbf{else} }\psi _{0}:=false \\ 
\text{\textbf{return} \textsc{Simplify}(}\sigma X=\varphi \lbrack \psi
_{0}/X]\text{)}%
\end{array}%
\end{equation*}

\newpage

\appendix

\paragraph{ATerm format}

\begin{equation*}
\begin{array}{ll}
\mathtt{<DataExpr>} & c \\ 
\mathtt{StateTrue} & true \\ 
\mathtt{StateFalse} & false \\ 
\mathtt{StateNot(<StateFrm>)} & \lnot \varphi  \\ 
\mathtt{StateAnd(<StateFrm>,<StateFrm>)} & \varphi \wedge \varphi  \\ 
\mathtt{StateOr(<StateFrm>,<StateFrm>)} & \varphi \vee \varphi  \\ 
\mathtt{StateImp(<StateFrm>,<StateFrm>)} & \varphi \Rightarrow \varphi  \\ 
\mathtt{StateForall(<DataVarId>+,<StateFrm>)} & \forall x{:}D.\varphi  \\ 
\mathtt{StateExists(<DataVarId>+,<StateFrm>)} & \exists x{:}D.\varphi  \\ 
\mathtt{StateMust(<RegFrm>,<StateFrm>)} & \langle \alpha \rangle \varphi  \\ 
\mathtt{StateMay(<RegFrm>,<StateFrm>)} & [\alpha ]\varphi  \\ 
\mathtt{StateYaled} & \nabla  \\ 
\mathtt{StateYaledTimed(<DataExpr>)} & \nabla (t) \\ 
\mathtt{StateDelay} & \Delta  \\ 
\mathtt{StateDelayTimed(<DataExpr>)} & \Delta (t) \\ 
\mathtt{StateVar(<String>,<DataExpr>\ast )} & X(d) \\ 
\mathtt{StateNu(<String>,<DataVarIdInit>\ast ,<StateFrm>)} & \nu X(x{:}%
D:=d).~\varphi  \\ 
\mathtt{StateMu(<String>,<DataVarIdInit>\ast ,<StateFrm>)} & \mu X(x{:}%
D:=d).~\varphi 
\end{array}%
\end{equation*}

\paragraph{Naming conventions}

\begin{equation*}
\begin{array}{lcl}
\mathsf{left}(\varphi \otimes \psi ) & = & \varphi \\ 
\mathsf{right}(\varphi \otimes \psi ) & = & \psi \\ 
\arg (\lnot \varphi ) & = & \varphi \\ 
\arg (\forall d:D.\varphi )=\arg (\exists d:D.\varphi ) & = & \varphi \\ 
\mathsf{var}(\forall d:D.\varphi )=\mathsf{var}(\exists d:D.\varphi ) & = & 
d:D \\ 
\arg (\left\langle \alpha \right\rangle \varphi )=\arg ([\alpha ]\varphi ) & 
= & \varphi \\ 
\mathsf{act}(\left\langle \alpha \right\rangle \varphi )=\mathsf{act}%
([\alpha ]\varphi ) & = & \alpha \\ 
\mathsf{time}(\nabla (t))=\mathsf{time}(\Delta (t)) & = & t \\ 
\mathsf{var}(X(d:D)) & = & d:D \\ 
\mathsf{\arg }(\sigma X(d:D:=e).\varphi ) & = & \varphi \\ 
\mathsf{name}(\sigma X(d:D:=e).\varphi ) & = & X \\ 
\mathsf{var}(\sigma X(d:D:=e).\varphi ) & = & d:D \\ 
\mathsf{val}(\sigma X(d:D:=e).\varphi ) & = & e%
\end{array}%
\end{equation*}%
where $\sigma $ is either $\mu $ or $\nu $, and $\otimes $ is either $\wedge 
$, $\vee $, or $\Rightarrow $.

\end{document}
