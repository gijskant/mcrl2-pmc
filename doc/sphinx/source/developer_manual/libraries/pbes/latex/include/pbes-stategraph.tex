%TCIDATA{Version=5.50.0.2890}
%TCIDATA{LaTeXparent=1,1,pbes-implementation-notes.tex}
                      

\section{Stategraph}

We denote the number of predicate variable instances occurring in a
predicate formula $\varphi $ by $\mathrm{npred}(\varphi )$. We assume that
predicate variable instances in $\varphi $ are assigned a unique natural
number between $1$ and $\mathrm{npred}(\varphi )$.

\begin{definition}
Let $\varphi $ be a predicate formula and let $i$ be between $1$ and $%
\mathrm{npred}(\varphi )$. The functions $\mathrm{pred}(\varphi ,i)$, $%
\mathrm{data}(\varphi ,i)$ and $\mathrm{PVI}(\varphi ,i)$ are such that the
predicate variable instance $\mathrm{PVI}(\varphi ,i)$ is the $i^{th}$
predicate variable instance in $\varphi $, syntactically present as $\mathrm{%
pred}(\varphi ,i)(\mathrm{data}(\varphi ,i))$. Let $\psi $ be a predicate
formula. We write $\varphi \lbrack i\rightarrow \psi ]$ to indicate that the
predicate variable instance at position $i$ is replaced syntactically by $%
\psi $ in $\varphi $.
\end{definition}

\begin{definition}
Let $\varphi $ be a predicate formula. We define the guard of predicate
variable instantiation $\mathrm{PVI}(\varphi ,i)$ for $i\leq \mathrm{npred}%
(\varphi )$ inductively as follows:
\end{definition}

\begin{equation*}
\begin{array}{lll}
guard^{i}(c) & = & false \\ 
guard^{i}(Y) & = & true \\ 
guard^{i}(\forall d:D.\varphi ) & = & guard^{i}(\varphi ) \\ 
guard^{i}(\exists d:D.\varphi ) & = & guard^{i}(\varphi ) \\ 
guard^{i}(\varphi \wedge \psi ) & = & \left\{ 
\begin{array}{lll}
s(\varphi )\wedge guard^{i-\mathrm{npred}(\varphi )}(\psi ) &  & \text{if }i>%
\mathrm{npred}(\varphi ) \\ 
s(\psi )\wedge guard^{i}(\varphi ) &  & \text{if }i\leq \mathrm{npred}%
(\varphi )%
\end{array}%
\right. \\ 
guard^{i}(\varphi \vee \psi ) & = & \left\{ 
\begin{array}{lll}
n(\varphi )\wedge guard^{i}(\psi ) &  & \text{if }i>\mathrm{npred}(\varphi )
\\ 
ns(\psi )\wedge guard^{i}(\varphi ) &  & \text{if }i\leq \mathrm{npred}%
(\varphi )%
\end{array}%
\right.%
\end{array}%
\end{equation*}%
where%
\begin{equation*}
\begin{array}{lll}
s(\varphi ) & = & \left\{ 
\begin{array}{lll}
\varphi &  & \text{if }\mathrm{npred}(\varphi )=0 \\ 
true &  & \text{otherwise}%
\end{array}%
\right. \\ 
ns(\varphi ) & = & \left\{ 
\begin{array}{lll}
\lnot \varphi &  & \text{if }\mathrm{npred}(\varphi )=0 \\ 
true &  & \text{otherwise}%
\end{array}%
\right.%
\end{array}%
\end{equation*}%
We define the function $sig$ recursively as follows:%
\begin{equation*}
\begin{array}{lll}
sig(b) & = & FV(b) \\ 
sig(\varphi \wedge \psi ) & = & sig(\varphi )\cup sig(\psi ) \\ 
sig(\varphi \vee \psi ) & = & sig(\varphi )\cup sig(\psi ) \\ 
sig(X(e)) & = & \emptyset \\ 
sig(\exists d:D.\varphi ) & = & sig(\varphi )\setminus \{d\} \\ 
sig(\forall d:D.\varphi ) & = & sig(\varphi )\setminus \{d\} \\ 
sig(\varphi \Rightarrow \psi ) & = & sig(\varphi )\cup sig(\psi ) \\ 
sig(\lnot \varphi ) & = & sig(\varphi )%
\end{array}%
\end{equation*}

\subsection{The functions source, dest and copy}

Let $X(d:D)=\varphi $ be a PBES equation. Let $\mathrm{source}$ be a
function with the property that 
\begin{equation*}
\mathrm{source}(X,i,j)=\left\{ 
\begin{array}{lll}
e &  & \text{if }guard^{i}(\mathrm{PVI}(\varphi _{X},i))\Rightarrow
d[j]\approx e \\ 
\bot &  & \text{otherwise}%
\end{array}%
\right.
\end{equation*}%
A possible heuristic for obtaining a source function is to look for positive
occurrences of constraints of the form $d[j]\approx e$ in the guards; these
can be used to define the source function. Let $\mathrm{sigma}(X,i)$ be the
substitution defined as%
\begin{equation*}
\mathrm{sigma}(X,i)(v)=%
\begin{array}{lll}
e &  & \text{if }\mathrm{source}(X,i,j)=e\text{ for some }j \\ 
v &  & \text{otherwise}%
\end{array}%
\end{equation*}

We define the function $\mathrm{dest}$ as follows:%
\begin{equation*}
\mathrm{dest}(X,i,j)=\left\{ 
\begin{array}{lll}
c &  & \text{if }rewrite(\mathrm{sigma}(X,i)(\mathrm{PVI}(\varphi ,i))[j]=c
\\ 
\bot &  & \text{otherwise}%
\end{array}%
\right.
\end{equation*}%
with $c$ a constant. We define the function $\mathrm{copy}$ as follows:%
\begin{equation*}
\mathrm{copy}(X,i,j)=\left\{ 
\begin{array}{lll}
k &  & \text{if }\mathrm{PVI}(\varphi ,i)[k]=d[j] \\ 
\bot &  & \text{otherwise}%
\end{array}%
\right.
\end{equation*}%
We define the function $\mathrm{used}$ as follows:%
\begin{equation*}
\mathrm{used}(X,i,j)=d_{X}[j]\in FV(guard^{i}(\mathrm{PVI}(\varphi _{X},i)))
\end{equation*}%
We define the function $\mathrm{changed}$ as follows:%
\begin{equation*}
\mathrm{changed}(X,i,j)=\mathrm{pred}(\varphi _{X},i)=X\wedge d_{X}[j]\neq 
\mathrm{data}(\varphi _{X},i)[j]
\end{equation*}

Let $par(X)$ be the set of parameters of the equation corresponding to $X$.
Let $pos(X,i)$ denote the $i$-th parameter of the equation corresponding to $%
X$.\newpage

\subsection{Control flow parameters}

Control flow parameters are computed in phases. First the function $LCFP$ is
computed, then the function $GCFP$, and finally they are related using $\sim 
$.

\subsubsection{LCFP computation}

There are two versions of the computation of LCFP.

\begin{equation*}
\begin{tabular}{l}
\textsc{ComputeLocalControlflowParametersDefault(}$\mathcal{E}$\textsc{)} \\ 
$\text{\textbf{for }}X\in bnd(\mathcal{E})\text{ \textbf{do}}$ \\ 
$\qquad \text{\textbf{for }}n=1,\cdots ,\left\vert par(X)\right\vert \text{ 
\textbf{do}}$ \\ 
$\qquad \qquad LCFP(X,n):=true$ \\ 
$\text{\textbf{for }}X\in bnd(\mathcal{E})\text{ \textbf{do}}$ \\ 
$\qquad \text{\textbf{for }}i=1,\cdots ,npred(\varphi _{X})\text{ \textbf{do}%
}$ \\ 
$\qquad \qquad \text{\textbf{if }}pred(\varphi _{X},i)=X\text{ \textbf{then}}
$ \\ 
$\qquad \qquad \qquad \text{\textbf{for }}n=1,\cdots ,\left\vert
par(X)\right\vert \text{ \textbf{do}}$ \\ 
$\qquad \qquad \qquad \qquad \text{\textbf{if }}source(X,i,n)=\bot \text{ }%
\wedge dest(X,i,n)=\bot \wedge copy(X,i,n)\neq n\text{ \textbf{then}}$ \\ 
$\qquad \qquad \qquad \qquad \qquad LCFP(X,n):=false$ \\ 
$\text{\textbf{return} }LCFP$%
\end{tabular}%
\end{equation*}

\begin{equation*}
\begin{tabular}{l}
\textsc{ComputeLocalControlflowParametersAlternative(}$\mathcal{E}$\textsc{)}
\\ 
$\text{\textbf{for }}X\in bnd(\mathcal{E})\text{ \textbf{do}}$ \\ 
$\qquad \text{\textbf{for }}n=1,\cdots ,\left\vert par(X)\right\vert \text{ 
\textbf{do}}$ \\ 
$\qquad \qquad LCFP(X,n):=true$ \\ 
$\text{\textbf{for }}X\in bnd(\mathcal{E})\text{ \textbf{do}}$ \\ 
$\qquad \text{\textbf{for }}i=1,\cdots ,npred(\varphi _{X})\text{ \textbf{do}%
}$ \\ 
$\qquad \qquad \text{\textbf{if }}pred(\varphi _{X},i)=X\text{ \textbf{then}}
$ \\ 
$\qquad \qquad \qquad \text{\textbf{for }}n=1,\cdots ,\left\vert
par(X)\right\vert \text{ \textbf{do}}$ \\ 
$\qquad \qquad \qquad \qquad \text{\textbf{if }}(source(X,i,n)=\bot \text{ }%
\wedge dest(X,i,n)=\bot \wedge copy(X,i,n)\neq \bot )$ \\ 
$\qquad \qquad \qquad \qquad $\textbf{or} $(source(X,i,n)\neq \bot \text{ }%
\wedge dest(X,i,n)\neq \bot \wedge copy(X,i,n)=\bot )$ \\ 
$\qquad \qquad \qquad \qquad \text{\textbf{then}}$ \\ 
$\qquad \qquad \qquad \qquad \qquad LCFP(X,n):=false$ \\ 
$\text{\textbf{return} }LCFP$%
\end{tabular}%
\end{equation*}

\subsubsection{GCFP computation}

\begin{equation*}
\begin{tabular}{l}
\textsc{ComputeGlobalControlflowParameters(}$\mathcal{E},LCFP$\textsc{)} \\ 
$\text{\textbf{for }}X\in bnd(\mathcal{E})\text{ \textbf{do}}$ \\ 
$\qquad \text{\textbf{for }}n=1,\cdots ,\left\vert par(X)\right\vert \text{ 
\textbf{do}}$ \\ 
$\qquad \qquad GCFP(X,n):=LCFP(X,n)$ \\ 
$\text{\textbf{for }}X\in bnd(\mathcal{E})\text{ \textbf{do}}$ \\ 
$\qquad \text{\textbf{for }}i=1,\cdots ,npred(\varphi _{X})\text{ \textbf{do}%
}$ \\ 
$\qquad \qquad \text{\textbf{let }}Y=pred(\varphi _{X},i)$ \\ 
$\qquad \qquad \text{\textbf{if }}Y\neq X\text{ \textbf{then}}$ \\ 
$\qquad \qquad \qquad \text{\textbf{for }}n=1,\cdots ,\left\vert
par(X)\right\vert \text{ \textbf{do}}$ \\ 
$\qquad \qquad \qquad \qquad \text{\textbf{if }}dest(X,i,n)=\bot \wedge
\forall m:copy(X,i,m)\neq n\text{ \textbf{then}}$ \\ 
$\qquad \qquad \qquad \qquad \qquad GCFP(X,n):=false$ \\ 
$\text{\textbf{return} }GCFP$%
\end{tabular}%
\end{equation*}%
\newline

\subsubsection{Related control flow parameters}

GCFP parameters can be related using the relation $\sim $. There are two
versions of the computation of $\sim $.%
\begin{equation*}
\begin{tabular}{l}
\textsc{ComputeRelatedGlobalControlflowParametersDefault(}$\mathcal{E},GCFP$%
\textsc{)} \\ 
$\text{\textbf{for }}X\in bnd(\mathcal{E})\text{ \textbf{do}}$ \\ 
$\qquad \text{\textbf{for }}i=1,\cdots ,npred(\varphi _{X})\text{ \textbf{do}%
}$ \\ 
$\qquad \qquad \text{\textbf{let }}Y=pred(\varphi _{X},i)$ \\ 
$\qquad \qquad \text{\textbf{for }}n=1,\cdots ,\left\vert par(X)\right\vert 
\text{ \textbf{do}}$ \\ 
$\qquad \qquad \qquad \text{\textbf{if }}copy(X,i,n)=m\neq \bot \text{ 
\textbf{then}}$ \\ 
$\qquad \qquad \qquad \qquad \text{\textbf{if }}GCFP(X,n)\wedge GCFP(Y,m)%
\text{ \textbf{then }}(X,n)\sim (Y,m)$%
\end{tabular}%
\end{equation*}%
\begin{equation*}
\begin{tabular}{l}
\textsc{ComputeRelatedGlobalControlflowParametersAlternative(}$\mathcal{E}%
,GCFP$\textsc{)} \\ 
$\text{\textbf{for }}X\in bnd(\mathcal{E})\text{ \textbf{do}}$ \\ 
$\qquad \text{\textbf{for }}i=1,\cdots ,npred(\varphi _{X})\text{ \textbf{do}%
}$ \\ 
$\qquad \qquad \text{\textbf{let }}Y=pred(\varphi _{X},i)$ \\ 
$\qquad \qquad \text{\textbf{for }}n=1,\cdots ,\left\vert par(X)\right\vert 
\text{ \textbf{do}}$ \\ 
$\qquad \qquad \qquad \text{\textbf{if }}copy(X,i,n)=m\neq \bot \text{ 
\textbf{then}}$ \\ 
$\qquad \qquad \qquad \qquad \text{\textbf{if }}GCFP(X,n)\wedge GCFP(Y,m)%
\text{ }\wedge dest(X,i,m)=\bot \text{ \textbf{then }}(X,n)\sim (Y,m)$%
\end{tabular}%
\end{equation*}

\subsubsection{Control flow graphs}

The relation $\sim $ defines a graph on the set of vertices $V=\{(X,i)\mid
X\in bnd(\mathcal{E})\wedge 1\leq i\leq \left\vert d_{X}\right\vert \}$.
This graph is called the global control flow graph. The connected components
in this graph are the local control flow graphs. A local control flow graph
is called \emph{invalid} if it contains two vertices $(X,i)$ and $(X,j)$
with $i\neq j$. \newpage

\subsection{Global control flow graph}

\subsubsection{Notations}

In the following algorithm we use a projection that removes all parameters
of a predicate variable that do not correspond to a control flow parameter.
This projection applied to the predicate variable $X_{k}(e_{1},\cdots
,e_{m}) $ is denoted as $\overline{X_{k}(e_{1},\cdots ,e_{m})}$. The mapping 
$R$ is a rewriter that takes a substitution to data variables as second
argument.

\begin{equation*}
\begin{tabular}{l}
\textsc{ComputeGlobalControlflowGraph(}$\mathcal{E}$, $X_{init}(e_{init})$%
\textsc{)} \\ 
$u_{0}:=X_{init}(\widehat{e_{init}})$ \\ 
$V:=\{u_{0}\}$ \\ 
$E:=\emptyset $ \\ 
$todo:=\{u_{0}\}$ \\ 
$sig(u_{0}):=\emptyset $ \\ 
\textbf{while} $todo\neq \emptyset $ \textbf{do} \\ 
$\qquad $\textbf{choose} $u=X(d=e)\in todo$ \\ 
$\qquad todo:=todo\setminus \{u\}$ \\ 
$\qquad sig(u):=sig(R(\varphi _{X},[d:=e])$ \\ 
$\qquad \text{\textbf{for }}i=1\cdots \mathrm{npred}(\varphi )\text{ \textbf{%
do}}$ \\ 
$\qquad \qquad g:=R(guard^{i}(\varphi _{X}),[d_{X}:=e])$ \\ 
$\qquad \qquad \text{\textbf{if }}g\neq false$ \\ 
$\qquad \qquad \qquad Y(f):=\mathrm{PVI}(\varphi _{X},i)[d:=e]$ \\ 
$\qquad \qquad \qquad v:=Y(\widehat{f})$ \\ 
$\qquad \qquad \qquad \text{\textbf{if }}v\notin V$ \\ 
$\qquad \qquad \qquad \qquad V:=V\cup \{v\}$ \\ 
$\qquad \qquad \qquad \qquad todo:=todo\cup \{v\}$ \\ 
$\qquad \qquad \qquad E:=E\cup \{(u,v)\}$ \\ 
$\qquad \qquad \qquad label(u,v):=i$ \\ 
$\text{\textbf{return} }(V,E)$%
\end{tabular}%
\end{equation*}

\newpage

\subsection{\protect\bigskip Control flow marking}

The following algorithm computes the function $marking$ that denotes which
parameters are marked in a vertex of the control flow graph $(V,E)$.

\begin{equation*}
\begin{tabular}{l}
\textsc{ComputeControlFlowMarking(}$\mathcal{E}$, $X_{init}(e_{init})$, $V$, 
$E$\textsc{)} \\ 
\textbf{for} $u=X(e)\in V$ \textbf{do }$marking(u):=sig(u)\cap par(X)$ \\ 
$todo:=V$ \\ 
\textbf{while} $todo\neq \emptyset $ \textbf{do} \\ 
$\qquad $\textbf{choose} $v=X(e)\in todo$ \\ 
$\qquad todo:=todo\setminus \{v\}$ \\ 
$\qquad \text{\textbf{for }}(u,v)\in E\text{ \textbf{do}}$ \\ 
$\qquad \qquad $\textbf{let} $X(f)=label(u,v)$ \\ 
$\qquad \qquad \text{\textbf{for }}d_{X}[j]\in marking(v)$ \\ 
$\qquad \qquad \qquad M:=(FV(f[j])\setminus marking(u))\cap par(X)$ \\ 
$\qquad \qquad \qquad \text{\textbf{if }}M\neq \emptyset $ \\ 
$\qquad \qquad \qquad \qquad marking(u):=marking(u)\cup M$ \\ 
$\qquad \qquad \qquad \qquad todo:=todo\cup \{u\}$%
\end{tabular}%
\end{equation*}

\begin{equation*}
\begin{tabular}{l}
\textsc{ComputeControlFlowMarkingLocal(}$\mathcal{E}$, $X_{init}(e_{init})$, 
$(V_{1},E_{1},B_{1}),\cdots ,(V_{K},E_{K},B_{K})$\textsc{)} \\ 
\textbf{for} $k=1\cdots K$ \textbf{do} \\ 
$\qquad $\textbf{for} $u=X(e)\in V_{k}$ \textbf{do} \\ 
$\qquad \qquad marking(u):=sig(u)\cap \{d\mid (X,d)\in B_{k}\}$ \\ 
$stable:=false$ \\ 
\textbf{while} $\lnot stable$ \textbf{do} \\ 
$\qquad stable:=true$ \\ 
$\qquad $\textbf{for} $k=1\cdots K$ \textbf{do} \\ 
$\qquad \qquad todo:=V_{k}$ \\ 
$\qquad \qquad $\textbf{while} $todo\neq \emptyset $ \textbf{do} \\ 
$\qquad \qquad \qquad $\textbf{choose} $v=X(e)\in todo$ \\ 
$\qquad \qquad \qquad todo:=todo\setminus \{v\}$ \\ 
$\qquad \qquad \qquad \text{\textbf{for }}(u,v,i)\in E_{k}\text{ \textbf{do}}
$ \\ 
$\qquad \qquad \qquad \qquad $\textbf{let} $u=Y(f)$ \\ 
$\qquad \qquad \qquad \qquad \text{\textbf{for }}d_{X}[j]\in marking(v)\text{
\textbf{do}}$ \\ 
$\qquad \qquad \qquad \qquad \qquad M:=(FV(data(\varphi _{Y},i)[j]\setminus
marking(u))\cap \{d\mid (Y,d)\in B_{k}\}$ \\ 
$\qquad \qquad \qquad \qquad \qquad \text{\textbf{if }}M\neq \emptyset $ \\ 
$\qquad \qquad \qquad \qquad \qquad \qquad marking(u):=marking(u)\cup M$ \\ 
$\qquad \qquad \qquad \qquad \qquad \qquad todo:=todo\cup \{u\}$ \\ 
$\qquad \qquad \qquad \qquad \qquad \qquad stable:=false$ \\ 
$\qquad \qquad $\textbf{for} $u=X(e)\in V_{k}$ \textbf{do} \\ 
$\qquad \qquad \qquad \text{\textbf{for }}i=1\cdots \mathrm{npred}(\varphi )%
\text{ \textbf{do}}$ \\ 
$\qquad \qquad \qquad \qquad $\textbf{let} $Y(f)=\mathrm{PVI}(\varphi
_{X},i) $ \\ 
$\qquad \qquad \qquad \qquad \text{\textbf{for }}l$ such that $%
(Y,d_{Y}[l])\notin B_{k}\text{ \textbf{do}}\wedge d_{X}[j]\notin marking(u)$
\\ 
$\qquad \qquad \qquad \qquad \qquad M:=(\{d\mid (X,d)\in B_{k}\}\cap
FV(f[l]))\setminus marking(u)$ \\ 
$\qquad \qquad \qquad \qquad \qquad \text{\textbf{if }}M\neq \emptyset
\wedge \exists k^{\prime }$ such that $(Y,d_{Y}[l])\in B_{k^{\prime }}$ \\ 
$\qquad \qquad \qquad \qquad \qquad \qquad \wedge \exists w=Y(g)\in
V_{k^{\prime }}$ \textbf{such that} $d_{Y}[l]\in marking(w)$ \\ 
$\qquad \qquad \qquad \qquad \qquad \qquad \qquad marking(u):=marking(u)\cup
M$ \\ 
$\qquad \qquad \qquad \qquad \qquad \qquad \qquad stable:=false$ \\ 
$\qquad \qquad \qquad \qquad \qquad \qquad \qquad $\textbf{break}%
\end{tabular}%
\end{equation*}

\newpage

\subsection{Reset variables}

Let $\mathrm{PVI}(\varphi _{X},i)=Y(e)$ and let $V$ be the global control
flow graph. Then we define

\begin{equation*}
\begin{tabular}{l}
\textsc{ResetVariableGlobal(}$Y(e)$, $i,$ $V$\textsc{)} \\ 
$\varphi :=true$ \\ 
\textbf{for} $u=Y(f)\in V$ \textbf{do} \\ 
$\qquad c:=true$ \\ 
$\qquad k:=1$ \\ 
$\qquad $\textbf{for} $j=1\cdots \left\vert par(Y)\right\vert $ \textbf{do}
\\ 
$\qquad \qquad r:=[]$ \\ 
$\qquad \qquad $\textbf{if} $CFP(Y,j)$ \textbf{then} \\ 
$\qquad \qquad \qquad $\textbf{if} $dest(X,i,j)=\bot $ \\ 
$\qquad \qquad \qquad \qquad c:=c\wedge (e[j]=f[k])$ \\ 
$\qquad \qquad \qquad r:=r\vartriangleleft f[k]$ \\ 
$\qquad \qquad $\textbf{else if} $e[j]\in marking(u)$ \textbf{then} \\ 
$\qquad \qquad \qquad r:=r\vartriangleleft e[j]$ \\ 
$\qquad \qquad $\textbf{else} \\ 
$\qquad \qquad \qquad r:=r\vartriangleleft default\_value(e[j])$ \\ 
$\qquad \qquad k:=k+1$ \\ 
$\qquad \varphi :=\varphi \wedge (c\Rightarrow Y(r))$ \\ 
\textbf{return} $\varphi $%
\end{tabular}%
\end{equation*}%
\begin{equation*}
\begin{tabular}{l}
\textsc{ResetVariableLocal(}$Y(e)$, $i,$ $V_{1},\cdots ,V_{K},B_{1},\cdots
,B_{K}$\textsc{)} \\ 
$\varphi :=true$ \\ 
$I:=\{j\mid \mathrm{dest}(X,i,j)=\bot \}$ \\ 
\textbf{for all} $v^{\prime }$ \textbf{such that }$\{i_{1},\cdots
,i_{\left\vert I\right\vert }\}=I\wedge v^{\prime }[m]\in \mathrm{values}%
(d_{Y}[i_{m}])$ \textbf{do} \\ 
$\qquad c:=true$ \\ 
$\qquad k:=1$ \\ 
$\qquad v:=[]$ \\ 
$\qquad $\textbf{for} $j=1\cdots \left\vert par(Y)\right\vert $ \textbf{do}
\\ 
$\qquad \qquad $\textbf{if} $\mathrm{CFP}(Y,j)$ \\ 
$\qquad \qquad \qquad $\textbf{if} $\mathrm{dest}(X,i,j)=\bot $ \\ 
$\qquad \qquad \qquad \qquad v:=v\vartriangleleft v^{\prime }[k]$ \\ 
$\qquad \qquad \qquad \qquad k:=k+1$ \\ 
$\qquad \qquad \qquad $\textbf{else} \\ 
$\qquad \qquad \qquad \qquad v:=v\vartriangleleft \mathrm{dest}(X,i,j)$ \\ 
$\qquad k:=1$ \\ 
$\qquad $\textbf{for} $j=1\cdots \left\vert par(Y)\right\vert $ \textbf{do}
\\ 
$\qquad \qquad r:=[]$ \\ 
$\qquad \qquad $\textbf{if} $\mathrm{CFP}(Y,j)$ \textbf{then} \\ 
$\qquad \qquad \qquad $\textbf{if} $\mathrm{dest}(X,i,j)=\bot $ \\ 
$\qquad \qquad \qquad \qquad c:=c\wedge (d_{Y}[j]=v[k])$ \\ 
$\qquad \qquad \qquad r:=r\vartriangleleft v[k]$ \\ 
$\qquad \qquad \qquad k:=k+1$ \\ 
$\qquad \qquad $\textbf{else} \\ 
$\qquad \qquad \qquad relevant:=true$ \\ 
$\qquad \qquad \qquad $\textbf{for} $k$ \textbf{such} \textbf{that} $%
(Y,d_{Y}[j])\in B_{k}$ \\ 
$\qquad \qquad \qquad \qquad $\textbf{let} $m$ \textbf{be such that} $%
d_{Y}[m]$ \textbf{corresponds to} $G_{k}$ \\ 
$\qquad \qquad \qquad \qquad relevant:=relevant\wedge d_{Y}[j]\in
marking(Y(v[m]))$ \\ 
$\qquad \qquad \qquad $\textbf{if} $relevant$ \textbf{then} $%
r:=r\vartriangleleft e[j]$ \textbf{else} $r:=r\vartriangleleft
default\_value(e[j])$ \\ 
$\qquad \varphi :=\varphi \wedge (c\Rightarrow Y(r))$ \\ 
\textbf{return} $\varphi $%
\end{tabular}%
\end{equation*}

\newpage

\subsection{Belongs relation}

The following algorithm computes the belongs relation $B_{k}$ that
corresponds with the local control flow graph $(V_{k}$, $\longrightarrow
_{k})$.

\begin{equation*}
\begin{tabular}{l}
\textsc{ComputeBelongs(} $\longrightarrow _{must}$, $X_{0}$, $p_{0}$, $V_{k}$%
, $\longrightarrow _{k}$\textsc{)} \\ 
$B_{k}:=\emptyset $ \\ 
$todo:=\{(X_{0},p_{0})\}$ \\ 
$visited:=\emptyset $ \\ 
\textbf{while} $todo\neq \emptyset $ \textbf{do} \\ 
$\qquad $\textbf{choose} $(X,p)\in todo$ \\ 
$\qquad todo:=todo\setminus \{(X,p)\}$ \\ 
$\qquad visited:=visited\cup \{(X,p)\}$ \\ 
$\qquad \text{\textbf{if }}(X,p)\nrightarrow _{must}$\textbf{\ then continue}
\\ 
$\qquad belongs:=\{j\mid d[j]$ is data parameter of $X\}$ \\ 
$\qquad $\textbf{for} $(X,p)\longrightarrow _{must}(Y,q)\text{ \textbf{do}}$
\\ 
$\qquad \qquad \text{\textbf{for }}i=1\cdots \mathrm{npred}(\varphi )\text{ 
\textbf{do}}$ \\ 
$\qquad \qquad \qquad \text{\textbf{if }}\mathrm{pred}(\varphi _{X},i)\neq Y$%
\textbf{\ then continue} \\ 
$\qquad \qquad \qquad \text{\textbf{if }}(Y,q)\notin visited$ \\ 
$\qquad \qquad \qquad \qquad todo:=todo\cup \{(Y,q)\}$ \\ 
$\qquad \qquad \qquad \text{\textbf{for }}j\in belongs$ \textbf{do} \\ 
$\qquad \qquad \qquad \qquad $\textbf{if }$(\mathrm{used}(X,i,j)\vee \mathrm{%
changed}(X,i,j))\wedge (X,\_,\_)\overset{i}{\nrightarrow }_{k}$ \\ 
$\qquad \qquad \qquad \qquad \qquad belongs:=belongs\setminus \{j\}$ \\ 
$\qquad B_{k}:=B_{k}\cup \{(X,d_{X}[j])\mid j\in belongs\}$ \\ 
\textbf{return} $B_{k}$%
\end{tabular}%
\end{equation*}%
\newpage
