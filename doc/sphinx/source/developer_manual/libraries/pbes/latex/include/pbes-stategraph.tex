%TCIDATA{Version=5.50.0.2890}
%TCIDATA{LaTeXparent=1,1,pbes-implementation-notes.tex}
                      

\section{Stategraph}

We denote the number of predicate variable instances occurring in a
predicate formula $\varphi $ by $\mathrm{npred}(\varphi )$. We assume that
predicate variable instances in $\varphi $ are assigned a unique natural
number between $1$ and $\mathrm{npred}(\varphi )$.

\begin{definition}
Let $\varphi $ be a predicate formula and let $i$ be between $1$ and $%
\mathrm{npred}(\varphi )$. The functions $\mathrm{pred}(\varphi ,i)$, $%
\mathrm{data}(\varphi ,i)$ and $\mathrm{PVI}(\varphi ,i)$ are such that the
predicate variable instance $\mathrm{PVI}(\varphi ,i)$ is the $i^{th}$
predicate variable instance in $\varphi $, syntactically present as $\mathrm{%
pred}(\varphi ,i)(\mathrm{data}(\varphi ,i))$. Let $\psi $ be a predicate
formula. We write $\varphi \lbrack i\rightarrow \psi ]$ to indicate that the
predicate variable instance at position $i$ is replaced syntactically by $%
\psi $ in $\varphi $.
\end{definition}

\begin{definition}
Let $\varphi $ be a predicate formula. We define the guard of predicate
variable instantiation $\mathrm{PVI}(\varphi ,i)$ for $i\leq \mathrm{npred}%
(\varphi )$ inductively as follows:
\end{definition}

\begin{equation*}
\begin{array}{lll}
guard^{i}(c) & = & false \\ 
guard^{i}(Y) & = & true \\ 
guard^{i}(\forall d:D.\varphi ) & = & guard^{i}(\varphi ) \\ 
guard^{i}(\exists d:D.\varphi ) & = & guard^{i}(\varphi ) \\ 
guard^{i}(\varphi \wedge \psi ) & = & \left\{ 
\begin{array}{lll}
s(\varphi )\wedge guard^{i-\mathrm{npred}(\varphi )}(\psi ) &  & \text{if }i>%
\mathrm{npred}(\varphi ) \\ 
s(\psi )\wedge guard^{i}(\varphi ) &  & \text{if }i\leq \mathrm{npred}%
(\varphi )%
\end{array}%
\right. \\ 
guard^{i}(\varphi \vee \psi ) & = & \left\{ 
\begin{array}{lll}
n(\varphi )\wedge guard^{i}(\psi ) &  & \text{if }i>\mathrm{npred}(\varphi )
\\ 
ns(\psi )\wedge guard^{i}(\varphi ) &  & \text{if }i\leq \mathrm{npred}%
(\varphi )%
\end{array}%
\right.%
\end{array}%
\end{equation*}%
where%
\begin{equation*}
\begin{array}{lll}
s(\varphi ) & = & \left\{ 
\begin{array}{lll}
\varphi &  & \text{if }\mathrm{npred}(\varphi )=0 \\ 
true &  & \text{otherwise}%
\end{array}%
\right. \\ 
ns(\varphi ) & = & \left\{ 
\begin{array}{lll}
\lnot \varphi &  & \text{if }\mathrm{npred}(\varphi )=0 \\ 
true &  & \text{otherwise}%
\end{array}%
\right.%
\end{array}%
\end{equation*}%
We define the function $sig$ recursively as follows:%
\begin{equation*}
\begin{array}{lll}
sig(b) & = & FV(b) \\ 
sig(\varphi \wedge \psi ) & = & sig(\varphi )\cup sig(\psi ) \\ 
sig(\varphi \vee \psi ) & = & sig(\varphi )\cup sig(\psi ) \\ 
sig(X(e)) & = & \emptyset \\ 
sig(\exists d:D.\varphi ) & = & sig(\varphi )\setminus \{d\} \\ 
sig(\forall d:D.\varphi ) & = & sig(\varphi )\setminus \{d\} \\ 
sig(\varphi \Rightarrow \psi ) & = & sig(\varphi )\cup sig(\psi ) \\ 
sig(\lnot \varphi ) & = & sig(\varphi )%
\end{array}%
\end{equation*}

\subsection{The functions source, dest and copy}

Let $X(d:D)=\varphi $ be a PBES equation. Let $\mathrm{source}$ be a
function with the property that 
\begin{equation*}
\mathrm{source}(X,i,j)=\left\{ 
\begin{array}{lll}
e &  & \text{if }guard^{i}(\mathrm{PVI}(\varphi _{X},i))\Rightarrow
d[j]\approx e \\ 
\bot &  & \text{otherwise}%
\end{array}%
\right.
\end{equation*}%
A possible heuristic for obtaining a source function is to look for positive
occurrences of constraints of the form $d[j]\approx e$ in the guards; these
can be used to define the source function. Let $\mathrm{sigma}(X,i)$ be the
substitution defined as%
\begin{equation*}
\mathrm{sigma}(X,i)(v)=%
\begin{array}{lll}
e &  & \text{if }\mathrm{source}(X,i,j)=e\text{ for some }j \\ 
v &  & \text{otherwise}%
\end{array}%
\end{equation*}

We define the function $\mathrm{dest}$ as follows:%
\begin{equation*}
\mathrm{dest}(X,i,j)=\left\{ 
\begin{array}{lll}
c &  & \text{if }rewrite(\mathrm{sigma}(X,i)(\mathrm{PVI}(\varphi ,i))[j]=c
\\ 
\bot &  & \text{otherwise}%
\end{array}%
\right.
\end{equation*}%
with $c$ a constant. We define the function $\mathrm{copy}$ as follows:%
\begin{equation*}
\mathrm{copy}(X,i,j)=\left\{ 
\begin{array}{lll}
k &  & \text{if }\mathrm{PVI}(\varphi ,i)[k]=d[j] \\ 
\bot &  & \text{otherwise}%
\end{array}%
\right.
\end{equation*}%
We define the function $\mathrm{used}$ as follows:%
\begin{equation*}
\mathrm{used}(X,i,j)=d_{X}[j]\in FV(guard^{i}(\mathrm{PVI}(\varphi _{X},i)))
\end{equation*}%
We define the function $\mathrm{changed}$ as follows:%
\begin{equation*}
\mathrm{changed}(X,i,j)=\mathrm{pred}(\varphi _{X},i)=X\wedge d_{X}[j]\neq 
\mathrm{data}(\varphi _{X},i)[j]
\end{equation*}

Let $par(X)$ be the set of parameters of the equation corresponding to $X$.
Let $pos(X,i)$ denote the $i$-th parameter of the equation corresponding to $%
X$.\newpage

\subsection{Control flow parameters}

Control flow parameters are computed in phases. First the function $LCFP$ is
computed, then the function $GCFP$, and finally they are related using $\sim 
$.

\subsubsection{LCFP computation}

There are two versions of the computation of LCFP.

\begin{equation*}
\begin{tabular}{l}
\textsc{ComputeLocalControlflowParametersDefault(}$\mathcal{E}$\textsc{)} \\ 
$\text{\textbf{for }}X\in bnd(\mathcal{E})\text{ \textbf{do}}$ \\ 
$\qquad \text{\textbf{for }}n=1,\cdots ,\left\vert par(X)\right\vert \text{ 
\textbf{do}}$ \\ 
$\qquad \qquad LCFP(X,n):=true$ \\ 
$\text{\textbf{for }}X\in bnd(\mathcal{E})\text{ \textbf{do}}$ \\ 
$\qquad \text{\textbf{for }}i=1,\cdots ,npred(\varphi _{X})\text{ \textbf{do}%
}$ \\ 
$\qquad \qquad \text{\textbf{if }}pred(\varphi _{X},i)=X\text{ \textbf{then}}
$ \\ 
$\qquad \qquad \qquad \text{\textbf{for }}n=1,\cdots ,\left\vert
par(X)\right\vert \text{ \textbf{do}}$ \\ 
$\qquad \qquad \qquad \qquad \text{\textbf{if }}source(X,i,n)=\bot \text{ }%
\wedge dest(X,i,n)=\bot \wedge copy(X,i,n)\neq n\text{ \textbf{then}}$ \\ 
$\qquad \qquad \qquad \qquad \qquad LCFP(X,n):=false$ \\ 
$\text{\textbf{return} }LCFP$%
\end{tabular}%
\end{equation*}

\begin{equation*}
\begin{tabular}{l}
\textsc{ComputeLocalControlflowParametersAlternative(}$\mathcal{E}$\textsc{)}
\\ 
$\text{\textbf{for }}X\in bnd(\mathcal{E})\text{ \textbf{do}}$ \\ 
$\qquad \text{\textbf{for }}n=1,\cdots ,\left\vert par(X)\right\vert \text{ 
\textbf{do}}$ \\ 
$\qquad \qquad LCFP(X,n):=true$ \\ 
$\text{\textbf{for }}X\in bnd(\mathcal{E})\text{ \textbf{do}}$ \\ 
$\qquad \text{\textbf{for }}i=1,\cdots ,npred(\varphi _{X})\text{ \textbf{do}%
}$ \\ 
$\qquad \qquad \text{\textbf{if }}pred(\varphi _{X},i)=X\text{ \textbf{then}}
$ \\ 
$\qquad \qquad \qquad \text{\textbf{for }}n=1,\cdots ,\left\vert
par(X)\right\vert \text{ \textbf{do}}$ \\ 
$\qquad \qquad \qquad \qquad \text{\textbf{if }}(source(X,i,n)=\bot \text{ }%
\wedge dest(X,i,n)=\bot \wedge copy(X,i,n)\neq \bot )$ \\ 
$\qquad \qquad \qquad \qquad $\textbf{or} $(source(X,i,n)\neq \bot \text{ }%
\wedge dest(X,i,n)\neq \bot \wedge copy(X,i,n)=\bot )$ \\ 
$\qquad \qquad \qquad \qquad \text{\textbf{then}}$ \\ 
$\qquad \qquad \qquad \qquad \qquad LCFP(X,n):=false$ \\ 
$\text{\textbf{return} }LCFP$%
\end{tabular}%
\end{equation*}

\subsubsection{GCFP computation}

\begin{equation*}
\begin{tabular}{l}
\textsc{ComputeGlobalControlflowParameters(}$\mathcal{E},LCFP$\textsc{)} \\ 
$\text{\textbf{for }}X\in bnd(\mathcal{E})\text{ \textbf{do}}$ \\ 
$\qquad \text{\textbf{for }}n=1,\cdots ,\left\vert par(X)\right\vert \text{ 
\textbf{do}}$ \\ 
$\qquad \qquad GCFP(X,n):=LCFP(X,n)$ \\ 
$\text{\textbf{for }}X\in bnd(\mathcal{E})\text{ \textbf{do}}$ \\ 
$\qquad \text{\textbf{for }}i=1,\cdots ,npred(\varphi _{X})\text{ \textbf{do}%
}$ \\ 
$\qquad \qquad \text{\textbf{let }}Y=pred(\varphi _{X},i)$ \\ 
$\qquad \qquad \text{\textbf{if }}Y\neq X\text{ \textbf{then}}$ \\ 
$\qquad \qquad \qquad \text{\textbf{for }}n=1,\cdots ,\left\vert
par(X)\right\vert \text{ \textbf{do}}$ \\ 
$\qquad \qquad \qquad \qquad \text{\textbf{if }}dest(X,i,n)=\bot \wedge
\forall m:copy(X,i,m)\neq n\text{ \textbf{then}}$ \\ 
$\qquad \qquad \qquad \qquad \qquad GCFP(X,n):=false$ \\ 
$\text{\textbf{return} }GCFP$%
\end{tabular}%
\end{equation*}%
\newline

\subsubsection{Related control flow parameters}

GCFP parameters can be related using the relation $\sim $. There are two
versions of the computation of $\sim $.%
\begin{equation*}
\begin{tabular}{l}
\textsc{ComputeRelatedGlobalControlflowParametersDefault(}$\mathcal{E},GCFP$%
\textsc{)} \\ 
$\text{\textbf{for }}X\in bnd(\mathcal{E})\text{ \textbf{do}}$ \\ 
$\qquad \text{\textbf{for }}i=1,\cdots ,npred(\varphi _{X})\text{ \textbf{do}%
}$ \\ 
$\qquad \qquad \text{\textbf{let }}Y=pred(\varphi _{X},i)$ \\ 
$\qquad \qquad \text{\textbf{for }}n=1,\cdots ,\left\vert par(X)\right\vert 
\text{ \textbf{do}}$ \\ 
$\qquad \qquad \qquad \text{\textbf{if }}copy(X,i,n)=m\neq \bot \text{ 
\textbf{then}}$ \\ 
$\qquad \qquad \qquad \qquad \text{\textbf{if }}GCFP(X,n)\wedge GCFP(Y,m)%
\text{ \textbf{then }}(X,n)\sim (Y,m)$%
\end{tabular}%
\end{equation*}%
\begin{equation*}
\begin{tabular}{l}
\textsc{ComputeRelatedGlobalControlflowParametersAlternative(}$\mathcal{E}%
,GCFP$\textsc{)} \\ 
$\text{\textbf{for }}X\in bnd(\mathcal{E})\text{ \textbf{do}}$ \\ 
$\qquad \text{\textbf{for }}i=1,\cdots ,npred(\varphi _{X})\text{ \textbf{do}%
}$ \\ 
$\qquad \qquad \text{\textbf{let }}Y=pred(\varphi _{X},i)$ \\ 
$\qquad \qquad \text{\textbf{for }}n=1,\cdots ,\left\vert par(X)\right\vert 
\text{ \textbf{do}}$ \\ 
$\qquad \qquad \qquad \text{\textbf{if }}copy(X,i,n)=m\neq \bot \text{ 
\textbf{then}}$ \\ 
$\qquad \qquad \qquad \qquad \text{\textbf{if }}GCFP(X,n)\wedge GCFP(Y,m)%
\text{ }\wedge dest(X,i,m)=\bot \text{ \textbf{then }}(X,n)\sim (Y,m)$%
\end{tabular}%
\end{equation*}

\subsubsection{Control flow graphs}

The relation $\sim $ defines a graph on the set of vertices $V=\{(X,i)\mid
X\in bnd(\mathcal{E})\wedge 1\leq i\leq \left\vert d_{X}\right\vert \}$.
This graph is called the global control flow graph. The connected components
in this graph are the local control flow graphs. A local control flow graph
is called \emph{invalid} if it contains two vertices $(X,i)$ and $(X,j)$
with $i\neq j$. \newpage

\subsection{Global control flow graph}

\subsubsection{Notations}

In the following algorithm we use a projection that removes all parameters
of a predicate variable that do not correspond to a control flow parameter.
This projection applied to the predicate variable $X_{k}(e_{1},\cdots
,e_{m}) $ is denoted as $\overline{X_{k}(e_{1},\cdots ,e_{m})}$. The mapping 
$R$ is a rewriter that takes a substitution to data variables as second
argument.

\begin{equation*}
\begin{tabular}{l}
\textsc{ComputeGlobalControlflowGraph(}$\mathcal{E}$, $X_{init}(e_{init})$%
\textsc{)} \\ 
$u_{0}:=X_{init}(\widehat{e_{init}})$ \\ 
$V:=\{u_{0}\}$ \\ 
$E:=\emptyset $ \\ 
$todo:=\{u_{0}\}$ \\ 
$sig(u_{0}):=\emptyset $ \\ 
\textbf{while} $todo\neq \emptyset $ \textbf{do} \\ 
$\qquad $\textbf{choose} $u=X(d=e)\in todo$ \\ 
$\qquad todo:=todo\setminus \{u\}$ \\ 
$\qquad sig(u):=sig(R(\varphi _{X},[d:=e])$ \\ 
$\qquad \text{\textbf{for }}i=1\cdots \mathrm{npred}(\varphi )\text{ \textbf{%
do}}$ \\ 
$\qquad \qquad g:=R(guard^{i}(\varphi _{X}),[d_{X}:=e])$ \\ 
$\qquad \qquad \text{\textbf{if }}g\neq false$ \\ 
$\qquad \qquad \qquad Y(f):=\mathrm{PVI}(\varphi _{X},i)[d:=e]$ \\ 
$\qquad \qquad \qquad v:=Y(\widehat{f})$ \\ 
$\qquad \qquad \qquad \text{\textbf{if }}v\notin V$ \\ 
$\qquad \qquad \qquad \qquad V:=V\cup \{v\}$ \\ 
$\qquad \qquad \qquad \qquad todo:=todo\cup \{v\}$ \\ 
$\qquad \qquad \qquad E:=E\cup \{(u,v)\}$ \\ 
$\qquad \qquad \qquad label(u,v):=i$ \\ 
$\text{\textbf{return} }(V,E)$%
\end{tabular}%
\end{equation*}

\newpage

\subsection{\protect\bigskip Control flow marking}

The following algorithm computes the function $marking$ that denotes which
parameters are marked in a vertex of the control flow graph $(V,E)$.

\begin{equation*}
\begin{tabular}{l}
\textsc{ComputeControlFlowMarking(}$\mathcal{E}$, $X_{init}(e_{init})$, $V$, 
$E$\textsc{)} \\ 
\textbf{for} $u=X(e)\in V$ \textbf{do }$marking(u):=sig(u)\cap par(X)$ \\ 
$todo:=V$ \\ 
\textbf{while} $todo\neq \emptyset $ \textbf{do} \\ 
$\qquad $\textbf{choose} $v=X(e)\in todo$ \\ 
$\qquad todo:=todo\setminus \{v\}$ \\ 
$\qquad \text{\textbf{for }}(u,v)\in E\text{ \textbf{do}}$ \\ 
$\qquad \qquad $\textbf{let} $X(f)=label(u,v)$ \\ 
$\qquad \qquad \text{\textbf{for }}d_{X}[j]\in marking(v)$ \\ 
$\qquad \qquad \qquad M:=(FV(f[j])\setminus marking(u))\cap par(X)$ \\ 
$\qquad \qquad \qquad \text{\textbf{if }}M\neq \emptyset $ \\ 
$\qquad \qquad \qquad \qquad marking(u):=marking(u)\cup M$ \\ 
$\qquad \qquad \qquad \qquad todo:=todo\cup \{u\}$%
\end{tabular}%
\end{equation*}

\newpage

\subsection{Reset variables}

Let $\mathrm{PVI}(\varphi _{X},i)=Y(e)$ and let $V$ be the global control
flow graph. Then we define

\begin{equation*}
\begin{tabular}{l}
\textsc{ResetVariableGlobal(}$Y(e)$, $i,$ $V$\textsc{)} \\ 
$\varphi :=true$ \\ 
\textbf{for} $u=Y(f)\in V$ \textbf{do} \\ 
$\qquad c:=true$ \\ 
$\qquad k:=1$ \\ 
$\qquad $\textbf{for} $j=1\cdots \left\vert par(Y)\right\vert $ \textbf{do}
\\ 
$\qquad \qquad r:=[]$ \\ 
$\qquad \qquad $\textbf{if} $CFP(Y,j)$ \textbf{then} \\ 
$\qquad \qquad \qquad $\textbf{if} $dest(X,i,j)=\bot $ \\ 
$\qquad \qquad \qquad \qquad c:=c\wedge (e[j]=f[k])$ \\ 
$\qquad \qquad \qquad r:=r\vartriangleleft f[k]$ \\ 
$\qquad \qquad $\textbf{else if} $e[j]\in marking(u)$ \textbf{then} \\ 
$\qquad \qquad \qquad r:=r\vartriangleleft e[j]$ \\ 
$\qquad \qquad $\textbf{else} \\ 
$\qquad \qquad \qquad r:=r\vartriangleleft default\_value(e[j])$ \\ 
$\qquad \qquad k:=k+1$ \\ 
$\qquad \varphi :=\varphi \wedge (c\Rightarrow Y(r))$ \\ 
\textbf{return} $\varphi $%
\end{tabular}%
\end{equation*}

\newpage

\subsection{Compute values}

Let $C$ be a component containing related CFPs.

\begin{equation*}
\begin{tabular}{l}
\textsc{ComputeValues(}$\mathcal{E}$, $X_{init}(e_{init}),C$\textsc{)} \\ 
$result:=$ $\emptyset $ \\ 
$\text{\textbf{for }}(X,j)\in C$ \textbf{do} \\ 
$\qquad $\textbf{if} $(X=X_{init})$\textbf{\ then} $result:=result\cup
\{e_{init}[j]\}$ \\ 
\textbf{for} $(X,k)\in C$ \textbf{do} \\ 
$\qquad \text{\textbf{for }}i=1\cdots \mathrm{npred}(\varphi _{X})\text{ 
\textbf{do}}$ \\ 
$\qquad \qquad \text{\textbf{if }}\mathrm{source}(X,i,k)=v$\textbf{\ then} $%
result:=result\cup \{v\}$ \\ 
\textbf{for} $(Y,k)\in C$ \textbf{do} \\ 
$\qquad \text{\textbf{for }}i=1\cdots \mathrm{npred}(\varphi _{Y})\text{ 
\textbf{do}}$ \\ 
$\qquad \qquad \text{\textbf{if }}\mathrm{pred}(\varphi _{Y},i)=Y\wedge 
\mathrm{dest}(Y,i,k)=v$\textbf{\ then} $result:=result\cup \{v\}$ \\ 
\textbf{return} $result$%
\end{tabular}%
\end{equation*}

\begin{equation*}
\begin{tabular}{l}
\textsc{ComputeLocalControlFlowGraph(}$U,C$\textsc{)} \\ 
$V,E:=U,\emptyset $ \\ 
$todo:=U$ \\ 
\textbf{while} $todo\neq \emptyset $ \textbf{do} \\ 
$\qquad \text{\textbf{choose }}u=(X,k,d=e)\in todo$ \\ 
$\qquad todo:=todo\setminus \{u\}$ \\ 
$\qquad \text{\textbf{for }}i=1\cdots \mathrm{npred}(\varphi _{X})\text{ 
\textbf{do}}$ \\ 
$\qquad \qquad \text{\textbf{let }}Y=$ $\mathrm{pred}(\varphi _{X},i)$ \\ 
$\qquad \qquad \text{\textbf{if }}d=\bot $ $\text{\textbf{then}}$ \\ 
$\qquad \qquad \qquad \text{\textbf{if }}(Y,k^{\prime })\in C$ for some $%
k^{\prime }$ \textbf{then} \\ 
$\qquad \qquad \qquad \qquad \text{\textbf{if }}\mathrm{dest}(X,i,k^{\prime
})=e^{\prime }$ \textbf{then} \\ 
$\qquad \qquad \qquad \qquad \qquad v:=(Y,k^{\prime },d_{Y}[k^{\prime
}]=e^{\prime })$ \\ 
$\qquad \qquad \qquad \qquad \qquad insert(V,E,todo,u,i,v)$ \\ 
$\qquad \qquad \qquad \text{\textbf{else}}$ \\ 
$\qquad \qquad \qquad \qquad $\textbf{if} $X\neq Y$ \textbf{then} \\ 
$\qquad \qquad \qquad \qquad \qquad v:=(Y,\bot ,?=?)$ \\ 
$\qquad \qquad \qquad \qquad \qquad insert(V,E,todo,u,i,v)$ \\ 
$\qquad \qquad \text{\textbf{else}}$ \\ 
$\qquad \qquad \qquad \text{\textbf{if }}(Y,k^{\prime })\in C$ for some $%
k^{\prime }$ \textbf{then} \\ 
$\qquad \qquad \qquad \qquad \text{\textbf{if }}\left( \mathrm{source}%
(X,i,k)=e\mathbf{\ }\wedge \mathrm{dest}(X,i,k^{\prime })=e^{\prime }\right) 
$ \\ 
$\qquad \qquad \qquad \qquad \vee \left( Y\neq X\wedge \mathrm{source}%
(X,i,k)=\bot \mathbf{\ }\wedge \mathrm{dest}(X,i,k^{\prime })=e^{\prime
}\right) $ \\ 
$\qquad \qquad \qquad \qquad \vee \left( Y\neq X\wedge \mathrm{source}%
(X,i,k)=\bot \mathbf{\ }\wedge \mathrm{copy}(X,i,k)=e^{\prime }\right) $ 
\textbf{then} \\ 
$\qquad \qquad \qquad \qquad \qquad v:=(Y,k^{\prime },d_{Y}[k^{\prime
}]=e^{\prime })$ \\ 
$\qquad \qquad \qquad \qquad \qquad insert(V,E,todo,u,i,v)\qquad $ \\ 
$\qquad \qquad \qquad \text{\textbf{else}}$ \\ 
$\qquad \qquad \qquad \qquad v:=(Y,\bot ,?=?)$ \\ 
$\qquad \qquad \qquad \qquad insert(V,E,todo,u,i,v)\qquad $ \\ 
\textbf{return} $(V,E),$%
\end{tabular}%
\end{equation*}%
where $insert(V,E,todo,u,i,v)$ is shorthand for the statements%
\begin{equation*}
\begin{array}{l}
\text{\textbf{if} }v\notin todo\text{ \textbf{then} }todo:=todo\cup \{v\} \\ 
V:=V\cup \{v\} \\ 
E:=E\cup \{(u,i,v)\}%
\end{array}%
\end{equation*}%
\newpage

Let $(V$, $\mathrm{\longrightarrow })$ be a local control flow graph, and $%
\mathrm{rules}$ be a predicate defined as%
\begin{equation*}
\mathrm{rules}(V,X,i)=\exists _{(X,j,e)\in V}:(X,j,e)\overset{i}{\mathrm{%
\longrightarrow }}.
\end{equation*}%
\begin{equation*}
\begin{tabular}{l}
\textsc{ComputeBelongs(} $V$, $\mathrm{\longrightarrow }$, $\mathrm{rules}$%
\textsc{)} \\ 
$B_{k}:=\emptyset $ \\ 
\textbf{for} $(X,j,v)\in V$ \textbf{do} \\ 
$\qquad belongs:=\{m\mid d_{X}[m]$ is a data parameter of $X\}$ \\ 
$\qquad \text{\textbf{for }}i=1\cdots \mathrm{npred}(\varphi )\text{ \textbf{%
do}}$ \\ 
$\qquad \qquad \text{\textbf{for }}m\in belongs\text{ \textbf{do}}$ \\ 
$\qquad \qquad \qquad $\textbf{if }$\left( \mathrm{used}(X,i,m)\vee \mathrm{%
changed}(X,i,m)\right) \wedge \lnot \mathrm{rules}(V,X,i)$ \textbf{then} \\ 
$\qquad \qquad \qquad \qquad belongs:=belongs\setminus \{m\}$ \\ 
$\qquad B_{k}:=B_{k}\cup \{(X,d_{X}[m])\mid m\in belongs\}$ \\ 
\textbf{return} $B_{k}$%
\end{tabular}%
\end{equation*}%
\begin{equation*}
\begin{tabular}{l}
\textsc{ResetVariableLocal(}$i,$ $\sigma X(d_{X})=\varphi _{X},V_{1},\cdots
,V_{J},B_{1},\cdots ,B_{J},\mathrm{rules}$\textsc{)} \\ 
\textbf{let }$Y(e)=\mathrm{PVI}(\varphi _{X},i)$ \\ 
$e^{\prime }:=e$ \\ 
\textbf{for} $k=1\cdots \left\vert e\right\vert $ \textbf{do} \\ 
$\qquad $\textbf{if} $\mathrm{CFP}(Y,k)$ \textbf{then continue} \\ 
$\qquad relevant:=true$ \\ 
$\qquad $\textbf{for} $j=1\cdots J$ \textbf{do} \\ 
$\qquad \qquad $\textbf{if} $\mathrm{rules}(V_{j},X,i)$ \textbf{then} \\ 
$\qquad \qquad \qquad $\textbf{let }$p,q$ be such that $(Y,p,q)\in V_{j}$ \\ 
$\qquad \qquad \qquad $\textbf{if} $\mathrm{dest}(X,i,p)\neq \bot $ \textbf{%
then} \\ 
$\qquad \qquad \qquad \qquad $\textbf{let} $q^{\prime }=\mathrm{dest}(X,i,p)$
\\ 
$\qquad \qquad \qquad \qquad relevant:=relevant\wedge \left( \left(
Y,d_{Y}[k]\right) \in B_{j}\Rightarrow d_{Y}[k]\in marking(Y,p,q^{\prime
})\right) $ \\ 
$\qquad \qquad \qquad $\textbf{else} \\ 
$\qquad \qquad \qquad \qquad relevant:=relevant\wedge (\left(
Y,d_{Y}[k]\right) \in B_{j}\Rightarrow \exists r:d_{Y}[k]\in marking(Y,p,r))$
\\ 
$\qquad $\textbf{if} $\lnot relevant$ \textbf{then} $e^{\prime
}[k]:=default\_value(d_{Y}[k])$ \\ 
\textbf{return }$Y(e^{\prime })$%
\end{tabular}%
\end{equation*}%
\newpage

Let $\left( V,E\right) $ be a local control flow graph, and $B$ the
corresponding belongs relation.%
\begin{eqnarray*}
&&%
\begin{tabular}{l}
\textsc{UpdateMarkingRule1(}$(V,E,B)$, $u$, $i$, $v$\textsc{)} \\ 
\textbf{let} $u=\left( X,n,d_{X}[n]=z\right) $ \\ 
\textbf{let} $v=\left( Y,m,d_{Y}[m]=w\right) $ \\ 
\textbf{let} $Y(e)=\mathrm{PVI}(\varphi _{X},i)$ \\ 
$M:=\emptyset $ \\ 
$\text{\textbf{for }}d_{Y}[l]\in marking(v)\text{ \textbf{do}}$ \\ 
$\qquad M:=M\cup \left( \mathrm{FV}\left( \mathrm{rewr}(e[l],\left[
d_{X}[n]:=z\right] )\right) \cap \{d\mid (X,d)\in B\}\right) $ \\ 
\textbf{return }$marking(u)\cup M$%
\end{tabular}
\\
&& \\
&&%
\begin{tabular}{l}
\textsc{UpdateMarkingRule2(}$\left( V,E,B\right) $, $u$, $i$, $v$\textsc{)}
\\ 
\textbf{let} $u=\left( X,n,d_{X}[n]=z\right) $ \\ 
\textbf{let} $v=\left( Y,m,d_{Y}[m]=w\right) $ \\ 
\textbf{let} $Y(e)=\mathrm{PVI}(\varphi _{X},i)$ \\ 
$M:=\emptyset $ \\ 
$\text{\textbf{for }}d_{Y}[l]\in marking(v)\text{ \textbf{do}}$ \\ 
$\qquad $\textbf{if} $\left( Y,d_{Y}[l]\right) \notin B$ \textbf{then} \\ 
$\qquad \qquad M:=M\cup \left( \mathrm{FV}\left( \mathrm{rewr}(e[l],\left[
d_{X}[n]:=z\right] )\right) \cap \{d\mid (X,d)\in B\}\right) $ \\ 
\textbf{return }$marking(u)\cup M$%
\end{tabular}%
\end{eqnarray*}

\bigskip

\newpage

\begin{equation*}
\begin{array}{l}
\begin{tabular}{l}
\textsc{ComputeMarkingLocal(}$\mathcal{E}$, $(V_{1},E_{1},B_{1}),\cdots
,(V_{J},E_{J},B_{J})$\textsc{)} \\ 
\textbf{for} $j=1\cdots J$ \textbf{do} \\ 
$\qquad $\textbf{for} $u=(X,n,d_{X}[n]=z)\in V_{j}$ \textbf{do} \\ 
$\qquad \qquad marking(u):=significant(u)\cap \{d\mid (X,d)\in B_{j}\}$ \\ 
$stable:=false$ \\ 
\textbf{while} $\lnot stable$ \textbf{do} \\ 
$\qquad stableint:=false$ \\ 
$\qquad $\textbf{while} $\lnot stableint$ \textbf{do} \\ 
$\qquad \qquad stableint:=true$ \\ 
$\qquad \qquad $\textbf{for} $j=1\cdots J$ \textbf{do} \\ 
$\qquad \qquad \qquad todo:=V_{j}$ \\ 
$\qquad \qquad \qquad $\textbf{while} $todo\neq \emptyset $ \textbf{do} \\ 
$\qquad \qquad \qquad \qquad $\textbf{choose} $u=(X,n,d_{X}[n]=z)\in todo$
\\ 
$\qquad \qquad \qquad \qquad todo:=todo\setminus \{u\}$ \\ 
$\qquad \qquad \qquad \qquad $\textbf{if} $marking(u)=\{d\mid (X,d)\in
B_{j}\}$ \textbf{then continue} \\ 
$\qquad \qquad \qquad \qquad $\textbf{for} $(u,i,v)\in E_{j}$ \textbf{do} \\ 
$\qquad \qquad \qquad \qquad \qquad m:=marking(u)$ \\ 
$\qquad \qquad \qquad \qquad \qquad marking(u):=$ \textsc{UpdateMarkingRule1(%
}$(V_{j},E_{j},B_{j})$, $u$, $i$, $v$\textsc{)} \\ 
$\qquad \qquad \qquad \qquad \qquad \text{\textbf{if }}m\neq marking(u)$ 
\textbf{then} \\ 
$\qquad \qquad \qquad \qquad \qquad \qquad todo:=todo\cup \{v^{\prime }\mid
\exists i^{\prime }:(v^{\prime },i^{\prime },u)\in E_{j}\}$ \\ 
$\qquad \qquad \qquad \qquad \qquad \qquad stableint:=false$ \\ 
$\qquad stableext:=false$%
\end{tabular}
\\ 
\begin{tabular}{l}
$\qquad $\textbf{while} $\lnot stableext$ \textbf{do} \\ 
$\qquad \qquad stableext:=true$ \\ 
$\qquad \qquad $\textbf{for} $j=1\cdots J$ \textbf{do} \\ 
$\qquad \qquad \qquad $\textbf{for} $u=(X,n,d_{X}[n]=z)\in V_{j}$ \textbf{do}
\\ 
$\qquad \qquad \qquad \qquad $\textbf{if} $marking(u)=\{d\mid (X,d)\in
B_{j}\}$ \textbf{then continue} \\ 
$\qquad \qquad \qquad \qquad \text{\textbf{for }}i=1\cdots \mathrm{npred}%
(\varphi _{X})$ \textbf{do} \\ 
$\qquad \qquad \qquad \qquad \qquad $\textbf{let} $Y(e)=\mathrm{PVI}(\varphi
_{X},i)$ \\ 
$\qquad \qquad \qquad \qquad \qquad $\textbf{for} $k=1\cdots J$ \textbf{do}
\\ 
$\qquad \qquad \qquad \qquad \qquad \qquad \text{\textbf{if }}k\neq j$ 
\textbf{then} \\ 
$\qquad \qquad \qquad \qquad \qquad \qquad \qquad $\textbf{for} $%
v=(Y,m,d_{Y}[m]=w)\in V_{k}$ \textbf{do} \\ 
$\qquad \qquad \qquad \qquad \qquad \qquad \qquad \qquad $\textbf{if} $%
\exists v^{\prime }=(X,m^{\prime },D_{X}[m^{\prime }])\in V_{k}$ \textbf{%
such that} $(v^{\prime },i,v)\in E_{k}$ \textbf{then} \\ 
$\qquad \qquad \qquad \qquad \qquad \qquad \qquad \qquad \qquad m:=marking(u)
$ \\ 
$\qquad \qquad \qquad \qquad \qquad \qquad \qquad \qquad \qquad marking(u):=$
\textsc{UpdateMarkingRule2(}$(V_{j},E_{j},B_{j})$, $u$, $i$, $v$\textsc{)}
\\ 
$\qquad \qquad \qquad \qquad \qquad \qquad \qquad \qquad \qquad \text{%
\textbf{if }}m\neq marking(u)$ \textbf{then} \\ 
$\qquad \qquad \qquad \qquad \qquad \qquad \qquad \qquad \qquad \qquad
stableint:=false$ \\ 
$\qquad \qquad \qquad \qquad \qquad \qquad \qquad \qquad \qquad \qquad
stableext:=false$ \\ 
$\qquad stable:=stableint\wedge stableext$%
\end{tabular}%
\end{array}%
\end{equation*}%
\newpage \newpage 
