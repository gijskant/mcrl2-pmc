%TCIDATA{Version=5.50.0.2890}
%TCIDATA{LaTeXparent=1,1,pbes-implementation-notes.tex}
                      

\section{Parity game generator}

Let $\mathcal{E=(\sigma }_{1}X_{1}(d_{1}:D_{1})=\varphi _{1})\ldots \mathcal{%
(\sigma }_{n}X_{n}(d_{n}:D_{n})=\varphi _{n})$ be a PBES with initial state $%
X_{init}(e_{init})$, and let $R:PbesTerm\rightarrow PbesTerm$ be a rewriter.
The PBES must be in normal form, i.e. it may not contain negations or
implications. The following algorithm computes a BES. The generated
equations are in a restricted format, such that the BES can be taken as
input for a parity game solver.

{\small 
\begin{equation*}
\begin{array}{l}
\text{\textsc{GenerateBES(}}\mathcal{E}\text{, }X_{init}(e_{init})\text{, }R%
\text{\textsc{)}} \\ 
\text{result}:=\{(\nu Y_{\top }=Y_{\top }),(\mu Y_{\bot }=Y_{\bot })\} \\ 
\text{visited}:=\{R(X_{init}(e_{init}))\} \\ 
\text{explored}:=\{\top ,\bot \} \\ 
\text{\textbf{while }}\text{visited}\neq \emptyset \text{ \textbf{do}} \\ 
\qquad \text{\textbf{choose }}\psi \in \text{visited}\setminus \text{explored%
} \\ 
\qquad \text{visited}:=\text{visited}\ \backslash \ \{\psi \} \\ 
\qquad \text{explored}:=\text{explored}\cup \{\psi \} \\ 
\qquad \text{\textbf{if} }\{\psi =X_{k}(e)\}\text{ \textbf{then}} \\ 
\qquad \qquad \xi :=R(\varphi _{k}[d_{k}:=e]) \\ 
\qquad \text{\textbf{else}} \\ 
\qquad \qquad \xi :=\psi \\ 
\qquad \text{\textbf{if} }\{\xi =X_{k}(e)\}\text{ \textbf{then}} \\ 
\qquad \qquad \text{result}:=\text{ result}\cup (\sigma _{\psi }Y_{\psi
}=Y_{\xi }) \\ 
\qquad \qquad \sigma _{\xi }:=\sigma _{k} \\ 
\qquad \qquad \text{visited}:=\text{visited}\cup \{\xi \} \\ 
\qquad \text{\textbf{else if} }\{\xi =\dbigwedge_{j\in J}\phi _{j}\}\text{ 
\textbf{then}} \\ 
\qquad \qquad \text{result}:=\text{ result}\cup (\sigma _{\psi }Y_{\psi
}=\dbigwedge_{j\in J}Y_{\phi _{j}}) \\ 
\qquad \qquad \text{\textbf{for} }j\in J\text{ \textbf{do}} \\ 
\qquad \qquad \qquad \text{\textbf{if} }\{\xi =X_{k}(e)\}\text{ \textbf{then}
}\sigma _{\phi _{j}}:=\sigma _{k}\text{ \textbf{else} }\sigma _{\phi
_{j}}:=\sigma _{\psi } \\ 
\qquad \qquad \text{visited}:=\text{visited}\cup \{\phi _{j}\}_{j\in J} \\ 
\qquad \text{\textbf{else if} }\{\xi =\dbigvee_{j\in J}\phi _{j}\}\text{ 
\textbf{then}} \\ 
\qquad \qquad \text{result}:=\text{ result}\cup (\sigma _{\psi }Y_{\psi
}=\dbigvee_{j\in J}Y_{\phi _{j}}) \\ 
\qquad \qquad \text{\textbf{for} }j\in J\text{ \textbf{do}} \\ 
\qquad \qquad \qquad \text{\textbf{if} }\{\xi =X_{k}(e)\}\text{ \textbf{then}
}\sigma _{\phi _{j}}:=\sigma _{k}\text{ \textbf{else} }\sigma _{\phi
_{j}}:=\sigma _{\psi } \\ 
\qquad \qquad \text{visited}:=\text{visited}\cup \{\phi _{j}\}_{j\in J} \\ 
\qquad \text{\textbf{else if} }\{\xi =\top \}\text{ \textbf{then}} \\ 
\qquad \qquad \text{result}:=\text{ result}\cup (\sigma _{\psi }Y_{\psi
}=Y_{\top }) \\ 
\qquad \text{\textbf{else if} }\{\xi =\bot \}\text{ \textbf{then}} \\ 
\qquad \qquad \text{result}:=\text{ result}\cup (\sigma _{\psi }Y_{\psi
}=Y_{\bot }) \\ 
\text{\textbf{return }result}%
\end{array}%
\end{equation*}%
}

In every step of the while loop the equation for $Y_{\psi }$ is computed. If
the right hand side of the equation for $Y_{\psi }$ is a propositional
variable instantiation, it is expanded into the right hand side of the
corresponding PBES equation. Otherwise it is converted into a disjunction or
conjunction by introducing new BES variables. The rewriter $R$ is expected
to eliminate all quantifiers, so the while loop does not contain cases for
handling them. The order of the equations in the BES is significant.
Therefore in the implementation instead of fixpoint symbols $\sigma _{\psi }$
priority values are used. The BES variables $Y_{\psi }$ are represented by
integers.

An alternative for inserting the equations $(\nu Y_{\top }=Y_{\top })$ and $%
(\mu Y_{\bot }=Y_{\bot })$ at the beginning of the resulting BES is to
replace $\mu Y_{\psi }=\top $ and $\nu Y_{\psi }=\top $ by $\nu Y_{\psi
}=Y_{\psi }$, and to replace $\mu Y_{\psi }=\bot $ and $\nu Y_{\psi }=\bot $
by $\mu Y_{\psi }=Y_{\psi }$. This eliminates the need to introduce special
equations for true and false.

This algorithm is implemented in the class \texttt{parity\_game\_generator}.
The choice for $\psi $ in the while loop is left to the user of the
class.\newpage 
