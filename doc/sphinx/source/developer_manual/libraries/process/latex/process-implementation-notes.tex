% at
% left merge
% \RequirePackage{stmaryrd}


\documentclass{article}
%%%%%%%%%%%%%%%%%%%%%%%%%%%%%%%%%%%%%%%%%%%%%%%%%%%%%%%%%%%%%%%%%%%%%%%%%%%%%%%%%%%%%%%%%%%%%%%%%%%%%%%%%%%%%%%%%%%%%%%%%%%%%%%%%%%%%%%%%%%%%%%%%%%%%%%%%%%%%%%%%%%%%%%%%%%%%%%%%%%%%%%%%%%%%%%%%%%%%%%%%%%%%%%%%%%%%%%%%%%%%%%%%%%%%%%%%%%%%%%%%%%%%%%%%%%%
\usepackage{amssymb}
\usepackage{geometry}
\usepackage{stmaryrd}

%TCIDATA{OutputFilter=LATEX.DLL}
%TCIDATA{Version=5.50.0.2890}
%TCIDATA{<META NAME="SaveForMode" CONTENT="1">}
%TCIDATA{BibliographyScheme=Manual}
%TCIDATA{Created=Friday, June 15, 2012 17:58:29}
%TCIDATA{LastRevised=Thursday, June 23, 2016 15:51:58}
%TCIDATA{<META NAME="GraphicsSave" CONTENT="32">}
%TCIDATA{<META NAME="DocumentShell" CONTENT="Standard LaTeX\Blank - Standard LaTeX Article">}
%TCIDATA{CSTFile=40 LaTeX article.cst}

\newtheorem{theorem}{Theorem}
\newtheorem{acknowledgement}[theorem]{Acknowledgement}
\newtheorem{algorithm}[theorem]{Algorithm}
\newtheorem{axiom}[theorem]{Axiom}
\newtheorem{case}[theorem]{Case}
\newtheorem{claim}[theorem]{Claim}
\newtheorem{conclusion}[theorem]{Conclusion}
\newtheorem{condition}[theorem]{Condition}
\newtheorem{conjecture}[theorem]{Conjecture}
\newtheorem{corollary}[theorem]{Corollary}
\newtheorem{criterion}[theorem]{Criterion}
\newtheorem{definition}[theorem]{Definition}
\newtheorem{example}[theorem]{Example}
\newtheorem{exercise}[theorem]{Exercise}
\newtheorem{lemma}[theorem]{Lemma}
\newtheorem{notation}[theorem]{Notation}
\newtheorem{problem}[theorem]{Problem}
\newtheorem{proposition}[theorem]{Proposition}
\newtheorem{remark}[theorem]{Remark}
\newtheorem{solution}[theorem]{Solution}
\newtheorem{summary}[theorem]{Summary}
\newenvironment{proof}[1][Proof]{\noindent\textbf{#1.} }{\ \rule{0.5em}{0.5em}}
\geometry{left=1in,right=1in,top=1in,bottom=1in}
% Macros for Scientific Word 4.0 documents saved with the LaTeX filter.
% Copyright (C) 2002 Mackichan Software, Inc.

\typeout{TCILATEX Macros for Scientific Word 5.0 <13 Feb 2003>.}
\typeout{NOTICE:  This macro file is NOT proprietary and may be 
freely copied and distributed.}
%
\makeatletter

%%%%%%%%%%%%%%%%%%%%%
% pdfTeX related.
\ifx\pdfoutput\relax\let\pdfoutput=\undefined\fi
\newcount\msipdfoutput
\ifx\pdfoutput\undefined
\else
 \ifcase\pdfoutput
 \else 
    \msipdfoutput=1
    \ifx\paperwidth\undefined
    \else
      \ifdim\paperheight=0pt\relax
      \else
        \pdfpageheight\paperheight
      \fi
      \ifdim\paperwidth=0pt\relax
      \else
        \pdfpagewidth\paperwidth
      \fi
    \fi
  \fi  
\fi

%%%%%%%%%%%%%%%%%%%%%
% FMTeXButton
% This is used for putting TeXButtons in the 
% frontmatter of a document. Add a line like
% \QTagDef{FMTeXButton}{101}{} to the filter 
% section of the cst being used. Also add a
% new section containing:
%     [f_101]
%     ALIAS=FMTexButton
%     TAG_TYPE=FIELD
%     TAG_LEADIN=TeX Button:
%
% It also works to put \defs in the preamble after 
% the \input tcilatex
\def\FMTeXButton#1{#1}
%
%%%%%%%%%%%%%%%%%%%%%%
% macros for time
\newcount\@hour\newcount\@minute\chardef\@x10\chardef\@xv60
\def\tcitime{
\def\@time{%
  \@minute\time\@hour\@minute\divide\@hour\@xv
  \ifnum\@hour<\@x 0\fi\the\@hour:%
  \multiply\@hour\@xv\advance\@minute-\@hour
  \ifnum\@minute<\@x 0\fi\the\@minute
  }}%

%%%%%%%%%%%%%%%%%%%%%%
% macro for hyperref and msihyperref
%\@ifundefined{hyperref}{\def\hyperref#1#2#3#4{#2\ref{#4}#3}}{}

\def\x@hyperref#1#2#3{%
   % Turn off various catcodes before reading parameter 4
   \catcode`\~ = 12
   \catcode`\$ = 12
   \catcode`\_ = 12
   \catcode`\# = 12
   \catcode`\& = 12
   \y@hyperref{#1}{#2}{#3}%
}

\def\y@hyperref#1#2#3#4{%
   #2\ref{#4}#3
   \catcode`\~ = 13
   \catcode`\$ = 3
   \catcode`\_ = 8
   \catcode`\# = 6
   \catcode`\& = 4
}

\@ifundefined{hyperref}{\let\hyperref\x@hyperref}{}
\@ifundefined{msihyperref}{\let\msihyperref\x@hyperref}{}




% macro for external program call
\@ifundefined{qExtProgCall}{\def\qExtProgCall#1#2#3#4#5#6{\relax}}{}
%%%%%%%%%%%%%%%%%%%%%%
%
% macros for graphics
%
\def\FILENAME#1{#1}%
%
\def\QCTOpt[#1]#2{%
  \def\QCTOptB{#1}
  \def\QCTOptA{#2}
}
\def\QCTNOpt#1{%
  \def\QCTOptA{#1}
  \let\QCTOptB\empty
}
\def\Qct{%
  \@ifnextchar[{%
    \QCTOpt}{\QCTNOpt}
}
\def\QCBOpt[#1]#2{%
  \def\QCBOptB{#1}%
  \def\QCBOptA{#2}%
}
\def\QCBNOpt#1{%
  \def\QCBOptA{#1}%
  \let\QCBOptB\empty
}
\def\Qcb{%
  \@ifnextchar[{%
    \QCBOpt}{\QCBNOpt}%
}
\def\PrepCapArgs{%
  \ifx\QCBOptA\empty
    \ifx\QCTOptA\empty
      {}%
    \else
      \ifx\QCTOptB\empty
        {\QCTOptA}%
      \else
        [\QCTOptB]{\QCTOptA}%
      \fi
    \fi
  \else
    \ifx\QCBOptA\empty
      {}%
    \else
      \ifx\QCBOptB\empty
        {\QCBOptA}%
      \else
        [\QCBOptB]{\QCBOptA}%
      \fi
    \fi
  \fi
}
\newcount\GRAPHICSTYPE
%\GRAPHICSTYPE 0 is for TurboTeX
%\GRAPHICSTYPE 1 is for DVIWindo (PostScript)
%%%(removed)%\GRAPHICSTYPE 2 is for psfig (PostScript)
\GRAPHICSTYPE=\z@
\def\GRAPHICSPS#1{%
 \ifcase\GRAPHICSTYPE%\GRAPHICSTYPE=0
   \special{ps: #1}%
 \or%\GRAPHICSTYPE=1
   \special{language "PS", include "#1"}%
%%%\or%\GRAPHICSTYPE=2
%%%  #1%
 \fi
}%
%
\def\GRAPHICSHP#1{\special{include #1}}%
%
% \graffile{ body }                                  %#1
%          { contentswidth (scalar)  }               %#2
%          { contentsheight (scalar) }               %#3
%          { vertical shift when in-line (scalar) }  %#4

\def\graffile#1#2#3#4{%
%%% \ifnum\GRAPHICSTYPE=\tw@
%%%  %Following if using psfig
%%%  \@ifundefined{psfig}{\input psfig.tex}{}%
%%%  \psfig{file=#1, height=#3, width=#2}%
%%% \else
  %Following for all others
  % JCS - added BOXTHEFRAME, see below
    \bgroup
	   \@inlabelfalse
       \leavevmode
       \@ifundefined{bbl@deactivate}{\def~{\string~}}{\activesoff}%
        \raise -#4 \BOXTHEFRAME{%
           \hbox to #2{\raise #3\hbox to #2{\null #1\hfil}}}%
    \egroup
}%
%
% A box for drafts
\def\draftbox#1#2#3#4{%
 \leavevmode\raise -#4 \hbox{%
  \frame{\rlap{\protect\tiny #1}\hbox to #2%
   {\vrule height#3 width\z@ depth\z@\hfil}%
  }%
 }%
}%
%
\newcount\@msidraft
\@msidraft=\z@
\let\nographics=\@msidraft
\newif\ifwasdraft
\wasdraftfalse

%  \GRAPHIC{ body }                                  %#1
%          { draft name }                            %#2
%          { contentswidth (scalar)  }               %#3
%          { contentsheight (scalar) }               %#4
%          { vertical shift when in-line (scalar) }  %#5
\def\GRAPHIC#1#2#3#4#5{%
   \ifnum\@msidraft=\@ne\draftbox{#2}{#3}{#4}{#5}%
   \else\graffile{#1}{#3}{#4}{#5}%
   \fi
}
%
\def\addtoLaTeXparams#1{%
    \edef\LaTeXparams{\LaTeXparams #1}}%
%
% JCS -  added a switch BoxFrame that can 
% be set by including X in the frame params.
% If set a box is drawn around the frame.

\newif\ifBoxFrame \BoxFramefalse
\newif\ifOverFrame \OverFramefalse
\newif\ifUnderFrame \UnderFramefalse

\def\BOXTHEFRAME#1{%
   \hbox{%
      \ifBoxFrame
         \frame{#1}%
      \else
         {#1}%
      \fi
   }%
}


\def\doFRAMEparams#1{\BoxFramefalse\OverFramefalse\UnderFramefalse\readFRAMEparams#1\end}%
\def\readFRAMEparams#1{%
 \ifx#1\end%
  \let\next=\relax
  \else
  \ifx#1i\dispkind=\z@\fi
  \ifx#1d\dispkind=\@ne\fi
  \ifx#1f\dispkind=\tw@\fi
  \ifx#1t\addtoLaTeXparams{t}\fi
  \ifx#1b\addtoLaTeXparams{b}\fi
  \ifx#1p\addtoLaTeXparams{p}\fi
  \ifx#1h\addtoLaTeXparams{h}\fi
  \ifx#1X\BoxFrametrue\fi
  \ifx#1O\OverFrametrue\fi
  \ifx#1U\UnderFrametrue\fi
  \ifx#1w
    \ifnum\@msidraft=1\wasdrafttrue\else\wasdraftfalse\fi
    \@msidraft=\@ne
  \fi
  \let\next=\readFRAMEparams
  \fi
 \next
 }%
%
%Macro for In-line graphics object
%   \IFRAME{ contentswidth (scalar)  }               %#1
%          { contentsheight (scalar) }               %#2
%          { vertical shift when in-line (scalar) }  %#3
%          { draft name }                            %#4
%          { body }                                  %#5
%          { caption}                                %#6


\def\IFRAME#1#2#3#4#5#6{%
      \bgroup
      \let\QCTOptA\empty
      \let\QCTOptB\empty
      \let\QCBOptA\empty
      \let\QCBOptB\empty
      #6%
      \parindent=0pt
      \leftskip=0pt
      \rightskip=0pt
      \setbox0=\hbox{\QCBOptA}%
      \@tempdima=#1\relax
      \ifOverFrame
          % Do this later
          \typeout{This is not implemented yet}%
          \show\HELP
      \else
         \ifdim\wd0>\@tempdima
            \advance\@tempdima by \@tempdima
            \ifdim\wd0 >\@tempdima
               \setbox1 =\vbox{%
                  \unskip\hbox to \@tempdima{\hfill\GRAPHIC{#5}{#4}{#1}{#2}{#3}\hfill}%
                  \unskip\hbox to \@tempdima{\parbox[b]{\@tempdima}{\QCBOptA}}%
               }%
               \wd1=\@tempdima
            \else
               \textwidth=\wd0
               \setbox1 =\vbox{%
                 \noindent\hbox to \wd0{\hfill\GRAPHIC{#5}{#4}{#1}{#2}{#3}\hfill}\\%
                 \noindent\hbox{\QCBOptA}%
               }%
               \wd1=\wd0
            \fi
         \else
            \ifdim\wd0>0pt
              \hsize=\@tempdima
              \setbox1=\vbox{%
                \unskip\GRAPHIC{#5}{#4}{#1}{#2}{0pt}%
                \break
                \unskip\hbox to \@tempdima{\hfill \QCBOptA\hfill}%
              }%
              \wd1=\@tempdima
           \else
              \hsize=\@tempdima
              \setbox1=\vbox{%
                \unskip\GRAPHIC{#5}{#4}{#1}{#2}{0pt}%
              }%
              \wd1=\@tempdima
           \fi
         \fi
         \@tempdimb=\ht1
         %\advance\@tempdimb by \dp1
         \advance\@tempdimb by -#2
         \advance\@tempdimb by #3
         \leavevmode
         \raise -\@tempdimb \hbox{\box1}%
      \fi
      \egroup%
}%
%
%Macro for Display graphics object
%   \DFRAME{ contentswidth (scalar)  }               %#1
%          { contentsheight (scalar) }               %#2
%          { draft label }                           %#3
%          { name }                                  %#4
%          { caption}                                %#5
\def\DFRAME#1#2#3#4#5{%
  \vspace\topsep
  \hfil\break
  \bgroup
     \leftskip\@flushglue
	 \rightskip\@flushglue
	 \parindent\z@
	 \parfillskip\z@skip
     \let\QCTOptA\empty
     \let\QCTOptB\empty
     \let\QCBOptA\empty
     \let\QCBOptB\empty
	 \vbox\bgroup
        \ifOverFrame 
           #5\QCTOptA\par
        \fi
        \GRAPHIC{#4}{#3}{#1}{#2}{\z@}%
        \ifUnderFrame 
           \break#5\QCBOptA
        \fi
	 \egroup
  \egroup
  \vspace\topsep
  \break
}%
%
%Macro for Floating graphic object
%   \FFRAME{ framedata f|i tbph x F|T }              %#1
%          { contentswidth (scalar)  }               %#2
%          { contentsheight (scalar) }               %#3
%          { caption }                               %#4
%          { label }                                 %#5
%          { draft name }                            %#6
%          { body }                                  %#7
\def\FFRAME#1#2#3#4#5#6#7{%
 %If float.sty loaded and float option is 'h', change to 'H'  (gp) 1998/09/05
  \@ifundefined{floatstyle}
    {%floatstyle undefined (and float.sty not present), no change
     \begin{figure}[#1]%
    }
    {%floatstyle DEFINED
	 \ifx#1h%Only the h parameter, change to H
      \begin{figure}[H]%
	 \else
      \begin{figure}[#1]%
	 \fi
	}
  \let\QCTOptA\empty
  \let\QCTOptB\empty
  \let\QCBOptA\empty
  \let\QCBOptB\empty
  \ifOverFrame
    #4
    \ifx\QCTOptA\empty
    \else
      \ifx\QCTOptB\empty
        \caption{\QCTOptA}%
      \else
        \caption[\QCTOptB]{\QCTOptA}%
      \fi
    \fi
    \ifUnderFrame\else
      \label{#5}%
    \fi
  \else
    \UnderFrametrue%
  \fi
  \begin{center}\GRAPHIC{#7}{#6}{#2}{#3}{\z@}\end{center}%
  \ifUnderFrame
    #4
    \ifx\QCBOptA\empty
      \caption{}%
    \else
      \ifx\QCBOptB\empty
        \caption{\QCBOptA}%
      \else
        \caption[\QCBOptB]{\QCBOptA}%
      \fi
    \fi
    \label{#5}%
  \fi
  \end{figure}%
 }%
%
%
%    \FRAME{ framedata f|i tbph x F|T }              %#1
%          { contentswidth (scalar)  }               %#2
%          { contentsheight (scalar) }               %#3
%          { vertical shift when in-line (scalar) }  %#4
%          { caption }                               %#5
%          { label }                                 %#6
%          { name }                                  %#7
%          { body }                                  %#8
%
%    framedata is a string which can contain the following
%    characters: idftbphxFT
%    Their meaning is as follows:
%             i, d or f : in-line, display, or floating
%             t,b,p,h   : LaTeX floating placement options
%             x         : fit contents box to contents
%             F or T    : Figure or Table. 
%                         Later this can expand
%                         to a more general float class.
%
%
\newcount\dispkind%

\def\makeactives{
  \catcode`\"=\active
  \catcode`\;=\active
  \catcode`\:=\active
  \catcode`\'=\active
  \catcode`\~=\active
}
\bgroup
   \makeactives
   \gdef\activesoff{%
      \def"{\string"}%
      \def;{\string;}%
      \def:{\string:}%
      \def'{\string'}%
      \def~{\string~}%
      %\bbl@deactivate{"}%
      %\bbl@deactivate{;}%
      %\bbl@deactivate{:}%
      %\bbl@deactivate{'}%
    }
\egroup

\def\FRAME#1#2#3#4#5#6#7#8{%
 \bgroup
 \ifnum\@msidraft=\@ne
   \wasdrafttrue
 \else
   \wasdraftfalse%
 \fi
 \def\LaTeXparams{}%
 \dispkind=\z@
 \def\LaTeXparams{}%
 \doFRAMEparams{#1}%
 \ifnum\dispkind=\z@\IFRAME{#2}{#3}{#4}{#7}{#8}{#5}\else
  \ifnum\dispkind=\@ne\DFRAME{#2}{#3}{#7}{#8}{#5}\else
   \ifnum\dispkind=\tw@
    \edef\@tempa{\noexpand\FFRAME{\LaTeXparams}}%
    \@tempa{#2}{#3}{#5}{#6}{#7}{#8}%
    \fi
   \fi
  \fi
  \ifwasdraft\@msidraft=1\else\@msidraft=0\fi{}%
  \egroup
 }%
%
% This macro added to let SW gobble a parameter that
% should not be passed on and expanded. 

\def\TEXUX#1{"texux"}

%
% Macros for text attributes:
%
\def\BF#1{{\bf {#1}}}%
\def\NEG#1{\leavevmode\hbox{\rlap{\thinspace/}{$#1$}}}%
%
%%%%%%%%%%%%%%%%%%%%%%%%%%%%%%%%%%%%%%%%%%%%%%%%%%%%%%%%%%%%%%%%%%%%%%%%
%
%
% macros for user - defined functions
\def\limfunc#1{\mathop{\rm #1}}%
\def\func#1{\mathop{\rm #1}\nolimits}%
% macro for unit names
\def\unit#1{\mathord{\thinspace\rm #1}}%

%
% miscellaneous 
\long\def\QQQ#1#2{%
     \long\expandafter\def\csname#1\endcsname{#2}}%
\@ifundefined{QTP}{\def\QTP#1{}}{}
\@ifundefined{QEXCLUDE}{\def\QEXCLUDE#1{}}{}
\@ifundefined{Qlb}{\def\Qlb#1{#1}}{}
\@ifundefined{Qlt}{\def\Qlt#1{#1}}{}
\def\QWE{}%
\long\def\QQA#1#2{}%
\def\QTR#1#2{{\csname#1\endcsname {#2}}}%
\long\def\TeXButton#1#2{#2}%
\long\def\QSubDoc#1#2{#2}%
\def\EXPAND#1[#2]#3{}%
\def\NOEXPAND#1[#2]#3{}%
\def\PROTECTED{}%
\def\LaTeXparent#1{}%
\def\ChildStyles#1{}%
\def\ChildDefaults#1{}%
\def\QTagDef#1#2#3{}%

% Constructs added with Scientific Notebook
\@ifundefined{correctchoice}{\def\correctchoice{\relax}}{}
\@ifundefined{HTML}{\def\HTML#1{\relax}}{}
\@ifundefined{TCIIcon}{\def\TCIIcon#1#2#3#4{\relax}}{}
\if@compatibility
  \typeout{Not defining UNICODE  U or CustomNote commands for LaTeX 2.09.}
\else
  \providecommand{\UNICODE}[2][]{\protect\rule{.1in}{.1in}}
  \providecommand{\U}[1]{\protect\rule{.1in}{.1in}}
  \providecommand{\CustomNote}[3][]{\marginpar{#3}}
\fi

\@ifundefined{lambdabar}{
      \def\lambdabar{\errmessage{You have used the lambdabar symbol. 
                      This is available for typesetting only in RevTeX styles.}}
   }{}

%
% Macros for style editor docs
\@ifundefined{StyleEditBeginDoc}{\def\StyleEditBeginDoc{\relax}}{}
%
% Macros for footnotes
\def\QQfnmark#1{\footnotemark}
\def\QQfntext#1#2{\addtocounter{footnote}{#1}\footnotetext{#2}}
%
% Macros for indexing.
%
\@ifundefined{TCIMAKEINDEX}{}{\makeindex}%
%
% Attempts to avoid problems with other styles
\@ifundefined{abstract}{%
 \def\abstract{%
  \if@twocolumn
   \section*{Abstract (Not appropriate in this style!)}%
   \else \small 
   \begin{center}{\bf Abstract\vspace{-.5em}\vspace{\z@}}\end{center}%
   \quotation 
   \fi
  }%
 }{%
 }%
\@ifundefined{endabstract}{\def\endabstract
  {\if@twocolumn\else\endquotation\fi}}{}%
\@ifundefined{maketitle}{\def\maketitle#1{}}{}%
\@ifundefined{affiliation}{\def\affiliation#1{}}{}%
\@ifundefined{proof}{\def\proof{\noindent{\bfseries Proof. }}}{}%
\@ifundefined{endproof}{\def\endproof{\mbox{\ \rule{.1in}{.1in}}}}{}%
\@ifundefined{newfield}{\def\newfield#1#2{}}{}%
\@ifundefined{chapter}{\def\chapter#1{\par(Chapter head:)#1\par }%
 \newcount\c@chapter}{}%
\@ifundefined{part}{\def\part#1{\par(Part head:)#1\par }}{}%
\@ifundefined{section}{\def\section#1{\par(Section head:)#1\par }}{}%
\@ifundefined{subsection}{\def\subsection#1%
 {\par(Subsection head:)#1\par }}{}%
\@ifundefined{subsubsection}{\def\subsubsection#1%
 {\par(Subsubsection head:)#1\par }}{}%
\@ifundefined{paragraph}{\def\paragraph#1%
 {\par(Subsubsubsection head:)#1\par }}{}%
\@ifundefined{subparagraph}{\def\subparagraph#1%
 {\par(Subsubsubsubsection head:)#1\par }}{}%
%%%%%%%%%%%%%%%%%%%%%%%%%%%%%%%%%%%%%%%%%%%%%%%%%%%%%%%%%%%%%%%%%%%%%%%%
% These symbols are not recognized by LaTeX
\@ifundefined{therefore}{\def\therefore{}}{}%
\@ifundefined{backepsilon}{\def\backepsilon{}}{}%
\@ifundefined{yen}{\def\yen{\hbox{\rm\rlap=Y}}}{}%
\@ifundefined{registered}{%
   \def\registered{\relax\ifmmode{}\r@gistered
                    \else$\m@th\r@gistered$\fi}%
 \def\r@gistered{^{\ooalign
  {\hfil\raise.07ex\hbox{$\scriptstyle\rm\text{R}$}\hfil\crcr
  \mathhexbox20D}}}}{}%
\@ifundefined{Eth}{\def\Eth{}}{}%
\@ifundefined{eth}{\def\eth{}}{}%
\@ifundefined{Thorn}{\def\Thorn{}}{}%
\@ifundefined{thorn}{\def\thorn{}}{}%
% A macro to allow any symbol that requires math to appear in text
\def\TEXTsymbol#1{\mbox{$#1$}}%
\@ifundefined{degree}{\def\degree{{}^{\circ}}}{}%
%
% macros for T3TeX files
\newdimen\theight
\@ifundefined{Column}{\def\Column{%
 \vadjust{\setbox\z@=\hbox{\scriptsize\quad\quad tcol}%
  \theight=\ht\z@\advance\theight by \dp\z@\advance\theight by \lineskip
  \kern -\theight \vbox to \theight{%
   \rightline{\rlap{\box\z@}}%
   \vss
   }%
  }%
 }}{}%
%
\@ifundefined{qed}{\def\qed{%
 \ifhmode\unskip\nobreak\fi\ifmmode\ifinner\else\hskip5\p@\fi\fi
 \hbox{\hskip5\p@\vrule width4\p@ height6\p@ depth1.5\p@\hskip\p@}%
 }}{}%
%
\@ifundefined{cents}{\def\cents{\hbox{\rm\rlap c/}}}{}%
\@ifundefined{tciLaplace}{\def\tciLaplace{\ensuremath{\mathcal{L}}}}{}%
\@ifundefined{tciFourier}{\def\tciFourier{\ensuremath{\mathcal{F}}}}{}%
\@ifundefined{textcurrency}{\def\textcurrency{\hbox{\rm\rlap xo}}}{}%
\@ifundefined{texteuro}{\def\texteuro{\hbox{\rm\rlap C=}}}{}%
\@ifundefined{euro}{\def\euro{\hbox{\rm\rlap C=}}}{}%
\@ifundefined{textfranc}{\def\textfranc{\hbox{\rm\rlap-F}}}{}%
\@ifundefined{textlira}{\def\textlira{\hbox{\rm\rlap L=}}}{}%
\@ifundefined{textpeseta}{\def\textpeseta{\hbox{\rm P\negthinspace s}}}{}%
%
\@ifundefined{miss}{\def\miss{\hbox{\vrule height2\p@ width 2\p@ depth\z@}}}{}%
%
\@ifundefined{vvert}{\def\vvert{\Vert}}{}%  %always translated to \left| or \right|
%
\@ifundefined{tcol}{\def\tcol#1{{\baselineskip=6\p@ \vcenter{#1}} \Column}}{}%
%
\@ifundefined{dB}{\def\dB{\hbox{{}}}}{}%        %dummy entry in column 
\@ifundefined{mB}{\def\mB#1{\hbox{$#1$}}}{}%   %column entry
\@ifundefined{nB}{\def\nB#1{\hbox{#1}}}{}%     %column entry (not math)
%
\@ifundefined{note}{\def\note{$^{\dag}}}{}%
%
\def\newfmtname{LaTeX2e}
% No longer load latexsym.  This is now handled by SWP, which uses amsfonts if necessary
%
\ifx\fmtname\newfmtname
  \DeclareOldFontCommand{\rm}{\normalfont\rmfamily}{\mathrm}
  \DeclareOldFontCommand{\sf}{\normalfont\sffamily}{\mathsf}
  \DeclareOldFontCommand{\tt}{\normalfont\ttfamily}{\mathtt}
  \DeclareOldFontCommand{\bf}{\normalfont\bfseries}{\mathbf}
  \DeclareOldFontCommand{\it}{\normalfont\itshape}{\mathit}
  \DeclareOldFontCommand{\sl}{\normalfont\slshape}{\@nomath\sl}
  \DeclareOldFontCommand{\sc}{\normalfont\scshape}{\@nomath\sc}
\fi

%
% Greek bold macros
% Redefine all of the math symbols 
% which might be bolded	 - there are 
% probably others to add to this list

\def\alpha{{\Greekmath 010B}}%
\def\beta{{\Greekmath 010C}}%
\def\gamma{{\Greekmath 010D}}%
\def\delta{{\Greekmath 010E}}%
\def\epsilon{{\Greekmath 010F}}%
\def\zeta{{\Greekmath 0110}}%
\def\eta{{\Greekmath 0111}}%
\def\theta{{\Greekmath 0112}}%
\def\iota{{\Greekmath 0113}}%
\def\kappa{{\Greekmath 0114}}%
\def\lambda{{\Greekmath 0115}}%
\def\mu{{\Greekmath 0116}}%
\def\nu{{\Greekmath 0117}}%
\def\xi{{\Greekmath 0118}}%
\def\pi{{\Greekmath 0119}}%
\def\rho{{\Greekmath 011A}}%
\def\sigma{{\Greekmath 011B}}%
\def\tau{{\Greekmath 011C}}%
\def\upsilon{{\Greekmath 011D}}%
\def\phi{{\Greekmath 011E}}%
\def\chi{{\Greekmath 011F}}%
\def\psi{{\Greekmath 0120}}%
\def\omega{{\Greekmath 0121}}%
\def\varepsilon{{\Greekmath 0122}}%
\def\vartheta{{\Greekmath 0123}}%
\def\varpi{{\Greekmath 0124}}%
\def\varrho{{\Greekmath 0125}}%
\def\varsigma{{\Greekmath 0126}}%
\def\varphi{{\Greekmath 0127}}%

\def\nabla{{\Greekmath 0272}}
\def\FindBoldGroup{%
   {\setbox0=\hbox{$\mathbf{x\global\edef\theboldgroup{\the\mathgroup}}$}}%
}

\def\Greekmath#1#2#3#4{%
    \if@compatibility
        \ifnum\mathgroup=\symbold
           \mathchoice{\mbox{\boldmath$\displaystyle\mathchar"#1#2#3#4$}}%
                      {\mbox{\boldmath$\textstyle\mathchar"#1#2#3#4$}}%
                      {\mbox{\boldmath$\scriptstyle\mathchar"#1#2#3#4$}}%
                      {\mbox{\boldmath$\scriptscriptstyle\mathchar"#1#2#3#4$}}%
        \else
           \mathchar"#1#2#3#4% 
        \fi 
    \else 
        \FindBoldGroup
        \ifnum\mathgroup=\theboldgroup % For 2e
           \mathchoice{\mbox{\boldmath$\displaystyle\mathchar"#1#2#3#4$}}%
                      {\mbox{\boldmath$\textstyle\mathchar"#1#2#3#4$}}%
                      {\mbox{\boldmath$\scriptstyle\mathchar"#1#2#3#4$}}%
                      {\mbox{\boldmath$\scriptscriptstyle\mathchar"#1#2#3#4$}}%
        \else
           \mathchar"#1#2#3#4% 
        \fi     	    
	  \fi}

\newif\ifGreekBold  \GreekBoldfalse
\let\SAVEPBF=\pbf
\def\pbf{\GreekBoldtrue\SAVEPBF}%
%

\@ifundefined{theorem}{\newtheorem{theorem}{Theorem}}{}
\@ifundefined{lemma}{\newtheorem{lemma}[theorem]{Lemma}}{}
\@ifundefined{corollary}{\newtheorem{corollary}[theorem]{Corollary}}{}
\@ifundefined{conjecture}{\newtheorem{conjecture}[theorem]{Conjecture}}{}
\@ifundefined{proposition}{\newtheorem{proposition}[theorem]{Proposition}}{}
\@ifundefined{axiom}{\newtheorem{axiom}{Axiom}}{}
\@ifundefined{remark}{\newtheorem{remark}{Remark}}{}
\@ifundefined{example}{\newtheorem{example}{Example}}{}
\@ifundefined{exercise}{\newtheorem{exercise}{Exercise}}{}
\@ifundefined{definition}{\newtheorem{definition}{Definition}}{}


\@ifundefined{mathletters}{%
  %\def\theequation{\arabic{equation}}
  \newcounter{equationnumber}  
  \def\mathletters{%
     \addtocounter{equation}{1}
     \edef\@currentlabel{\theequation}%
     \setcounter{equationnumber}{\c@equation}
     \setcounter{equation}{0}%
     \edef\theequation{\@currentlabel\noexpand\alph{equation}}%
  }
  \def\endmathletters{%
     \setcounter{equation}{\value{equationnumber}}%
  }
}{}

%Logos
\@ifundefined{BibTeX}{%
    \def\BibTeX{{\rm B\kern-.05em{\sc i\kern-.025em b}\kern-.08em
                 T\kern-.1667em\lower.7ex\hbox{E}\kern-.125emX}}}{}%
\@ifundefined{AmS}%
    {\def\AmS{{\protect\usefont{OMS}{cmsy}{m}{n}%
                A\kern-.1667em\lower.5ex\hbox{M}\kern-.125emS}}}{}%
\@ifundefined{AmSTeX}{\def\AmSTeX{\protect\AmS-\protect\TeX\@}}{}%
%

% This macro is a fix to eqnarray
\def\@@eqncr{\let\@tempa\relax
    \ifcase\@eqcnt \def\@tempa{& & &}\or \def\@tempa{& &}%
      \else \def\@tempa{&}\fi
     \@tempa
     \if@eqnsw
        \iftag@
           \@taggnum
        \else
           \@eqnnum\stepcounter{equation}%
        \fi
     \fi
     \global\tag@false
     \global\@eqnswtrue
     \global\@eqcnt\z@\cr}


\def\TCItag{\@ifnextchar*{\@TCItagstar}{\@TCItag}}
\def\@TCItag#1{%
    \global\tag@true
    \global\def\@taggnum{(#1)}}
\def\@TCItagstar*#1{%
    \global\tag@true
    \global\def\@taggnum{#1}}
%
%%%%%%%%%%%%%%%%%%%%%%%%%%%%%%%%%%%%%%%%%%%%%%%%%%%%%%%%%%%%%%%%%%%%%
%
\def\QATOP#1#2{{#1 \atop #2}}%
\def\QTATOP#1#2{{\textstyle {#1 \atop #2}}}%
\def\QDATOP#1#2{{\displaystyle {#1 \atop #2}}}%
\def\QABOVE#1#2#3{{#2 \above#1 #3}}%
\def\QTABOVE#1#2#3{{\textstyle {#2 \above#1 #3}}}%
\def\QDABOVE#1#2#3{{\displaystyle {#2 \above#1 #3}}}%
\def\QOVERD#1#2#3#4{{#3 \overwithdelims#1#2 #4}}%
\def\QTOVERD#1#2#3#4{{\textstyle {#3 \overwithdelims#1#2 #4}}}%
\def\QDOVERD#1#2#3#4{{\displaystyle {#3 \overwithdelims#1#2 #4}}}%
\def\QATOPD#1#2#3#4{{#3 \atopwithdelims#1#2 #4}}%
\def\QTATOPD#1#2#3#4{{\textstyle {#3 \atopwithdelims#1#2 #4}}}%
\def\QDATOPD#1#2#3#4{{\displaystyle {#3 \atopwithdelims#1#2 #4}}}%
\def\QABOVED#1#2#3#4#5{{#4 \abovewithdelims#1#2#3 #5}}%
\def\QTABOVED#1#2#3#4#5{{\textstyle 
   {#4 \abovewithdelims#1#2#3 #5}}}%
\def\QDABOVED#1#2#3#4#5{{\displaystyle 
   {#4 \abovewithdelims#1#2#3 #5}}}%
%
% Macros for text size operators:
%
\def\tint{\mathop{\textstyle \int}}%
\def\tiint{\mathop{\textstyle \iint }}%
\def\tiiint{\mathop{\textstyle \iiint }}%
\def\tiiiint{\mathop{\textstyle \iiiint }}%
\def\tidotsint{\mathop{\textstyle \idotsint }}%
\def\toint{\mathop{\textstyle \oint}}%
\def\tsum{\mathop{\textstyle \sum }}%
\def\tprod{\mathop{\textstyle \prod }}%
\def\tbigcap{\mathop{\textstyle \bigcap }}%
\def\tbigwedge{\mathop{\textstyle \bigwedge }}%
\def\tbigoplus{\mathop{\textstyle \bigoplus }}%
\def\tbigodot{\mathop{\textstyle \bigodot }}%
\def\tbigsqcup{\mathop{\textstyle \bigsqcup }}%
\def\tcoprod{\mathop{\textstyle \coprod }}%
\def\tbigcup{\mathop{\textstyle \bigcup }}%
\def\tbigvee{\mathop{\textstyle \bigvee }}%
\def\tbigotimes{\mathop{\textstyle \bigotimes }}%
\def\tbiguplus{\mathop{\textstyle \biguplus }}%
%
%
%Macros for display size operators:
%
\def\dint{\mathop{\displaystyle \int}}%
\def\diint{\mathop{\displaystyle \iint}}%
\def\diiint{\mathop{\displaystyle \iiint}}%
\def\diiiint{\mathop{\displaystyle \iiiint }}%
\def\didotsint{\mathop{\displaystyle \idotsint }}%
\def\doint{\mathop{\displaystyle \oint}}%
\def\dsum{\mathop{\displaystyle \sum }}%
\def\dprod{\mathop{\displaystyle \prod }}%
\def\dbigcap{\mathop{\displaystyle \bigcap }}%
\def\dbigwedge{\mathop{\displaystyle \bigwedge }}%
\def\dbigoplus{\mathop{\displaystyle \bigoplus }}%
\def\dbigodot{\mathop{\displaystyle \bigodot }}%
\def\dbigsqcup{\mathop{\displaystyle \bigsqcup }}%
\def\dcoprod{\mathop{\displaystyle \coprod }}%
\def\dbigcup{\mathop{\displaystyle \bigcup }}%
\def\dbigvee{\mathop{\displaystyle \bigvee }}%
\def\dbigotimes{\mathop{\displaystyle \bigotimes }}%
\def\dbiguplus{\mathop{\displaystyle \biguplus }}%

\if@compatibility\else
  % Always load amsmath in LaTeX2e mode
  \RequirePackage{amsmath}
\fi

\def\ExitTCILatex{\makeatother\endinput}

\bgroup
\ifx\ds@amstex\relax
   \message{amstex already loaded}\aftergroup\ExitTCILatex
\else
   \@ifpackageloaded{amsmath}%
      {\if@compatibility\message{amsmath already loaded}\fi\aftergroup\ExitTCILatex}
      {}
   \@ifpackageloaded{amstex}%
      {\if@compatibility\message{amstex already loaded}\fi\aftergroup\ExitTCILatex}
      {}
   \@ifpackageloaded{amsgen}%
      {\if@compatibility\message{amsgen already loaded}\fi\aftergroup\ExitTCILatex}
      {}
\fi
\egroup

%Exit if any of the AMS macros are already loaded.
%This is always the case for LaTeX2e mode.


%%%%%%%%%%%%%%%%%%%%%%%%%%%%%%%%%%%%%%%%%%%%%%%%%%%%%%%%%%%%%%%%%%%%%%%%%%
% NOTE: The rest of this file is read only if in LaTeX 2.09 compatibility
% mode. This section is used to define AMS-like constructs in the
% event they have not been defined.
%%%%%%%%%%%%%%%%%%%%%%%%%%%%%%%%%%%%%%%%%%%%%%%%%%%%%%%%%%%%%%%%%%%%%%%%%%
\typeout{TCILATEX defining AMS-like constructs in LaTeX 2.09 COMPATIBILITY MODE}
%%%%%%%%%%%%%%%%%%%%%%%%%%%%%%%%%%%%%%%%%%%%%%%%%%%%%%%%%%%%%%%%%%%%%%%%
%  Macros to define some AMS LaTeX constructs when 
%  AMS LaTeX has not been loaded
% 
% These macros are copied from the AMS-TeX package for doing
% multiple integrals.
%
\let\DOTSI\relax
\def\RIfM@{\relax\ifmmode}%
\def\FN@{\futurelet\next}%
\newcount\intno@
\def\iint{\DOTSI\intno@\tw@\FN@\ints@}%
\def\iiint{\DOTSI\intno@\thr@@\FN@\ints@}%
\def\iiiint{\DOTSI\intno@4 \FN@\ints@}%
\def\idotsint{\DOTSI\intno@\z@\FN@\ints@}%
\def\ints@{\findlimits@\ints@@}%
\newif\iflimtoken@
\newif\iflimits@
\def\findlimits@{\limtoken@true\ifx\next\limits\limits@true
 \else\ifx\next\nolimits\limits@false\else
 \limtoken@false\ifx\ilimits@\nolimits\limits@false\else
 \ifinner\limits@false\else\limits@true\fi\fi\fi\fi}%
\def\multint@{\int\ifnum\intno@=\z@\intdots@                          %1
 \else\intkern@\fi                                                    %2
 \ifnum\intno@>\tw@\int\intkern@\fi                                   %3
 \ifnum\intno@>\thr@@\int\intkern@\fi                                 %4
 \int}%                                                               %5
\def\multintlimits@{\intop\ifnum\intno@=\z@\intdots@\else\intkern@\fi
 \ifnum\intno@>\tw@\intop\intkern@\fi
 \ifnum\intno@>\thr@@\intop\intkern@\fi\intop}%
\def\intic@{%
    \mathchoice{\hskip.5em}{\hskip.4em}{\hskip.4em}{\hskip.4em}}%
\def\negintic@{\mathchoice
 {\hskip-.5em}{\hskip-.4em}{\hskip-.4em}{\hskip-.4em}}%
\def\ints@@{\iflimtoken@                                              %1
 \def\ints@@@{\iflimits@\negintic@
   \mathop{\intic@\multintlimits@}\limits                             %2
  \else\multint@\nolimits\fi                                          %3
  \eat@}%                                                             %4
 \else                                                                %5
 \def\ints@@@{\iflimits@\negintic@
  \mathop{\intic@\multintlimits@}\limits\else
  \multint@\nolimits\fi}\fi\ints@@@}%
\def\intkern@{\mathchoice{\!\!\!}{\!\!}{\!\!}{\!\!}}%
\def\plaincdots@{\mathinner{\cdotp\cdotp\cdotp}}%
\def\intdots@{\mathchoice{\plaincdots@}%
 {{\cdotp}\mkern1.5mu{\cdotp}\mkern1.5mu{\cdotp}}%
 {{\cdotp}\mkern1mu{\cdotp}\mkern1mu{\cdotp}}%
 {{\cdotp}\mkern1mu{\cdotp}\mkern1mu{\cdotp}}}%
%
%
%  These macros are for doing the AMS \text{} construct
%
\def\RIfM@{\relax\protect\ifmmode}
\def\text{\RIfM@\expandafter\text@\else\expandafter\mbox\fi}
\let\nfss@text\text
\def\text@#1{\mathchoice
   {\textdef@\displaystyle\f@size{#1}}%
   {\textdef@\textstyle\tf@size{\firstchoice@false #1}}%
   {\textdef@\textstyle\sf@size{\firstchoice@false #1}}%
   {\textdef@\textstyle \ssf@size{\firstchoice@false #1}}%
   \glb@settings}

\def\textdef@#1#2#3{\hbox{{%
                    \everymath{#1}%
                    \let\f@size#2\selectfont
                    #3}}}
\newif\iffirstchoice@
\firstchoice@true
%
%These are the AMS constructs for multiline limits.
%
\def\Let@{\relax\iffalse{\fi\let\\=\cr\iffalse}\fi}%
\def\vspace@{\def\vspace##1{\crcr\noalign{\vskip##1\relax}}}%
\def\multilimits@{\bgroup\vspace@\Let@
 \baselineskip\fontdimen10 \scriptfont\tw@
 \advance\baselineskip\fontdimen12 \scriptfont\tw@
 \lineskip\thr@@\fontdimen8 \scriptfont\thr@@
 \lineskiplimit\lineskip
 \vbox\bgroup\ialign\bgroup\hfil$\m@th\scriptstyle{##}$\hfil\crcr}%
\def\Sb{_\multilimits@}%
\def\endSb{\crcr\egroup\egroup\egroup}%
\def\Sp{^\multilimits@}%
\let\endSp\endSb
%
%
%These are AMS constructs for horizontal arrows
%
\newdimen\ex@
\ex@.2326ex
\def\rightarrowfill@#1{$#1\m@th\mathord-\mkern-6mu\cleaders
 \hbox{$#1\mkern-2mu\mathord-\mkern-2mu$}\hfill
 \mkern-6mu\mathord\rightarrow$}%
\def\leftarrowfill@#1{$#1\m@th\mathord\leftarrow\mkern-6mu\cleaders
 \hbox{$#1\mkern-2mu\mathord-\mkern-2mu$}\hfill\mkern-6mu\mathord-$}%
\def\leftrightarrowfill@#1{$#1\m@th\mathord\leftarrow
\mkern-6mu\cleaders
 \hbox{$#1\mkern-2mu\mathord-\mkern-2mu$}\hfill
 \mkern-6mu\mathord\rightarrow$}%
\def\overrightarrow{\mathpalette\overrightarrow@}%
\def\overrightarrow@#1#2{\vbox{\ialign{##\crcr\rightarrowfill@#1\crcr
 \noalign{\kern-\ex@\nointerlineskip}$\m@th\hfil#1#2\hfil$\crcr}}}%
\let\overarrow\overrightarrow
\def\overleftarrow{\mathpalette\overleftarrow@}%
\def\overleftarrow@#1#2{\vbox{\ialign{##\crcr\leftarrowfill@#1\crcr
 \noalign{\kern-\ex@\nointerlineskip}$\m@th\hfil#1#2\hfil$\crcr}}}%
\def\overleftrightarrow{\mathpalette\overleftrightarrow@}%
\def\overleftrightarrow@#1#2{\vbox{\ialign{##\crcr
   \leftrightarrowfill@#1\crcr
 \noalign{\kern-\ex@\nointerlineskip}$\m@th\hfil#1#2\hfil$\crcr}}}%
\def\underrightarrow{\mathpalette\underrightarrow@}%
\def\underrightarrow@#1#2{\vtop{\ialign{##\crcr$\m@th\hfil#1#2\hfil
  $\crcr\noalign{\nointerlineskip}\rightarrowfill@#1\crcr}}}%
\let\underarrow\underrightarrow
\def\underleftarrow{\mathpalette\underleftarrow@}%
\def\underleftarrow@#1#2{\vtop{\ialign{##\crcr$\m@th\hfil#1#2\hfil
  $\crcr\noalign{\nointerlineskip}\leftarrowfill@#1\crcr}}}%
\def\underleftrightarrow{\mathpalette\underleftrightarrow@}%
\def\underleftrightarrow@#1#2{\vtop{\ialign{##\crcr$\m@th
  \hfil#1#2\hfil$\crcr
 \noalign{\nointerlineskip}\leftrightarrowfill@#1\crcr}}}%
%%%%%%%%%%%%%%%%%%%%%

\def\qopnamewl@#1{\mathop{\operator@font#1}\nlimits@}
\let\nlimits@\displaylimits
\def\setboxz@h{\setbox\z@\hbox}


\def\varlim@#1#2{\mathop{\vtop{\ialign{##\crcr
 \hfil$#1\m@th\operator@font lim$\hfil\crcr
 \noalign{\nointerlineskip}#2#1\crcr
 \noalign{\nointerlineskip\kern-\ex@}\crcr}}}}

 \def\rightarrowfill@#1{\m@th\setboxz@h{$#1-$}\ht\z@\z@
  $#1\copy\z@\mkern-6mu\cleaders
  \hbox{$#1\mkern-2mu\box\z@\mkern-2mu$}\hfill
  \mkern-6mu\mathord\rightarrow$}
\def\leftarrowfill@#1{\m@th\setboxz@h{$#1-$}\ht\z@\z@
  $#1\mathord\leftarrow\mkern-6mu\cleaders
  \hbox{$#1\mkern-2mu\copy\z@\mkern-2mu$}\hfill
  \mkern-6mu\box\z@$}


\def\projlim{\qopnamewl@{proj\,lim}}
\def\injlim{\qopnamewl@{inj\,lim}}
\def\varinjlim{\mathpalette\varlim@\rightarrowfill@}
\def\varprojlim{\mathpalette\varlim@\leftarrowfill@}
\def\varliminf{\mathpalette\varliminf@{}}
\def\varliminf@#1{\mathop{\underline{\vrule\@depth.2\ex@\@width\z@
   \hbox{$#1\m@th\operator@font lim$}}}}
\def\varlimsup{\mathpalette\varlimsup@{}}
\def\varlimsup@#1{\mathop{\overline
  {\hbox{$#1\m@th\operator@font lim$}}}}

%
%Companion to stackrel
\def\stackunder#1#2{\mathrel{\mathop{#2}\limits_{#1}}}%
%
%
% These are AMS environments that will be defined to
% be verbatims if amstex has not actually been 
% loaded
%
%
\begingroup \catcode `|=0 \catcode `[= 1
\catcode`]=2 \catcode `\{=12 \catcode `\}=12
\catcode`\\=12 
|gdef|@alignverbatim#1\end{align}[#1|end[align]]
|gdef|@salignverbatim#1\end{align*}[#1|end[align*]]

|gdef|@alignatverbatim#1\end{alignat}[#1|end[alignat]]
|gdef|@salignatverbatim#1\end{alignat*}[#1|end[alignat*]]

|gdef|@xalignatverbatim#1\end{xalignat}[#1|end[xalignat]]
|gdef|@sxalignatverbatim#1\end{xalignat*}[#1|end[xalignat*]]

|gdef|@gatherverbatim#1\end{gather}[#1|end[gather]]
|gdef|@sgatherverbatim#1\end{gather*}[#1|end[gather*]]

|gdef|@gatherverbatim#1\end{gather}[#1|end[gather]]
|gdef|@sgatherverbatim#1\end{gather*}[#1|end[gather*]]


|gdef|@multilineverbatim#1\end{multiline}[#1|end[multiline]]
|gdef|@smultilineverbatim#1\end{multiline*}[#1|end[multiline*]]

|gdef|@arraxverbatim#1\end{arrax}[#1|end[arrax]]
|gdef|@sarraxverbatim#1\end{arrax*}[#1|end[arrax*]]

|gdef|@tabulaxverbatim#1\end{tabulax}[#1|end[tabulax]]
|gdef|@stabulaxverbatim#1\end{tabulax*}[#1|end[tabulax*]]


|endgroup
  

  
\def\align{\@verbatim \frenchspacing\@vobeyspaces \@alignverbatim
You are using the "align" environment in a style in which it is not defined.}
\let\endalign=\endtrivlist
 
\@namedef{align*}{\@verbatim\@salignverbatim
You are using the "align*" environment in a style in which it is not defined.}
\expandafter\let\csname endalign*\endcsname =\endtrivlist




\def\alignat{\@verbatim \frenchspacing\@vobeyspaces \@alignatverbatim
You are using the "alignat" environment in a style in which it is not defined.}
\let\endalignat=\endtrivlist
 
\@namedef{alignat*}{\@verbatim\@salignatverbatim
You are using the "alignat*" environment in a style in which it is not defined.}
\expandafter\let\csname endalignat*\endcsname =\endtrivlist




\def\xalignat{\@verbatim \frenchspacing\@vobeyspaces \@xalignatverbatim
You are using the "xalignat" environment in a style in which it is not defined.}
\let\endxalignat=\endtrivlist
 
\@namedef{xalignat*}{\@verbatim\@sxalignatverbatim
You are using the "xalignat*" environment in a style in which it is not defined.}
\expandafter\let\csname endxalignat*\endcsname =\endtrivlist




\def\gather{\@verbatim \frenchspacing\@vobeyspaces \@gatherverbatim
You are using the "gather" environment in a style in which it is not defined.}
\let\endgather=\endtrivlist
 
\@namedef{gather*}{\@verbatim\@sgatherverbatim
You are using the "gather*" environment in a style in which it is not defined.}
\expandafter\let\csname endgather*\endcsname =\endtrivlist


\def\multiline{\@verbatim \frenchspacing\@vobeyspaces \@multilineverbatim
You are using the "multiline" environment in a style in which it is not defined.}
\let\endmultiline=\endtrivlist
 
\@namedef{multiline*}{\@verbatim\@smultilineverbatim
You are using the "multiline*" environment in a style in which it is not defined.}
\expandafter\let\csname endmultiline*\endcsname =\endtrivlist


\def\arrax{\@verbatim \frenchspacing\@vobeyspaces \@arraxverbatim
You are using a type of "array" construct that is only allowed in AmS-LaTeX.}
\let\endarrax=\endtrivlist

\def\tabulax{\@verbatim \frenchspacing\@vobeyspaces \@tabulaxverbatim
You are using a type of "tabular" construct that is only allowed in AmS-LaTeX.}
\let\endtabulax=\endtrivlist

 
\@namedef{arrax*}{\@verbatim\@sarraxverbatim
You are using a type of "array*" construct that is only allowed in AmS-LaTeX.}
\expandafter\let\csname endarrax*\endcsname =\endtrivlist

\@namedef{tabulax*}{\@verbatim\@stabulaxverbatim
You are using a type of "tabular*" construct that is only allowed in AmS-LaTeX.}
\expandafter\let\csname endtabulax*\endcsname =\endtrivlist

% macro to simulate ams tag construct


% This macro is a fix to the equation environment
 \def\endequation{%
     \ifmmode\ifinner % FLEQN hack
      \iftag@
        \addtocounter{equation}{-1} % undo the increment made in the begin part
        $\hfil
           \displaywidth\linewidth\@taggnum\egroup \endtrivlist
        \global\tag@false
        \global\@ignoretrue   
      \else
        $\hfil
           \displaywidth\linewidth\@eqnnum\egroup \endtrivlist
        \global\tag@false
        \global\@ignoretrue 
      \fi
     \else   
      \iftag@
        \addtocounter{equation}{-1} % undo the increment made in the begin part
        \eqno \hbox{\@taggnum}
        \global\tag@false%
        $$\global\@ignoretrue
      \else
        \eqno \hbox{\@eqnnum}% $$ BRACE MATCHING HACK
        $$\global\@ignoretrue
      \fi
     \fi\fi
 } 

 \newif\iftag@ \tag@false
 
 \def\TCItag{\@ifnextchar*{\@TCItagstar}{\@TCItag}}
 \def\@TCItag#1{%
     \global\tag@true
     \global\def\@taggnum{(#1)}}
 \def\@TCItagstar*#1{%
     \global\tag@true
     \global\def\@taggnum{#1}}

  \@ifundefined{tag}{
     \def\tag{\@ifnextchar*{\@tagstar}{\@tag}}
     \def\@tag#1{%
         \global\tag@true
         \global\def\@taggnum{(#1)}}
     \def\@tagstar*#1{%
         \global\tag@true
         \global\def\@taggnum{#1}}
  }{}

\def\tfrac#1#2{{\textstyle {#1 \over #2}}}%
\def\dfrac#1#2{{\displaystyle {#1 \over #2}}}%
\def\binom#1#2{{#1 \choose #2}}%
\def\tbinom#1#2{{\textstyle {#1 \choose #2}}}%
\def\dbinom#1#2{{\displaystyle {#1 \choose #2}}}%

% Do not add anything to the end of this file.  
% The last section of the file is loaded only if 
% amstex has not been.
\makeatother
\endinput

\font \aap cmmi10
\providecommand{\at}{\mathbin{\mbox{\aap ,}}}
\providecommand{\leftmerge}{\mathbin{\mathrel{\llfloor}}}


\begin{document}

\title{Process Library Implementation Notes}
\author{Wieger Wesselink}
\maketitle

\section{Process Library Implementation Notes}

\subsection{Processes}

Process expressions in mCRL2 are expressions built according to the
following syntax:%
\[
\begin{array}{ccc}
\text{expression} & \text{C++ equivalent} & \text{ATerm grammar} \\
a(e) & \text{action(}a\text{,}e\text{)} & \text{Action} \\
P(e) & \text{process(}P\text{,}e\text{)} & \text{Process} \\
P(d:=e) & \text{process\_assignment(}P\text{,}d:=e\text{)} & \text{%
ProcessAssignment} \\
\delta & \text{delta()} & \text{Delta} \\
\tau & \text{tau()} & \text{Tau} \\
\dsum\limits_{d}x & \text{sum(}d\text{,}x\text{)} & \text{Sum} \\
\partial _{B}(x) & \text{block(}B\text{,}x\text{)} & \text{Block} \\
\tau _{B}(x) & \text{hide(}B\text{,}x\text{)} & \text{Hide} \\
\rho _{R}(x) & \text{rename(}R\text{,}x\text{)} & \text{Rename} \\
\Gamma _{C}(x) & \text{comm(}C\text{,}x\text{)} & \text{Comm} \\
\bigtriangledown _{V}(x) & \text{allow(}V\text{,}x\text{)} & \text{Allow} \\
x\mid y & \text{sync(}x\text{,}y\text{)} & \text{Sync} \\
x\at t & \text{at\_time(}x\text{,}t\text{)} & \text{AtTime} \\
x\cdot y & \text{seq(}x\text{,}y\text{)} & \text{Seq} \\
c\rightarrow x & \text{if\_then(}c\text{,}x\text{)} & \text{IfThen} \\
c\rightarrow x\diamond y & \text{if\_then\_else(}c\text{,}x\text{,}y\text{)}
& \text{IfThenElse} \\
x\ll y & \text{binit(}x\text{,}y\text{)} & \text{BInit} \\
x\ \parallel \ y & \text{merge(}x\text{,}y\text{)} & \text{Merge} \\
x\ \leftmerge \ y & \text{lmerge(}x\text{,}y\text{)} & \text{LMerge} \\
x+y & \text{choice(}x\text{,}y\text{)} & \text{Choice}%
\end{array}%
\]

where the types of the symbols are as follows:%
\[
\begin{array}{cl}
a,b & \text{strings (action names)} \\
P & \text{a process identifier} \\
e & \text{a sequence of data expressions} \\
d & \text{a sequence of data variables} \\
B & \text{a set of strings (action names) } \\
R & \text{a sequence of rename expressions} \\
C & \text{a sequence of communication expressions} \\
V & \text{a sequence of multi actions} \\
t & \text{a data expression of type real} \\
x,y & \text{process expressions} \\
c & \text{ a data expression of type bool}%
\end{array}%
\]%
A rename expression is of the form $a\rightarrow b$, with $a$ and $b$ action
names. A multi action is of the form $a_{1}\ |\ \cdots \ |\ a_{n}$, with $%
a_{i}$ actions. A communication expression is of the form $b_{1}\ |\ \cdots
\ |\ b_{n}\rightarrow b$, with $b$ and $b_{i}$ action names.

\subsubsection{Restrictions}

A multi action is a multi set of actions. The left hand sides of the
communication expressions in $C$ must be unique. Also the left hand sides of
the rename expressions in $R$ must be unique.

\newpage

\subsubsection{Linear process expressions}

Linear process expressions are a subset of process expresions satisfying the
following grammar:
\begin{verbatim}
<linear process expression> ::= choice(<linear process expression>, <linear process expression>)
                              | <summand>

<summand>                   ::= sum(<variables>, <alternative>)
                              | <conditional action prefix>
                              | <conditional deadlock>

<conditional action prefix> ::= if_then(<condition>, <action prefix>)
                              | <action prefix>

<action prefix>             ::= seq(<timed multiaction>, <process reference>)
                              | <timed multiaction>

<timed multiaction>         ::= at_time(<multiaction>, <time stamp>)
                              | <multiaction>

<multiaction>               ::= tau()
                              | <action>
                              | sync(<multiaction>, <multiaction>)

<conditional deadlock>      ::= if_then(<condition>, <timed deadlock>)
                              | <timed deadlock>

<timed deadlock>            ::= delta()
                              | at_time(delta(), <time stamp>)

<process reference>         ::= process(<process identifier>, <data expressions>)
                              | process_assignment(<process identifier>, <data assignments>)
\end{verbatim}

\newpage

\subsection{Guarded process expressions}

We define the predicate $is\_guarded$ for process expressions as follows: $%
is\_guarded(p)=is\_guarded(p,\emptyset )$%
\[
\begin{array}{lll}
is\_guarded(a(e),W) & = & true \\
is\_guarded(\delta ,W) & = & true \\
is\_guarded(\tau ,W) & = & true \\
is\_guarded(P(e),W) & = &
\begin{array}{l}
\left\{
\begin{array}{ll}
false & \text{if }P\in W \\
is\_guarded(p,W\cup \{P\}) & \text{if }P\notin W%
\end{array}%
\right. \\
\text{where }P(d)=p\text{ is the equation corresponding to }P(e)%
\end{array}
\\
is\_guarded(p+q,W) & = & is\_guarded(p,W)\wedge is\_guarded(q,W) \\
is\_guarded(p\cdot q,W) & = & is\_guarded(p,W) \\
is\_guarded(c\rightarrow p,W) & = & is\_guarded(p,W) \\
is\_guarded(c\rightarrow p\diamond q,W) & = & is\_guarded(p,W)\wedge
is\_guarded(q,W) \\
is\_guarded(\Sigma _{d:D}\ p,W) & = & is\_guarded(p,W) \\
is\_guarded(p\at t,W) & = & is\_guarded(p,W) \\
is\_guarded(p\ll q,W) & = & is\_guarded(p,W) \\
is\_guarded(p\parallel q,W) & = & is\_guarded(p,W)\wedge is\_guarded(q,W) \\
is\_guarded(p\leftmerge q,W) & = & is\_guarded(p,W) \\
is\_guarded(p\mid q,W) & = & is\_guarded(p,W)\wedge is\_guarded(q,W) \\
is\_guarded(\rho _{R}(p),W) & = & is\_guarded(p,W) \\
is\_guarded(\partial _{B}(p),W) & = & is\_guarded(p,W) \\
is\_guarded(\tau _{I}(p),W) & = & is\_guarded(p,W) \\
is\_guarded(\Gamma _{C}(p),W) & = & is\_guarded(p,W) \\
is\_guarded(\nabla _{V}(p),W) & = & is\_guarded(p,W)%
\end{array}%
\]%
\newline
N.B. This specification assumes that process names are unique. In mCRL2
process names can be overloaded, therefore in the implemenation $W$ contains
\emph{process identifiers} (i.e. both the process name and the sorts of the
arguments) instead of process names.

\newpage

\subsection{Alphabet reduction}

Alphabet reduction is a preprocessing step for linearization. It is a
transformation on process expressions that preserves branching bisimulation.

\subsubsection{Notations}

In this text action names are represented using $a,b,\ldots $ and multi
action names using $\alpha ,\beta ,\ldots $ So in general we have $\alpha
=a_{1}\mid \ldots \mid a_{n}$. In alphabet reduction data parameters play a
minor role, therefore we choose a notation in which data parameters are
omitted. We use the abbreviation $\overline{a}=a(e_{1},\ldots ,e_{n})$ to
denote an action, and $\overline{\alpha }=\overline{a_{1}}\mid \ldots \mid
\overline{a_{n}}$ to denote a multi action, where $e_{1},\ldots ,e_{n}$ are
data expressions.Note that a multi action is a multiset (or bag) of actions
and a multi action name is a multiset of names. We write $\alpha \beta $ as
shorthand for $\alpha \cup \beta $ and $a\beta $ for $\{a\}\cup \beta $.
Sets of multi action names are represented using $A,A_{1},A_{2},\ldots $ A
communication $C$ maps multi action names to action names, and is denoted as
$\{\alpha _{1}\rightarrow a_{1},\ldots ,\alpha _{n}\rightarrow a_{n}\}$. A
renaming $R$ is a substitution on action names, and is denoted as $%
R=\{a_{1}\rightarrow b_{1},\ldots ,a_{n}\rightarrow b_{n}\}$. A block set $B$
is a set of action names. A hide set $I$ is a set of action names.

\subsubsection{Definitions}

We define multi actions $\overline{\alpha }$ using the following grammar:%
\[
\overline{\alpha }:=\overline{a}\shortmid \overline{\alpha }\mid \overline{a}%
,
\]%
where $\overline{a}$ is an action, and where $\shortmid $ is used to
distinguish alternatives.

We define pCRL terms $p$ using the following grammar:%
\[
p::=\overline{a}\shortmid P\shortmid \delta \shortmid \tau \shortmid
p+p\shortmid p\cdot p\shortmid c\rightarrow p\shortmid c\rightarrow
p\diamond p\shortmid \Sigma _{d:D}p\shortmid p\at t\shortmid p\ll p,
\]%
and parallel mCRL terms $q$ using the following grammar:%
\[
q::=p\shortmid q\parallel q\shortmid q\leftmerge q\shortmid q\mid q\shortmid
\rho _{R}(q)\shortmid \partial _{B}(q)\shortmid \tau _{I}(q)\shortmid \Gamma
_{C}(q)\shortmid \nabla _{V}(q).
\]

\paragraph{Remark 1}

Note that there is an unfortunate overload of the $\boldsymbol{\mid }$%
-operator in both multi actions and process expressions. This has
consequences for the implementation, since it there is no clean distinction
between parallel and non-parallel operators.

\paragraph{Remark 2}

The mCRL2 language also has a construct $P(d_{i_{1}}=e_{i_{1}},\ldots
,d_{i_{k}}=e_{i_{k}})$, but this is just a shorthand notation. Therefore we
will ignore it in this text.

\subsubsection{Alphabet operations}

Let $A,A_{1}$ and $A_{2}$ be sets of multi action names. Then we define%
\[
\begin{array}{lll}
A^{\subseteq } & = & \{\alpha \mid \exists \beta .\alpha \beta \in A\} \\
A_{1}A_{2} & = & \{\alpha \beta \mid \alpha \in A_{1}\text{ and }\beta \in
A_{2}\} \\
A_{1}\leftarrowtail A_{2} & = & \{\alpha \mid \exists \beta .\alpha \beta
\in A_{1}\text{ and }\beta \in A_{2}\}%
\end{array}%
\]%
Note that $\beta $ can take the value $\tau $ in the definition of $%
A_{1}\leftarrowtail A_{2}$, which implies $A_{1}\subset A_{1}\leftarrowtail
A_{2}$. The set $A^{\subseteq }$ has an exponential size, so whenever
possible it should not be computed explicitly.

Let $C$ be a communication set, then we define%
\[
\begin{array}{lll}
C(A) & = & \cup _{\alpha \in A}\text{\textsc{Comm}(}C\text{, }\alpha \text{)}
\\
C^{-1}(A) & = & \cup _{\alpha \in A}\text{\textsc{CommInverse}(}C\text{, }%
\alpha \text{)} \\
filter_{\nabla }(C,A) & = & \{\gamma \rightarrow c\in C\mid \exists _{\alpha
\in A}.\gamma \subset \alpha \}%
\end{array}%
\]%
where \textsc{Comm} and \textsc{CommInverse} are defined using pseudo code
as follows:%
\[
\begin{array}{l}
\text{\textsc{Comm}(}C\text{, }\alpha \text{)} \\
R:=\{\alpha \} \\
\text{\textbf{for }}\gamma \rightarrow c\in C\text{ \textbf{do}} \\
\qquad \text{\textbf{if }}\exists \beta .\alpha =\beta \gamma \text{ \textbf{%
then} }R:=R\cup \text{\textsc{Comm}(}C\text{, }\beta c\text{)} \\
\text{\textbf{return }}R%
\end{array}%
\]%
\[
\begin{array}{l}
\text{\textsc{CommInverse}(}C\text{, }\alpha _{1}\text{,}\alpha _{2}\text{)}
\\
R:=\{\alpha _{1}\alpha _{2}\} \\
\text{\textbf{for }}\gamma \rightarrow c\in C\text{ \textbf{do}} \\
\qquad \text{\textbf{if }}\exists \beta .\alpha _{1}=\beta c\text{ \textbf{%
then} }R:=R\cup \text{\textsc{CommInverse}(}C\text{, }\beta \text{,}\alpha
_{2}\gamma \text{)} \\
\text{\textbf{return }}R%
\end{array}%
\]%
Note that $C^{-1}(\alpha )=$\textsc{CommInverse}($C$, $\alpha $,$\tau $).

Let $R$ be a rename set, then we define%
\[
\begin{array}{lll}
R(\alpha ) & = & \{R(\alpha _{i})\mid \alpha _{i}\in \alpha \} \\
R^{-1}(\alpha ) & = & \{\beta \mid R(\beta )=\alpha \} \\
R(A) & = & \{R(\alpha )\mid \alpha \in A\} \\
R^{-1}(A) & = & \{R^{-1}(\alpha )\mid \alpha \in A\}%
\end{array}%
\]%
Let $I$ be a hide set, then we define%
\[
\begin{array}{ccc}
\tau _{I}(A) & = & \{\beta \mid \exists _{\alpha \in A,\gamma \in I^{\ast
}}.\alpha =\beta \gamma \wedge \beta \cap I=\emptyset \} \\
\tau _{I}^{-1}(A) & = & \partial _{I}(A)I^{\ast }%
\end{array}%
\]%
Let $B$ be a block set, then we define%
\[
\partial _{B}(A)=\{\alpha \in A\mid \alpha \cap B=\emptyset \}
\]%
We define a mapping $act$ that extracts the individual action names of a set
of multi action names:%
\[
\begin{array}{lll}
act\left( a_{1}\mid \ldots \mid a_{n}\right) & = & \left\{ a_{1}\mid \ldots
\mid a_{n}\right\} \\
act\left( A\right) & = & \bigcup_{\alpha \in A}act\left( \alpha \right)%
\end{array}%
\]

\subsubsection{The mapping $\protect\alpha $}

We define the mapping $\alpha $ as follows. The value $\alpha (p,\emptyset )$
is an over approximation of the alphabet of process expression $p$.%
\[
\begin{array}{lll}
\alpha (\overline{a},W) & = & \{a\} \\
\alpha (P,W) & = &
\begin{array}{l}
\left\{
\begin{array}{ll}
\emptyset  & \text{if }P\in W \\
\alpha (p,W\cup \{P\}) & \text{if }P\notin W,%
\end{array}%
\right.  \\
\text{ where }P=p\text{ is the equation of }P%
\end{array}
\\
\alpha (\delta ,W) & = & \emptyset  \\
\alpha (\tau ,W) & = & \{\tau \} \\
\alpha (p+q,W) & = & \alpha (p,W)\cup \alpha (q,W) \\
\alpha (p\cdot q,W) & = & \alpha (p,W)\cup \alpha (q,W) \\
\alpha (c\rightarrow p,W) & = & \alpha (p,W) \\
\alpha (c\rightarrow p\diamond q,W) & = & \alpha (p,W)\cup \alpha (q,W) \\
\alpha (\Sigma _{d:D}p,W) & = & \alpha (p,W) \\
\alpha (p\at t,W) & = & \alpha (p,W) \\
\alpha (p\ll q,W) & = & \alpha (p,W)\cup \alpha (q,W) \\
\alpha (p\parallel q,W) & = & \alpha (p,W)\cup \alpha (q,W)\cup \alpha
(p,W)\alpha (q,W) \\
\alpha (p\leftmerge q,W) & = & \alpha (p,W)\cup \alpha (q,W)\cup \alpha
(p,W)\alpha (q,W) \\
\alpha (p\mid q,W) & = & \alpha (p,W)\alpha (q,W) \\
\alpha (\rho _{R}(p),W) & = & R(\alpha (p,W)) \\
\alpha (\partial _{B}(p),W) & = & \partial _{B}(\alpha (p,W)) \\
\alpha (\tau _{I}(p),W) & = & \tau _{I}(\alpha (p,W)) \\
\alpha (\Gamma _{C}(p),W) & = & C(\alpha (p,W)) \\
\alpha (\nabla _{V}(p),W) & = & \alpha (p,W)\cap (V\cup \{\tau \})%
\end{array}%
\]%
Example 1

If $C=\{a\mid b\rightarrow c\}$, then $\alpha (\Gamma _{C}(a(1)\mid
b(2)))=\{a,b,c,a\mid b\}$. Note that the action $c$ does not occur in the
transition system of this process expression.

\paragraph{Example 2}

In the computation of $\left\{ a_{1},a_{2},\ldots ,a_{20}\right\} \cap
\alpha \left( a_{1}\parallel a_{2}\parallel \ldots \parallel a_{20}\right) $
the above mentioned optimization is really needed.

\subsubsection{Computation of the alphabet}

When computing $A\cap \alpha (p,W)$ for some multi action name set $A$, it
may be beneficial to apply an optimization. This is done to keep
intermediate expressions small. We introduce $\alpha (p,W,A)$ $=A\cap \alpha
(p,W)$, and define it as follows:%
\[
\begin{array}{lll}
\alpha (\overline{a},W,A) & = & \left\{
\begin{array}{ll}
\{a\} & \text{if }a\in A \\
\emptyset  & \text{if }a\notin A%
\end{array}%
\right.  \\
\alpha (P,W,A) & = &
\begin{array}{l}
\left\{
\begin{array}{ll}
\emptyset  & \text{if }P\in W \\
\alpha (p,W\cup \{P\},A) & \text{if }P\notin W,%
\end{array}%
\right.  \\
\text{ where }P=p\text{ is the equation of }P%
\end{array}
\\
\alpha (p+q,W,A) & = & \alpha (p,W,A)\cup \alpha (q,W,A) \\
\alpha (p\cdot q,W,A) & = & \alpha (p,W,A)\cup \alpha (q,W,A) \\
\alpha (c\rightarrow p,W,A) & = & \alpha (p,W,A) \\
\alpha (c\rightarrow p\diamond q,W,A) & = & \alpha (p,W,A)\cup \alpha (q,W,A)
\\
\alpha (\Sigma _{d:D}p,W,A) & = & \alpha (p,W,A) \\
\alpha (p\at t,W,A) & = & \alpha (p,W,A) \\
\alpha (p\ll q,W,A) & = & \alpha (p,W,A)\cup \alpha (q,W,A) \\
\alpha (p\parallel q,W,A) & = & \alpha (p,W,A)\cup \alpha (q,W,A)\cup \alpha
(p,W,A^{\subseteq })\alpha (q,W,A^{\subseteq }) \\
\alpha (p\leftmerge q,W,A) & = & \alpha (p,W,A)\cup \alpha (q,W,A)\cup
\alpha (p,W,A^{\subseteq })\alpha (q,W,A^{\subseteq }) \\
\alpha (p\mid q,W,A) & = & \alpha (p,W,A^{\subseteq })\alpha
(q,W,A^{\subseteq })%
\end{array}%
\]

\subsubsection{More efficient computation of the alphabet}

The computation of $\alpha (p,W,A)$ can be done more efficiently. We define
the function $proc(p,W)$ as follows:%
\[
\begin{array}{lll}
proc(\overline{a},W) & = & \emptyset  \\
proc(P,W) & = & \left\{
\begin{array}{ll}
\emptyset  & \text{if }P\in W \\
\{P\}\cup proc(p,W) & \text{if }P\notin W%
\end{array}%
\right.  \\
proc(p+q,W) & = & proc(p,W)\cup proc(q,W) \\
proc(p\cdot q,W) & = & proc(p,W)\cup proc(q,W) \\
proc(c\rightarrow p,W) & = & proc(p,W) \\
proc(c\rightarrow p\diamond q,W) & = & proc(p,W)\cup proc(q,W) \\
proc(\Sigma _{d:D}p,W) & = & proc(p,W) \\
proc(p\at t,W) & = & proc(p,W)%
\end{array}%
\]%
Using this function we can change the computation of $\alpha (p,W,A)$ at
three places:%
\[
\begin{array}{lll}
\alpha (p+q,W,A) & = & \alpha (p,W,A)\cup \alpha (q,W\cup proc(p,W),A) \\
\alpha (p\cdot q,W,A) & = & \alpha (p,W,A)\cup \alpha (q,W\cup proc(p,W),A)
\\
\alpha (c\rightarrow p\diamond q,W,A) & = & \alpha (p,W,A)\cup \alpha
(q,W\cup proc(p,W),A)%
\end{array}%
\]%
Note that the value $proc(p,W)$ can be computed on the fly during the
computation of $\alpha (p,W,A)$.

\subsubsection{The mappings $push$, $push_{\protect\nabla }$ and $%
push_{\partial }$}

We define mappings $push$, $push_{\nabla }$ and $push_{\partial }$ such that
$push(p)$ is bisimulation equivalent to $p$, $push_{\nabla }(A,p)$ is
bisimulation equivalent to $\nabla _{A}(p)$, and $push_{\partial }(B,p)$ is
bisimulation equivalent to $\partial _{B}(p)$. The goal of these mappings is
to push allow and block expressions deeply inside process expressions. It is
important to know that an allow set $A$ in the expression $\nabla _{A}(p)$
implicitly contains the empty multi action $\tau $. Let $\mathcal{E}$ $%
=\{P_{1}(d)=p_{1},\ldots ,P_{n}(d)=p_{n}\}$ be a sequence of process
equations.%
\[
\begin{tabular}{lll}
$push(p)$ & $=$ & $p$ if $p$ is a pCRL expression \\
$push(p\parallel q)$ & $=$ & $push\left( p\right) \parallel push\left(
q\right) $ \\
$push(p\leftmerge q)$ & $=$ & $push\left( p\right) \leftmerge push\left(
q\right) $ \\
$push(p\mid q)$ & $=$ & $push\left( p\right) \mid push\left( q\right) $ \\
$push(\rho _{R}(p))$ & $=$ & $\rho _{R}(push\left( p\right) )$ \\
$push(\partial _{B}(p))$ & $=$ & $push_{\partial }(B,p)$ \\
$push(\tau _{I}(p))$ & $=$ & $\tau _{I}(push\left( p\right) )$ \\
$push(\Gamma _{C}(p))$ & $=$ & $\Gamma _{C}\left( push\left( p\right)
\right) $ \\
$push(\nabla _{V}(p))$ & $=$ & $push_{\nabla }(V,p)$%
\end{tabular}%
\]%
We assume that $P_{A,e}^{\nabla }$ is a unique name for every $P\in
\{P_{1},\ldots ,P_{n}\}$, multi action name set $A$ and sequence of data
expressions $e$.%
\[
\begin{tabular}[t]{lll}
$push_{\nabla }\left( A,\overline{a}\right) $ & $=$ & $\left\{
\begin{array}{ll}
\overline{a} & \text{if }N(\overline{a})\in A \\
\delta & \text{otherwise}%
\end{array}%
\right. $ \\
$push_{\nabla }\left( A,P\left( e\right) \right) $ & $=$ & $%
\begin{array}{l}
P_{A}^{\nabla }(e)\text{, where }P(d)=p\text{ is the equation of }P\text{,
and} \\
\text{where }P_{A}^{\nabla }(d)=push_{\nabla }\left( A,p\right) \text{ is a
new equation}%
\end{array}%
$ \\
$push_{\nabla }\left( A,\delta \right) $ & $=$ & $\delta $ \\
$push_{\nabla }\left( A,\tau \right) $ & $=$ & $\tau $ \\
$push_{\nabla }\left( A,p+q\right) $ & $=$ & $push_{\nabla }\left(
A,p\right) +push_{\nabla }\left( A,q\right) $ \\
$push_{\nabla }\left( A,p\cdot q\right) $ & $=$ & $push_{\nabla }\left(
A,p\right) \cdot push_{\nabla }\left( A,q\right) $ \\
$push_{\nabla }\left( A,c\rightarrow p\right) $ & $=$ & $c\rightarrow
push_{\nabla }\left( A,p\right) $ \\
$push_{\nabla }\left( A,c\rightarrow p\diamond q\right) $ & $=$ & $%
c\rightarrow push_{\nabla }\left( A,p\right) \diamond push_{\nabla }\left(
A,q\right) $ \\
$push_{\nabla }\left( A,\Sigma _{d:D}p\right) $ & $=$ & $\Sigma
_{d:D}push_{\nabla }\left( A,p\right) $ \\
$push_{\nabla }\left( A,p\at t\right) $ & $=$ & $push_{\nabla }\left(
A,p\right) \at t$ \\
$push_{\nabla }\left( A,p\ll q\right) $ & $=$ & $push_{\nabla }\left(
A,p\right) \ll push_{\nabla }\left( A,q\right) $ \\
$push_{\nabla }(A,p\parallel q)$ & $=$ & $\mathsf{allow}(A,p^{\prime
}\parallel q^{\prime })\text{ where}\left\{ \text{ }%
\begin{array}{lll}
p^{\prime } & = & push_{\nabla }(A^{\subseteq },p) \\
q^{\prime } & = & push_{\nabla }(A\leftarrowtail \alpha (p^{\prime }),q)%
\end{array}%
\right. $ \\
$push_{\nabla }(A,p\leftmerge q)$ & $=$ & $\mathsf{allow}(A,p^{\prime }%
\leftmerge q^{\prime })\text{ where}\left\{ \text{ }%
\begin{array}{lll}
p^{\prime } & = & push_{\nabla }(A^{\subseteq },p) \\
q^{\prime } & = & push_{\nabla }(A\leftarrowtail \alpha (p^{\prime }),q)%
\end{array}%
\right. $ \\
$push_{\nabla }(A,p\mid q)$ & $=$ & $\mathsf{allow}(A,p^{\prime }\mid
q^{\prime })$ where$\left\{ \text{ }%
\begin{array}{lll}
p^{\prime } & = & push_{\nabla }(A^{\subseteq },p) \\
q^{\prime } & = & push_{\nabla }(A\leftarrowtail \alpha (p^{\prime }),q)%
\end{array}%
\right. $ \\
$push_{\nabla }(A,\rho _{R}(p))$ & $=$ & $\rho _{R}(p^{\prime })$ where $%
p^{\prime }=$ $push_{\nabla }(R^{-1}(A),p)$ \\
$push_{\nabla }(A,\partial _{B}(p))$ & $=$ & $push_{\nabla }(\partial
_{B}(A),p)$ \\
$push_{\nabla }(A,\tau _{I}(p))$ & $=$ & $\tau _{I}(p^{\prime })$ where $%
p^{\prime }=push_{\nabla }(\tau _{I}^{-1}(A),p)$ \\
$push_{\nabla }(A,\Gamma _{C}(p))$ & $=$ & $\mathsf{allow}(A,\Gamma
_{C}(p^{\prime }))$ where $p^{\prime }=push_{\nabla }(C^{-1}(A),p)$ \\
$push_{\nabla }(A,\nabla _{V}(p))$ & $=$ & $push_{\nabla }(A\cap V,p),$%
\end{tabular}%
\]%
where%
\[
\begin{array}{lll}
\mathsf{allow}(A,p) & = & \left\{
\begin{array}{ll}
p & \text{if }(A\cup \{\tau \})\cap \alpha (p)=\alpha (p) \\
\nabla _{A\cap \alpha (p)}(p) & \text{otherwise}%
\end{array}%
\right.%
\end{array}%
\]%
Note that the alphabet $\alpha (p)$ is computed on the fly during the
computation of $push_{\nabla }\left( A,p\right) $.

\paragraph{Example 1}

Let $P=(a+b)\cdot P$. Then $push_{\nabla }\left( \{a\},P,\emptyset \right)
=P^{\prime }$, with $P^{\prime }=push_{\nabla }\left( \{a\},(a+b)\cdot
P,\{(P,\{a\},P^{\prime })\}\right) =push_{\nabla }\left(
\{a\},(a+b),\{(P,\{a\},P^{\prime })\}\right) \cdot push_{\nabla }\left(
\{a\},P,\{(P,\{a\},P^{\prime })\}\right) =\cdots =a\cdot P^{\prime }$.

\paragraph{Example 2}

Let $P=a\cdot \nabla _{\{a\}}(P)$. Then $push_{\nabla }\left(
\{a\},P,\emptyset \right) =P^{\prime }$, with $P^{\prime }=push_{\nabla
}\left( \{a\},a\cdot \nabla _{\{a\}}(P),\{(P,\{a\},P^{\prime })\}\right)
=push_{\nabla }\left( \{a\},a,\{(P,\{a\},P^{\prime })\}\right) \cdot
push_{\nabla }\left( \{a\},\nabla _{\{a\}}(P),\{(P,\{a\},P^{\prime
})\}\right) =\cdots =a\cdot P^{\prime }$.

We assume that $P_{A,e}^{\partial }$ is a unique name for every $P\in
\{P_{1},\ldots ,P_{n}\}$, multi action name set $A$ and sequence of data
expressions $e$.%
\[
\begin{tabular}{lll}
$push_{\partial }(B,\overline{a})$ & $=$ & $\left\{ \text{ }%
\begin{array}{lll}
\overline{a} &  & \text{if }N(\overline{a})\cap B=\emptyset  \\
\delta  &  & \text{otherwise}%
\end{array}%
\right. $ \\
$push_{\partial }(B,P(e))$ & $=$ & $%
\begin{array}{l}
P_{B,e}^{\partial }(e) \\
\text{where }P(d)=p\text{ is the equation of }P\text{, and} \\
\text{where }P_{B,e}^{\partial }(d)=push_{\partial }\left( B,p\right) \text{
is a new equation}%
\end{array}%
$ \\
$push_{\partial }(B,\delta )$ & $=$ & $\delta $ \\
$push_{\partial }(B,\tau )$ & $=$ & $\tau $ \\
$push_{\partial }(B,p+q)$ & $=$ & $push_{\partial }(B,p)+push_{\partial
}(B,q)$ \\
$push_{\partial }(B,p\cdot q)$ & $=$ & $push_{\partial }(B,p)\cdot
push_{\partial }(B,q)$ \\
$push_{\partial }(B,c\rightarrow p)$ & $=$ & $c\rightarrow push_{\partial
}(B,p)$ \\
$push_{\partial }(B,c\rightarrow p\diamond q)$ & $=$ & $c\rightarrow
push_{\partial }(B,p)\diamond push_{\partial }(B,q)$ \\
$push_{\partial }(B,\Sigma _{d:D}p)$ & $=$ & $\Sigma _{d:D}push_{\partial
}(B,p)$ \\
$push_{\partial }(B,p\at t)$ & $=$ & $push_{\partial }(B,p)\at t$ \\
$push_{\partial }(B,p\ll q)$ & $=$ & $push_{\partial }(B,p)\ll
push_{\partial }(B,q)$ \\
$push_{\partial }(B,p\parallel q)$ & $=$ & $push_{\partial }\left(
B,p\right) \parallel push_{\partial }\left( B,q\right) $ \\
$push_{\partial }(B,p\leftmerge q)$ & $=$ & $push_{\partial }\left(
B,p\right) \leftmerge push_{\partial }\left( B,q\right) $ \\
$push_{\partial }(B,p\mid q)$ & $=$ & $push_{\partial }\left( B,p\right)
\mid push_{\partial }\left( B,q\right) $ \\
$push_{\partial }(B,\rho _{R}(p))$ & $=$ & $\rho _{R}\left( R^{-1}\left(
B\right) ,p\right) $ \\
$push_{\partial }(B,\partial _{B_{1}}(p))$ & $=$ & $push_{\partial }(B\cup
B_{1},p)$ \\
$push_{\partial }(B,\tau _{I}(p))$ & $=$ & $\tau _{I}\left( push_{\partial
}\left( B\setminus I,p\right) \right) $ \\
$push_{\partial }(B,\Gamma _{C}(p))$ & $=$ & $\mathsf{block}(B,\Gamma
_{C}\left( push_{\partial }\left( B^{\prime },p\right) \right) $ where $%
B^{\prime }=B\setminus \left\{ b\in B\mid \exists _{\gamma \rightarrow c\in
C}.b\in \gamma \wedge c\notin B\right\} $ \\
$push_{\partial }(B,\nabla _{V}(p))$ & $=$ & $push_{\nabla }(\partial
_{B}(A),p,\emptyset ),$%
\end{tabular}%
\]%
where%
\[
\begin{array}{lll}
\mathsf{block}(B,p) & = & \left\{
\begin{array}{ll}
p & \text{if }B=\emptyset  \\
\partial _{B}(p) & \text{otherwise}%
\end{array}%
\right.
\end{array}%
\]

\paragraph{Example 3}

The presence of $R^{-1}(\partial _{B}(A))$ instead of just $R^{-1}(A)$ in
the right hand side of the rename operator is explained by the example $%
push_{\nabla }(\{b\},\rho _{\{b\rightarrow c\}}b)$. We see that $\rho
_{\{b\rightarrow c\}}push_{\nabla }(R^{-1}(A),p)=\rho _{\{b\rightarrow
c\}}push_{\nabla }(\{b\},b)=\rho _{\{b\rightarrow c\}}b=c$, which is clearly
the wrong answer.

\paragraph{Optimizations}

During the computation of $push_{\bigtriangledown }$ the following
optimizations are applied in the right hand side of each equation:%
\[
\begin{array}{lll}
\nabla _{A}(p) & = & \left\{
\begin{array}{ll}
p & \text{if }(A\cup \{\tau \})\cap \alpha (p)=\alpha (p) \\
\nabla _{A\cap \alpha (p)}(p) & \text{otherwise}%
\end{array}%
\right. \\
\nabla _{\emptyset }(p) & = & \left\{
\begin{array}{ll}
\tau & \text{if }p=\tau \\
\delta & \text{otherwise}%
\end{array}%
\right. \\
\Gamma _{C}(p) & = & \Gamma _{filter_{\nabla }(C,\alpha (p))}(p) \\
\delta \mid \delta & = & \delta \\
\delta \parallel \delta & = & \delta%
\end{array}%
\]

\subsubsection{Allow sets}

There are two rules in the definition of $push_{\nabla }$ where the allow
set can/should not be computed explicitly. The computation of $push_{\nabla
}(A,p\parallel q)$ involves computation of $push_{\nabla }(p,A^{\subseteq }%
\dot{)}$. We want to avoid the computation of $A^{\subseteq }$, since it can
become very large. The computation of $push_{\nabla }(A,\tau _{I}(p))$
involves computation of $push_{\nabla }(p,\tau _{I}^{-1}(A)\dot{)}$. The set
$\tau _{I}^{-1}(A)=AI^{\ast }$ is infinite.

In the implementation we use allow sets of the form $A^{\subseteq }I^{\ast
}, $ where $A$ is a set of multi action names and $I$ is a set of action
names. The $^{\subseteq }$ is optional and $I$ may be empty. Such an allow
set is stored as two sets $A$ and $I$, together with an attribute that tells
if $^{\subseteq }$ is appicable. We need to show that allow sets are closed
under the operations in $push_{\nabla }$.%
\[
\begin{array}{lll}
\partial _{B}(A^{\subseteq }I^{\ast }) & = & \tau _{B}(A)^{\subseteq }\tau
_{B}(I)^{\ast } \\
\tau _{I_{1}}^{-1}\left( A^{\subseteq }I^{\ast }\right) & = & \partial
_{I_{1}}(A^{\subseteq })\left( I\cup I_{1}\right) ^{\ast } \\
\left( A^{\subseteq }I^{\ast }\right) \cap V & = & \{\beta \in V\mid \exists
_{\alpha \in A}.\tau _{I}(\beta )\sqsubseteq \alpha \} \\
R^{-1}\left( A^{\subseteq }I^{\ast }\right) & = & R^{-1}\left( A^{\subseteq
}\right) R^{-1}\left( I\right) ^{\ast } \\
C^{-1}\left( A^{\subseteq }I^{\ast }\right) & \subseteq & C^{-1}\left(
A\right) ^{\subseteq }act\left( C^{-1}\left( I\right) \right) ^{\ast } \\
\left( A^{\subseteq }I^{\ast }\right) \leftarrowtail A_{1} & = &
A^{\subseteq }I^{\ast } \\
\left( A^{\subseteq }I^{\ast }\right) ^{\subseteq } & = & A^{\subseteq
}I^{\ast } \\
\partial _{B}(AI^{\ast }) & = & \partial _{B}(A)\tau _{B}(I)^{\ast } \\
\tau _{I_{1}}^{-1}\left( AI^{\ast }\right) & = & \partial _{I_{1}}(A)\left(
I\cup I_{1}\right) ^{\ast } \\
\left( AI^{\ast }\right) \cap V & = & \{\beta \in V\mid \exists _{\alpha \in
A}.\tau _{I}(\beta )=\alpha \} \\
R^{-1}\left( AI^{\ast }\right) & = & R^{-1}\left( A\right) R^{-1}\left(
I\right) ^{\ast } \\
C^{-1}\left( AI^{\ast }\right) & \subseteq & C^{-1}\left( A\right) act\left(
C^{-1}\left( I\right) \right) ^{\ast } \\
\left( AI^{\ast }\right) ^{\subseteq } & = & A^{\subseteq }I^{\ast }%
\end{array}%
\]%
where we used the following properties:%
\[
\begin{array}{lll}
\partial _{B}\left( A_{1}A_{2}\right) & = & \partial _{B}\left( A_{1}\right)
\partial _{B}\left( A_{1}\right) \\
\partial _{B}\left( A^{\subseteq }\right) & = & \tau _{B}(A)^{\subseteq } \\
R^{-1}\left( A_{1}A_{2}\right) & = & R^{-1}\left( A_{1}\right) R^{-1}\left(
A_{2}\right) \\
R^{-1}\left( A^{\ast }\right) & = & R^{-1}\left( A\right) ^{\ast } \\
C^{-1}\left( A^{\subseteq }\right) & \subseteq & C^{-1}\left( A\right)
^{\subseteq } \\
C^{-1}\left( A_{1}A_{2}\right) & = & C^{-1}\left( A_{1}\right) C^{-1}\left(
A_{2}\right) \\
C^{-1}\left( A^{\ast }\right) & = & C^{-1}\left( A\right) ^{\ast } \\
A^{\subseteq }\leftarrowtail A_{1} & = & A^{\subseteq }%
\end{array}%
\]%
Note that in case of the communication we only have an inclusion relation
instead of equality. This is done to stay within the format $A^{\subseteq
}I^{\ast }$. As a consequence the implementation uses an over-approximation
of $C^{-1}\left( A^{\subseteq }I^{\ast }\right) $ and $C^{-1}\left( AI^{\ast
}\right) $. Furthermore note that the property $R^{-1}\left( A^{\subseteq
}\right) =R^{-1}\left( A\right) ^{\subseteq }$ does not hold. A counter
example is $R=\{b\rightarrow a\}$ and $A=\{a,b\mid c\}$. In that case we
have $R^{-1}\left( A^{\subseteq }\right) =\{a,b,c\}^{\subseteq }$ and $%
R^{-1}\left( A\right) ^{\subseteq }=\{a,b\}^{\subseteq }$. Another property
that was initially assumed, but that does not hold is $\left( AI^{\ast
}\right) \leftarrowtail A_{1}=\left( A\leftarrowtail \tau _{I}(A_{1})\right)
I^{\ast }$.

\newpage

\subsection{Optimization for $push_{\protect\nabla }$}

In some cases the $push_{\nabla }$ operator produces expressions that are
too large. This section proposes an optimization for the case $push_{\nabla
}(A,\Gamma _{C}(p))$ that can help to prevent this problem for certain
practical cases.

\[
\begin{tabular}[t]{lll}
$push_{\nabla }(A,\Gamma _{C}(p))$ & $=$ & $\left\{
\begin{array}{ll}
\mathsf{allow}(A,\Gamma _{C\setminus C^{\prime }}(push_{\nabla \Gamma
}(A^{\prime },C^{\prime },p))) & \text{if }C\neq C^{\prime } \\
push_{\nabla \Gamma }(A,C,p)) & \text{otherwise,}%
\end{array}%
\right. $%
\end{tabular}%
\]%
with $C^{\prime }=\{\beta \rightarrow b\in C\mid b\notin
\bigcup\limits_{\beta ^{\prime }\rightarrow b^{\prime }\in C}\beta ^{\prime
}\}$ and $A^{\prime }=((C\setminus C^{\prime })(A))^{\subseteq }$ and%
\[
\begin{tabular}[t]{lll}
$push_{\nabla \Gamma }(A,C,p\parallel q)$ & $=$ & $\mathsf{allow}\left(
A,\Gamma _{C}\left( \mathsf{allow}\left( C^{-1}(A),p^{\prime }\parallel
q^{\prime }\right) \right) \right) \text{ where}\left\{ \text{ }%
\begin{array}{lll}
p^{\prime } & = & push_{\nabla \Gamma }(A^{\prime },C,p) \\
q^{\prime } & = & push_{\nabla \Gamma }(A^{\prime \prime },C,q) \\
A^{\prime } & = & C^{-1}(A)^{\subseteq }\setminus \left( C^{-1}(A)\setminus
A\right) \\
A^{\prime \prime } & = & \left( C^{-1}(A)\leftarrowtail \alpha (p^{\prime
})\right) \setminus \left( C^{-1}(A)\setminus A\right)%
\end{array}%
\right. $ \\
$push_{\nabla \Gamma }(A,C,p\leftmerge q)$ & $=$ & $\mathsf{allow}\left(
A,\Gamma _{C}\left( \mathsf{allow}\left( C^{-1}(A),p^{\prime }\leftmerge %
q^{\prime }\right) \right) \right) \text{ where}\left\{ \text{ }%
\begin{array}{lll}
p^{\prime } & = & push_{\nabla \Gamma }(A^{\prime },C,p) \\
q^{\prime } & = & push_{\nabla \Gamma }(A^{\prime \prime },C,q) \\
A^{\prime } & = & C^{-1}(A)^{\subseteq }\setminus \left( C^{-1}(A)\setminus
A\right) \\
A^{\prime \prime } & = & \left( C^{-1}(A)\leftarrowtail \alpha (p^{\prime
})\right) \setminus \left( C^{-1}(A)\setminus A\right)%
\end{array}%
\right. $ \\
$push_{\nabla \Gamma }(A,C,p\mid q)$ & $=$ & $\mathsf{allow}\left( A,\Gamma
_{C}\left( \mathsf{allow}\left( C^{-1}(A),p^{\prime }\mid q^{\prime }\right)
\right) \right) \text{ where}\left\{ \text{ }%
\begin{array}{lll}
p^{\prime } & = & push_{\nabla \Gamma }(A^{\prime },C,p) \\
q^{\prime } & = & push_{\nabla \Gamma }(A^{\prime \prime },C,q) \\
A^{\prime } & = & C^{-1}(A)^{\subseteq }\setminus \left( C^{-1}(A)\setminus
A\right) \\
A^{\prime \prime } & = & \left( C^{-1}(A)\leftarrowtail \alpha (p^{\prime
})\right) \setminus \left( C^{-1}(A)\setminus A\right)%
\end{array}%
\right. $ \\
$push_{\nabla \Gamma }(A,C,\partial _{B}(p))$ & $=$ & $push_{\nabla \Gamma
}(\partial _{B}(A),C,p)$ \\
$push_{\nabla \Gamma }(A,C,\nabla _{V}(p))$ & $=$ & $push_{\nabla \Gamma
}(A\cap V,C,p)$ \\
$push_{\nabla \Gamma }(A,C,p)$ & $=$ & $\mathsf{allow}(A,\Gamma
_{C}(p^{\prime }))$ where $p^{\prime }=push_{\nabla }(C^{-1}(A),p)$ for all
other cases of $p$%
\end{tabular}%
\]%
Note that in this case the allow set $A$ has the general shape $\left(
A_{1}^{\subseteq }\setminus A_{2}^{\subseteq }\right) I^{\ast }$ (?), with
the subset operator $\subseteq $ optional, and with $I$ possibly empty. To
implement this optimization, it needs to be investigated if such a set $A$
is closed under the operations $\partial _{B}(A)$, $\tau _{I_{1}}^{-1}(A)$, $%
A\cap V$, $R^{-1}(A)$, $C^{-1}(A)$, $A\leftarrowtail A_{1}$, $A^{\subseteq }$
and $C(A)$.

\end{document}
