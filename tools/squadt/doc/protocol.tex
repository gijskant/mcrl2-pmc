\documentclass{article}
\usepackage[UKenglish]{babel}
\usepackage{pslatex}
\usepackage{graphicx}
\usepackage{calc}
\usepackage{latexsym,amssymb,amsfonts,amsthm,amsmath}
\usepackage{float}
\usepackage{verbatim}
\usepackage{xspace}

\usepackage[top=3cm,left=3.3cm,bottom=3cm,right=3.3cm]{geometry}

\title{A Protocol for Interactively Controlling Software Tools\\DRAFT}
\author{J. van der Wulp}

\newtheorem{example}{Example}

\newcommand{\msg}[1]{\texttt{#1}}
\newcommand{\squadt}{SQuADT\xspace}

\makeatletter
\renewcommand{\paragraph}{\@startsection
  {paragraph}%
  {4}%
  {-1.3em}%
  {-\baselineskip}%
  {0.1\baselineskip}%
  {\normalfont\normalsize\textbf}}
\makeatother

%\setlength{\parindent}{0cm}
%\setlength{\parskip}{\baselineskip}

\bibliographystyle{alpha}

\begin{document}

\maketitle
\thispagestyle{empty}

 \section{Introduction}

  The mCRL2 toolset (see \cite{groote_et_al:DSP:2007:862}) is a collection of
  tools around the formal modelling language mCRL2 that can be used for formal
  verification and analysis of process behaviour.  Most of the tools have a
  traditional command line interface and today not everyone is comfortable with
  this way of working.  Therefore we started working on a tool integration
  framework to make the toolset usable for a broader audience. The idea is that
  a uniform graphical user interface should make it easier to use tools without
  having too much knowledge about the specifics of every tool.  The focus is
  simplifying the use of individual tools as well as combinations of tools and
  to automate frequently occurring tasks that involve the use of multiple
  tools.
  
  The \squadt desktop application is a graphical user interface layer around a
  new tool integration framework, within which a central part is played by the
  communication protocol described in this text.  The name \squadt, stands for
  (Systems Quality, Analysis and Design Toolset), which refers to the kind of
  tasks that can be performed with the connected tools. The connected tools are
  those found in the mCRL2 toolset. Most of these the tools can be used
  stand-alone by means of a traditional command line interface and some with a
  graphical user interface.  The only communication between the tools is
  uni-directional by means of files or file streams (also known as piping). The
  design of the \squadt graphical user interface as well as much of its
  functionality have, so far been targeted at tools with this specific
  behaviour.
  
  The idea of using a graphical user interface to simplify the use of a toolset
  is is not new. The \squadt application is very much influenced by the
  Eucalyptus application (see \cite{CADP}) in the CADP toolset. Eucalyptus was
  developed (around 1996) in the context of CADP for a very similar purpose as
  \squadt is for the mCRL2 toolset. However, this does not mean that all the
  underlying ideas of \squadt are the same as those of Eucalyptus.
  
  %  Ever since the first version, \squadt communicates with tools using the
  %  communication protocol described in this document.
  
  Contrary to Eucalyptus, in \squadt every action is performed in the context
  of a project. This approach is adapted from integrated development
  environments (IDE).  An IDE is an integration framework for software
  development; it integrates a number of often stand-alone software tools that
  are used for software development. In the project context \squadt manages a
  collection of files and a collection of tools that can be used to add new
  files to a project. The user observes and directs this process through a
  graphical user interface.  The core of that user interface is focussed around
  an interactive visual overview of all data dependencies within a project.
  Every dependency represents an application of a tool on a set of files with
  another (disjoint) set of files files as the result (where output depends on
  input).  Within a project \squadt keeps track of all file inter dependencies
  and uses this to monitor file consistency (which will be explained in a moment).

  Actually the adoption of the notion of project was necessary as a context for
  preserving information about how files depend on one another. One important
  use of those dependencies is the detection of possible inconsistencies
  between files. As an illustration of the purpose consider the following
  scenario. Picture a project containing a file representing a log generated by
  a tool as a result of the task it performed. Now another tool is applied in
  the context of the same project to generate a report of this log file. For
  some reason the user decides to reapply the tool that generates the log file
  thereby updating this file.  Depending on the intend of the user the report
  is possibly no longer up-to-date (or inconsistent) with the data in the log
  file. Inconsistencies arise frequently as a result of working with multiple
  tools on multiple files. Sometimes such inconsistencies are overlooked and as
  a result the user analyses the wrong results. The idea is to help the user
  to become aware of such inconsistencies when they arise.

  Consistency is a property between two disjoint sets of files that is
  established by the application of a tool to the files of one of the sets and
  the output of the tool being the files in the other set.  Every file in a
  project is either added by the user, or it is produced by a single tool with
  a single configuration from a single set of input files. So every file in the
  project can be (re)produced in a unique way by applying tools to files. A
  file in a project is \textit{consistent} if and only if reapplication of the
  tool does not change the contents of the file, provided that the set of input
  files is consistent.  A tool produces a set of output files from a given set
  of input files, at which point the output files are all consistent. Without
  inspecting the contents of files, the framework can only detect when changes
  have occurred to files.  Using file dependencies available from the project
  context it can then deduce which files may no longer be consistent.
  
  Inconsistency can result from legitimate tool application within the scope of
  a project, but also from unexpected tool behaviour or system/hardware
  failure.  The integration framework as well as the individual connected tools
  must cooperate in order to maximise the effectiveness of any inconsistency
  detection mechanism. Firstly, creation and modification of files by tools must be
  restricted, i.e. directed pro-actively by the integration framework.  This is
  in line with the assumption that a file in a project must be reproducible in
  a unique way. A direct consequence is that input files cannot also be output
  files for the same tool application.  Secondly, tools must communicate a
  complete specification of input and output files.  When tools behave
  accordingly the set of dependencies among files in a project is complete and
  the dependencies together form a directed acyclic graph.
 
  Better integration is possible when a tool is `aware' of the integration
  context. Different user interfaces to the same functionality may have different
  requirements. For example in the context of a graphical user interface it is
  convenient and accepted practise to show progress indication. For the purpose
  of integration it may be desirable to adapt a tool such that it shows
  progress when it is operating within the context of the integration
  framework. To conveniently use any functionality provided by the integration
  framework there must be access to use such functionality directly from
  within a tool.  The purpose of this text is to describe the communication
  protocol used by the integration framework that supports \squadt.

%  The access to functionality provided by the integration
%  framework will be provided by means of a custom communication protocol.

%%%  All functionality described above can in essence be obtained by an
%%%  integration system that just uses existing tools and their command line
%%%  interfaces. Making tools available in such a system would be a matter of
%%%  ad-hoc connection for instance by means of a plugin framework. This is the
%%%  approach chosen for Eucalyptus, the graphical front-end of the CADP toolset.
%%%  This approach is portable and works fine but connecting tools often takes a
%%%  lot of effort. One step further than Eucalyptus is to consider a facility at
%%%  user interface level that allows a tool to visualise task progress, and
%%%  interact with the user. This offers additional opportunities that were not
%%%  available to a tool with just a command line interface. So we decided on
%%%  using a custom communication protocol instead of existing command-line
%%%  interfaces in order to force tool builder to design new interfaces for
%%%  operating in a \squadt context.
%%%
%%%  As a matter of reference to an existing solution, a popular approach to an
%%%  interface is WSDL (see \cite{2001-WSDL}). Which is an interface description
%%%  language that is used to power web services, based on XML.  A web service is
%%%  ``a software system designed to support interoperable machine to machine
%%%  interaction over a network''. Technically it is possible to use the web
%%%  services model as intermediate interface between tool and integration system.
%%%  However, because of the static nature of interface description in WSDL it is
%%%  not easily possible to read additional input or produce additional output.
%%%  Several of the mCRL2 tools create new output files based on the input.
%%%  This is not something that is easily described in WSDL.
%%%  
%%%  For tools without a graphical user interface
%%%  it would be possible to create a web service layer around the tool. For tools
%%%  with a graphical user interface this would only work. Tools with graphical user Our
%%%  focus is on a protocol for controlling tools from start to finish.

%  In abstract, the platform provides a number of facilities that can be used by
%  tools to communicate with the user. The tools provide services that are made
%  accessible to the user of the platform. The platform acts as a controller
%  (possibly for multiple tools at the same time) and therefore the system will
%  mostly be referred to as (the) controller in the remainder of this document.
%  Similarly, a tool is the communication partner of the controller.

  The remainder of this text is structured as follows. To give an overview of
  things involved, the next section introduces the important concepts. This is
  followed by a high level overview of the protocol. From this the reader
  should get a picture of the structure of communication: what is communicated,
  in what way and why. This is followed by a more detailed description of the
  protocol. It consists of a description of the contents of messages, their
  syntax and effects (rough semantics) and some of the important design
  choices.  At the end is a quick comparison with interfaces used by other
  integration frameworks.
% IDE comparison
% note control aspect: user asserts control over a tool by means of the system

 \section{Concepts}

   There are lots of examples of software tools that all have different
   functionality and interfaces. It is only useful to consider integration
   between tools when there is a meaningful way in which the tools can be used
   together in order to obtain a result that any of the individual tools
   cannot.  To create value, a tool integration framework must simplify the use
   of connected tools or offer functionality transcends the functionality
   provided by individual connected tools.
 
   The main purpose of solving the tool integration problem is assisting a user
   in utilising the connected tools. Making tools accessible from a unified
   user interface already serves this goal. More interesting however, is adding
   functionality that crosses the boundaries of the individual tools. The prime
   example of such functionality in our framework is that of consistency
   monitoring.  The consistency monitoring functionality has profoundly
   influenced protocol design at several points.
   
   This section functions as introduction to the important concepts around
   which a communication protocol will be built in later sections.

  \subsection{Tool}

%   A regular English dictionary defines the meaning of the word tool as: the
%   means whereby some act is accomplished.
   A \textit{(software) tool} is a program that processes input and produces
   output that functionally depends on that input. The output of a tool is the
   result or accomplishment. Both input and output of a tool are sets of
   references to sources of binary data. Considered so far are: data from files
   in a local filesystem or potentially unbounded streams of data sent or received
   over a network. Alternate sources of input or output namely data associated
   with user interaction through connected human interface devices are treated
   specially.
   
   Any tool is always used with a particular purpose in mind. Let's assume that
   the use serves the purpose (i.e. the tool is right for the job). A tool may
   serve different purposes and for each unique purpose the tool is said to
   have a \emph{function} for that particular purpose. For all thinkable
   purposes, the largest set of functions for a particular tool makes up it's
   \textit{total functionality}.

%   Modern operating systems allow programs to be executed as processes either
%   concurrently or in a time-sharing fashion. The consequence of is that a tool
%   can be running multiple times at the same moment on the same machine. Notice
%   that a tool's input and output function as communication channels between
%   processes. % where to put unidirectional requirement?

  \subsection{Task}

   A \textit{task} for a tool is the use of a specific combination of functions
   of that tool. This combination determines the requirements on input and
   output. The input of a tool needed for a task, called \textit{task input},
   is a non-empty set of resource identifiers (discussed shortly). Similarly
   output of a tool for a task, called \textit{task output}, is a non-empty set
   of resource identifiers that is disjoint with task input. Notice that this
   precludes manipulation of input as part as output. 

   A resource identifier is a name of a file or stream associated with a type
   that describes the data format (or file type).  The principle method of
   specifying input/output files or streams is the Uniform Resource Identifier
   (or URI, see \cite{rfc3305}). A URI that is specified as part of input must
   identify an existing resource before a tool can be applied. Similarly, a URI
   is specified as output must identify an existing resource after the tool
   completes its task.

   Every input as well as output is associated with a type that is specified
   using the MIME format (Multipurpose Internet Mail Extensions, \cite{rfc2822}).
   The type of outputs must be classified carefully because output of one tool
   may serve as input to other tools that may have requirements on types.

  \subsection{Task Configuration} \label{concepts::task_configuration}

%Whether the use actually satisfies the purpose depends on the
%   functionality the tool provides and how the tool is used.
   
   The process of bringing a tool in the state that it can perform a specified
   task is called \textit{task configuration}. The state of a tool after task
   configuration is complete can be made explicit by capturing it as a task
   specification.  A \textit{task specification} for a tool is a concrete
   specification that uniquely defines a task for that tool (up to
   user-interaction).

   \noindent A task specification consists of a description of task input and
   task output and it specifies the specific combination of functions that
   define the task. The task input is set of URI's that all identify an
   existing resource. The task output is a set of URI's that is disjoint from
   the task input. Every input as well as output is associated with a MIME
   type, if the actual resource identified by the URI does not match the type a
   tool is expected to fail in completing the task.

   \begin{figure}[H]
    \begin{center}
     \includegraphics{task_configuration.eps}
    \end{center}
    \caption{Graphic overview of the contents of a task specification}
   \end{figure}

   % creation and manipulation (tool and system)
 
   A tool creates its own task specification and communicates it afterward
   with the integration framework. The framework can read and modify the part
   that specifies the task input and output, e.g. it can rename input files.
   The remainder of the task specification is tool-specific, the framework can
   only store this information but not interpret. The purpose of communicating
   the configuration with the framework is to have a means to preserve it.
   
   Task execution is the process of using a tool to fulfil a configured task.
   To configure a task the tool must create a task specification in cooperation
   with the user and communicate it with the integration framework. To actually
   start execution the framework communicates a task specification with the
   tool that must either accept or reject it. Once a task specification is
   accepted actual task execution may commence.

   % note on user interaction
%   Some tools always require interaction with the user. In such a case a task cannot
%   be unambiguously specified. The task specification only specifies a starting
%   point from where the user interaction . The actions of the user may
%   completely change the configuration in any way.

   % note on failure
%   Since tools are programs, and programs can fail, task execution may fail.
%   Nothing can be concluded about a task that failed. If output exists it
%   must be assumed to be corrupted.

  \subsection{Display}

   Since tools create task specifications themselves, with the user as
   beneficiary, the tool must have means of communicating with the user. The
   \textit{(interaction) display} is a tool-controlled graphical user interface
   that acts as a direct communication channel between a running tool and the
   user. Every running instance of a tool has its own display. The display can
   also be used for instance to show task progress or to query a user during
   task execution.

   There are no limits to the use of the display. Once a tool is running it can
   make use of the display facility.

%%   Ideal integration would present a single user interface for all supported
%%   tools. Naturally a tool integration framework sits between the user and a
%%   set of tools. Tools would use the facilities of the integration framework to
%%   create a (graphical) user interface for communication with the user. This
%%   level of integration is very nice for a user but is too restrictive for our
%%   purposes.  Tools that have not been developed with integration in mind, can
%%   still be adapted to work in the context of an integration framework.  Many
%%   such existing tools have their own (graphical) user interface which cannot
%%   easily be recreated just for the purpose of obtaining a more uniform
%%   interface when the tool is used from the integration framework.

%%   As an example on the use of the display consider the following scenario. In
%%   order to arrive at a task-configuration, a tool must communicate with the
%%   user. As a result \textit{the tool} produces this task-configuration.  The
%%   complete procedure is as follows. First the user selects a tool that has the
%%   functionality to complete the task at hand.  The tool is started and must be
%%   configured to actually complete that task.  The tool can gather the
%%   information it needs to produce a task specification from the user through
%%   its interaction display. After the process of task configuration a task can
%%   be started. The tool will commence task execution during which it can use
%%   the display to inform the user about progress, or request further input.

   % added value
%   As hinted at before, the interaction display can be used for other purposes
%   than configuration. It could be used to add a graphical user interface to a
%   tool that otherwise does not have one. The tool controls the contents of
%   it's display as long as it runs. An example of its use in a
%   non-configuration setting may be giving progress indication during task
%   execution.

%  \subsection{Tool Integration}
%
%   The \squadt application is a tool for interactive software tool
%   integration.  It is built on top of a tool integration framework that
%   currently only consists of a tool control interface. The main tasks of the
%   framework are running tools and directing communication between tools
%   whether running or not. The integration level functionality that \squadt
%   provides using the framework are:
%   
%    \begin{itemize}
%     \item help guide users to find tools that offer the desired functionality
%     \item automate frequently (re)occurring tasks that involves running tools
%     \item offer a uniform way in which users can interact with tools
%    \end{itemize}
%
%   There are some more but these are the most significant ones and all have
%   directly affected the design of the tool control interface.

 \section{Communication Protocol (high level)} \label{s:message_definitions}

  The \squadt application is built on top of a portable tool integration
  framework built in C++. At the heart of this framework is the communication
  protocol we are going to sketch in this section. The protocol represents the
  interface between two different parties, the framework or controlling-side
  (called \textit{controller}), and the side of a software program representing
  a tool (called \textit{tool}). The role of a tool is that a configurable
  service, that of a controller is to orchestrate on behalf of a user.

  The smallest unit of communication is a message. Messages are sent and
  received in a particular order and the protocol specifies how each message
  must be interpreted. Every message is equipped with type that identifies its
  purpose.  The type indicates how the message should be interpreted and
  provides a means to specify restrictions on message order. Interpretation of
  a message is based on both it's type and the role of the party that receives
  it.

  Message interpretation, besides type and role of the receiving party, is also
  affected by the messaging context.  A \textit{message context} is a
  chronologically ordered list of messages that were all either sent or
  received by the same communication partner in that particular order.  More
  formal: let $p$ and $q$ be communication partners. The \textit{receiving
  context} of $p$ (pertaining to $q$) is the sequence of \emph{all} messages
  $m_{1}, m_{2}, m_{3}, \ldots$ that were sent by $q$ and received by $p$ in
  this particular order.  The \textit{sending context} of $p$ (pertaining to
  $q$) is the sequence of \emph{all} messages $m_{1}, m_{2}, m_{3}, \ldots$
  that were sent by $p$ and received by $q$ in this particular order. From
  hereon we will assume that every message that is sent will also be received
  and that message order is preserved. More precise the receiving context of
  $p$ pertaining to $q$ is a prefix of the sending context of $q$ pertaining to
  $p$.

  A basic pattern used in the protocol is a request-response sequence. The
  tool, as well as the controller, can issue a request that the other party
  \emph{must} respond to. So a request in the sending context can always be
  uniquely paired to a response in the receiving context. Besides this clear
  pattern there are also notification messages, e.g. message that do not
  require any response.

  This section provides a high level overview of the protocol in terms of: what
  is communicated when and for what purpose. The next section zooms in on the
  different messages (by type) and the purpose and structure of data
  communicated by means of such a message.

  \subsection{Notes on Presentation}

   Messages have a name and are described by giving type and direction and a
   specification of the structure of its contents. For example table
   \ref{table:example_message} describes a message called ``example message''
   with type ``example'' it is sent by a controller and received by a tool and
   carries some data (discussed shortly). A message without data has the data
   field unspecified. If both communication partners can send the same message
   (message with the same name) the direction field is unspecified.
   
   The contents of a message is the combination of type, direction and data.
   Message types are names that are introduced on demand. The direction
   determines which of the communication partners can send or receive such
   messages. Message names are not communicated; every message can be
   identified by the combination of type data and direction. In this section
   the focus lies on the information carried by a message and its purpose; not
   on the representation and details of interpretation.
   
   The structure of a message is described in terms of named types that may be
   composed of other (nested) types. The type notation is a mix using
   mCRL2-data syntax for types to do BNF-style specification.  The names hint
   at the purpose and data inter-dependency and help to form an intuition about
   how this data is used to fulfill the purpose.  There is a small number of
   predefined types which are discussed next. The first type is \texttt{Id}
   representing a textual identifier (scope is determined at use). The other
   types are \texttt{URI}, \texttt{MIME-type}, $\mathbb{B}$, $\mathbb{N}$,
   and \texttt{String}, with the usual meaning.

   Tuples are specified using the $\times$ operator as usual e.g. $\texttt{Id}
   \times \mathbb{N}$ is an ordered pair of an identifier and a natural number.
   Lastly there are the set and list types, with constructors $Set$ and $List$
   that are parametrised by the type of the elements, e.g.  $Set(\mathbb{N})$
   represents a set of natural numbers.
   
   The following table specifies a message sent by a tool with name `example
   message' of type \msg{example}.  The contents (or data) of the message is a
   list of finite length with pairs of Token $\times$ Number; where token is of
   type String and Number is a natural number.

   \begin{table}[H]
    \begin{center}
     \begin{tabular}{|ll|}
      \hline
       \multicolumn{2}{|l|}{\textbf{example request}} \\
      \hline
       message type:    & \msg{example} \\
      \hline
       direction:       & tool to controller \\
       data:            & List(Token $\times$ Number) \\
                        & \ Token  = String \\
                        & \ Number = $\mathbb{N}$ \\
      \hline
     \end{tabular}
     \vspace{-0.3cm}
    \end{center}
    \label{table:example_message}
   \end{table}
   \noindent Consequently the contents of this message is a sequence of pairs
   consisting of a string and a natural number.

   \noindent Pictures of state machines (inspired by regular automata notation)
   are used to specify the allowed communication behaviour with regard to a
   message. When an action label on a transition is the name of a message then
   this transition represents a communication action, otherwise it is a local
   action by \emph{one} of the tool partners. The non-communication actions
   illustrate local choice that influences communication behaviour. The
   incoming arrow on the left identifies the initial state. Figure
   \ref{fig:process_example} shows an example process.

   \begin{figure}[H]
    \begin{center}
     \includegraphics{example.eps}
    \end{center}
    \vspace{-0.3cm}
    \caption{Example of a process based around communication behaviour}
    \label{fig:process_example}
   \end{figure}

   From the initial state an example request takes place, meaning a
   communication action occurs between the partners. Subsequently an example
   response must occur after which a state is reached from which a start
   notification can occur. A start notification can only be followed by a stop
   notification and vice versa. 

  \subsection{Instance identification} \label{s:instance_identification}

   The purpose of instance identification is to identify a tool that starts
   communicating with the controller and match it with the purpose to which
   this tool was started. The integration framework that plays the part of
   controller starts tools on demand, e.g. to allow a user to configure a task,
   and waits for each tool to start communicating. When started the integration
   framework must pass it a unique token that identifies the purpose to which
   the tool was started.  The tool will subsequently initiate communication
   with the controller and communicate the token. After a token is received the
   integration framework must decide on its validity and thereby the identity
   of the communication partner.   

   When a tool is started it will initiate communication with a controller.  In
   practise this means that a new messaging context (connection) is established
   at the side of controller and tool.  After the connection is established the
   tool \emph{must} send an identification message containing the token it
   received.
   
   \begin{figure}[H]
    \begin{center}
     \begin{tabular}{|ll|}
      \hline
       \multicolumn{2}{|l|}{\textbf{identification notification}} \\
      \hline
       message type:    & \msg{identification} \\
      \hline
       direction:       & tool to controller \\
       data:            & Token \\
                        & \ Token = String \\
      \hline
     \end{tabular}
     \vspace{-0.3cm}
    \end{center}
   \end{figure}

   \noindent A message of this type \emph{must} be the first message in any
   receiving context of a controller. The environment must evaluate an incoming
   token and must make a judgement on its validity. If the token is not valid
   then communication with the tool \emph{should} be aborted.

   \begin{figure}[H]
    \begin{center}
     \includegraphics{connection_and_identification.eps}
    \end{center}
    \vspace{-0.3cm}
    \caption{Process of instance identification}
    \label{fig::identification}
   \end{figure}

   \noindent Figure \ref{fig::identification} shows the combined behaviour of a
   controller and a tool with regard to instance identification. A tool
   initiates a connection, sends an identification notification and starts
   waiting for incoming messages (controller has initiative).  At the side of
   controller the value of the token is somehow used to choose between breaking
   the connection and accepting the identity of the tool on the other side. If
   the connection persists and the tool receives an arbitrary other message
   from the controller its identify was accepted.

  \subsection{Capabilities}

   Capabilities represent both an information facility for one communication
   partner to learn about the capabilities of the other, as well as a protocol
   extension mechanism. The extension mechanism currently only consists of a
   means to check for the protocol version supported by a communication
   partner. This provides some limited backward and forward compatibility
   between protocol versions.  For example it allows a tool developer to check
   controller side support for facilities that have been introduced in specific
   protocol versions.
   
   \noindent The exchange of capabilities follows a request response sequence
   as illustrated by the following figure.
   
   \begin{figure}[H]
    \begin{center}
     \includegraphics{exchange_of_capabilities.eps}
    \end{center}
    \vspace{-0.3cm}
    \caption{Process of exchange of capabilities}
   \end{figure}

   \noindent The process follows the basic request response pattern and is
   symmetric for both communication parties. A request for capabilities
   message looks as follows.
   
   \begin{table}[H]
    \begin{center}
     \begin{tabular}{|ll|}
      \hline
       \multicolumn{2}{|l|}{\textbf{capabilities request}} \\
      \hline
       message type:    & \msg{capabilities} \\
      \hline
       data:            & \\
      \hline
     \end{tabular}
     \vspace{-0.3cm}
    \end{center}
   \end{table}

   \noindent The partner that receives \emph{must} send a response. For a
   controller the response only contains the protocol version number and looks
   as follows.

   \begin{table}[H]
    \begin{center}
     \begin{tabular}{|ll|}
      \hline
       \multicolumn{2}{|l|}{\textbf{capabilities response (controller)}} \\
      \hline
       message type:   & \msg{capabilities} \\
      \hline
       direction:      & controller to tool \\
       data:           & Version = Major $\times$ Minor \\
                       & \ Major = $\mathbb{N}$ \\
                       & \ Minor = $\mathbb{N}$ \\
      \hline
     \end{tabular}
    \end{center}
    \vspace{-0.3cm}
   \end{table}

   \noindent For a tool the response in addition contains an abstract
   `advertisement' of the tools functionality. This advertisement is a
   non-empty set of input configurations.  An \textit{input configuration} is a
   pair of a \textit{category} (introduced next), and a non-empty list of names
   for inputs each of which associated with a type (as MIME-type).  The
   category is a descriptive name for the type of functionality that a tool
   offers for that specific input configuration.
   
   \begin{table}[H]
    \begin{center}
     \begin{tabular}{|ll|}
      \hline
       \multicolumn{2}{|l|}{\textbf{capabilities response (tool)}} \\
      \hline
       message type:   & \msg{capabilities} \\
      \hline
       direction:      & tool to controller \\
       data:           & Version $\times$ Set(InputConfiguration) \\
                       & \ Version = Major $\times$ Minor \\
                       & \ \ Major = $\mathbb{N}$ \\
                       & \ \ Minor = $\mathbb{N}$ \\
                       & \ InputConfiguration = Category $\times$ List(Id $\times$ Format) \\
                       & \ \ Category = String \\
                       & \ \ Format = MIME-type \\
      \hline
     \end{tabular}
    \end{center}
    \vspace{-0.3cm}
   \end{table}

   The set of input configurations partition the entirety of tasks that a tool
   can perform into classes that have the same input requirements and whose
   functionality falls in the same category. Every task specification is based
   on an input combination. The input combination expresses a set of basic
   requirements for configuration of tools using some specific tool
   functionality.
   
%   It is important to realise that the content of capabilities response
%   messages described here represent only a bare minimum. Protocol extension
%   requires extension to contents and therefore the structure of these
%   messages (see section \ref{ss:structure}).

  \subsection{Task Configuration} \label{ss::task_configuration}

%   The controller makes use of the services provided by a tool through task
%   configuration. After that, the task that is performed makes use of a
%   selection of the functionality that the tool provides.  Configuration and
%   task execution are separated, the latter is discussed in section
%   \ref{ss::task_execution}.

   A task specification can be obtained in two ways, either by deriving it from
   from an input configuration or by receiving one as result of a previous task
   configuration process. Task configuration is considered to be an incremental
   activity, the result of which is a configuration that is uniquely
   represented by a task configuration. A tool's configuration can always be
   recreated from a task specification so it is always possible to modify or
   refine the configuration. In fact the task configuration process always
   starts with communicating an existing task specification with a tool.
   
   Configuration details, as part of task specifications, can now exist outside
   the tool.  A consequence of making a task specification explicit is that it
   opens up the opportunity that a tool receives a task specification that does
   not uniquely specify a configuration. Simple causes are tools that change
   over time or task specifications get corrupted.  It is necessary for tools
   to check whether a task specifications are usable.  As a consequence tool
   developer must think about validating a task specification and in the case
   the configuration is unusable must resolve this problem through
   communication with the user.

   The following figure depicts the basic (isolated) communication behaviour of
   the task configuration process.
   
   \begin{figure}[H]
    \begin{center}
     \includegraphics{configuration_execution.eps}
    \end{center}
    \vspace{-0.3cm}
    \label{fig:configuration_execution}
    \caption{Schematic overview of the process of task configuration and execution}
   \end{figure}

   The process of task configuration is a straight-forward application of the
   request-response pattern. The state with label \texttt{C} represents the
   configured state, i.e. the state from which task execution may commence.
   Similarly the state with label \texttt{E} represents the state in which the
   tool is executing a task. Section \ref{ss::task_execution} introduces the
   messages that deal with task execution.

%   Let $sc$ be the sending context of a controller and $rc$ be the receiving
%   context (pertaining to one tool). If both controller and tool are still
%   running and every configuration message in $sc$ can be paired to a unique
%   configuration message in the receiving context then \emph{no} task
%   configuration is in progress.
   
   The following figure specifies the content of the message that initiates the
   task configuration process.
   \begin{table}[H]
    \begin{center}
     \begin{tabular}{|ll|}
      \hline
       \multicolumn{2}{|l|}{\textbf{configuration request}} \\
      \hline
       message type:   & \msg{configuration} \\
      \hline
       direction       & controller to tool \\
       data:           & Interactive $\times$ Configuration \\
                       & \ Interactive = $\mathbb{B}$ \\
                       & \ Configuration = Category $\times$ List(ConfigurationItem) \\
                       & \ \ Category = String \\
                       & \ \ ConfigurationItem = Id $\times$ (Option $|$ Object) \\
                       & \ \ Option = List(DataType $\times$ String) \\
                       & \ \ \ DataType \\
                       & \ \ Object = (Input $|$ Output) $\times$ Format $\times$ Location \\
                       & \ \ \ Location = URI \\
                       & \ \ \ Format   = MIME-type \\
      \hline
     \end{tabular}
     \vspace{-0.3cm}
    \end{center}
   \end{table}

   \noindent The message \emph{must} specify contain the interactivity flag; it
   specifies whether or not further user configuration is desired. The tool may
   initiate interaction with the user regardless of the value for the
   interactivity flag. The intended meaning however is that the tool only
   verifies usability of the task specification (Configuration) and only
   initiates communication when this fails.

   The Configuration type represents a task specification. A minimal task
   specification must have a non-empty list of objects, a category name, a list
   of options and a possibly empty list of outputs. The category gives an
   indication of what functionality of the tool is used to perform the task.
   The options represent the language for task specification. A single option
   is a parametrised entity that represents the smallest part of optional
   configurable behaviour. The list of options identifies a combination of
   functions that make up the task. 

   Before sending a response the tool may initiate arbitrary interaction with
   the user (see subsection \ref{ss:user_interaction}). The user as beneficiary
   is supposed to direct the process of task-configuration if interactive
   communication is the purpose. A response message looks as follows:
   (\textit{Configuration} is reused)

   \begin{table}[H]
    \begin{center}
     \begin{tabular}{|ll|}
      \hline
       \multicolumn{2}{|l|}{\textbf{configuration response}} \\
      \hline
       message type:   & \msg{configuration} \\
      \hline
       direction       & tool to controller \\
       data:           & Validity $\times$ Configuration \\
                       & \ Validity = $\mathbb{B}$ \\
      \hline
     \end{tabular}
    \end{center}
    \vspace{-0.4cm}
   \end{table}

   \noindent The response carries a judgement and the final configuration.
   Depending on the Validity flag the embedded configuration was judged usable
   and is accepted. Notice that the configuration sent in the request is not
   necessarily the same as the configuration in the response. The tool may have
   made changes to the configuration hopefully after having consulting the
   user.

%   It is assumed that the task-configuration that is returned in a response has
%   a strong resemblance to the task-configuration sent in the request. At the
%   very least they should share the same main-input, the rest is up to the
%   creativity of the tool developer.

%   \noindent About the configuration specifications: every option is uniquely
%   identified, and so is every object. The reason is to make it easier for a
%   tool developer to test for availability of options/objects.  An option
%   represents an atomic unit in the configurable behaviour of a tool. For
%   validation purposes a data type can be specified against which the values
%   for the option are matched. An object is a file associated with a format and
%   a location.

  \subsection{Task Execution} \label{ss::task_execution}

   When configuration is complete the controller may start task execution by
   sending a task start command. Configuration is complete when the controller
   receives a configuration response message with an accepted configuration and
   it has not sent a new configuration request.  A task is called \textit{in
   progress} as soon as a task start signal is sent, and as long as no task
   stop signal has been received. See figure \ref{fig:configuration_execution}
   for a schematic overview of the process of task execution.

   A task start command message looks as follows:

   \begin{table}[H]
    \begin{center}
     \begin{tabular}{|ll|}
      \hline
       \multicolumn{2}{|l|}{\textbf{task start command}} \\
      \hline
       message type:   & \msg{task} \\
      \hline
       direction       & controller to tool \\
      \hline
     \end{tabular}
    \end{center}
    \vspace{-0.3cm}
   \end{table}

   \noindent When a tool receives a task start command it must start executing
   the configured task. As task execution completes the tool must send a task stop
   notification as shown in table \ref{table:task_stop}.

   \begin{table}[H]
    \begin{center}
     \begin{tabular}{|ll|}
      \hline
       \multicolumn{2}{|l|}{\textbf{task stop notification}} \\
      \hline
       message type:   & \msg{task} \\
      \hline
       direction       & tool to controller \\
       data            & Result \\
                       & \ Result = $\mathbb{B}$ \\
      \hline
     \end{tabular}
     \label{table:task_stop}
    \end{center}
    \vspace{-0.3cm}
   \end{table}

   \noindent The data in the stop notification signifies success or failure
   of task execution. In case of failure the user should probably be notified
   of the details of the failure using the display or the reporting facility
   both of which will be discussed shortly.
   
%%   A side remark on task
%%   execution: the controller has the responsibility to ensure that the inputs
%%   in a configuration exist prior to starting a task that depends on them. Also
%%   it must ensure that they remain unchanged (by the environment) during task
%%   execution. Likewise, the outputs must not exist or be modifiable by the tool
%%   prior to starting a task, and it must ensure that the environment leaves the
%%   outputs unchanged during task execution.

  \subsection{Reporting}

   The purpose of the reporting facility is to inform the user (through the
   controller) of individual task activities and their progress. A report may
   be send from any context and signifies either a warning, error or just
   notification of some event. The facility is intended as secondary source of
   information (next to the display) that a user may consult to get more
   feedback on configuration or task execution. A report message looks as follows:
   
   \begin{figure}[H]
    \begin{center}
     \begin{tabular}{|ll|}
      \hline
       \multicolumn{2}{|l|}{\textbf{report notification}} \\
      \hline
       message type:   & \msg{report} \\
      \hline
       direction:      & tool to controller \\
       data:           & ReportType $\times$ Description \\
                       & \ ReportType = 'notice' $|$ 'warning' $|$ 'error' \\
                       & \ Description = String \\
      \hline
     \end{tabular}
    \end{center}
   \end{figure}
   \vspace{-0.3cm}

   \noindent This facility is meant as an indirect method of communication with
   the user. The information from the reporting facility ends up in a log that
   is only visible when the user happens to want it that way. So the reporting
   facility must not be relied on as a part of the user regular user interface.
   The reporting facility is intended as additional source of information for
   the user and \emph{not} an exception handling facility for the tool
   developer.

  \subsection{Termination}

   The termination facility allows the controller to terminate a tool in a
   controlled fashion.  In this way a tool is allowed time to free resources
   and remove inconsistent outputs.
   
   \noindent A termination request is a message that looks as follows.

   \begin{figure}[H]
    \begin{center}
     \begin{tabular}{|ll|}
      \hline
       \multicolumn{2}{|l|}{\textbf{termination command}} \\
      \hline
       message type:    & \msg{termination} \\
      \hline
       direction:       & controller to tool \\
      \hline
     \end{tabular}
    \end{center}
   \end{figure}
   \vspace{-0.5cm}

   \noindent The response must be a termination notification, as shown in the
   table below, which signifies that the tool is shutting down and will
   terminate soon. A tool may send a termination notification from any context.

   \begin{figure}[H]
    \begin{center}
     \begin{tabular}{|ll|}
      \hline
       \multicolumn{2}{|l|}{\textbf{termination notification}} \\
      \hline
       message type:   & \msg{termination} \\
      \hline
       direction:      & tool to controller \\
      \hline
     \end{tabular}
    \end{center}
   \end{figure}
   \vspace{-0.5cm}

   \enlargethispage*{30pt}

   \noindent The message represents a general notification of termination and
   may also be sent by a tool in case of emergency when it must terminate
   before completing a task.
   The following figure shows the communication behaviour with regard to
   termination command/notification behaviour.

   \begin{figure}[H]
    \begin{center}
     \includegraphics{termination.eps}
    \end{center}
    \vspace{-0.5cm}
    \caption{termination behaviour}
   \end{figure} 
   
   \noindent When a tool fails to respond to a termination request, the
   controller is assumed to force a tool to terminate by other means.

  \subsection{Display} \label{ss:user_interaction}

   The display facility represents the primary means of a tool to communicate
   with the user.  Think of it as an interactive bulletin board controlled by
   the tool. The bulletin board notifies the tool after any change a user
   makes. Of course to user is not allowed to make arbitrary changes to
   content, he/she can only add or change data where it is the intention of the
   developer of the tool that owns the display.

   The display facility shows an arrangement of widgets (user interface
   primitives, e.g. a button), with which the user can interact. To keep it
   simple, widgets are chosen from a preselected set of the set of very basic
   widgets. A simple relative positioning scheme is used for widget arrangement
   on the space provided by the display. A widget arrangement, also called a
   layout, cannot be manipulated. From a tool perspective, the only
   manipulation possible is the complete replacement of a layout or a change of
   the state of an individual widget the existing layout (the one on display).

   To make use of the display a tool must communicate a set of widgets and a
   layout.  A \textit{layout specification} is a description of a set of
   widgets and how they are positioned relative to each other on the space made
   available by the display. User interaction with widgets on the display
   is relayed to the associated tool immediately after the fact. On the other
   hand a tool can change the internal state of widgets, e.g. change the label
   of a button from "okay" to "cancel". In both cases information from
   individual widgets, called \textit{display data}, is exchanged between the
   communication partners.

   \begin{table}[H]
    \begin{center}
     \begin{tabular}{|ll|}
      \hline
       \multicolumn{2}{|l|}{\textbf{display change command}} \\
      \hline
       message type:   & \msg{display\_layout} \\
      \hline
       direction:      & tool to controller \\
       data:           & LayoutManager \\
                       & \ LayoutManager = BoxLayoutManager \\
                       & \ \ BoxLayoutManager = Direction $\times$ List(Properties $\times$ id $\times$ LayoutElement)) \\
                       & \ \ \ Direction = 'horizontal' $|$ 'vertical' \\
                       & \ \ \ Properties = Visibility $\times$ Input $\times$ Margins $\times$ Alignment \\
                       & \ \ \ \ Visibility = 'visible' $|$ 'hidden' \\
                       & \ \ \ \ Status = 'enabled' $|$ 'disabled' \\
                       & \ \ \ \ Margins = Left $\times$ Top $\times$ Right $\times$ Bottom \\
                       & \ \ \ \ \ Left   = $\mathbb{N}$ \\
                       & \ \ \ \ \ Top    = $\mathbb{N}$ \\
                       & \ \ \ \ \ Right  = $\mathbb{N}$ \\
                       & \ \ \ \ \ Bottom = $\mathbb{N}$ \\
                       & \ \ \ \ Alignment = HorizontalAlignment $\times$ VerticalAlignment \\
                       & \ \ \ \ \ HorizontalAlignment = 'left' $|$ 'centre' $|$ 'right' \\
                       & \ \ \ \ \ VerticalAlignment = 'bottom' $|$ 'middle' $|$ 'top' \\
                       & \ \ \ LayoutElement = LayoutManager $|$ Control \\
                       & \ \ \ \ Widget = ProgressBar $|$ RadioButton $|$ Button $|$ Label $|$ TextField $|$ CheckBox \\
                       & \ \ \ \ \ ProgressBar = $\mathbb{N} \times \mathbb{N} \times \mathbb{N}$ \\
                       & \ \ \ \ \ RadioButton = $\mathbb{B}$ \\
                       & \ \ \ \ \ Button      = String \\
                       & \ \ \ \ \ Label       = String \\
                       & \ \ \ \ \ TextField   = String \\
      \hline
     \end{tabular}
    \end{center}
   \end{table}
   \vspace{-0.4cm}
   \noindent Every widget is associated with an identifier ($id$), that
   \emph{must} uniquely identify it the layout. Communication of changes to the
   state of a widget rely on the identifier as widget specifier.  The $box$ and
   $properties$ elements represent a set of constraints on the relative layout
   on what is contained in the box. A layout consists of set of nested box
   elements that contain widgets.  The horizontal or vertical direction of a
   box determines the way it lays out the elements it contains above or beside
   each other respectively.

   The layout properties further affect layout of elements relative to each
   other or the containing box. Visibility determines whether a layout element
   is visible or not. Input determines whether a widget is active, i.e.
   whether it responds to user interaction. Margins control the distance
   between the directly adjacent elements. For first and last elements this
   means the distance to the borders of the containing box. A box equally
   divides the amount of available space over the available widgets. When
   there is plenty of space after deduction of margins the alignment can be
   used to control either the vertical or horizontal position of a widget 
   within the available space.

   To communicate changes to a widget on the display the following messages
   are used.

   \begin{table}[H]
    \begin{center}
     \begin{tabular}{|ll|}
      \hline
       \multicolumn{2}{|l|}{\textbf{display interaction notification}} \\
      \hline
       message type:   & \msg{display\_data} \\
      \hline
       direction:      & controller to tool \\
       data:           & Id $\rightarrow$ State \\
                       & \ State = String \\
      \hline
     \end{tabular}
    \end{center}
   \vspace{-0.4cm}
   \end{table}

   \noindent When a tool receives such a message it is interpreted as a state
   change of a widget that matches the identifier. When the identifier is not
   known the message must be ignored. Similarly a tool can request a change to
   a widget on the display using the following message.

   \begin{table}[H]
    \begin{center}
     \begin{tabular}{|ll|}
      \hline
       \multicolumn{2}{|l|}{\textbf{display manipulation command}} \\
      \hline
       message type:   & \msg{display\_data} \\
      \hline
       direction:      & tool to controller \\
       data:           & Id $\rightarrow$ State \\
                       & \ State = String \\
      \hline
     \end{tabular}
    \end{center}
   \vspace{-0.4cm}
   \end{table}

   \noindent When a controller receives such a message it is interpreted as a
   state change of a widget that matches the identifier. When the identifier is
   not known the message must be ignored. The state actually depends on the
   type of the widget identified by the identifier and is assumed to be
   representable as a string.  For example the structure of the state of a
   progress bar widget state is $ProgressBar$ or $\mathbb{N} \times \mathbb{N}
   \times \mathbb{N}$ (which can also be represented as String of course).

%    A tool can change the state of widgets on a display by means of their
%    identifier. The following message must be used for this purpose.

   The following figure shows the communication behaviour with regard to the
   use of the display.
   
   \begin{figure}[H]
    \begin{center}
     \includegraphics{display.eps}
    \end{center}
    \vspace{-0.4cm}
    \caption{Process of display manipulation}
   \end{figure}

   \noindent Initially the display shows the empty layout. User interaction
   without widgets is not possible, so display interaction notification
   messages will not be sent by the controller and similarly display
   manipulation requests will not be sent by the tool. A display manipulation
   command is ignored when the controller cannot identify the widget that was
   targeted or when the state is not a valid state description for the targeted
   widget. The display becomes non-empty when a layout change command with a
   valid non-empty layout specification is communicated.  A display interaction
   notification is ignored when the tool cannot identify the widget that was
   targeted or when the state is not a valid state description for the targeted
   widget. 
   
 \section{Implementation details} \label{s:protocol_implementation}

   The previous section specified a high level view of the protocol. At this
   point it should be clear what concepts are involved and how those concepts
   fit together. Communication is based on typed messages. And message
   interpretation depends on the type, and the partner that receives the
   message. This section focusses on implementation details and visits
   important design decisions.
   
   An important design goal was to create a communication interface that can be
   used from a wide variety of programming languages. Meaning that we looked
   only at standardized inter-process communication mechanisms available as
   operating system resource such as sockets and pipes. To further simplify
   the use of the interface from other programming languages it was decided
   to make the protocol text-based using XML \cite{XML-1_0-4}.

%   Not explicitly mentioned so far, but very important in this section is that
%   we rely on the property that the order in which messages are sent is the
%   same as the order in which those messages are received (or actually
%   delivered). Another such property is that communication between partners
%   uses a notion of connection.
   Another important general design decision was to not put focus on protocol
   security aspects. We feel that security as well as performance are key
   aspects in the design of any system. By this we mean that as with
   performance it is often not possible to implement or to simply extend a
   design to add security at a later stage. Putting a focus on security would
   too much slow down the development process. The main goal has been on
   getting a proof-of-concept implementation of a protocol stack on which to
   built the \squadt application.

   The OSI model \cite{Day1983} is a popular way of analysing and describing
   communication protocols. Our use of this model only serves to provide a
   frame of reference. The OSI model divides the communication into subproblems
   (using seven layers) that can be solved independently. Layers 1 through 5
   represent basic functionality covered by widely available standard
   communication protocols. The \squadt application covers the 7th
   (application) layer and our communication protocol covers the 6th
   (presentation) layer.
   
  \subsection{Transport} \label{ss:transport}

   For transport of data layers 1 through 5 of the OSI model the internet (or
   TCP/IP) protocol suite is used (see \cite{rfc793} and \cite{rfc791} for
   TCP(-v4) respectively IP). The TCP/IP protocol provides (bi-directional)
   reliable order preserving delivery of a byte stream (layers 1 through 4 of
   OSI). In addition it creates session functionality, or a stateful connection
   between communication peers which allows one-time identification for tools
   (section \ref{s:instance_identification}).  The use of TCP/IP is ubiquitous
   and usable from a wide variety of programming languages.

%   An alternative is to use
%   standard input/output streams (or piping).  But the main disadvantage of
%   this facility is that there is not built-in support in many programming
%   languages to do non-blocking communication. This is a strict requirement
%   because of the asynchronous nature of the protocol.  A further disadvantage
%   over TCP/IP is that all communication is limited to the same machine.

%   The TCP/IP protocol suite is chosen as the recommended means of
%   transport. It is well-known and supported by a lot of programming languages.
%   The protocol requires connection-state, a tool is authenticated once and on
%   failure the connection is terminated. In addition, because meaning is
%   assigned to the order of messages, the protocol requires that messages are
%   delivered in the same order as which they were offered to the sender. All
%   requirements are all supported by TCP/IP, note that for UDP/IP this is not
%   the case.

  \subsection{Messaging}

   The presentation layer of the OSI model is about mapping between application
   level concepts (with their own syntax and semantics) and data representation
   in communication (data in messages). The topic of this subsection is the
   representation of messages (see section \ref{s:message_definitions}) and
   their interpretation in the domain of the application.
   
%   XMPP has the notion of a message that can be sent from client to client, on
%   top of it our protocol can be implemented.   Unfortunately no single
%   client/server side implementations exist at the time that are usable on all
%   target platforms without also introducing quite a number of other
%   dependencies. Creating our own implementation would have taken to much time,
%   especially when we would have implemented all of XMPP.  So we choose a lightweight
%   custom implementation, instead of using XMPP with 3rd-party client and
%   server implementations.

%   The choice for XML is obvious we aim for extensibility and XML to some
%   extend allows changes to the format extensions to the format there is wide
%   support in many programming languages. With XML it is possible to create a
%   good parser that is to some extent resistant to extensions, meaning it will
%   also work on future versions of the format that may contain additional
%   information.  Another benefit of using XML is it makes the messages
%   readable, making it easy to print and manually verify the structure and
%   contents.

   \subsubsection{Basic Structure: Envelope} \label{ss:structure}

    \noindent Messages are wrapped in the \textit{message} element. A mandatory
    attribute is \textit{type} that specifies the type of the message. The type
    attribute can occur only once and its value must be among those introduced
    in the previous section: \textit{identification}, \textit{capabilities},
    \textit{configuration}, \textit{display\_layout}, \textit{display\_data},
    \textit{termination}, \textit{task},
    \textit{report}. As an example consider the following message with type
    `termination'.
 
    \begin{verbatim}
     <message type="termination"><![CDATA[message content]]></message>\end{verbatim}

    \noindent The content of a message is wrapped in a so-called CDATA section. The
    contents of a CDATA section is treated as character-only data and not
    parsed as markup. This allows embedding arbitrary character data into XML
    documents. To deal between data that contains fragments that match the
    end-marker \verb']]>', any instance of \verb']]>' in the message content
    \textit{must} be replaced by \verb']]]><![CDATA[]>'.
    
    \noindent Message structure will be specified subsequently for all the
    message types and for some of interesting variations. Usually the structure
    of an XML document is (formally) specified using XML Document Type
    Declaration (\cite{Sperberg-McQueen:06:EML}) or the XML Schema standard
    \cite{Malhotra:06:XSP}. An XML Schema specification is available for the
    complete set of messages. Many people find both of them difficult to read
    and understand. The treatment in this section consists of presenting a set
    of XML fragments and text to illustrate translation.

%  \subsubsection{Authentication}

%   Instead of a full-featured authentication scheme we chose a simple instance
%   identification scheme. Of course security is something that should be part
%   of the design of the protocol. Security is not a main concern, so it is
%   postponed until it can be added as an extension somewhere in the future.
   
%   The only authentication that is of importance to the functioning of the
%   system that uses the protocol is identifying the peer as one of the tools
%   that was started. For this purpose the instance identification message,
%   described in section \ref{s:instance_identification} was devised. When
%   starting a tool the system must somehow pass the tool a token that can be
%   used later to uniquely identify it.

  \subsubsection{Capabilities}

   Messages for communicating the capabilities of each of the communication
   partners contain at the very least the version of the protocol. For example:

   \small \begin{verbatim}
  <capabilities>
   <protocol-version major="1" minor="0" />
  </capabilities>\end{verbatim}
  \normalsize

   \noindent A capabilities response message sent in reply to a request for
   capabilities by the controller in addition must contain the set of
   input-configurations. An input-configuration is represented by means of the
   \textit{input-configuration} element. The non-empty set is modelled as an
   non-empty ordered list without duplicates, meaning that elements with the
   same values for the category and format attributes are ignored. For example:

   \small \begin{verbatim}
  <capabilities>
   <protocol-version major="1" minor="0" />
   <input-configuration category="editing">
    <object id="mcrl2_in" format="text/mcrl2" />
   </input-configuration>
   <input-configuration category="visualisation">
    <object id="mcrl2_in" format="text/mcrl2" />
   </input-configuration>
   <input-configuration category="editing">
    <object id="lps_in" format="application/lps" />
   </input-configuration>
  </capabilities>\end{verbatim}
  \normalsize

   \noindent The attribute \textit{category} is mandatory and represents the
   category of functionality associated with this input configuration. The
   \textit{input-configuration} section contains an arbitrary amount of
   \textit{object} sections each representing a source of input. An
   \textit{object} element has two mandatory attributes: \textit{id} a unique
   identifier within the scope of an input-configuration section and
   \textit{format} a MIME-type (see Multipurpose Internet Mail Extensions,
   \cite{rfc2822}) that specifies the type of the input.

   For the example above the interpretation is as follows. As input the tool
   takes a file that either uses the text based format called mcrl2, then
   behaves either as an editor or as a visualiser, or it uses the binary lps
   format and behave as an editor.

  \subsubsection{Configuration} \label{ss:implementation_configuration}

   The purpose of a configuration specification is to differentiate between
   behaviours of a tool in order to allow for selection of behaviour. Depending
   on the developer of a tool different behaviour can be observed from the
   outside. It also depends on the developer to what extend this behaviour can
   be selected before task execution is started.

   Traditionally most programs with a command line interface take arguments
   that are used to (re)produce this state non-interactively by means of a
   command which is a single specially formatted string. The model behind this
   method of expressing a task-specification is based on the idea that tasks
   consist of a sequence of operations or sub tasks each of which can be
   parametrised.  A configuration is a selection between available operations
   and values for the parameters for operations or the combinations thereof. On
   the command line such a configuration is represented as a string that
   consists of so-called options followed by values for the arguments of this
   option. We chose to reuse this model but will not use the same
   implementation.

   The controller is assumed to be oblivious to any data that is produced by
   tools other than the fact that there is a name associated to this data and
   the `type' of the data.  In particular the controller cannot inspect the
   data in order to obtain more knowledge about the data.

   The integration framework is the only party with knowledge about available
   data sources. This knowledge is obtained from configurations and
   communication with the user. By means of a configuration a tool informs the
   framework on what data (and its type) are produced by the task. To represent
   the type of the data the MIME standard is used. The use of MIME is
   wide-spread among desktop applications and in other communication protocols.
   The application of a tool to do a specific task is restricted by the kind of
   data the task requires as input. An example of an initial configuration
   looks as follows.

   \small \begin{verbatim}
  <configuration interactive="true" category="debugging">
   <option id="-v">
    <argument type="integer">1</argument>
   </option>
   <object id="in" type="input" location="/dev/random" format="application/octet-stream"/>
   <object id="out" type="output" location="/tmp/out" format="application/octet-stream"/>
  </configuration>\end{verbatim}
  \normalsize

   \noindent The \textit{category} attribute is mandatory, the interactive
   attribute must be true when it is part of a message that serves as a request
   for interactive configuration.  A \textit{configuration} element may contain
   an arbitrary number of \textit{object} and \textit{option} elements in any
   order, they represent data sources and options respectively.
   
   The \textit{id} attributes of \textit{option} and \textit{object} elements
   represent the identifiers and all must be unique within the context of a
   task specification (the containing \textit{configuration} element).  An
   \textit{object} element must contain the \textit{type} attribute, which
   specifies whether the tool takes it as input or produces it as output. The
   \textit{location} attribute must specify a URI (see \cite{rfc3305}), and the
   \textit{format} attribute contains a data format specifier using the MIME
   standard.  An \textit{option} element may contain an arbitrary number of
   \textit{argument} elements that each represent a single typed-argument to
   the option. A number of predefined types is available for arguments to
   options: string, integer, natural, positive, real.

   A valid (initial) configuration can be obtained from an input configuration
   as follows.  Create an empty \textit{configuration} section and add an
   attribute \textit{interactive} set to true and add the contents of an
   input-configuration section (section \ref{ss:implementation_configuration}).

  \subsubsection{Display}

   Design of a good user interface may require more flexibility such as
   separate manipulation of layout properties. In the previous section display
   manipulation was restricted to replacing the entire content of the display
   at once or modifying the state of individual widgets on the display.  In
   particular it is not possible to manipulate the layout itself. More complete
   manipulation capabilities rapidly increase complexity. A conservative
   approach was chosen to save on implementation time. 

%%   A good example is the combination of HTML,
%%   CSS, and JavaScript because it essentially offers this flexibility by means
%%   of open standards. The complexity of these standards and lack of easy-to-use
%%   standard portable implementations make them unsuitable to directly rely on
%%   at this time.

   We have preselected a limited set of widgets that are available to tools to
   construct display layouts. Again the reason is to limit complexity although
   the display layout construction and manipulation by itself are by far the
   most complex functionality the protocol has to offer.  The downside of this
   limitation of choice is that a tool developer has less control over the
   exact layout of widgets on the display. Instead we have aimed to make it
   easy to implement new widgets and hope to extend the protocol when there is
   a specific demand.
   
%   The available elements are: \textit{box-layout-manager}, \textit{progress-bar},
%   \textit{radio-button}, \textit{button}, \textit{label}, \textit{checkbox},
%   \textit{text\_field}.  A detailed introduction of the individual widgets 
%   comes next followed by a description on how the widgets are positioned by
%   means of a layout specification.

   \paragraph{Widgets}

    Because the use of graphical user interfaces today is so ubiquitous it is
    assumed that the reader is familiar with the purpose and basic functions of
    each of the widgets presented earlier. The name and function of widgets 
    and the way a layout is built are based on concepts and terminology used in
    Java Swing.  Every widget has a mandatory \textit{id} attribute that must
    be unique within the scope of the containing \textit{display-layout} section
    (introduced shortly).
   
    The following example shows a specification for a label, a button and a
    checkbox with text Cancel.
   
    \small \begin{verbatim}
  <label id="x"><![CDATA[Cancel]]></label>
  <button id="y"><![CDATA[Cancel]]></button>
  <checkbox id="z" checked="true"><![CDATA[Cancel]]></checkbox>\end{verbatim}
  \normalsize

    \noindent When the button is pressed or the checkbox is toggled this fact is
    communicated by sending a \msg{display\_data} message with the complete
    respective button or checkbox specification. The checkbox has a description
    and is always in one of two states: checked or not. A more complex example
    is the radio-button widget, because it is not a stand-alone widget.

    \small \begin{verbatim}
  <radio-button id="x" connected="y"><![CDATA[first]]></radio-button>
  <radio-button id="y" connected="x" selected="true"><![CDATA[second]]></radio-button>\end{verbatim}
  \normalsize

   \noindent Radio buttons are always grouped and only a single button in the
   group is selected (pressed).  By default the first radio button is selected,
   if another radio button should be selected then the \textit{select}
   attribute must have value true. The radio buttons are connected by means of
   the \textit{connected} attribute that contains the value of the \textit{id}
   attribute of another radio button in the group. Every radio button in the
   group can be found by repeatedly following the \textit{connected} attribute
   to find the connected radio-button by its identifier. If the selection
   changes then only the specification of the radio button that gets selected
   must be sent by means of a \msg{display\_data} message to inform the other
   side of this event.

   The text field displays an input widget for the user to input text. It is
   like the button widget, but it contains an element \textit{text} that
   holds the content of the widget. The following fragment shows example of
   the specification of a text field.
   
   \small \begin{verbatim}
  <text-field id="x"><text><![CDATA[100]]><text></text-field>\end{verbatim}
  \normalsize

   Input validation may be added in the future. The framework then has the means
   to check and inform the user whether the data entered by the user matches
   the expectations of the tool developer.  For example if the input box should
   contain a number then it can be automatically checked to not contain
   non-digit characters.

   The progress bar is used to show progress to a user. It models progress by
   means of a sub range of the integer domain, specified by a minimum and
   maximum value and shows progress by colouring part of this domain up to some
   `current' value that \emph{must} be in the domain $[ minimum \ldots
   maximum ]$. The example is self explanatory:

   \small \begin{verbatim}
  <progress-bar id="x" minimum="10" maximum="20" current="15"/>\end{verbatim}
  \normalsize

   \noindent Updates to the state of a widget are specified in the same way as
   in the layout specification. The \textit{id} attribute identifies the
   widget of which the state is to be updated. The attributes then specify the
   new value for the attribute with the same name and child elements specify
   other aspects of the state. When attributes are missing, their value remains
   unchanged.

   \paragraph{Layout}

   A display layout specification is represented by a \textit{display-layout}
   element that contains a single \textit{layout-manager} element, called the
   \textit{top layout manager}.

   \small \begin{verbatim}
  <display-layout>
   <layout-manager>
    <box-layout-manager variant="vertical" id="x">
     ...
    </box-layout-manager>
   </layout-manager>
  </display-layout>\end{verbatim}
  \normalsize

   \noindent The layout manager specifies the way in which elements are laid
   out across the display.  The \textit{box-layout-manager} has a
   \textit{variant} attribute that specifies the direction in which the
   elements directly contained in it are laid out on the available space.
   Child elements are laid out on the screen horizontally, or vertically and
   are expanded to fill space. The top layout manager completely fills the
   available space of the display. The following figure shows an example of how
   widgets can be laid out using nested layout managers.

   \begin{figure}[H]
    \begin{center}
     \includegraphics{example_layout.eps}
    \end{center}
    \caption{Example layout with annotation}
   \end{figure}

   \noindent The dotted lines mark the boundaries of space allocated to
   different layout managers at the same nesting level. The picture illustrates
   the use of layout properties and how they influence the layout. A complete
   listing of the XML specification that can be used to generate the layout in
   the picture will be given later after layout properties are discussed in
   more detail.

   A number of layout properties provide more control over how elements are
   positioned and if they are visible and active (enabled/disabled). The
   available properties are: alignment, margins (as pixels) one of (top, right,
   bottom, left), vertical alignment (top, middle, bottom),
   horizontal-alignment (left, center, right), element visibility and element
   activity.
   More attention is in order for element activity. Every widget can be
   enabled or disabled for user interaction. A widget is \textit{active} when
   the user can interact with it. This notion is extended to layout managers
   and thereby arbitrary elements as follows. If a layout manager is inactive
   all elements it widgets are inactive.
   
   Every element that is a child of a layout manager is associated with a value
   for each of the available layout properties. An implicit set of default
   values is assumed that can be used to reduce specification size.
   The default properties are as follows: alignment is left, no margins,
   elements are enabled and visible. The effective properties of an element are
   relative to that of the previous child. For example:

   \enlargethispage{-\baselineskip}

   \small \begin{verbatim}
 <box-layout-manager variant="vertical" id="">
  <properties margin-top="1" margin-bottom="1" horizontal-alignment="right" />
  <button><![CDATA[Ok]]></button>
  <properties />
  <button><![CDATA[Cancel]]></button>
 </box-layout-manager>\end{verbatim}
 \normalsize

   \noindent The layout properties for both buttons are the same, top and bottom margins
   are one pixel, vertical alignment is middle and horizontal alignment is right
   and both elements are visible and enabled. The \textit{properties} element
   directly preceding a widget specify the layout properties for that widget. If
   there is no properties element then the previous \textit{properties} element
   within the same layout manager defines the layout properties for that
   element.

   The following figures shows a non-trivial layout specification that belongs
   to the overview picture presented earlier.

   \small \begin{verbatim}
 <display-layout>
  <layout-manager>
   <box-layout-manager variant="horizontal" id="top">
    <box-layout-manager variant="vertical" id="top_top">
     <box-layout-manager variant="vertical" id="top_top_left">
      <properties margin-left="5" margin-top="10">
      <label id="alabel"><![CDATA[This is a label]]></label>
     </box-layout-manager>
     <box-layout-manager variant="vertical" id="top_top_right">
      <properties margin-left="0" margin-top="0" horizontal-alignment="center">
      <box-layout-manager variant="vertical" id="top_top_right_top">
       <properties horizontal-alignment="center">
       <button id="abutton"><![CDATA[B Button]]></button>
      </box-layout-manager>
      <box-layout-manager variant="vertical" id="top_top_right_bottom">
       <properties horizontal-alignment="left">
       <button id="abutton"><![CDATA[A Button]]></button>
      </box-layout-manager>
     </box-layout-manager>
    </box-layout-manager>
    <box-layout-manager variant="vertical" id="top_bottom">
     <properties horizontal-alignment="center">
     <progress-bar id="progress" minimum="0" maximum="1000" current="350" />
    </box-layout-manager>
   </box-layout-manager>
  </layout-manager>
 </display-layout>\end{verbatim}
  \normalsize

  \noindent Pay careful attention to the use of properties to control margins and
  alignment. Especially the default values of attributes relative to the
  previous \textit{properties} element.

  \subsection{Extensibility}

   Protocol extension is performed by increasing the major version number and
   introducing the changes. The increase in version number is supposed to make
   it easy to test whether additional functionality with respect to previous
   versions is available and/or to signal compatibility mode. The capabilities
   request-response mechanism can be used check finer degrees of compatibility.
   
  \section{Comparison}
  
   How does our approach measure up to other approaches to tool integration?  In
   Eucalyptus, the graphical front-end to CADP, detailed knowledge about the
   capabilities of individual tools as well as file formats seems to be
   integrated. Such coupling is very tight and limits its applicability. We set
   our target somewhat higher. We have chosen to avoid building in knowledge
   about particular file formats or tools.
   
   Other approaches we know of are the electronic tool integration platform
   (ETI) \cite{RICVT} and repository (\cite{SFAV}). The former is a platform
   that uses web-services (using SOAP \cite{SOAP} and WSDL \cite{2001-WSDL})
   in order to loosely connect tools in a way very similar to ours. A tool can
   be connected by means of filling in a web-form that generates an XML file
   that represents the tools' interface.  This is very similar to the XML
   formatted message on tools capabilities. Connected tools are aware of the
   integration context and ETI offers facilities that are usable to tools via
   Java specific remote procedure calls.
   
   Repository also uses web-services but in contrast to ETI, the connection
   between tool and framework is ad-hoc by means of scripts. This puts it
   somewhere in between Eucalyptus and ETI. The tools are not aware of the
   integration context so a script is needed to make a tool behave properly in
   the integration context.
 
   For all of the above approaches it seems that tools communicate through
   files.  A serious consideration on our side was that files could grow very
   big and that you do not want to copy those unnecessarily across a network.
   For this reason the use of web-services as interface between tool was not
   really considered until a protocol implementation was already available.

  \section{Afterthoughts}

   After implementation work had started on a implementation of a protocol
   stack, an existing XML-based protocol called XMPP core came into the view.
   This protocol offers basic messaging functionality including a request
   response mechanism and little more than that.  Quickly summarised: is a
   relatively compact open communication protocol that that relies on TCP/IP to
   transport two XML streams (one for each direction), also see
   \cite{Sperberg-McQueen:06:EML}.  At some point XMPP core (Extensible
   Messaging and Presence Protocol, \cite{rfc3920}) even became an official
   standard. In hind-sight we would have liked to consider the use of XMPP
   core.

   Our messaging functionality is very similar in approach to XMPP. We would
   like to conclude with the message that with a little effort a future
   possible reimplementation of the protocol stack can be based on an XMPP-core
   implementation.

  \enlargethispage*{4pt}
  \bibliography{references}

  \section{Appendix}
  \pagestyle{empty}

  \subsection{Behavioural model}
  The following listing shows an mCRL2 model of the combined communication
  behaviour of the protocol. Besides communication tool start and termination
  only communications actions are visible. The model features a single
  controller and a single tool.

  \small \verbatiminput{protocol.mcrl2} \normalsize
%  \rotatebox{90}{\verbatiminput{protocol.mcrl2}}

  \pagebreak

  The following figure is a graphical representation of the communication
  behaviour using the mCRL2 model presented previously. The picture was created
  automatically by first linearising and instantiating to create a labelled
  transition system and minimising that modulo branching bisimulation. The
  minimised labelled transition system then was visualised as a labelled graph
  and the resulting picture is the result of some manual polishing.

  \begin{figure}[H]
   \includegraphics[width=\textheight-3.5cm,angle=90]{protocol.eps}
   \caption{Graphical representation of the state-space}
  \end{figure}

  \subsection{XML Schema for Messages}

  A formal specification of the syntax of messages is given by the following
  XML schema listing. The main element is a message with a mandatory attribute
  type. The value of the message type attribute determines the contents of the
  message.

  \small \verbatiminput{protocol.xsd} \normalsize
%   The purpose of the protocol is make it possible for a user to control of a
%   tool through facilities offered by a separate system that acts as
%   intermediary. The protocol describes how a tool can be configured to perform
%   a task, how it can be made to perform this task and report the results.  A
%   task specification comes into being by interaction with the user through the
%   system. The protocol describes this process.  The most important
%   functionality offered by the protocol is repeatability of the configuration
%   process based on a previous configuration.
   
%   The deskSQuADT application currently uses the protocol as its only method
%   for controlling tools. The most important features provided by the current
%   version are:
%    \begin{itemize}
%     \item dependencies generated by application of tools on files are
%     visualised
%     \item change propagation through (semi-)automated task execution for
%     repeating tasks; changes in input are detected and tools are re-executed
%     on request to ensure up-to-date outputs
%     \item data consistency is guarded by avoiding concurrent execution of
%     tasks that share inputs or outputs
%    \end{itemize}
%   All of these features are the result of functionality purposefully built
%   into the protocol.
   
%   Experience so far has told us that the current ability to fill and
%   manipulate the display is rather limited. Interaction with the user would
%   improve with a broader choice in controls and more fine-grained control over
%   the layout. A good example of useful additional control over layout would be
%   hiding or disabling controls in a layout when they are not needed.

%   Since the decision was made to create a custom implementation for the
%   protocol, see section \ref{s:protocol_implementation}, the XMPP protocol has
%   been formalised by the Internet Engineering Task Force (IETF). This means it
%   is now an open internet standard. Because he notion of a message is
%   approximately the same for the protocols, It was (and remains) an option to
%   use our protocol on top of XMPP.

%   It is still interesting to consider implementing this protocol on top of
%   XMPP. The latter has additional functionality that can be used to help solve
%   other tool integration problems. An example of this is active communication
%   between multiple tools through a publish-subscribe mechanism.  A future
%   extension to the protocol or perhaps even another protocol can offer such
%   functionality.

\end{document}
