%%%%%%%%%%%%%%%%%%%%%%%%%%%%%%%%%%%%%%%%%%%%%%%%%%%%%%%%%%%%%%%%%%
%%%  MTLPEs                                                    %%%
%%%%%%%%%%%%%%%%%%%%%%%%%%%%%%%%%%%%%%%%%%%%%%%%%%%%%%%%%%%%%%%%%%
\documentclass[twoside,fleqn,a4paper,dvips]{article}
\usepackage{textcomp,amsmath,amssymb,amsthm,stmaryrd}
\usepackage{geometry}
\usepackage[ps2pdf]{hyperref} % remove for printing
%\usepackage[active]{srcltx}
\usepackage{mymath,mythm,mcrl}
  
\setlength{\mathindent}{0.5em}

\def\YSU{\texttt{ YSU: TODO :USY\marginpar{!!!} }}
\def\JFG{\texttt{ JFG: TODO :GFJ\marginpar{!!!} }}
\newcommand{\afunc}[3]{\mathsf{#1}:#2\rightarrow#3}

\title{LPE definitions and an ATerm representation format for \mcrl\ 
  with multiactions and time}
  
\author{Jan Friso Groote \and Yaroslav S. Usenko}

\begin{document}
\maketitle

\section{LPE definitions}

The equation below represents a Linear Process Equation for \mcrl\ 
with multiactions and time (MTLPE).
\begin{gather}
\begin{split}
\pr{X}(\vect{d\ap D})=&\sum_{i\in I}\sum_{\vect{e_i\ap E_i}} c_i(\vect{d,e_i})\to\act{a}^0_i(\vect{f_{i,0}}(\vect{d,e_i}))\comm\dots\comm\act{a}^{n(i)}_i(\vect{f_{i,n(i)}}(\vect{d,e_i}))\at t_i(\vect{d,e_i})\seqc
\pr{X}_i(\vect{g_i}(\vect{d,e_i}))\notag\\
&+\sum_{j\in J}\sum_{\vect{e_j\ap E_j}} c_j(\vect{d,e_j})\to\act{a}^0_j(\vect{f_{j,0}}(\vect{d,e_j}))\comm\dots\comm\act{a}^{n(j)}_j(\vect{f_{j,n(j)}}(\vect{d,e_j}))\at t_j(\vect{d,e_j})\\
&+\sum_{\vect{e_\delta\ap E_\delta}} c_\delta(\vect{d,e_\delta})\to\delta\at t_\delta(\vect{d,e_\delta})
\end{split}
\end{gather}
where $I$ and $J$ are disjoint.

It is possible to translate multiactions to regular \mcrl\ actions
(with longer parameter lists).  In this way a MTLPE can be translated
to a TLPE, preserving equivalence.  The TLPE that corresponds to the
above MTLPE is defined in the following way.
\begin{gather}
\begin{split}
\pr{X}(\vect{d\ap D})=&\sum_{i\in I}\sum_{\vect{e_i\ap E_i}} c_i(\vect{d,e_i})\to\act{a}^0_i\_\act{a}^1_i\_\dots\_\act{a}^{n(i)}_i(\vect{f_{i,0}}(\vect{d,e_i}),\dots,\vect{f_{i,n(i)}}(\vect{d,e_i}))\at t_i(\vect{d,e_i})\seqc
\pr{X}_i(\vect{g_i}(\vect{d,e_i}))\notag\\
&+\sum_{j\in J}\sum_{\vect{e_j\ap E_j}} c_j(\vect{d,e_j})\to\act{a}^0_j\_\act{a}^1_j\_\dots\_\act{a}^{n(j)}_j(\vect{f_{j,0}}(\vect{d,e_j}),\dots,\vect{f_{j,n(j)}}(\vect{d,e_j}))\at t_j(\vect{d,e_j})\\
&+\sum_{\vect{e_\delta\ap E_\delta}} c_\delta(\vect{d,e_\delta})\to\delta\at t_\delta(\vect{d,e_\delta})
\end{split}
\end{gather}
where $I$ and $J$ are disjoint, and
$\act{a}^0_i\_\act{a}^1_i\_\dots\_\act{a}^{n(i)}_i$ and
$\act{a}^0_j\_\act{a}^1_j\_\dots\_\act{a}^{n(j)}_j$ are new actions
(for each $i$ and $j$), parameterized by the concatenation of the
parameter lists of the contained actions.

\YSU formalize below

\begin{thm}
  Given 
  $\mathrm{MTLPE1}=(\mathrm{genspect})=\mathrm{MTLPE2}$,\\
  $\mathrm{TLPE(MTLPE1)}=(\mathrm{timed\ mcrl})=\mathrm{TLPE(MTLPE2)}$.
\end{thm}

Time can be eliminated from TLPEs in a similar way (see page 106 of
the thesis).

\appendix

\section{ATerm representation format for MTLPEs}
A MTLPE is stored as an ATerm with the following functions. The sort
of stored MTLPE is $\nm{MTLPE}$.

\begin{gather*}
\afunc{spec2gen}{\nm{DataTypes}\x\nm{ActionSpec}^*\x\nm{InitProcSpec}}{\nm{MTLPE}}\\
\afunc{actspec}{\nm{String}\x\nm{String}^*}{\nm{ActionSpec}}\\
\afunc{initprocspec}{\nm{Term}^*\x\nm{Variable}^*\x\nm{Summand}^*}{\nm{InitProcSpec}}\\
\afunc{smd}{\nm{Variable}^*\x\nm{Action}^*\x\nm{Term}\x\nm{NextState}\x\nm{Term}}{\nm{Summand}}\\
\afunc{act}{\nm{String}\x\nm{Term}^*}{\nm{Action}}\\
\afunc{terminated}{}{\nm{NextState}}\\
\afunc{i}{\nm{IndexedTerm}^*}{\nm{NextState}}\\
\afunc{d}{\nm{Signature}\x\nm{Equation}^*}{\nm{DataTypes}}\\
\afunc{e}{\nm{Variable}^*\x\nm{Term}\x\nm{Term}}{\nm{Equation}}\\
\afunc{v}{\nm{String}\x\nm{String}}{\nm{Variable}}\\
\afunc{it}{\Nat\x\nm{Term}}{\nm{IndexedTerm}}\\
\afunc{dc}{\Nat}{\nm{IndexedTerm}}\\
\afunc{s}{\nm{String}^*\x\nm{Function}^*\x\nm{Function}^*}{\nm{Signature}}\\
\afunc{f}{\nm{String}\x\nm{String}^{*}\x\nm{String}}{\nm{Function}}
\end{gather*}
The sort $\nm{Term}$ consists of arbitrary ATerm terms where all function
symbols must be quoted. The sort $\nm{String}$ consists of quoted constants,
i.e.\ function symbols of arity 0. The sort $\Nat$ is the built in
sort of natural numbers in the ATerm library. The list
of elements of sort $D$ is denoted by $D^{*}$.

The constructor of sort $\nm{InitProcSpec}$ contains the actual LPE
parameters (from \texttt{init}) as the first parameter, the formal LPE
parameters as the second argument, and the list of summands as the
third parameter. The third parameter of $\mathsf{cmd}$ is the term of
sort $\nm{Time}$ representing the time at which the multiaction
happens.  The last parameter of $\mathsf{cmd}$ is the boolean term
representing the condition.

If the delta summand of the TLPE is present, $\delta$ has to be
represented by the ATerm string \texttt{"Delta"}, and actions with
this name should not be allowed.  An alternative is in using a special
summand construction.

\newpage
\section{ATerm representation format for LPEs (for \mcrl\ v1)}
An LPE is stored as an ATerm with the following functions. The sort
of stored LPE is $\nm{LPE}$.

\begin{gather*}
\afunc{spec2gen}{\nm{DataTypes}\x\nm{InitProcSpec}}{\nm{LPE}}\\
\afunc{initprocspec}{\nm{Term}^*\x\nm{Variable}^*\x\nm{Summand}^*}{\nm{InitProcSpec}}\\
\afunc{smd}{\nm{Variable}^*\x\nm{String}\x\nm{Term}^*\x\nm{NextState}\x\nm{Term}}{\nm{Summand}}\\
\afunc{terminated}{}{\nm{NextState}}\\
\afunc{i}{\nm{Term}^*}{\nm{NextState}}\\
\afunc{d}{\nm{Signature}\x\nm{Equation}^*}{\nm{DataTypes}}\\
\afunc{e}{\nm{Variable}^*\x\nm{Term}\x\nm{Term}}{\nm{Equation}}\\
\afunc{v}{\nm{String}\x\nm{String}}{\nm{Variable}}\\
\afunc{s}{\nm{String}^*\x\nm{Function}^*\x\nm{Function}^*}{\nm{Signature}}\\
\afunc{f}{\nm{String}\x\nm{String}^{*}\x\nm{String}}{\nm{Function}}
\end{gather*}
The sort $\nm{Term}$ consists of arbitrary ATerm terms where all function
symbols must be quoted. The sort $\nm{String}$ consists of quoted constants,
i.e.\ function symbols of arity 0. The list
of elements of sort $D$ is denoted by $D^{*}$.

The constructor of sort $\nm{InitProcSpec}$ contains the actual LPE
parameters (from \texttt{init}) as the first parameter, the formal LPE
parameters as the second argument, and the list of summands as the
third parameter. The last parameter of $\mathsf{cmd}$ is the boolean term
representing the condition.


%\bibliographystyle{plain}
%\bibliography{genspect}

\end{document}
