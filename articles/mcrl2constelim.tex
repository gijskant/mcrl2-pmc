\documentclass[a4paper,9pt]{article}
\usepackage{vmargin}

\setmarginsrb{1.5cm}{2cm}{1.5cm}{2cm}{0cm}{0cm}{0cm}{1cm}

%opening
\title{mCRL2-constelim \\ DRAFT}
\author{Frank Stappers}

\begin{document}

\maketitle

\begin{abstract}
This documentation describes the usage and implementation of  the tool \verb"constelim" within the mCRL2 toolset. 
Basically, \verb"constelim" is a tool which eliminates constant parameters in a linear process equation (LPE).
\end{abstract}

\section{Description} \label{sec:dec}

\subsection{Algorithm}
Before the algorithm is given we declare some of the used keywords.
\begin{itemize}
\item \textit{init vector} \\ A vector which is constructed to detect changes during during each iteration. 
 The elements in the vector represent the values of the process parameter.
\item \textit{flag vector} \\ This vector is a representation for indicating which elements of the \textit{init vector} 
are constant and variable. 
\item \textit{state vector} \\ This vector is calculated from the \textit{init vector}. 
Each summand of the LPE has is own state vector.
\item \textit{change vector} \\ This vector is constructed from the \textit{init vector}, \textit{flag vector} 
and the \textit{state vector}. This vector is a representation for indicating which element of the \textit{state vector}
are constant and variable. 
\end{itemize}

\textbf{The Algorithm}
\begin{enumerate}
\item[\textbf{1}] Construct the \textit{init vector} out of the \verb"init" process.
All variables which are not included in the \textit{init vector} are free variables and marked as don't care.
\item[\textbf{2}] Mark the process parameters form the \textit{init vector} \textit{constant} and 
store the result in \textit{flag vector}.
\item[\textbf{3}] Rewrite all data arguments of \textit{constant} marked process parameters.
\item[\textbf{4}] Calculate the for each summand, which the guard evalutes true, its \textit{state vector} and 
for each \textit{state vector} its \textit{change vector}.
\item[\textbf{5}] Construct out of all calculted \textit{state vectors} the new \textit{init vector}.
Construct out of all the corresponding \textit{change vectors} the new \textit{flag vector}. 
If the newly constructed \textit{flag vector} 
differs from the old \textit{flag vector} substitute the old \textit{flag vector} with new \textit{flag vector} and
 substitute the old \textit{init vector} with the new  \textit{init vector}. 
 If the old and the \textit{flag vector} differs continu with steps 3, 4 and 5.
 If the \textit{change vector} does not change from the \textit{flag vector} continue to the next step.
\item[\textbf{6}] Substitute all parameters which are marked \textit{constant} with the given constant values and remove
the parameters from the LPE.
\end{enumerate}

\subsection{Decisions}

If a process parameter becomes flagged \textit{variable} the value of this parameter is set to an arbitray value.
To indicate that for process $x$ a arbitrary value can be chosen we introduce the notation $x_{nc}$.
    
%of the variable $x$ to $x_{nc}$. This means an value is already assigned and cannot act as a don't care value.
% If a $x_{nc}$ is substituted this will always change that process parameter flag to \textit{variable}. 
%If a \_ is substituted the process parameter flag stays unchanged. \\ 
If a process parameter is flagged \textit{constant} and there are multiple assignments (e.g. i:=j $\wedge$ i:=\_) 
the assignment with the most specific value will be used. \\
Let $v_{1}$ be a value of the process parameter of a \textit{init vector} and let $v_{2}$ be a value of a 
\textit{state vector} of a summand. 
Shorthand we denote \textit{constant} as $C$ and \textit{variable} as $V$.
The rules for constructing the \textit{change vector} on the values upon $v_{1}$ and $v_{2}$ are:

\begin{enumerate}
\item $ v_{1} = v_{2} \rightarrow C $
\item $ v_{1} \neq v_{2} \rightarrow V $
\item $ v_{1} = v_{2_{nc}} \vee v_{2} == v_{2_{nc}} \rightarrow V$
\item $ v_{1} = \_  \vee v_{2} = \_ \rightarrow C $
\end{enumerate}
Note: The rule with the lowerst number has the highest priority.

\subsubsection*{Flag Rules} 
Initially all process parameters are flagged \textit{constant}.
If the value of process parameter is equal to a \textit{don't care} the corresponding flag stays unchanged.
If process parameter $x$ has a value equal to $x_{nc}$ the corresponding flag is set \textit{variable}.
If a process parameter after a rewriting in not equal to its initial value the corresponding flag is set \textit{variable}.
In case the value of the process parameter is not changed during rewriting the corresponding flag stays unchanged.
When comparing flags we use these rules: 
\begin{tabbing}
\verb"    " \= C $\wedge$ C $\rightarrow$ C \\
\> C $\wedge$ V $\rightarrow$ V \\
\> V $\wedge$ C $\rightarrow$ V \\
\> V $\wedge$ V $\rightarrow$ V \\
\end{tabbing}

Variables which act as a \textit{don't care} are denoted with a "\_". 
Because the rules of changed process parameters differ from \textit{don't care} variables we introduce 
a special variable $x_{nc}$. This is a special value which indicates a that variable $x$ is changed. 
When the algorithm detects a changed variable $x$, it indicates the variable with the special 
value $x_{nc}$. $x_{nc}$ acts like a sort of flag. $x_{nc}$ differs from a \textit{don't care} value. 

\section{Examples}
\subsection{Example 1}
To illustrate how constelim behaves we look at the example below:
\begin{verbatim}
act  action: Nat;

proc P(i,j: Nat) =
       true -> action(i) . P(i := i + Pos2Nat(1));

init P(i := 0, j := 0);
\end{verbatim}

First we write the construct the  \textit{init vector} form \verb"init": $\langle 0 ,0 rangle$. 
Next we mark all  process parameters to be \textit{constant}. This gives us the following \textit{flag vector}: $\langle$C,C$\rangle$.
We now process all data arguments of marked process parameters. 
We construct \textit{state vectors} for each summand which yields $true$. In this example we have only one summand which always yield $true$, so we get the following \textit{state vector}: $\langle 1, 0 \rangle$. Next we compute the \textit{state vector}. The state vector will result in: $\langle V, C \rangle$
If the the \ldots over all \textit{change vectors} differs from the \textit{flag vector}, we compute the new \textit{init} and \textit{flag vector} and continue with steps 3, 4, and 5 in the algorithm. We keep on doing until the \textit{change vector} does not differ from the \textit{flag vector}. The iteration terminates. In our case we then have to substitute \verb"j" by \verb"0".\\

The tabular below will show the taken computable steps: \\
\begin{tabular}{|l|l|l|l|l|l|}
\hline
step 	&  init vector 			& flag vector			& state vector 			& change vector 		& variables \\
\hline	
\hline
1 	&  $\langle 0,0 \rangle$ 	& \_				& \_				& \_				& $i=0, j=0 $ \\
\hline
2	&  $\langle 0,0 \rangle$ 	&  $\langle C,C \rangle$ 	& \_				& \_				& $i=0, j=0 $\\
\hline
3	&  $\langle 0,0 \rangle$ 	&  $\langle C,C \rangle$ 	&  $\langle i+1,j \rangle$ 	& \_				& $i=0, j=0 $\\
\hline
4	&  $\langle 0,0 \rangle$ 	&  $\langle C,C \rangle$ 	&  $\langle 1,0 \rangle$ 	&  $\langle V,C \rangle$ 	& $i=1, j=0 $\\
\hline
5	&  $\langle i_{nc},0 \rangle$ 	&  $\langle V,C \rangle$ 	&  \_				& \_ 				&  $i= i_{nc} , j=0$\\
\hline
3	&  $\langle   i_{nc} ,0 \rangle$ 	&  $\langle V,C \rangle$ 	&  $\langle   i_{nc}+1 ,0 \rangle$  			&  \_			 	& $i= i_{nc} , j=0$ \\
\hline
4	&  $\langle   i_{nc} ,0 \rangle$ 	&  $\langle V,C \rangle$ 	&  $\langle   i_{nc} ,0 \rangle$	&  $\langle V,C \rangle$  		& $i= i_{nc} , j=0 $\\
\hline
5	&  $\langle   i_{nc} ,0 \rangle$ 	&  $\langle V,C \rangle$ 	&  \_ 					& \_  					& $i= i_{nc} , j=0 $\\
\hline
6	& 					&				&					&					& \verb"j" is substituted by \verb"0" and remove all \verb"j"\\	
\hline
\end{tabular}\\

We get the following LPE:
\begin{verbatim}
    Init P(i:=0, 0);
    P(i,0) = action(i).P(i := i+1, 0); 
\end{verbatim}

\subsection{Example 2}
This example show why we need have to iteration over steps 3, 4 and 5 in the algorithm. We take the following linear process equation:
\begin{verbatim}
    Init P(i:=0, j:=0);
    P(i,j) = action(i).P(i := i+1, j := i); 
\end{verbatim} % \footnote{Generated from: \verb"Init: P(0,0)" and \verb"P(i,i:Nat) = P(i+1,i)"}
This will lead to the following sequence of steps:\\
\begin{tabular}{|l|l|l|l|l|l|}
\hline
step 	&  init vector & flag vector& state vector & change vector & variables \\
\hline
\hline
1	&  $\langle 0,0 \rangle$  		& -				& \_					& \_				& $i=0, j=0 $ \\
\hline
2	&  $\langle 0,0 \rangle$  		&  $\langle C,C \rangle$  	& \_					& \_				& $i=0, j=0 $\\
\hline
3	&  $\langle 0,0 \rangle$  		&  $\langle C,C \rangle$  	&  $\langle i+1,i \rangle$ 		& \_				& $i=0, j=0 $\\
\hline
4	&  $\langle 0,0 \rangle$ 		&  $\langle C,C \rangle$ 	&  $\langle 1,0 \rangle$  		&  $\langle V,C \rangle$ 	& $i=1, j=0 $\\
\hline
5	&  $\langle i_{nc},0 \rangle$ 		&  $\langle V,C \rangle$ 	&  \_			  		&  \_			 	& $i= i_{nc} , j=0$ \\
\hline
3	&  $\langle   i_{nc} ,0 \rangle$  		&  $\langle V,C \rangle$ 	&  $\langle   i_{nc}+1 , i_{nc}$		&  \_			  	& $i= i_{nc} , j=0$ \\
\hline
4	&  $\langle   i_{nc} ,0 \rangle$  		&  $\langle V,C \rangle$  	&  $\langle   i_{nc} , i_{nc}   \rangle$  	&  $\langle V,V \rangle$  	& $i= i_{nc} , j= i_{nc}  $\\
\hline
5	&  $\langle   i_{nc} ,j_{nc} \rangle$  		&  $\langle V,V \rangle$  	&  \_  						&  \_  				& $i= i_{nc} , j= j_{nc}  $\\
\hline
3 	&  $\langle   i_{nc} , j_{nc}   \rangle$  	&  $\langle V,V \rangle$ 	&  $\langle   i_{nc}+1 , j_{nc}   \rangle$  	&  \_			  	& $i= i_{nc} , j= j_{nc}  $\\
\hline
4	&  $\langle   i_{nc} , j_{nc}   \rangle$  	&  $\langle V,V \rangle$  	&  $\langle   i_{nc} , j_{nc}   \rangle$  	&  $\langle V,V \rangle$  	& $i= i_{nc} , j= j_{nc} $ \\
\hline
5	&  $\langle   i_{nc} , j_{nc}   \rangle$  	&  $\langle V,V \rangle$  	&  \_					  	&  \_			 	& $i= i_{nc} , j= j_{nc} $ \\
\hline
6 	& 			 			& 				& 						&				& No substitution\\
\hline
\end{tabular} \\

Now we can see why the iteration of steps 3, 4 and 5 of the algorithm are needed. The linear process equation cannot be improved by \verb"constelim".

\subsection{Example 3} 
Next we will study a case with free variables. We take the following linear process equation:
\begin{verbatim}
    act  action: Nat;

    var  k,l,m: Nat;
    proc P(state: Pos, j,i: Nat) =
           state == 1           -> action(j) . P(state := 1, j := j, i := k)
         + state == 2 && i <  5 -> action(i) . P(state := 2, j := l, i := i + Pos2Nat(1))
         + state == 2 && i == 5 -> action(i) . P(state := 1, j := i, i := m);

    var  n: Nat;
    init P(state := 2, j := n, i := 0);
\end{verbatim}
If we take a close look to this example we can see we see that \verb"n" is not initialized. No particular value is assigned, so we deal with a free variable. \newline
This will lead to the following sequence of steps:\\
\begin{tabular}{|l|l|l|l|l|l|}
\hline
step & init vector & flag vector & state vectors & change vector & variables \\
1	& $\langle 2,\_,0\rangle$ 		& \_				& \_ 										& \_				& $state=2, i=0, j=\_, k=\_, l=\_, m=\_, n=\_$ \\
\hline
2	& $\langle 2,\_,0\rangle$ 		& $\langle C,C,C\rangle$	& \_ 										& \_ 				& $state=2, i=0, j=\_, k=\_, l=\_, m=\_, n=\_$ \\
\hline
3	& $\langle 2,\_,0\rangle$ 		& $\langle C,C,C\rangle$	& $\{state=1\}\rightarrow \langle 1, j, k \rangle$				& \_				& $state=2, i=0, j=\_, k=\_, l=\_, m=\_, n=\_$ \\
	&					&				& $\{state=2 \wedge i < 5 \} \rightarrow \langle 2, l , i+1 \rangle$		& 				& \\
	&					&				& $\{state=2 \wedge i =5 \} \rightarrow \langle 1, i, m \rangle$		& 				& \\
\hline
4	& $\langle 2,\_,0\rangle$ 		& $\langle C,C,C\rangle$	& $false \rightarrow \delta$							& \_				& \_ \\
	&					&				& $\langle 2,\_,1\rangle$							& $\langle C,C,V\rangle$	& $ \bullet state=2, i=1, j=\_, k=\_, l=\_, m=\_, n=\_$ \\
	&					&				& $false \rightarrow \delta$							& \_				& \_ \\
\hline
5	&$\langle 2,\_,i=i_{nc}\rangle$		& $\langle C,C,V\rangle$	& \_										& \_				& $ state =2, i=i_{nc}, j=5, k=\_, l=\_, m=\_, n=\_$ \\
\hline
3	&$\langle 2,\_,i=i_{nc}\rangle$		& $\langle C,C,V\rangle$	& $\{state=1\}\rightarrow \langle 1, j, k \rangle$				& \_				& $ state =2, i=i_{nc}, j=5, k=\_, l=\_, m=\_, n=\_$ \\
	&					&				& $\{state=2 \wedge i < 5 \} \rightarrow \langle 2, l , i+1 \rangle$		& 				& \\
	&					&				& $\{state=2 \wedge i =5 \} \rightarrow \langle 1, i, m \rangle$		& 				& \\
\hline
4	& $\langle 2,\_,i=_{nc}\rangle$ 	& $\langle C,C,V\rangle$	& $false \rightarrow \delta$							& \_				& \_ \\
	&					&				& $\langle 2,\_,i_{nc}\rangle$							& $\langle C,C,V\rangle$	& $ \bullet state=2, i=i_{nc}, j=\_, k=\_, l=\_, m=\_, n=\_$ \\
	&					&				& $\langle 1,5,\_ \rangle$							& $\langle V,C,C\rangle$	& $ \bullet state=1, i=i_{nc}, j=5, k=\_, l=\_, m=\_, n=\_$ \\
\hline
5	&$\langle state_{nc},5,i=i_{nc}\rangle$	& $\langle V,C,V\rangle$	& \_										& \_				& $ state = state_{nc}, i=i_{nc}, j=5, k=\_, l=\_, m=\_, n=\_$ \\
\hline
3	&$\langle state_{nc},5,i=i_{nc}\rangle$	& $\langle V,C,V\rangle$	& $\{state=1\}\rightarrow \langle 1, j, k \rangle$				& \_				& $ state = state_{nc}, i=i_{nc}, j=5, k=\_, l=\_, m=\_, n=\_$ \\
	&					&				& $\{state=2 \wedge i < 5 \} \rightarrow \langle 2, l , i+1 \rangle$		& 				& \\
	&					&				& $\{state=2 \wedge i =5 \} \rightarrow \langle 1, i, m \rangle$		& 				& \\
\hline
4	&$\langle state_{nc},5,i=i_{nc}\rangle$	& $\langle V,C,V\rangle$	& $\langle 1,\_,\_ \rangle $							& $\langle V,C,V\rangle$	& $ \bullet state=state_{nc}, i=i_{nc}, j=\_, k=\_, l=\_, m=\_, n=\_$ \\ 
	&					&				& $\langle 2,\_,i_{nc}\rangle$							& $\langle C,C,V\rangle$	& $ \bullet state=state_{nc}, i=i_{nc}, j=\_, k=\_, l=\_, m=\_, n=\_$ \\
	&					&				& $\langle 1,5,\_\rangle$							& $\langle V,C,C\rangle$	& $ \bullet state=state_{nc}, i=i_{nc}, j=5, k=\_, l=\_, m=\_, n=\_$ \\
\hline
5	&$\langle state_{nc},5,i=i_{nc}\rangle$	& $\langle V,C,V\rangle$	& \_										& \_				& $ state = state_{n}, i=i_{nc}, j=5, k=\_, l=\_, m=\_, n=\_$ \\
\hline
6	&					&				&										&				&  subtitute \verb"j:=5" and remove \verb"j" \\
\hline	
\end{tabular}
The resulting algorithm:
\begin{verbatim}
    act  action: Nat;

    var  k,l,m: Nat;
    proc P(state: Pos, i: Nat) =
           state == 1           -> action(5) . P(state := 1, i := k)
         + state == 2 && i <  5 -> action(i) . P(state := 2, i := i + Pos2Nat(1))
         + state == 2 && i == 5 -> action(i) . P(state := 1, i := m);

    var  n: Nat;
    init P(state := 2, i := 0);
\end{verbatim}

\end{document}
