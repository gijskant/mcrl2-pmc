% ++value of free variables may only be have one value on the righthand side
% constelim can reduce the  statespace if there are free variables in it (true?)
% initial values are don't care
% 

\documentclass[a4paper,10pt]{article}
\usepackage{textcomp,amsmath,amssymb,amsthm}%,stmaryrd}
%\usepackage{geometry}
%\usepackage[ps2pdf]{hyperref} % remove for printing
%\usepackage[active]{srcltx}
\usepackage{mymath,mythm}

\newcommand{\lpe}{linear process equation}
\newcommand{\tool}{\textit{lpeconstelm}}
\newcommand{\ovr}{\overrightarrow}
\newcommand{\mcrl}{mCRL2}
\newcommand{\framework}{\textit{LPE framework} \cite{LPEframework}}
\newcommand{\pp}{process parameter}
\newcommand{\pps}{process parameters}
\newcommand{\ti}{\textit}
\newcommand{\tb}{\textbf}
\newcommand{\sv}{\textit{state vector}}
\newcommand{\fv}{\textit{flag vector}}
\newcommand{\svs}{\textit{new state vector set}}
\newcommand{\nfv}{\textit{new flag vector}}
\newcommand{\la}{$\leftarrow$}
\newcommand{\ra}{$\rightarrow$}
\newcommand{\sq}{$\square$}

\newcommand{\csvs}{new\_state\_vector\_set}
\newcommand{\cfvs}{new\_flag\_vector\_set}


%opening
\title{lpeconstelm \\ DRAFT \\ work in progress}
\author{F.P.M. Stappers}

\begin{document}

\maketitle

\begin{abstract}
This documentation describes the usage and implementation of the tool \tool\ within the \mcrl\ toolset.
Basically, \tool\ is a tool which eliminates constant parameters in a \lpe\ (LPE).
\end{abstract}

\tableofcontents

\section{Introduction}
This \tool\ tool is a tool for the \mcrl\ studio. The tool is a
filter which reads form a \ti{input.lpe}. The file \ti{input.lpe} is
a file in \ti{.lpe} format \cite{LPEformat}. We make use of the
\framework\ to read the \ti{input.lpe}. For
each constant \pp\ The filter substitutes its constant value and removes the \pp\ from the
\lpe . After the algorithm (Section \ref{sec:alg}) terminates, \tool\
will write the output to an output file \ti{output.lpe} in the \ti{.lpe} format \cite{LPEformat}.

\section{definitions} \label{sec:def}

The equation below is a simplified\footnote{Multiactions and time are not taken into account} representation of a \lpe\ in \mcrl\ .
\begin{defn}[\lpe]\label{def:LPE}
\begin{displaymath}
X (\ovr{d}:\ovr{D}) = \sum_{i \in I} \sum_{e_i:E_i} \ovr{c_i} ( \ovr{d}, e_i ) \rightarrow a(\ovr{f_i}(\ovr{d},e_i)) \cdot X(\ovr{g_i}(\ovr{d},e_i)) $$
$$ X(\ovr{init})
\end{displaymath}\\
A \lpe\ is written as in Definition \ref{def:LPE}. The different states are $\ovr{d}$ of the process are represented by the data vector parameter $\ovr{d}:\ovr{D}$. $\ovr{D}$ may be a Cartesian product of $n$ data types, meaning that $\ovr{d}$ may consist of a tuple $(d_1, \ldots, d_n)$. The \lpe expresses that in state $\ovr{d}$ it preforms action $a$, carrying data parameter $\ovr{f_i}(\ovr{d},e_i)$ and it can reach the new state $\ovr{g_i}(\ovr{d},e_i)$ under the condition that $\ovr{c_i}(\ovr{d},e_i)$ is \ti{true}. So for each summand $i$ from $I$ we have a function $\ovr{g_i}: \ovr{D} \times E \rightarrow \ovr{D}$ and a function $\ovr{c_i}: \ovr{D} \times E \rightarrow \mathbb{B}$
%, which reduces to a normal form \ti{true}, \ti{false} or something else. 

The data type $E$ helps to give the \lpe s a more general form. Not only the data parameter $\ovr{d}:\ovr{D}$, but also the data parameter $e_i: E_i$ can influence the condition $\ovr{c_i}$ and the and the resulting state $\ovr{g_i}$. Such a $e_i$ is often used in a \lpe to do a read or send action range over a domain. 

Free variables are data parameters which are not bound to \pps\ of a \lpe\ but occur in a \lpe .

For an more detailed explanation of \lpe s  we refer to \cite{LPE_info}.
\end{defn}

\subsection{lpeconstelm definition}\label{subsec:lpedef}

In this definition we don't concider free variables.
A parameter of a \lpe\ can be replaced by its initial value if it can be determined that this parameter remains constant throughout any run of the process. The elimination of constant parameters does not reduce the resulting state space, however it may shorten the time needed tot generate a state space from a \lpe. \\
If we have infinite time and space we can inspect each state space and check if a \pp\ changes throughout the the run of the process. We define the reachable state set $R$ for each state ($\ovr{v_0}$). So:
\begin{defn}\label{def:lpe} Let there be an LPE, where\begin{itemize}
\item $I$ is the set of summands.
\item $\ovr{c_i}$ is the condition function of summand $i$.
\item $\ovr{g_i}$ is the next state function of summand $i$.
\item
$R(\ovr{v_0})$ is the smallest set which complies to:
  \begin{itemize}
    \item $\ovr{v_0} \in R(\ovr{v_0})$
    \item For all $i \in I$ and $w_i \in E_i$ if $v \in R(\ovr{v_0})$ and $c_i(v, w_i)$ holds, \\ then $\ovr{g_i}(v, w_i) \in R(\ovr{v_0})$ \\
   \end{itemize}
\end{itemize}

\end{defn}

$S(\ovr{v_0})$ is the set of indices of \pps\ which are constant is defined:

\begin{defn}\label{def:lpe_comp}
$S(\ovr{v_0}) = \lbrace j \in  \lbrace 1, \ldots, n \rbrace \vert \exists_{u : D_j} (\forall_{\ovr{v} \in R(\ovr{v_0})}( v_j = u)) \rbrace $
\end{defn}

 Definition \ref{def:lpe_comp} has to be fully computed  with a minimal fixed point approach. 

In order to computate  Definition \ref{def:lpe_comp}, we have to generate the entire state space for $R(\ovr{init})$. Generating the entire state space can take a lot of time and space. In some cases it can take infinite time and space. If we can make an approximation of definition \ref{def:lpe_comp} we don't have to inspect all states. However by approximating there can be cases in which not all constant parameters can be found. 

A normal form is needed, because we evaluate open terms. 
\begin{defn} 
$NF(p)$ is the normal form of data parameter $p$. 
\end{defn}

%$S \subseteq \lbrace 1, \ldots, n \rbrace$ is the set of indices of constant \pps . 

%If we rewrite a condition which is an open term, it is possible
%that the term will not reduce to a normal form \ti{true} or \ti{false}. If the term reduces to a
%term that is not \ti{true} or \ti{false}, the condition might be \ti{true}. Therefore the summands of the conditions
%which are \ti{false} are not inspected. 

Because $e_i$ is often used in a \lpe\ to let a read or send action range over a domain and it has influence on wether a \pp\ is constant or variable. If such a $e_i$ is not fixed the corresponding \pp\ is always variable.  This is why we assume $e_i$ is fixed and we use $e'$ as a fixed data parameter of $E_i$.
%\begin{defn}
%The subset of $\lbrace 1, \ldots, n \rbrace$, where the elements of $S$ represent the indices of $d_0, \ldots d_1$ constant \pps .
%\end{defn}

\subsection{Proposal 1} \label{sec:prop1}
Let $j : 0 < j \leqslant n $. We assume $d_j$ is in its normal form. If $d_j$ is equal to $NF(\ovr{g_i^j}(\ovr{d},e'))$ for all $i \in I$ we can say that $d_j$ is constant. However some conditions $NF(\ovr{c_i}(\ovr{d},e'))$ might never be \ti{true}, so it useless to compare those $d_j$ with $NF(\ovr{g_i^j}(\ovr{d},e'))$  if $NF(\ovr{c_i}(\ovr{d},e'))$ is not  $"true"$. \\

So we define $S$:
\begin{defn} \label{def:sug1}
\begin{displaymath}
\lbrace   
j \in \lbrace 1, \ldots , n \rbrace \vert \forall_{i\in I} (\forall_{(\ovr{d}: \ovr{D})}((NF(\ovr{c_i}(\ovr{d},e')) = "true")  \Rightarrow (d_j = NF(g_i^j(\ovr{d},e')))))\rbrace
\end{displaymath}
This approximation is very basic and does not cover all possible solutions. If we take example
\ref{cexample:sug1}, we see why.
\begin{example} \label{cexample:sug1} The example \\
\begin{verbatim}
proc P(x : Nat) = (true) -> P(x := 2 * x);
init P(x : = 0);
\end{verbatim}

\end{example}
If we use Definition \ref{def:sug1}, the result is $S = \emptyset$. 
In Definition \ref{def:sug1} it has to hold for every $\ovr{d}$, however
we don't want to take a look for every $\ovr{d}$ but for a given $\ovr{d}$, namely $\ovr{init}$.
%For all $x / 0 $ the outcome of the \pp\ is not constant. However for $x=0$ the \pp\ is constant in this \lpe . 
%If we use $e_i$, we also have to examine the whole statespace. So we have to create another solution which is less relax then definition \ref{def:sug1}.
% { j | forall d : d=2*d }
\end{defn}

\subsection{Proposal 2}
In section \ref{sec:prop1} we see that definition \ref{def:sug1} is not sufficient, so we try to improve that approximation, by giving $\ovr{d}$ a specific value.  

So we define $S(\ovr{d}:\ovr{D})=$:
\begin{defn} \label{def:sug2}
\begin{displaymath}
\lbrace   
j \in \lbrace 1, \ldots , n \rbrace \vert \forall_{i\in I} (NF(\ovr{c_i}(\ovr{d},e')) = "true") \rightarrow d_j = NF(g_i^j(\ovr{d},e'))))\rbrace
\end{displaymath}
%We elimate the problem for checking each state 
We now can instantiate the $S$ with a $\ovr{d}$. If we instantiate $S$ with $\ovr{init}$ this there is a problem. If we take a look at example \ref{cexample:sug2} we can see why:
\begin{example} \label{cexample:sug2} The example \\
\begin{verbatim} 
proc P(x,y: Nat) = (x=0) -> P(x:= 1, y:= 0) +
                   (x=1) -> P(x:= x, y:= 1);
init P(x := 0, y:= 0);
\end{verbatim}

\end{example}
If we use defintion \ref{def:sug2}, we get $S(0,0) = \lbrace 2 \rbrace$. This indicates that the second \pp\ is constant. However the result is not correct. The reason for that is that we only inspect those state spaces which are adjacent to the initial state.
\end{defn}

\subsection{Proposal 3}
If we rewrite a condition which is an open term, it is possible
that the term will not reduce to a normal form \ti{true} or \ti{false}. If the term reduces to a
term that is not \ti{true} or \ti{false}, the condition might be \ti{true} for some values of the free variables in the that term and should be inspected. The summands of the conditions which are \ti{false} are therefore not inspected.
 
If we use solution which is based upon a minimum fixed point, we have to inspect all state spaces in order to verify if a \pp\ is constant or variable. If we start of with a maximum fixed point (all \pps\ are constant) we are intrested in in the largest set in which for a given subset of \pps\ holds that those \pps\ are equal to their next state \pps . Let $j$ be $0 < j \leqslant $ and $\ovr{\otimes_S}$ be a vector generated from subset $S$. If for a all $NF(\ovr{c_i}(\ovr{\otimes_S},e'))$ are not equal to $"false"$, holds that all $\ovr{init}_j$ are equal to$ NF(g_i^j(\ovr{\otimes_S},e'))$, we have found a subset with constant \pps . If we start with the largest subset $S$ and only decrease subset $S$ if the comparison does not hold, we find the largest subset if the comparison holds with subset $S$.
 
This will lead to the Definition \ref{def:sug3} 

\begin{defn} \label{def:sug3} So we are interested in the largest subset:

\begin{flushleft}
$ S \subseteq \lbrace 1, \ldots, n \rbrace $ \\

such that:\\

$\forall_{i \in I} (NF( \ovr{c_i} ( \ovr{\otimes}, e' ) ) \not= "false" \rightarrow \forall_{j \in S}(d_j =  NF( g_i^j(\ovr{\otimes}, e' )))) $
\end{flushleft}
 
\begin{defn} \label{def:otimes} Let $\ovr{\otimes}$ be:

\begin{displaymath}
\ovr{init} \otimes_S \ovr{d}
\end{displaymath}

%\begin{displaymath}
%    \forall_{i:0<i\leqslant n}:\otimes_i = \left\{
%        \begin{array}{l l}
%            i \in S                      & d_i \\
%            i \not\in S                  & v_{0_i} \\
%        \end{array} \right.
%\end{displaymath}
\end{defn}
Let $j$ be $ 0 < j\leqslant n$. The notation is $\otimes_S$ to construct a \sv\ for $S$. 
If $j \not\in S$ then $\otimes_j := d_j$. If $j \in S$ then $\otimes_j := init_j$.

If we compare Definition \ref{def:sug3} to Definition \ref{def:lpe_comp} we see the following: 
$$S_{Def. \ref{def:sug3}} \subseteq S_{Def. \ref{def:lpe_comp}} $$  
For each condition that holds in Definition \ref{def:sug3} there is a condition $c_i$ that holds in Definition \ref{def:lpe_comp}. So for each summand $i$ in Definition \ref{def:sug3} there is also a summand $i$ in Definition \ref{def:lpe_comp}. However not all conditions that hold in Definition \ref{def:lpe_comp} will hold in in Definition \ref{def:sug3}. Therefore 
not all summands $i$ in Definition \ref{def:lpe_comp} will hold in Definition \ref{def:sug3}. Because Definition \ref{def:sug3} is based upon a maximum fixed point and Definition \ref{def:lpe_comp} is based upon a minimum fixed point, we can conclude that Definition \ref{def:sug3} is correct and sufficient.

%However free variables are not taken into account.
\end{defn}

\subsection{Proposal 4}

%Idea proposal: \\
%Use LPE constelim, after termination, build adjacency matrix , and do cross compare on same  


\subsection{Algorithm definitions}
To calculate the set of constant \pps\ we use the algorithm
\ref{def:sug3} Before we describe the algorithm we the introduce
used definitions. 
%These definitions are defined in \ref{sss:algdef}.
%A more general explanation is given in \ref{sss:algdesc}

\begin{defn} \textbf{- state vector} \label{sv}\\
This vector represents the \pps\ of the input of a \lpe . The vector is 
$\langle d_1, \ldots, d_n \rangle$, where $n$ is the is the number
of \pps . 
\end{defn}

\begin{defn} \textbf{- flag vector}\\
This vector of boolean values which represents if \pps\
are constant or variable. If an element of this vector is \ti{true}
this implies that the corresponding \pp\ is possibly constant. If an element
of this vector is \ti{false} this implies that the corresponding
\pp\ is variable.
\end{defn}

\begin{defn} \textbf{- new state vector set}\\
 This set of state vectors represents the result of the
\pps\ for which the normal form of the condition is not \ti{false}
\end{defn} 

\begin{defn}
\textbf{- new flag vector} \\
This vector of boolean values indicates if the
corresponding \pps\ are \ti{constant} or \ti{variable} after comparing each element from the \svs with all the other elements of \svs  and the \sv; and conjunction of the \fv\ and obtained result. 
%\begin{displaymath}
%\langle \cup_{i \in I}: c_i(\ovr{d, e_i}) : (\ovr{\Bbb{B}}) ) \rangle
%\end{displaymath}
\end{defn}

\subsubsection{Description}\label{sss:desc}
In this section the algorithm is explained informal. First \sv\ is constructed from the initial process of the \lpe . Next the \fv\ is "constructed".
% The number of vector elements in this vector is equal to the number of \pps . Each \pp\ corresponds with an element in the \fv .  
All elements in this vector are set \ti{true}.

We start the iteration. We rewrite and evaluated for each summand its condition. We "evaluate" the condition after substituting the values of the \pp\ from the \sv\ in the condition and rewrite it. Only those \pp\ from the \sv\ are substituted for which the corresponding element in the \fv\ is true. 

%After substitution the condition is rewritten. 
If the rewritten condition is in a normal form other then \ti{false} the condition might be \ti{true} for some instations of a contained free variable, so the summand will evaluated. 

In this step the \svs\ is constructed. This set contains the set of next state vectors which can be reached from the current \sv\ and for which the condition might be \ti{true}. 

After the \svs\ is constructed, we construct the \nfv . 

%Initally we set \ the size of the set to the number of non false conditions. Each element within  these vectors is set true. Each vector of the \fvs\ corresponds with a vector of \svs . Each element of a vector in the \fvs\ is a comparison between the \sv and the element of the corrresponding vector in the \svs . If a element from the vector in the \svs\ is not equal to an element out of the \sv\ , the corresponding \fvs\ element is set to false. 

%Next we construct a \ti{new} \fv . For each element in this vector we take the conjunction of all the values of the same element in the vectors of the \fvs . Each element of a vector from the \svs\ is checked if they differ from other elements  in vectors of the \svs . The boolean outcome of both results is the \ti{new} \fv . The reason for this second check is needed on the \svs , because free variables might occur. 
 
If the \nfv\ differs from the \fv , a \ti{new} \sv\ has to be created as well. The construction of the \ti{new} \sv\ contains all the elements from the \sv\ substituted with data parameters from the vectors of the \svs\ which are different . 
%These can be found in the \svs . 
A next iteration is started in which the old \sv\ is substituted by the \ti{new} \sv\ and the old \fv\ is substituted by the \nfv .
  
If the \ti{new} \fv\ does not differ from the \fv\ the algorithm ends and all \pps\ which have a corresponding value true in the \fv\ are substituted by their constant values. Then these \pps list and all occurrences of \pp\ 

\subsubsection{Free variables}
In \mcrl\ it is possible to use free variables. %Free variables are values which have no particular value and can therefore be threaded as don't care values.
Let $x$ be a free variable. $x$ can have any value out of domain $X$. However within the \lpe\ the value of free variable $x$ cannot be different from the value of another occurens of $x$.

\begin{example}
Let $x: \mathbb{B}$ be a free variable.
\begin{tabbing}
$P(x) =$ \verb"  " \= $ Q(\urcorner x) + Q(x) $ \\
$Q(x) =$ \> $ x \rightarrow C $\
\end{tabbing}

\end{example}


If a \pp\ is substituted by a free variable and the \pp\ has already got a value, the value remains the same. If a \pp\ , which is a free variable, is substituted by a variable with a specific value, the value is changed to that specific value. If a \pp , which is a free variable, is substituted by another free variable, the \pp\ remains a free variable. 

If we compare a free variable with another value, either a free variable or a specific value, the comparison always yields \ti{true}. Adding a check on the \svs\ solves the problem of the possibility of fake constants. At first this might seems strange, but is contributes to the validity of the algorithm. Let us concider example \ref{exmpl:freevar}:
\begin{example}\label{exmpl:freevar} The LPE: 
\begin{tabbing}
$P(y = \_,x = 0) =$ \= $(true) \rightarrow  P(y := x+1, x := x) + $ \\
	    \> $(true)
 \rightarrow P(y := x+2, x := x)$
\end{tabbing}

A free variable is denoted with $\_$. If we take a look at the LPE and build the \nfv\ we get:
\begin{center}
\begin{tabular}{cc} 
 $\lbrace \langle True, $ & $ True \rangle$\\
 $\langle True, $ & $ True \rangle \rbrace$\\
\end{tabular} 
\end{center}

However the when we build the \svs , we get:
\begin{center}
\begin{tabular}{cc} 
 $\lbrace \langle$ 1, & 0 $\rangle$\\
 $\langle$ 2, & 0 $\rangle \rbrace$\\
\end{tabular}  
\end{center}

If we take the conjunction of the elements of each vector within the \nfv\ the \ti{new} \fv\ will be:
\begin{center}
\begin{tabular}{cc} 
 $\langle True, $ & $ True \rangle$\\
\end{tabular}
\end{center}

However by examining the \svs\ this should be: 
\begin{center}
\begin{tabular}{cc} 
 $\langle False, $ & $ True \rangle$\\
\end{tabular}
\end{center}

This why have an additional check on the \svs\ to overcome this problem.
\end{example}

For further reading about \lpe\ and free variables please refer to \cite{LPEfreevar} 
\newpage
\section{Algorithm} \label{sec:alg}
Not Finished yet
\begin{tabbing} 
StateVector := $\ovr{init}$; \\
FlagVector := $\ovr{true}$; \\
Iteration := \ti{true}; \\
\tb{while} \= Iteration \tb{do} \ra \\
  \> NewStateVectorSet := $\emptyset$;\\
  \> NewFlagVector := FlagVector; \\
  \> \tb{forall} \= $i \in I$ \tb{do} \ra \\
  \>  \>   \tb{if} \= \verb" " \=\\
  \>  \>   \sq \> $(NV(c_i($Statevector$,e_i)) \not= false) \rightarrow$ \\
  \>  \>       \> \>NewStateVectorSet := \\
  \>  \>       \> \> \verb"    " \= NewStateVectorSet $\cup NV(g_i($Statevector$,e_i))$ ;\\  
%  \>  \>       \>  \>NewFlagVector := \\
  \>  \>       \>  \>             $\forall_{j : 0 \leqslant j \leq  n}:$ NewFlagVector$_j$ := \\
  \>  \>       \>  \>   \> NewFlagVector$_j \wedge (NV($Statevector$_j) = NV(g_i^j($Statevector$,e_i))$);\\
%  \>  \>       \> \> \>$\lbrace j \vert NV($Statevector$_j) = NV(g_i^j($Statevector$,e_i))) \rbrace$ \\
  \>  \>  \sq \>  $(NV(c_i($Statevector$,e_i)) = false) \rightarrow$ \\
  \>  \>  \> \>\tb{skip};\\
  \>  \tb{od} \\
  \>  \tb{if}  \\
  \>  \sq  \> (FlagVector $\neq$ Newflagvector) \ra \\ 
  \>       \>  \> $\forall_{j:0 \leqslant j \leq n} :(\exists_{s: s \in \text{NewStateVectorSet}}: ($StateVector$_j \not= s_j \rightarrow $StateVector$_j := s_j$)); \\
  \>       \>  \> FlagVector :=  NewFlagVector; \\
  \>  \sq \>  (flagvector $=$ NewFlagVector) \ra  \\
  \>      \> \> Iteration := \ti{false};\\
  \>  \tb{fi} \\
  \tb{od}\\
\end{tabbing} 

\section{Examples}
This section covers a few simple examples to illustrate the behavior of the algorithm. 
\subsection{Example 1} The example \\
\begin{verbatim}
proc P(i,j: Nat) = true -> P(i:= i + 1, j:=j);
init P(i := 0, j := 0)
\end{verbatim} \footnote{\mcrl specifaction language}

Construct state vector and flag vector: 
\begin{center}\begin{minipage}{250pt}
\sv \la  $\langle 0 , 0 \rangle$\\
\fv \la  $\langle True , True \rangle$\\
\end{minipage}\end{center}

Evaluate all the conditions of all summands.
The condition of the first summand is \ti{true}.\\

We constructed the \svs and the \nfv :
\begin{center}\begin{minipage}{250pt}
\svs \la  $\lbrace \langle 1 , 0 \rangle \rbrace $\\
\nfv \la $\langle False , True \rangle $
\end{minipage}\end{center}

We compare \ti{new} \fv\ with \fv\ and see they are not equal so we continue:
\begin{center}\begin{minipage}{250pt}
\sv \la $\langle 1 , 0 \rangle $\\
\fv \la $\langle False , True \rangle $\\
\end{minipage}\end{center}

In the second iteration the condition of the first summand is \ti{true}.\\
So we construct again the \svs, the \nfv\ : 
\begin{center}\begin{minipage}{250pt}
\svs \la  $\lbrace \langle 2 , 0 \rangle \rbrace $\\
\nfv \la  $\langle False , True \rangle $ \\
\end{minipage}\end{center}

If we compare the \fv\ with \ti{new} \fv\ we see they are equal and the iteration ends.
In this example the second \pp\ (\verb"j") is substituted by \verb"0" and \verb"j" is removed from the \lpe.
After applying the filter we get:\\
\begin{verbatim}
proc P(i) = true -> P(i := i + 1);
Init P(0);
\end{verbatim}

\subsection{Example 2}
This example shows the why iteration is needed.
\begin{verbatim}
proc P(i,j) = P(i := i + 1, j := i);
init P(i := 0, j := 0); 
\end{verbatim}
 
Construct state vector and flag vector: 
\begin{center}\begin{minipage}{250pt}
\sv \la  $\langle 0 , 0 \rangle$\\
\fv \la  $\langle True , True \rangle$\\
\end{minipage}\end{center}

Evaluate all the conditions of all summands.
The condition of the first summand is \ti{true}.\\

We constructed the \svs and the \nfv :
\begin{center}\begin{minipage}{250pt}
\svs \la  $\lbrace \langle 1 , 0 \rangle \rbrace $\\
\nfv \la  $\lbrace False , True \rbrace $\\
\end{minipage}\end{center}

We compare \ti{new} \fv\ with \fv\ and see they are not equal so we continue:
\begin{center}\begin{minipage}{250pt}
\sv \la $\langle 1 , 0 \rangle $\\
\fv \la $\langle False , True \rangle $\\
\end{minipage}\end{center}

In the second iteration the condition of the first summand is \ti{true}.\\
So we construct again the \svs and the \nfv\: 
\begin{center}\begin{minipage}{250pt}
\svs \la  $\lbrace \langle 2 , 1 \rangle \rbrace $\\
\nfv \la  $\langle False , False \rangle $ \\
\end{minipage}\end{center}

We compare \ti{new} \fv\ with \fv\ and see they are not equal so we continue:
\begin{center}\begin{minipage}{250pt}
\sv \la $\langle 2 , 1 \rangle $\\
\fv \la $\langle False , False \rangle $\\
\end{minipage}\end{center}

In the third iteration the condition of the first summand is \ti{true}.\\
So we construct again the \svs and the \nfv\ : 
\begin{center}\begin{minipage}{250pt}
\svs \la  $\lbrace \langle 3 , 2 \rangle \rbrace $\\
\nfv \la  $\langle False , False \rangle $ \\
\end{minipage}\end{center}

If we compare the \fv\ with \ti{new} \fv\ we see they are equal and the iteration ends.
Non of the elements in the \fv\ are \ti{true}, so there are no constants. The \lpe\ remains unchanged.


\newpage
\begin{thebibliography}{99}  \bibitem{LPE_info} auteur\\
   \textit{article},
   Extra info.
  \bibitem{LPEframework} J.W. Wesselink,
   \textit{http://www.win.tue.nl/~wieger/mcrl2/html/index.html}\\
   A C++ wrapper for the ATerm library.
  \bibitem{LPEformat} auteur,
   \textit{article}
   Extra info.
  \bibitem{LPEfreevar} auteur,
   \textit{article}
   Extra info.

\end{thebibliography}

\newpage



\end{document}
